\title{\bf Images}

\section{Basics \& Nomenclature}

Two-dimensional array-based detectors at the focal planes of
telescopes produce images --- two-dimension rasters of values.  Here
we consider the basic properties such images with an emphasis on the
sorts of issues affecting ultraviolet, optical, and infrared imaging
instruments. Many of the principles hold for other sorts of data sets.

These images can be thought of as noisy samplings of an {\it image
function} which is the convolution of the image coming from space with
a response function known as the {\it point spread function}
(PSF). The PSF can have contributions from the Earth's atmosphere, the
telescope's physical and geometric optics, and the detector.

The width of the PSF is usually characterized by its full-width
half-maximum (FWHM). An important property of an imaging system is its
sampling density relative to the FWHM. If this density is high enough,
typically greater than 2 pixels per FWHM, the PSF will be close to
{\it Nyquist sampled}, meaning that if the image were noiseless it
would preserve all of the information in the image function. The
exercises describe the origin of this criterion, which is related to
the true Nyquist sampling criterion of a particular band-limited
function (one with constant Fourier amplitudes out to some
wavenumber). Typically instruments do not truly Nyquist sample the
image function, but in certain circumstances it can be possible.

Images taken outside the Galactic Plane which are not extraordinarily
deep usually allow one to separate the flux into individual objects,
which are generally classifiable as ``point sources'' or ``extended
sources.'' The point sources are those whose images are consistent
with the PSF. With ordinary imaging capability, stars are point
sources. Galaxies are often extended sources, with FWHM larger than
the PSF FWHM. High redshift, luminous quasars usually appear as point
sources, though with high contrast imaging from space their extended
host galaxies can be detected.

The image locations can be quantified by their $x$ and $y$ pixel
values. The center of each object is typically determined by the mode
of the light distribution, or sometimes by a weighted centroid of the
flux distribution (which is less precise but for extended sources may
be more desirable at times). The relationship between these $x$ and
$y$ pixel values and the RA and Dec locations they correspond to is
called the {\it astrometric solution}. Usually the astrometric
solution is expressed in terms of a World Coordinate System
description of the image, which provides the metadata encoding an
approximate projection of the image and distortions in that
projection.

The raw image is usually obtained in some uncalibrated form. In terms
of the specific intensity of the image $I_\nu(\alpha, \delta)$ coming
from outside the Earth's atmosphere convolved with the instrumental
resolution, the raw image in DN$(x,y)$ in pixel coordinates may be
written:
\begin{equation}
\mathrm{DN}(x, y, t) = \mathrm{DN}_{\mathrm{back}}(x, y, t) + 
\int \frac{\dd{\nu}}{\nu} Q(\nu, {\rm alt}, {\rm az}, x, y, t)
\left[I_\nu(\nu, \alpha, \delta) + S_{\nu}(\nu, {\rm alt}, {\rm az},
t)\right]
\end{equation}
$Q$ is a general response function written to depend on the pixels
detecting the objects, the local sky coordinates (though it might have
a more general form in some cases), frequency, and time. $S_{\nu}$ is
background, generally due to sky emission, and written in terms of
altitude and azimuth, to suggest the important axes, but also in terms
of time (e.g. where the Moon is). DN$_{\mathrm{back}}$ is the
instrumental background.

Often the response $Q$ is at least roughly separable as follows
\begin{equation}
Q(\nu) = R(\nu) A({\rm alt}, {\rm az}, \nu) F(x, y)
\end{equation}
where $R(\nu)$ is the instrumental bandpass, $A$ is the atmospheric
throughput, and $F$ is the flat-field. $R$ is often defined to
incorporate most of the dependence of the atmosphere on wavelength, so
that $A$ would express only the residuals. $F$ is often defined to
include the determinant of the Jacobian between $(\alpha, \delta)$ and
$(x, y)$ which would otherwise need to appear and which primarily
depends on the instrument. It also is usually defined to have a mean
around unity.

A final common approximation is that the functions are sufficiently
separable or the dependence on angle is sufficient small over the
field of view of the image that we can write:
\begin{equation}
\mathrm{DN}(x, y, t) = \mathrm{DN}_{\mathrm{back}}(x, y, t) + 
\bar{A} F(x, y) \left(\mu(\alpha, \delta) + \mu_S({\rm alt}, {\rm az},
t)\right)
\end{equation}
We separate DN$_{\rm back}$ and $\mu_S$ because they are from
different physical sources. For example, in a CCD, DN$_{\rm back}$ is
due to dark current and and bias in the device, whereas $\mu_S$ is
emission from the sky itself.

The process of {\it photometric calibration} is the conversion of the
observed DN to $\mu$. In its most basic form, calibration requires
subtracting the instrumental background, scaling the remaining flux by
the overall factor $1/\bar{A}$, dividing by the flat field, and then
subtracting the sky. Typically, the instrumental background is
determined through bias frames taken with the shutter shut.  The flat
field is determined by observing either the sky in twilight or by
observing a screen (sometimes just the dome wall) illuminated such
that it mimics light uniformly entering the telescope aperture.

It is somewhat less standard how $\mu_S$ and $\bar{A}$ are
determined. In most cases, determining either one requires some
detection and measurement of objects in the image to occur, and their
determination is usually somewhat iterative.

For $\mu_s$, usually one takes an initial stab at the background level
(e.g. just a median), detecting the bright objects, and then
redetermining the background and its variation across the image when
excluding ``detected'' pixels; one can then iterate this
procedure. Typically this procedure subtracts not just the atmospheric
sky emission, but also other sources such as zodiacal light, and also
light from stars and galaxies. Whether these other sources should or
should not be subtracted depends on what is wanted out of the image.

For $\bar{A}$, the concept is to use detected objects in the field
whose $\mu$ is known from a catalog of standards to calibrate the
entire image. Sometimes the objects used for calibration are not in
the same field but are objects observed close in time to the field of
interest with the same instrument; in such cases the difference in
airmass of the field and the standard field needs to be accounted
for. Often in these cases the standards are not known in exactly the
same filters as the observations, and these color-dependent effects
must be taken into account. Sometimes there are no standards as such,
but instead a suite of overlapping observations can be calibrated onto
a self-consistent scale.

Once an image is calibrated, the stars, galaxies, or other objects may
be measured for their fluxes and other properties. The fluxes may be
measured through fixed apertures, or one may fit models to the data
and infer fluxes from those models.

\section{Commentary}

The description above of calibration procedures is roughly accurate,
but is more illustrative than anything. There are many different
techniques in use, driven by the nature of the instrument and
observations as well as individual tastes. In part because each
individual experiment is so different, astronomers (including myself)
tend to learn about this subject in the context of a particular set of
observations rather than from a generalized perspective. Keep that in
mind when people (including me!) are telling you how this process
works!

At the highest precision, the calibration effects described above are
not truly separable. For example, the astrometric solution depends on
the definition of the PSF, because that determines what exactly you
consider the location of each object, and the flat-fielding, which can
also affect the centroiding.

A more subtle example is that the flux definitions cannot truly be
separated from the calibration. The calibration must assume some
effective aperture for the fluxes of the calibrating sources, but some
flux will leak outside the aperture, in ways that can vary across the
images used in the calibration. These aperture effects need to be
accounted for for percent level calibration accuracy.

\section{Key References}

\begin{itemize}
  \item
    {\it Design and Construction of Large Telescopes},
      \citet{bely}
  \item
    {\it Kitchin}
\end{itemize}

\citet{gunn06a}

\section{Order-of-magnitude Exercises}

\begin{enumerate} 
\item Proper motion of galaxies 
\end{enumerate} 

\section{Analytic Exercises}

\begin{enumerate}
\item {\it Origin of the Nyquist criterion}. The Nyquist
    sampling criterion is based on the concept of a ``band
    limit.''
    \begin{enumerate}
    \item Show that if the 2D Fourier
    transform $f(\vec{k})$ of the point spread function has zero power
    higher than the band limit $k_{\rm max}$, then it can be perfectly
    described using a discrete Fourier transform that extends only up
    to that $k_{\rm max}$. What is the necessary configuration-space
    sampling for the discrete Fourier transform?
    \item Assume $f(\vec{k}) = {\rm constant}$
    below the band limit, and zero above it. What is the resulting PSF
    $f(\vec{x})$?
    \item Under what circumstances might the criterion in part (b)
    hold, given what you know about how telescopes work?
    \item What is the FWHM of $f(\vec{x})$ in units of the sampling?
    This sets the Nyquist criterion for sampling.
    \end{enumerate}
\item {\it Interpolation}. Imagine starting with an image sampled on a
    rectangular grid. Interpolation is the process of inferring
    the value of an image in between the given sampled points of the
    image. The case of Nyquist sampled, band limited images motivates
    a particular method of interpolation.
    \begin{enumerate}
    \item Explain how linear interpolation can be recast as
    constructing a model of the image using a set of basis functions,
    or ``kernels,'' centered on the original grid points.
    \item Explain why a (noiseless) Nyquist sampled, band limited
    image contains all the information necessary for perfect
    interpolation.
    \item If you interpolate perfectly in that situation, what is the
    effective kernel you are using? Note that this is called {\it
    sinc interpolation}.
    \item Explain why using exactly that interpolation kernel might be
    problematic.
    \end{enumerate}
\item Centroiding accuracy
\item Noise models
\end{enumerate}

\section{Numerics and Data Exercises}

\begin{enumerate}
\item {\it Interpolation}. Create an image of a critically sampled
    Gaussian PSF. Create another image of the same PSF, but with an
    offset that is some fraction of a pixel. We will test how well
    different types of interpolation work by trying to generate the
    offset by interpolating the first image and comparing it to the
    second image.
    \begin{enumerate}
     \item Test linear interpolation. 
     \item Test sinc interpolation.
     \item Test a ``damped'' version sinc interpolation, which
    multiplies the kernel by a broad Gaussian.
    \end{enumerate}
\item SDSS Calibration model
\item WCS headers
\item Centroiding method
\item Bias from using data as the noise estimate
\item Quick and dirty sky noise estimates
\end{enumerate}

\bibliographystyle{apj}
\bibliography{exex}  
