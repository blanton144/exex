\title{\bf Images}

\section{Basics \& Nomenclature}

Two-dimensional array-based detectors at the focal planes of
telescopes produce images. These images can be thought of as noisy
samplings of an {\it image function} which is the convolution of the
image coming from space with a response function known as the {\it
  point spread function} (PSF). The PSF can have contributions from
the Earth's atmosphere, the telescope's physical and geometric optics,
and the detector.

The width of the PSF is usually characterized by its full-width
half-maximum (FWHM). An important property of an imaging system is its
sampling density relative to the FWHM. If this density is high enough,
typically greater than 2 pixels per FWHM, the PSF will be close to
{\it Nyquist sampled}, meaning that if the image were noiseless it
would preserve all of the information in the image function.



\section{Commentary}

We will discuss spectra later. 

\section{Key References}

\begin{itemize}
  \item
    {\it Design and Construction of Large Telescopes},
      \citet{bely}
  \item
    {\it Kitchin}
\end{itemize}

\citet{gunn06a}

\section{Order-of-magnitude Exercises}

\begin{enumerate} 
\item Typical read noise in IR arrays vs optical and sampling
\item X-ray background
\end{enumerate} 

\section{Analytic Exercises}

\begin{enumerate}
\item Nyquist
\end{enumerate}

\section{Numerics and Data Exercises}

\begin{enumerate}
\item Sky brightness vs airmass during SDSS drift scan
\item OH variations during SDSS drift scan
\item Plot sky spectrum vs moon phase from SDSS
\item OH variations across SDSS fields
\item Find extreme airmass SDSS images, see refraction
\end{enumerate}

\bibliographystyle{apj}
\bibliography{exex}  
