\title{\bf Detectors}

\section{Basics \& Nomenclature}

Modern astronomical detectors come in several varieties depending on
wavelength and use case. The traditional detector introduced in the
1800s, the photographic plate, is too inefficient ($\sim$ 5\%) and too
inconvenient to convert into digital form to be of use today, though
it continued to be used until about 2000. Here, we will not review the
physics behind modern detectors except insofar as necessary to explain
how to interpret data taken with them.

In the optical, out to about a micron, the most common detector in use
is the {\it charge-coupled device} (CCD). Out to about 28 $\mu$m, the
most common detectors are infrared detector arrays. Although CCD and
infrared arrays are different technologies, both are
semiconductor-based detectors that detect photons of energy higher
than their band gap between their bound valence electron energies and
their conduction band. Silicon is the best-developed CCD technology
and has a band gap of 1.1 $\mu$m.  Germanium CCDs could in principle
extend this sensitivity to $1.8$ $\mu$m, but their technology is not
cost-effective today. HgCdTe (``MER-CAH-TEL'') infrared detectors have
a band gap of that can be designed anywhere between 0.4--12
$\mu$m. Si:As detectors can extend to 28 $\mu$m (and are for example
used in the two redder WISE bands).

Each pixel of a CCD detects individual photons that hit it, each of
which usually contributes one electron to the overall signal. For most
hardware, the charge is {\it read out} at amplifiers along the side of
the CCD, by transferring the charge from pixel to pixel to the edge of
the CCD. At the last pixel the charge is converted to a digital signal
by the amplifiers. CCDs may be readout by various numbers of
amplifiers. Because of the finite temperature of the devices, some
number of electrons are released even when there are no photons
entering the device; this phenomenon is known as the {\it dark
current}. To reduce the dark current, CCD detectors need to be cooled;
variations in temperature can cause variations in the dark current.

Infrared detector arrays operate similarly, but the infrared-sensitive
materials cannot today be used to support the circuits necessary for
CCD operation. Instead, the infrared sensitive material is bonded to a
silicon detector on which the read out occurs. These detectors are not
CCDs but multiplexers, which are read directly out on each pixel. The
charge can be read out nondestructively, allowing many reads on the
same pixel. However, the read out is intrinsically noisier.

From this basic understanding, there results an interpretation of the
digital numbers reported by a semiconductor device:
\begin{equation}
\mathrm{DN} =
\frac{n_e}{\mathrm{Gain~}} = 
\frac{n_p + \mathrm{Dark}}
{\mathrm{Gain~}}
\end{equation}
where $n_e$ is the number of electrons recorded by the device, $n_p$
is the number of photons actually detected, i.e. which are converted
to electrons (which will include all background photon sources). Gain
represents the {\it gain}, the number of electrons per DN reported by
the electronics. Dark represents the dark current in units of
electrons.

The noise in the DN is due primarily to two sources: Poisson noise
around the mean $n_e$ and {\it read noise} associated with the
electronics. The read noise does not usually depend on the
signal. Therefore:
\begin{equation}
\sigma_{\rm DN}^2 = \frac{n_e}{\mathrm{Gain}^2} + \mathrm{Read~Noise}^2
\end{equation}
The Poisson noise includes the ``object'' signal, the background
signal, and the dark current contributions to the expected number of
electrons $n_e$.

Over their usuable dynamic range in number of electrons, CCD devices
are remarkably but not perfectly linear.  CCD devices are limited in
their dynamic range by the number of electrons each pixel can
store. If the number of electron approaches or exceed this limit,
those electrons typically bleed to neighboring pixels; since the
electronics is not isotropic, they typically bleed along the same
direction the device is read out. They also may be limited by the
dynamic range of their analog-to-digital converter, which is often
16-bits.

CCDs are also sensitive to cosmic rays and these will release
electrons that contribute to the signal. A characteristic feature of a
cosmic ray is that it will be a sharper feature than the atmosphere
and optics allow. The exact nature of the cosmic ray distribution
depends on altitude (with more reaching higher altitude and space
detectors) and orientation of the detector relative to vertical (since
underneath the atmosphere the cosmic rays will be directed
preferentially downwards).

In the ultraviolet, other detectors are still in use. For example,
GALEX uses a position sensitive proportional counter. These devices
are essentially a crossed grid of wires under voltage inside a chamber
filled with an inert gas that can be ionized by UV photons. When a UV
photon causes a charge track, the voltage change is detected and
recorded. Unlike CCDs, these detectors can detect individual photons.

In the X-rays, CCD detectors are now typically used. CCDs lose
sensitivity in the UV due to absorption on their surfaces, but at
energies $> 120$ eV ($<100$ \AA) they become sensitive again. Unlike
in the UV and optical, X-ray photons can release many electrons. This
fact, and that the photons much rarer, allows X-ray CCDs to be energy
sensitive. At high count rates, they become complicated to analyze due
to latency effects; for some period of time after a photon is detected
subsequent photons cannot be.

In the low frequency radio, antennae of various sorts are generally
used as detectors, and the signal is read out as a changing
voltage. Radio detectors therefore preserve the phase information in
the electromagnetic signal, unlike higher frequency instruments. This
allows interoferometry on very wide baselines to achieve very high
angular resolution (low $\lambda/D$). In the higher frequency radio
(sub-mm), superconductor-insulator-superconductor (SIS) devices are
used, which are photon-sensing and so do not preserve the phase.  In
both regimes, feedhorns in the focal plane of the telescope are often
used to match the focal plane to the antennae. In some configurations,
antennae are used in arrays without any focusing element, and images
are reconstructed through cross-correlating in the resulting
signals. Noise in radio instruments is often dominated by thermal
noise due to the finite temperature of the receiver electronics.

No detectors capture all of the photons hitting each pixel. The
fraction of photons detected is known as the {\it quantum efficiency}
of the device. It is a function of wavelength (most notably near the
band gap). CCD and infrared devices have quantum efficiencies that are
typically quite high, e.g. up to 95\%.

\section{Commentary}

CCDs only became widely used in astronomy until the mid-1990s. Prior
to that time most imaging and spectroscopy was performed with
photographic plates. Photographic plates have only $\sim 5$\% quantum
efficiency typically, and obviously are harder to translate into
digital results. However, they remained competitive for wide field
imaging surprisingly long, because the high cost of CCDs kept them
from being used to completely cover the focal planes of wide field
telescopes.

\section{Key References}

\begin{itemize}
  \item
    {\it Design and Construction of Large Telescopes},
      \citet{bely03a}
  \item
    {\it Astrophysical Techniques},
      \citet{kitchin09a}
\end{itemize}

\section{Order-of-magnitude Exercises}

\begin{enumerate} 
\item Modern CCDs have read noise can be as low as 2 electrons,
    whereas infrared detectors often have read noise of $\sim 20$
    electrons. How many independent reads would have to be performed
    of an infrared detector to beat down the read noise to compete
    with CCD read noise?
\item The cosmic X-ray background at 1 keV is about $3\times 10^{-11}$
    W m$^{-2}$ sr$^{-1}$ (\citealt{fabian92a}). The Chandra X-ray
    telescope often takes exposures of $\sim 10^4$ s, and has a
    resolution of order an arcsec. What is the expected number of
    background photons within the a FWHM of a source that are
    contributed by this background for such an exposure?
\item What is the highest angular resolution that can be achieved by a
    ground-based interferometric radio experiment at $\sim 1.6$ GHz?

\begin{answer}[Author: Trey Jensen]
In interferometry, one can use aperture synthesis to create a larger
baseline aperture than traditional telescopes by placing an array of
multiple telescopes. The distance between any pair of two telescopes,
the baseline, sets the diffraction limit for images produced by
correlating the phase signal between the telescopes.  The diffraction
limit is $\theta\sim \frac{\lambda}{D}$ and the largest possible
distance we can separate two ground-based telescopes is of order the
diameter of the Earth $D\sim 10^7$ meters. For 1.6 GHz light, we have
an angular resolution of
\begin{align*}
    \theta\sim \frac{3\times 10^8}{(1.6\times 10^9)(10^7)}\sim 2\times 10^{-8} \textrm{ rad}\sim 4\times 10^{-3} \textrm{ arcseconds}.
\end{align*}
\end{answer}
\end{enumerate} 

\section{Analytic Exercises}

\begin{enumerate}
\item The detected electrons contributing to the noise include the
object, background, and dark current signal. Write the fractional
error in the object counts. Separate the terms involving the other
sources of noise. For low object signal, how does the fractional error
depend on signal? For high object signal, how does the fractional
error depend on signal?

\begin{answer}[Author: Dou Liu]
The noise $N$ includes the Poisson noise in the object, sky, dark
counts ($n_\mathrm{obj}$, $n_\mathrm{sky}$, and $n_\mathrm{dark}$),
added in quadrature with each other and the read noise
$N_\mathrm{read}$:
\begin{equation}
\label{ }
N
= \sqrt{N_\mathrm{obj}^2+N_\mathrm{sky}^2+N_\mathrm{dark}^2+N_\mathrm{read}^2}=\sqrt{n_\mathrm{obj}+n_\mathrm{sky}+n_\mathrm{dark}+N_\mathrm{read}^2}
\end{equation}
The fractional error in the object counts is:
\begin{equation}
\label{ }
\frac{N}{n_\mathrm{obj}}=\frac{\sqrt{n_\mathrm{obj}+n_\mathrm{sky}+n_\mathrm{dark}+N_\mathrm{read}^2}}{n_\mathrm{obj}}
= \sqrt{\frac{1}{n_\mathrm{obj}}\left(1+
\frac{n_\mathrm{sky}+n_\mathrm{dark}+N_\mathrm{read}^2}{n_\mathrm{obj}}\right)}
\end{equation}
Defining:
\begin{equation}
\sigma_\mathrm{other}^2 =
n_\mathrm{sky}+n_\mathrm{dark}+N_\mathrm{read}^2,
\end{equation}
we find that if $n_\mathrm{obj}\gg \sigma_\mathrm{other}^2$ (the
object-dominated regime), then
\begin{equation}
\frac{N}{n_\mathrm{obj}} \approx \frac{1}{\sqrt{n_\mathrm{obj}}}.
\end{equation}
If $n_\mathrm{obj}\ll \sigma_\mathrm{other}^2$ (the
background-dominated regime), then
\begin{equation}
\frac{N}{n_\mathrm{obj}} \approx \frac{\sigma_\mathrm{other}}{n_\mathrm{obj}}.
\end{equation}
Therefore, for the faintest fluxes (below the background) the
fractional errors are a stronger function of the flux than they are at
brighter fluxes (above the background).
\end{answer}
\item The noise in a data set is determined by the Poisson noise in
the expectation value of the signal. However, observationally we often
only have access to one realization of the signal, and often the
quoted errors are based on the Poisson noise in that signal.  If we
take multiple observations and combine them together weighting by the
inverse variance estimated in this way, it leads to a bias. Ignoring
any background contribution to the noise, estimate this bias as a
function of $n_e$.
\end{enumerate}

\section{Numerics and Data Exercises}

\begin{enumerate}
\item CCD images from SDSS are available
as \href{https://www.sdss.org/dr14/imaging/images/#corr}{corrected
frames}. Those images are calibrated in physical units, and have the
sky background subtracted. Read the documentation there, including the
data model. Download a corrected frame in the $r$-band. Use the
formulae in the documentation to translate the image into uncalibrated
DN units with the sky background reinstated (the files have all the
information necessary to do that). Is it all integers?  Use the result
to estimate the noise per pixel in a region without stars or
galaxies. Can you independently estimate that noise from the
distribution of flux values themselves, and do the estimates agree?
\end{enumerate}

\bibliographystyle{apj}
\bibliography{exex}  
