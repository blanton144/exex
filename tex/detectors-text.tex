\title{\bf Detectors}

\section{Basics \& Nomenclature}

Modern astronomical detectors come in several varieties depending on
wavelength and use case. The traditional detector introduced in the
1800s, the photographic plate, is too inefficient ($\sim$ 5\%) and too
inconvenient to convert into digital form to be of use today, though
it continued to be used until about 2000. Here, we will not review the
physics behind modern detectors except insofar as necessary to explain
how to interpret data taken with them.

In the optical, out to about a micron, the most common detector in use
is the {\it charge-coupled device} (CCD). Out to about 28 $\mu$m, the
most common detectors are infrared detector arrays. Although CCD and
infrared arrays are different technologies, both are
semiconductor-based detectors that detect photons of energy higher
than their band gap between their bound valence electron energies and
their conduction band. Silicon is the best-developed CCD technology
and has a band gap of 1.1 $\mu$m.  Germanium CCDs could in principle
extend this sensitivity to $1.8$ $\mu$m, but their technology is not
cost-effective today. HgCdTe (``MER-CAH-TEL'') infrared detectors have
a band gap of that can be designed anywhere between 0.4--12
$\mu$m. Si:As detectors can extend to 28 $\mu$m. {\bf check what WISE
is}.

Each pixel of a CCD detects individual photons that hit it, each of
which usually contributes one electron to the overall signal. For most
hardware, the charge is ``read out'' at amplifiers along the side of
the CCD, by transferring the charge from pixel to pixel to the edge of
the CCD. At the last pixel the charge is converted to a digital signal
by the amplifiers. CCDs may be readout by various numbers of
amplifiers. Because of the finite temperature of the devices, some
number of electrons are released even when there are no photons
entering the device; this phenomenon is known as the {\it dark
current}. To reduce the dark current, CCD detectors need to be cooled;
variations in temperature can cause variations in the dark current.

Infrared detector arrays operate similarly, but the infrared-sensitive
materials cannot today be used to support the circuits necessary for
CCD operation. Insted, the infrared sensitive material is bonded to a
silicon detector on which the read out occurs. These detectors are not
CCDs but multiplexers, which are read directly out on each pixel. The
charge can be read out nondestructively, allowing many reads on the
same pixel. However, the read out is intrinsically noisier.

From this basic understanding, there results an interpretation of the
digital numbers reported by a semiconductor device:
\begin{equation}
\mathrm{DN} =
\mathrm{Gain~} n_e =
\mathrm{Gain~} \left[\times n_p + \mathrm{Dark} \right]
\end{equation}
where $n_e$ is the number of electrons recorded by the device, $n_p$
is the number of photons actually detected, i.e. which are converted
to electrons (which will include all background photon sources),
``Gain'' represents the conversion of photons to the DN reported by
the electronics, and ``Dark'' represents the dark current in units.

The noise in the DN is due primarily to two sources: Poisson noise
around the mean $n_e$ and {\it read noise} associated with the
electronics. The read noise does not usually depend on the
signal. Therefore:
\begin{equation}
\sigma_{\rm DN}^2 = \mathrm{Gain}^2 n_e + \mathrm{Read~Noise}^2
\end{equation}
The Poisson noise includes the ``object'' signal, the background
signal, and the dark current contributions to the expected number of
electrons $n_e$.

Over their usuable dynamic range in number of electrons, CCD devices
are remarkably but not perfectly linear.  CCD devices are limited in
their dynamic range by the number of electrons each pixel can
store. If the number of electron approaches or exceed this limit,
those electrons typically bleed to neighboring pixels; since the
electronics is not isotropic, they typically bleed along the same
direction the device is read out. They also may be limited by the
dynamic range of their analog-to-digital converter, which is often
16-bits.

CCDs are also sensitive to cosmic rays and these will release
electrons that contribute to the signal. A characteristic feature of a
cosmic ray is that it will be a sharper feature than the atmosphere
and optics allow. The exact nature of the cosmic ray distribution
depends on altitude (with more reaching higher altitude and space
detectors) and orientation of the detector relative to vertical (since
underneath the atmosphere the cosmic rays will be directed
preferentially downwards).

In the ultraviolet, other detectors are still in use. For example,
GALEX uses a position dependent proportional counter. These devices
are essentially a crossed grid of wires under voltage inside a chamber
filled with an inert gas that can be ionized by UV photons. When a UV
photon causes a charge track, the voltage change is detected and
recorded. Unlike CCDs, these detectors can detect individual photons.

In the X-rays, CCD detectors are now used. CCDs lose sensitivity in
the UV due to absorption on their surfaces, but at energies $> 120$ eV
($<100$ \AA) they are sensitive again. Unlike in the UV and optical,
X-ray photons can release many electrons. This fact, and that the
photons much rarer, allows X-ray CCDs to be energy sensitive. At high
count rates, they become complicated to analyze due to .

\section{Commentary}

We will discuss spectra later. 

\section{Key References}

\begin{itemize}
  \item
    {\it Design and Construction of Large Telescopes},
      \citet{bely}
  \item
    {\it Kitchin}
\end{itemize}

\citet{gunn06a}

\section{Order-of-magnitude Exercises}

\begin{enumerate} 
\item Typical read noise in IR arrays vs optical and sampling
\item X-ray background
\end{enumerate} 

\section{Analytic Exercises}

\begin{enumerate}
\item Noise as a function of object and background 
\item Bias in estimating noise from data
\end{enumerate}

\section{Numerics and Data Exercises}

\begin{enumerate}
\item Sky brightness vs airmass during SDSS drift scan
\item OH variations during SDSS drift scan
\item Plot sky spectrum vs moon phase from SDSS
\item OH variations across SDSS fields
\item Find extreme airmass SDSS images, see refraction
\end{enumerate}

\bibliographystyle{apj}
\bibliography{exex}  
