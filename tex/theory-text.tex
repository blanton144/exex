\title{\bf Overview of Galaxy Formation Theory}

\section{Basics}

\subsection{General Story}

Galaxies form through the nonlinear gravitational collapse of dark
matter halos. Baryons participate in this collapse with the dark
matter, but unlike dark matter are able to radiate energy away and
therefore sink deeper into the potential well. They cannot radiate
away angular momentum, and they experience gas pressure; this
combination leads ultimately to the formation of rotating gas
disks. In this process, pockets of the gas cool, becoming neutral or
molecular. Inside the molecular regions, individual stars collapse and
cool. Through stellar processes, these stars enrich the interstellar
medium and return energy to it through feedback. At the centers of the
galaxies, black holes form and grow, and also exert feedback on the
gas. These feedback effects may help determine the star formation
rates of the forming galaxies. As these events occur, galaxies
experience further accretion and major mergers. Although there is not
an agreed-upon quantitative understanding of this whole process, there
is strong evidence that the dark matter halos with masses near
$10^{12}$ $M_\odot$.

Although this basic story is known, many of its details are unclear
and it is not known if the story is complete. The current theories of
galaxy formation do a good job reproducing the gross properties of
galaxies near $z\sim 0$. It remains to be seen if they consistently
predict the differences in the galaxy population at higher redshift,
more detailed properties of galaxies such as their mass profiles,
stellar kinematics, and internal chemical patterns, and their gaseous
environments.

\subsection{Halos and Galaxies}

An overarching challenge for physical galaxy formation theory can be
expressed as the relationship between halo mass and stellar mass of
galaxies. Approximately, this relationship expresses how effiently the
matter that fell into a halo was converted into star.

This relationship can be observationally constrained through {\it
  abundance matching}. Abundance matching assumes that halo mass and
stellar mass are at least approximately monotonically related. For any
halo mass $M_h$ for which the number density of halos above that mass
is predicted to be $\Phi_h(>M_h)$, one can find a corresponding
stellar mass $M_\ast$ for which the real universe is observed to have
$\Phi_\ast(>M_\ast) = \Phi(>M_h)$, and conclude that halos of mass
$M_h$ host galaxies with stellar mass $M_\ast$. There are numerous
refinements of this technique that account for scatter in the
relationship between $M_h$ and $M_\ast$, that use subhalos rather than
halos, that use circular velocity instead of $M_h$.

The other basic tool for understanding this relationship is the {\it
  halo occupation distribution} model. This model connects $N$-body
predictions of dark matter halo properties to observed galaxy
properties through $P(>M_\ast | M_h)$, the probability distribution of
the number of galaxies above stellar mass $M_\ast$ within a halo of
mass $M_h$. Many refinements of this model exist, the most important
of which is the distinction between {\it central galaxies} in a halo
and {\it satellite galaxies}. The framework also can be used to study
galaxies of different classes or properties, and as a function of
different environments around each halo.

\citet{wechsler18a} review the literature of the galaxy-halo
connection, including abundance matching and halo occupation
distributions. The major conclusion from these studies is that
$M_\ast/M_h$ peaks at $M_h\sim 10^{12}$. This ratio rises linearly or
even more steeply between $10^{10}$ and $10^{12}$ $M_\odot$ and
declines as $\sim M_h^{-1/2}$ at higher masses. These conclusions are
validated by using them to predict observations of weak lensing and
galaxy correlation functions. They can be further used to study the
scatter in the relationship between $M_h$ and $M_\ast$, how central
and satellite galaxies differ in their star formation histories, and
how these properties depend on environment.

These results outline a basic challenge of galaxy formation theories,
which is to explain how the efficiency of star formation depends on halo
mass.

\subsection{Physical Processes}

Theoretical models seek to explain galaxy-halo relation and the other
properties of galaxies from, most ambitiously, a first-principles
approach beginning with pattern of matter fluctuations at the time of
recombination. \citet{somerville15a} outline the major physical
processes at play:
\begin{itemize}
  \item Gravity, which drives gravitational growth leading to the
    collapse and clustering of dark matter halos. 
  \item Hydrodynamics, which controls the flow of the baryons and
    produces shocks. 
  \item Thermal processes, which control the cooling of gas and thus
    how it will flow into galaxies.
  \item Star formation, which in the context of galaxy formation
    models means how cool gas proceeds to fragment and form dense
    cloud cores that lead to individual stars, and which may affect
    the subsequent evolution of the galaxy through feedback due to
    stellar winds and supernovae.
  \item Black hole formation and growth, which occurs due to gas
    inflow to the very centers of galaxies, and which may affect the
    subsequent evolution of the galaxy through AGN feedback.
  \item Nucleosynthesis, which leads to chemical enrichment of
    interstellar gas, which affects the thermal processes because of
    the importance of metal cooling.
  \item Radiative transfer, which can heat and cool gas, as well as
    affect the observed nature of the galaxies.
\end{itemize}

In galaxy formation theories, gas cooling, inflow, and feedback play
critical roles, and as noted below can only be modeled through {\it
subgrid} physics---i.e. not from first principles. What simulations
can predict from first principles are the effects of gravity on
nonlinear collapse, the cooling of gas on large scales, and the flows
of gas.  However, these first principles calculations are still
limited by resolution, which may lead to qualitatively important
errors even on large scales.

\subsection{Methodology}

Simulations of galaxy formation utilize N-body simulations,
hydrodynamic simulations, and prescriptions for subgrid
physics, or rely on approximations to those simulations known as {it
semianalytic models}.

As used in the cosmological literature, the term {\it N-body
simulation} refers to a purely gravitational simulation of the
collisionless Boltzmann equation, approximated with particle
methods. These simulations nominally simulate pure cold dark matter
models, with no baryons. Traditional N-body simulations do not solve
the collisionless Boltzmann equation fully, but use particles to
sample the density field and study the gravitational interactions of
the particles. Because the number density of particles in the
simulation is far below the number density of expected cold dark
matter particles, if the gravitational interactions were solved
exactly they would lead to unrealistically short two-body relaxation
times in the simulations. Therefore, the gravitational force law needs
to be {\it softened} at some length.

To follow the evolution of baryons, N-body simulations are not
sufficient, because baryonic gas is a fluid, and also can radiate and
cool. Furthermore, a fraction of the baryonic gas can form stars
(which are dense enough that they behave like collisionless
particles). Cosmological {\it hydrodynamic simulations} include these
effects, but are correspondingly more expensive. Hydrodynamic
simulations come in two basic types: particle- and mesh-based. {\it
Smoothed particle hydrodynamics} methods (\citealt{springel10a}) treat
fluids as a set of particles carrying information on the gas
state. Mesh-based methods treat fluids as continuous
fields. From the point of view of galaxy formation, both types of
methods have developed sufficiently to agree with each other, and the
challenges in simulations are not thought to be dominated by this
methodological choice.

In both hydrodynamic and N-body simulations, there is a need for a
large dynamic range, which drives us to consider high resolution. On
the other hand, at late times much of the volume is nearly empty and
does not require as detailed simulation. Therefore, simulators are
driven to consider {\it adaptive resolution} techniques to focus
computational resources (grid cells and particles) in the regions of
greatest interest. Another technique is to run a large volume
simulation and choose a subset of regions to rerun with higher
resolution as a {\it zoom-in simulation}, with initial and boundary
conditions set by the original simulation.

Cosmological-scale hydrodynamic simulations do not model the physics
of the interstellar medium, star formation, and feedback directly. The
very highest resolution simulations reach about 100 pc in resolution,
and therefore barely resolve gaseous galactic disks. Therefore
typically the radiative cooling models used are those appropriate for
diffuse, collisionally ionized gas; with this cooling curve dense gas
is stable at $10^4$ K, but the additional cooling processes on
smaller-than-resolved scales that lead to molecular cloud and star
formation are not included. Nor are the subsequent processes that lead
to chemical evolution and feedback from stars.

The transition from dense, cold gas to stars, and subsequent feedback
on the interstellar medium from stellar processes, is therefore is
typically treated with {\it subgrid physics} modeling. Differences in
the subgrid physics are the dominant source of disagreement between
predictions for galaxy evolution from numerical simulations.

Subgrid physics needs to account for the multi-phase nature of the
interstellar medium; this multi-phase nature means that it does not
necessarily behave as a simple gasr In most cosmological simulations
this is taken into account in terms of empirically constrained
effective equations of state and rules for the production of molecule
gas.  To model star formation, typically simulators assume that gas
above some threshold density and that is converging
($\nabla\cdot\vec{v} < 0$), will form stars at a rate set by the
free-fall time:
\begin{equation}
\dot\rho_\ast = \frac{\epsilon_\ast \rho_{\rm gas}}{t_{\rm ff}}
\end{equation}
Since $t_{\rm ff} \propto \rho^{-1/2}$, this formulation has a
Kennicutt-Schmidt-type relationship built into it. $\epsilon_\ast$ is
observed in molecular clouds to be relative small ($\sim 0.01$)
probably due to turbulence well below the simulation resolution
scales, in molecular clouds and generated by stellar processes.

The subgrid physics also must account for the feedback from stellar
and AGN processes, both of which have been observed to cause outflows
and heating under the right circumstances. The stellar processes of
greatest effect should be supernovae, which should drive blast waves
into the interstellar medium, which both imparts momentum and heats
the gas. There is not a good theoretical understanding that predicts
bottom-up how these effects combine to drive a wind, and there is not
an agreed-upon way to implement an effective low resolution subgrid
model which behaves as bottom-up models would predict. Generally the
parameters are tuned with a wind velocity $v_{\rm wind}$ and a mass
loading factor $\eta = \dot M_{\rm wind}/{\rm SFR}$ that scales with
the galaxy velocity dispersion (i.e. depth of the potential well) as
$\eta\propto \sigma^{-\alpha}$ with $\alpha \sim 1$--2. $\alpha\sim 1$
is appropriate for {\it momentum-driven winds}, in which the gas cools
quickly relative to the dynamical time, and $\alpha\sim 2$ is
appropriate for {\it energy-driven winds}, in which the outflow is
driven by thermal heating that does not quickly radiate away.

Accreting supermassive black holes also can provide feedback. {\it
Radiative mode} feedback (associated with radiatively efficient
accretion onto the black hole) heats the gas up and ionizes it, and
can drive winds through radiation pressure (for example, coupling to
atomic lines, electrons, or dust).  Again, it is unclear how this
feedback should act exactly, or whether it is better approximated by a
momentum-driven or energy-driven model. {\it Jet mode} or {\it radio
mode} feedback is provided by the AGN jet, most often present without
the radiatively efficient accretion. The kinetic energy in the jets
are higher than the bolometric luminosity of the AGN. Although there
are hot bubbles observed in X-rays associated with some jets, which
indicate enough heating to offset cooling, the exact way the energy is
coupled remains unclear. To model all of this, simulations follow
parametrized models for black hole accretion and growth, depending on
the gas density and artificially seeding haloes with black holes.

All of the processes described above can also be followed with {\it
semianalytic models} (or SAMs), which have the advantage of being
faster and therefore allow more rapid testing of hypotheses and
fitting of parameters to observables. Semianalytic models start
typically with halo merger histories from N-body simulations (or more
rarely these days, from excursion set modeling). The merger histories
are then used to infer gas accretion and galaxy merger
histories. Within that context, a similar set of rules for star
formation, feedback, and quenching is applied.

Within any of these methods, nucleosynthesis along with stellar mass
loss and supernovae lead to chemical evolution. The enrichment is
modulated by pristine gas infall, and also chemically enriched
outflows (enriched by the very supernovae that cause the outflow).

\subsection{Features of Theoretical Predictions}

The implications of the comparison of the theoretical models with
observations is in flux, and much is not understood. However, there
are a number of results which have proven robust over many years and
are useful to understand, as they have shaped our understanding of the
ingredients needed for galaxy formation

The cooling of gas in halos as they formed is predicted to be highly
efficient, especially when metal cooling is accounted for, which
greatly enhances the gas cooling rates at halo temperatures. If this
cooling were not balanced by some process that prevented star
formation, it would lead to far more star formation at all halo masses
than observed. It is this {\it overcooling} problem, recognized even
in the 1980s before the Cold Dark Matter was fully established, that
motivates the extensive effort in modeling feedback.

In the simulations, the accretion mode of gas also is strongly
affected by the details of the cooling. In particular, when the
cooling time is short relative to the free-fall time, {\it cold mode
accretion} occurs directly down to the disk of a galaxy; in
simulations this tends to occur along streams fed by larger scale
filametnts. When the cooling time is long relative to the free-fall
time, a hot gas halo is formed that grows through {\it hot mode
accretion}. Gas falling in forms shocks near the edge of the halo, and
the system gradually cools. Generally larger mass halos ($>10^{12}
M_\odot$ experience hot mode accretion, and lower mass halos
experience cold mode accretion. 

In whatever manner gas cools, it leads to a decrease in its orbital
energy. This leads the gas to fall to the center, but if angular
momentum is conserved, it cannot fall all the way in. If we knew how
far the gas could fall in (i.e. its angular momentum) we could predict
the sizes of the disks that would form. We start by assuming the
specific angular momentum of baryonic material is the same as the dark
matter, and it cannot be efficiently transported outwards. The net
specific angular momentum $j_{\rm DM}$ of the dark matter in forming
halos is produced by tidal torques on forming halos, and can be
predicted from N-body simulations, and can be reexpressed as a
fraction of the average specific angular momentum based on the virial
quantities:
\begin{equation}
\lambda = \frac{j_{\rm DM}}{\sqrt{2} R_{\rm vir} v_{\rm vir}}.
\end{equation}
Typically $\lambda \sim 0.03$--$0.04$ in simulations. One can easily
calculate that for a flat rotation curve galaxies with an exponential
disk of scale length $R_d$, the specific angular momentum is $j_\ast =
2 R_d v_{\rm flat}$. If we set $j_{\rm DM} = j_\ast$ then we can solve
for $R_d$:
\begin{equation}
R_d = \frac{1}{\sqrt{2}} \lambda \left(\frac{v_{\rm vir}}{v_{\rm
flat}}\right) R_{\rm vir}
\end{equation}
Clearly for $R_{\rm vir} \sim 100$ kpc, $R_d$ will be a few kpc, as we
find for actual galaxies, indicating that angular momentum is roughly
conserved.

The formation of elliptical galaxies and spheroids and explaining
their scaling relations is more complex. The paradigm for a long time
has been that major mergers of disk galaxies can form elliptical
galaxies, which can grow by minor mergers with gas-rich or gas-poor
dwarf galaxies. But simulations find it is hard to prevent the
formation of a disk subsequently; this finding motivates the need to
presume that gas is heated or removed prior to the mergers. In
addition, it now appears possible that elliptical galaxies formed
largely in situ, driven by violent disk instabilities that drove gas
inward and formed a bulge; this would explain why ellipticals are
present at high redshift ($z\sim 2$) even though the progenitor
spirals that are required in the merger scenario are clearly very gas
rich at those redshifts.

In these massive systems, AGN feedback is essential to explaining why
their stellar-to-halo mass ratios are so low. In lower mass
systems, the low ratios cannot be due to AGN feedback, but supernovae
feedback may play a role. Dwarf galaxies appear to have had a nearly
constant, very low efficiency star formation, but simulations with
supernova feedback tend to have declining star formation with time,
with much lower star formation rates today than observed.

Theoretical models also of course can predict the variation of
properties with environment, the variation of metallicity with mass
and within galaxies, and many other observable properties. This is a
rich area of research with much remaining to be learned.

%\section{Commentary}
%
%[Successes and challenges]

% \section{Important numbers}

% \begin{itemize}
% \item 
% \end{itemize}

\section{Key References}

\begin{itemize}
  \item
    \href{https://ui.adsabs.harvard.edu/abs/2015ARA%26A..53...51S/abstract}
    {{\it Physical Models of Galaxy Formation in a Cosmological
        Framework} (\citealt{somerville15a})}
  \item
    \href{https://ui.adsabs.harvard.edu/abs/2018ARA%26A..56..435W/abstract}
      {{\it The Connection Between Galaxies and Their Dark Matter
          Halos} (\citealt{wechsler18a})}
\end{itemize}

\section{Order-of-magnitude Exercises}

\begin{enumerate} 
\item For a typical Milky Way sized halo, how many particles must it
    be modeled with for the two-body relaxation time to be longer than
    the age of the universe.
\end{enumerate} 

\section{Analytic Exercises}

\begin{enumerate}
\item Angular momentum and size
\item Minor merger growth
\end{enumerate}

\section{Numerics and Data Exercises}

\begin{enumerate}
\item Comparing halo and stellar mass function.
\item Dealing with a numerical simulation result
\end{enumerate}

\bibliographystyle{apj}
\bibliography{exex}  
