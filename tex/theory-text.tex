\title{\bf Overview of Galaxy Formation Theory}

\section{Basics}

\subsection{General Story}

Galaxies form through the nonlinear gravitational collapse of dark
matter halos. Baryons participate in this collapse with the dark
matter, but unlike dark matter are able to radiate energy away and
therefore sink deeper into the potential well. They cannot radiate
away angular momentum, and they experience gas pressure; this
combination leads ultimately to the formation of rotating gas
disks. In this process, pockets of the gas cool, becoming neutral or
molecular. Inside the molecular regions, individual stars collapse and
cool. Through stellar processes, these stars enrich the interstellar
medium and return energy to it through feedback. At the centers of the
galaxies, black holes form and grow, and also exert feedback on the
gas. These feedback effects may help determine the star formation
rates of the forming galaxies. As these events occur, galaxies
experience further accretion and major mergers. Although there is not
an agreed-upon quantitative understanding of this whole process, there
is strong evidence that the dark matter halos with masses near
$10^{12}$ $M_\odot$.

Although this basic story is known, many of its details are unclear
and it is not known if the story is complete. The current theories of
galaxy formation do a good job reproducing the gross properties of
galaxies near $z\sim 0$. It remains to be seen if they consistently
predict the differences in the galaxy population at higher redshift,
more detailed properties of galaxies such as their mass profiles,
stellar kinematics, and internal chemical patterns, and their gaseous
environments.

\subsection{Halos and Galaxies}

An overarching challenge for physical galaxy formation theory can be
expressed as the relationship between halo mass and stellar mass of
galaxies. Approximately, this relationship expresses how effiently the
matter that fell into a halo was converted into star.

This relationship can be observationally constrained through {\it
  abundance matching}. Abundance matching assumes that halo mass and
stellar mass are at least approximately monotonically related. For any
halo mass $M_h$ for which the number density of halos above that mass
is predicted to be $\Phi_h(>M_h)$, one can find a corresponding
stellar mass $M_\ast$ for which the real universe is observed to have
$\Phi_\ast(>M_\ast) = \Phi(>M_h)$, and conclude that halos of mass
$M_h$ host galaxies with stellar mass $M_\ast$. There are numerous
refinements of this technique that account for scatter in the
relationship between $M_h$ and $M_\ast$, that use subhalos rather than
halos, that use circular velocity instead of $M_h$.

The other basic tool for understanding this relationship is the {\it
  halo occupation distribution} model. This model connects $N$-body
predictions of dark matter halo properties to observed galaxy
properties through $P(>M_\ast | M_h)$, the probability distribution of
the number of galaxies above stellar mass $M_\ast$ within a halo of
mass $M_h$. Many refinements of this model exist, the most important
of which is the distinction between {\it central galaxies} in a halo
and {\it satellite galaxies}. The framework also can be used to study
galaxies of different classes or properties, and as a function of
different environments around each halo.

\citet{wechsler18a} review the literature of the galaxy-halo
connection, including abundance matching and halo occupation
distributions. The major conclusion from these studies is that
$M_\ast/M_h$ peaks at $M_h\sim 10^{12}$. This ratio rises linearly or
even more steeply between $10^{10}$ and $10^{12}$ $M_\odot$ and
declines as $\sim M_h^{-1/2}$ at higher masses. These conclusions are
validated by using them to predict observations of weak lensing and
galaxy correlation functions. They can be further used to study the
scatter in the relationship between $M_h$ and $M_\ast$, how central
and satellite galaxies differ in their star formation histories, and
how these properties depend on environment.

These results outline a basic challenge of galaxy formation theories,
which is to explain how the efficiency of star formation depends on halo
mass.

\subsection{Physical Processes}

Theoretical models seek to explain galaxy-halo relation and the other
properties of galaxies from, most ambitiously, a first-principles
approach beginning with pattern of matter fluctuations at the time of
recombination. \citet{somerville15a} outline the major physical
processes at play:
\begin{itemize}
  \item Gravity, which drives gravitational growth leading to the
    collapse and clustering of dark matter halos. 
  \item Hydrodynamics, which controls the flow of the baryons and
    produces shocks. 
  \item Thermal processes, which control the cooling of gas and thus
    how it will flow into galaxies.
  \item Star formation, which in the context of galaxy formation
    models means how cool gas proceeds to fragment and form dense
    cloud cores that lead to individual stars, and which may affect
    the subsequent evolution of the galaxy through feedback due to
    stellar winds and supernovae.
  \item Black hole formation and growth, which occurs due to gas
    inflow to the very centers of galaxies, and which may affect the
    subsequent evolution of the galaxy through AGN feedback.
  \item Nucleosynthesis, which leads to chemical enrichment of
    interstellar gas, which affects the thermal processes because of
    the importance of metal cooling.
  \item Radiative transfer, which can heat and cool gas, as well as
    affect the observed nature of the galaxies.
\end{itemize}

In galaxy formation theories, gas cooling, inflow, and feedback play
critical roles, and as noted below can only be modeled through {\it
subgrid} physics---i.e. not from first principles. What simulations
can predict from first principles are the effects of gravity on
nonlinear collapse, the cooling of gas on large scales, and the flows
of gas.  However, these first principles calculations are still
limited by resolution, which may lead to qualitatively important
errors even on large scales.

\subsection{Methodology}

Simulations of galaxy formation utilize N-body simulations,
hydrodynamic simulations, and prescriptions for subgrid
physics, or rely on approximations to those simulations known as {it
semianalytic models}.

As used in the cosmological literature, the term {\it N-body
simulation} refers to a purely gravitational simulation of the
collisionless Boltzmann equation, approximated with particle
methods. These simulations nominally simulate pure cold dark matter
models, with no baryons. Traditional N-body simulations do not solve
the collisionless Boltzmann equation fully, but use particles to
sample the density field and study the gravitational interactions of
the particles. Because the number density of particles in the
simulation is far below the number density of expected cold dark
matter particles, if the gravitational interactions were solved
exactly they would lead to unrealistically short two-body relaxation
times in the simulations. Therefore, the gravitational force law needs
to be {\it softened} at some length. 

[Semi-analytic]

[Hydro]

[Subgrid physics]

[Feedback]

\subsection{Features of Theoretical predictions}

[Cooling]

[hot vs cold mode]

\section{Commentary}

[Successes and challenges]

% \section{Important numbers}

% \begin{itemize}
% \item 
% \end{itemize}

\section{Key References}

\begin{itemize}
  \item
    \href{https://ui.adsabs.harvard.edu/abs/2015ARA%26A..53...51S/abstract}
    {{\it Physical Models of Galaxy Formation in a Cosmological
        Framework} (\citealt{somerville15a})}
  \item
    \href{https://ui.adsabs.harvard.edu/abs/2018ARA%26A..56..435W/abstract}
      {{\it The Connection Between Galaxies and Their Dark Matter
          Halos} (\citealt{wechsler18a})}
\end{itemize}

\section{Order-of-magnitude Exercises}

\begin{enumerate} 
\item Estimate the maximum efficiency of converting mass energy
    into emission for a non-rotating black hole of mass $M$.

\begin{answer}
In order to fall deeper into a potential well, matter has to lose
orbital energy and that energy is available for emission. In the case
of a black hole, once it reaches the innermost stable circular orbit
at $3r_s$, it can simply fall in without any further energy loss.

Using a Newtonian approximation, the orbital energy is $E = K + U =
-U/2$. From infinity to $3r_s$ implies an energy loss per unit mass of
$GM/6r_sc^2 = 1 / 12$, implying a maximum conversion efficiency of
$\sim 0.08$. In a general relativistic calculation, the real energy
available is slightly smaller than this ($\sim 0.06$).

For a spinning black hole, the efficiency can be considerably higher.
\end{answer}

\item Estimate efficiency of accretion
\item Estimate hottest gas in continuum
\item What is virial tempereature near disk
\item Doppler velocities to sizes
\item Critical density of CIII] and OIII
\item Estimate filling factor of BLR, NLR
\item Estimate duty cycle
\end{enumerate} 

\section{Analytic Exercises}

\begin{enumerate}
\item Eddington luminosity
\item Temperature profile of accretion disk
\item Relationship between energy distribution and synchrotron
\item Superluminal motion
\item Relativistic Beaming
\item Polarization
\end{enumerate}

\section{Numerics and Data Exercises}

\begin{enumerate}
\item Quasar SEDs
\item Quasar variability
\item Densities from line ratios
\item Temperatures from line ratios
\item OIII/Hbeta and ionization parameter
\item resolved spectroscopy of AGN
\end{enumerate}

\bibliographystyle{apj}
\bibliography{exex}  
