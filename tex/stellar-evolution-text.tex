\title{\bf Stellar Evolution}

\section{Basics \& Nomenclature}

Stars are formed from interstellar gas, through gravitational collapse
within molecular clouds. A spectrum of objects are formed from below
the hydrogen burning limit of 0.08 $M_\odot$ up to 100--200
$M_\odot$. Stars spend most of their lifetime on a {\it main
sequence}, burning hydrogen in their cores. Depending on their mass
they then proceed through a sequence of post-main sequence phases and
leave remnants in the form of white dwarfs, neutron stars, or black
holes (or in some cases no permanent remnant). In any system of stars,
their history is encoded in their distribution of luminosities and
colors, which can be measured directly in resolved stellar populations
or inferred from spectra and/or broad band imaging. Stellar evolution
processes are responsible for most of the elements higher mass than
helium.

The {\it stellar initial mass function}, or {\it IMF}, defines the
spectrum of initial masses of stars. This spectrum is difficult to
determine observationally, because almost all systems we observe have
been altered dynamically or by stellar evolution. For many decades,
the standard was the {\it Salpeter IMF}:
\begin{equation}
\Phi(M) \propto M^{-2.35}.
\end{equation}
This form leads to a large fraction of mass between $0.08$ and $0.5$
$M_\odot$. However, evidence from local systems implies that in many
cases, the true IMF has a flatter slope at lower masses. This
difference is significant because low mass stars emit very little
luminosity, so their presence is very difficult to directly detect,
and therefore inferences of total mass in stars depend strongly on the
assumptions of how many low mass stars are in the system.

The first and usually longest phase of stellar evolution is core
hydrogen burning on the main sequence. The cores are typically
millions to billions of Kelvin and fully ionized. Hydrogen is burned
to helium through two major processes, the {\it p-p chain} at low
masses and the {\it CNO cycle} at high masses ($M>2 M_\odot$). Each
process yields one ${}^4He$ nucleus with a mass of 3.96$m_p$ from 4
protons; the one percent difference yields the energy for stellar
luminosity. Numerous other nuclear processes are occurring
simultaneously that contribute to the luminosity (and to the neutrino
output). The nuclear processes depend on the tail of nuclei in the
Maxwell-Boltzman distribution that are sufficiently high energy to
tunnel through the Coulomb repulsion of the nuclei. They are therefore
strongly temperature sensitive.

Stellar structure is controlled by the following equations:
\begin{itemize}
\item Mass conservation:
\begin{equation}
\frac{{\rm d}M}{{\rm d}r} = 4 \pi r^2 \rho
\end{equation}
\item Energy conservation:
\begin{equation}
\frac{{\rm d}L}{{\rm d}r} = 4 \pi r^2 \rho \epsilon
\end{equation}
\item Hydrostatics:
\begin{equation}
\frac{{\rm d}P}{{\rm d}r} = - \frac{GM(r) \rho}{r^2}
\end{equation}
\item Energy transport, which comes in the form of radiative transfer:
\begin{equation}
\frac{{\rm d}T}{{\rm d}r} = \frac{L(r)\kappa(r) \rho(r)}{16\pi r^2 c a T^3}
\end{equation}
or in the form of convection.
\end{itemize}
$\kappa(r)$ is a critical parameter, as it strongly affects the
structure of the star, and therefore its size. The higher the
metallicity, the higher the opacity, the larger the star, and
therefore the lower the surface temperature at a given luminosity.

In the exercises, we will show that these equations imply a scaling of
luminosity with mass of approximately $L\propto M^4$ and with surface
temperature of $L\propto T_s^8$ on the main sequence.  The former
relationship implies that the stellar lifetimes on the main sequence
scale as $M^{-3}$.

After the main sequence, stellar evolution depends on the mass of the
star. At high luminosities (canonically above 8 $M_\odot$), stars have
short main sequence lifetimes (10s of Myrs) and thereafter undergo a
series of nuclear burning phases in their cores: He, C, Si, and so
on. Each burning phase is shorter than the last.  Shell burning is
occurring at the same time. Once Fe and Ni form in the core, energy
cannot be further released, and the core collapses. The result in many
and potentially all cases is a core-collapse supernova. The Fe and Ni
produced in the core is disintegrated in this process and most of that
mass becomes part of the neutron star or black hole that forms at the
center. However, the elements remaining from burning in the regions
outside the core, which are rich in $\alpha$ elements like O, Mg, and
so on, can be returned to the interstellar medium.

At lower masses, stars instead start burning hydrogen in a shell
around the inert helium core. This shell burning yields tremendous
luminosity and also induces the outer layers of the stars to expand up
to AU or greater size. The result is a red giant. {\bf get helium
flash right here and further phases}.

The net result of stellar evolution is that at early times the stellar
population is mostly on the main sequence and it is dominated by the
bluest stars. At late times it is dominated by the red giants, which
have recently (within $\sim$ a Gyr) left the main sequence.

Stars are normally classified according to their MK system: OBAFGKM,
which is in order of decreasing temperature. Subclassifications exist
(O1, O2, \ldots, O9, B1, \ldots). Hotter stars are referred to as
``early type'' and cooler stars are referred to as ``late type.''
These classifications are according to their spectra and the ordering
was originally based on the spectral phenomenology, which is why the
current nomenclature appears somewhat random.

Broadly speaking O and B stars have few lines, with He II lines in O
stars, He I lines in B stars, and weak Balmer lines in both. A stars
have strong Balmer lines. Balmer lines become weaker again for later
type stars. In F and G stars, lines of other neutral atoms become
important. Particularly in G stars the Ca II H and K lines (right
below 4000 \AA) appear. In G, K, and particularly M stars, molecular
lines become important as molecules like CH, CN, and TiO become able
to survive in the cooler atmospheres.

In galactic systems, the consequence of stellar evolution is that the
stellar continuum of young systems is dominated by hot stars, yielding
a blue spectrum with few clues to metallicity, whereas the stellar
continuum of old systems is dominated by cooler, old stars, with red
spectra that depend on metallicity (due to variations in the
abundances and their effects on the stellar opacity and therefore
temperature). 

\section{Commentary}

\section{Key References}

\begin{itemize}
  \item {\it Graham paper on Sersic}
\end{itemize}

\citet{gunn06a}

\section{Important numbers}

\begin{itemize}
\item $M_{\odot} = 1.989 \times 10^{30} {\rm ~kg} $
\item $R_{\odot} = 6.955 \times 10^{8} {\rm ~m} $
\item $T_{\odot}{\rm (surface)} = 5500 {\rm ~K} $
\item $T_{\odot}{\rm (core)} = 1.5 \ times 10^7 {\rm ~K} $
\item $L_{\odot} = 3.828 \times 10^{33} {\rm ~erg} {\rm ~s}^{-1}$
\end{itemize}

\section{Order-of-magnitude Exercises}

\begin{enumerate} 
\item Argue why higher mass stars produce more of their energy through
    the CNO cycle than lower mass stars do.
\item Detailed stellar evolution calculations predict a main sequence
    lifetime for the Sun of 10 billion years. What fraction of the
    total hydrogen in the Sun needs to be converted to helium to
    provide this lifetime?
\item Using the relations found in the text above, what is the 
\end{enumerate} 

\section{Analytic Exercises}

\begin{enumerate}
\item Under Salpeter, calculate luminosity as a function of time
\end{enumerate}

\section{Numerics and Data Exercises}

\begin{enumerate}
\item HR diagram for an open cluster
\item HR diagram for a globular cluster 
\item HR diagram locally
\item Compare K spectrum to E spectrum
\item Compare O/B spectrum to blue galaxy
\end{enumerate}

\bibliographystyle{apj}
\bibliography{exex}  
