\title{\bf Gravitational Lensing}

\section{Basics}

Under general relativity, in the presence of mass light is bent by the
curvature of spacetime. On astronomical scales this can cause the
phenomenon of {\it gravitational lensing}. 

\subsection{Point mass lensing}

Understanding lensing begins with the point mass case. It can be shown
that a photon traveling by a point mass, with an impact parameter $r$,
is in the small deflection limit deflected by an angle:
\begin{equation}
\theta_D = \frac{4GM}{rc^2} 
\end{equation}
This differs by a factor of two from the equivalent Newtonian
calculation. An important feature of lensing is that it is achromatic;
i.e., independent of wavelength.

Figure \ref{fig:symmetric} describes the symmetric point lens case and
defines the distances involved. In an analog to the optical thin lens
approximation, we define the {\it source plane} and the {\it lens
  plane}. In the perfectly aligned case the observer sees the source
as a ring surrounding the lens; perfect alignment means an offset
substantially than the source size.  A characteristic quantity of a
lens is radius of this ring, which is the {\it Einstein angle}:
\begin{equation}
\theta_E =  \sqrt{\frac{4GM}{c^2}} \sqrt{\frac{D_{LS}}{D_L D_S}}
\end{equation}
which can be related to the {\it Einstein radius} in the lens plane
$r_E = D_L \theta_E$.

Figure \ref{fig:offset} describes the offset point lens case. If the
source is a point source, this will result in two magnified (and one
highly demagnified) images for the observer. The condition on the
source angle:
\begin{equation}
\beta < \theta_E
\end{equation}
defines the {\it strong lensing} regime. In this regime, the two
images appear near the Einstein ring location. If the source is
extended instead of point-like, it can appear highly distorted in the
strong lensing case.

The opposite case is known as the {\it weak lensing} regime. 

In either case, the distortion of lensing has an effect on the
apparent brightness of the object. The total magnification can be
defined as the increase in the solid angle of the image. This solid
angle increase occurs even if our instrumentation still cannot detect
the extended nature of the image. Because surface brightness (more
technically specific intensity) is conserved in general relativity
this magnification leads to an increase in total flux density.

For multiply imaged sources, the images formed follow different paths
of different distances. This fact leads to a relative delay between
photons that travel different paths. In addition, a different general
relativistic delay is associated with each path, known as the {\it
Shapiro delay}. Fluctuations in the source will appear to the observer
at different times. A measured delay yields a measurement of physical
distance that can in principle be used to determine the distances of
the source and lens.

\subsection{Lensing from extended mass sheets}

On cosmological scales, weak lensing outside the Einstein radius of
individual groups and clusters occurs, but is not well described by
single point mass lensing. We will instead here describe the lensing
as due to a sheet of mass in the lens plane of varying surface
density. 

Let us consider a point in the source plane that (undeflected) would
be at angle $\vec{\beta}$. Let $\vec{x}$ represent the physical
position in the lens plane that the undeflected ray would have passed
through.  The deflection angle is the sum of the contributions of all
the mass in the lens plans:
\begin{equation}
\vec{\theta_D}\left(\vec{x}\right) = \frac{4G}{c^2
D_L}  \int \dd^2 \vec{x}' \Sigma\left(\vec{x}\right) \frac{\vec{x}
- \vec{x}'}{\left|\vec{x} - \vec{x}'\right|^2}
\end{equation}
We can relate the source plane position $\vec{\beta}$ to the observed
angle $\theta$ with the {\it lens equation}:
\begin{equation}
\vec{\beta} = \vec{\theta}
- \frac{D_{LS}}{D_S} \vec{\theta_D}\left(D_L \vec{\theta}\right)
  = \vec{\theta} - \vec{\alpha}
\end{equation}
We can use the above relations to show:
\begin{equation}
\vec{\alpha}
= \frac{1}{\pi} \int \dd^2\vec{\theta}' \kappa\left(\vec{\theta}'\right)
\frac{\vec{\theta} - \vec{\theta}'}
{\left|\vec{\theta} - \vec{\theta}'\right|^2}.
\end{equation}
where we define:
\begin{equation}
\kappa = \frac{\Sigma}{\Sigma_{\rm cr}},
\end{equation}
and:
\begin{equation}
\Sigma_{\rm cr} = \frac{c^2 D_S}{4\pi G D_{LS} D_L}
\end{equation}
The condition $\kappa>1$ leads to multiply imaged sources.

The form of $\vec{\alpha}$ suggests that it can be written as the
gradient of a potential,
\begin{equation}
\vec{\beta} = \vec{\theta} - \vec{\nabla}\psi,
\end{equation}
where
\begin{equation}
\psi
= \frac{1}{\pi} \int \dd^2\vec{\theta}' \kappa\left(\vec{\theta}'\right)
\ln \left|\vec{\theta} - \vec{\theta}'\right|.
\end{equation}
We can also show:
\begin{equation}
\nabla^2\psi = 2 \kappa
\end{equation}
We can define the Fermat time delay potential as
\begin{equation}
\tau\left(\vec{\theta}; \vec{\beta}\right) =
\frac{1}{2} \left(\vec{\theta} - \vec{\beta}\right)^2
- \psi\left(\vec{\theta}\right),
\end{equation}
and the lens equation can be rewritten as.
\begin{equation}
\vec{\nabla}\tau = 0.
\end{equation}
This result is an expression of the general relativistic version of
Fermat's principle.

\subsection{Weak lensing}

The lens equation can be locally linearized  around $\vec{\beta_0}$:
\begin{equation}
\vec{\beta} = \vec{\beta}_0
+ \frac{\partial \vec{\beta}}{\partial \vec{\theta}} \cdot \left(\vec{\theta}
- \vec{\theta_D} \right), 
\end{equation}
where the Jacobian can be written in index form:
\begin{equation}
{\bf A}\left(\vec{\theta}\right) = 
\frac{\partial \vec{\beta}}{\partial \vec{\theta}} = \left(\delta_{ij}
- \frac{\partial^2 \psi}{\partial\theta_i \partial\theta_j}\right)
{\hat e}_i {\hat e}_j
\end{equation}

{\bf NEED TO FINISH THIS}

\subsection{Microlensing}

A phenomenon called {\it microlensing} occurs when the lensing mass
and background source have a relative angular motion. The background
source increases as it moves through the Einstein radius of the
lens. This increase has a distinctive, achromatic signature, that can
be seen for individual stars in our Galaxy through monitoring.

A related phenomenon also known as microlensing occurs when viewing a
background source through a galactic system. The stars create a
lensing potential surface with distinct cusps that cause fluctuations
in the flux of the background source.

These phenomena can only occur if the background source is physically
smaller than the Einstein radius. Otherwise even if the center of the
source is aligned with the lens, most of the light is well outside the
Einstein radius in the lens plane and is not deflected. This fact
makes it possible to constrain the relative sizes of the background
source in different wavelengths (e.g. radio vs. optical) through
observations of its lensing.

\section{Important numbers}

\section{Key References}

\begin{itemize}
  \item
    \href{http://adsabs.harvard.edu/abs/2000asqu.book.....C}{
    {\it Binney \& Tremaine}
      \citet{cox00a}}, Chapter 5
\end{itemize}

\section{Order-of-magnitude Exercises}

\begin{enumerate} 
\item Typical Einstein angles for stars, galaxies, clusters.
\item Typical delay time
\item Typical shear values
\item Estimate probability of microlensing in galaxy
\end{enumerate}   

\section{Analytic Exercises}

\begin{enumerate}
\item GR calculation of lensing offset
\item Show Einstein angle
\item Calculate offset from Einstein angle for strong lensing
\item Calculate offset from source angle for weak lensing
\item Calculate magnification
\item Critical surface density case
\item Derive shear and magnification properties
\end{enumerate}

\section{Numerics and Data Exercises}

\begin{enumerate}
\item Modeling of lens system
\item Specific strong lenses
\item Measurements of shear
\end{enumerate}

\bibliographystyle{apj}
\bibliography{exex}  
