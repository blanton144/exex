\title{\bf Gravitational Lensing}

\section{Basics \& Nomenclature}

Under general relativity, in the presence of mass light is bent by the
curvature of spacetime. On astronomical scales this can cause the
phenomenon of {\it gravitational lensing}. 

Understanding lensing begins with the point mass case. It can be shown
that a photon traveling by a point mass, with an impact parameter $r$,
is in the small deflection limit deflected by an angle:
\begin{equation}
\theta_D = \frac{4GM}{rc^2} 
\end{equation}
This differs by a factor of two from the equivalent Newtonian
calculation. An important feature of lensing is that it is achromatic;
i.e., independent of wavelength.

Figure \ref{fig:symmetric} describes the symmetric point lens case and
defines the distances involved. In an analog to the optical thin lens
approximation, we define the {\it source plane} and the {\it lens
  plane}. In the perfectly aligned case the observer sees the source
as a ring surrounding the lens; perfect alignment means an offset
substantially than the source size.  A characteristic quantity of a
lens is radius of this ring, which is the {\it Einstein angle}:
\begin{equation}
\theta_E =  \sqrt{\frac{4GM}{c^2}} \sqrt{\frac{D_{LS}}{D_L D_S}}
\end{equation}
which can be related to the {\it Einstein radius} in the lens plane
$r_E = D_L \theta_E$.

Figure \ref{fig:offset} describes the offset point lens case. If the
source is a point source, this will result in two magnified (and one
highly demagnified) images for the observer. The condition on the
source angle:
\begin{equation}
\beta < \theta_E
\end{equation}
defines the {\it strong lensing} regime. In this regime, the two
images appear near the Einstein ring location. If the source is
extended instead of point-like, it can appear highly distorted in the
strong lensing case.

The opposite case is known as the {\it weak lensing} regime. 

In either case, the distortion of lensing has an effect on the
apparent brightness of the object. The total magnification can be
defined as the increase in the solid angle of the image. This solid
angle increase occurs even if our instrumentation still cannot detect
the extended nature of the image. Because surface brightness (more
technically specific intensity) is conserved in general relativity
this magnification leads to an increase in total flux density.

\section{Important numbers}

\section{Key References}

\begin{itemize}
  \item
    \href{http://adsabs.harvard.edu/abs/2000asqu.book.....C}{
    {\it Binney \& Tremaine}
      \citet{cox00a}}, Chapter 5
\end{itemize}

\section{Order-of-magnitude Exercises}

\begin{enumerate} 
\item Typical Einstein angles for stars, galaxies, clusters.
\item Typical shear values
\item Estimate probability of microlensing in galaxy
\end{enumerate}   

\section{Analytic Exercises}

\begin{enumerate}
\item GR calculation of lensing offset
\item Show Einstein angle
\item Calculate offset from Einstein angle for strong lensing
\item Calculate offset from source angle for weak lensing
\item Calculate magnification
\end{enumerate}

\section{Numerics and Data Exercises}

\begin{enumerate}

\item Specific strong lenses
\end{enumerate}

\bibliographystyle{apj}
\bibliography{exex}  
