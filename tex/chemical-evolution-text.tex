\title{\bf Chemical Evolution}

\section{Basics}

Chemical evolution models attempt to track the metallicity (or in the
more general case, abundance) enrichment over time. They rely on
nucleosynthetic yield predictions and in some cases inflow, outflow,
and mixing scenarios. In the most sophisticated cases, they are
incorporated into full hydrodynamic models for galaxy
formation. However, the simplest forms of chemical evolution models
can also be edifying.

\subsection{One-zone model}

The simplest version of chemical evolution is a {\it one-zone} model
of metal enrichment. Start with a mass $M_g$ in gas and form stars at
a rate $\dot{M_\ast}$. Define $m_Z$ as the {\it yield}; the mass in
metals created for each mass of star formation. Define the {\it
recycling rate} $r$ where $r\sim 0.5$, the fraction of gas that is
returned to the ISM. Define $M_\ast$ as the actual mass of stars
(accounting for the fact that some of the mass gets recycled).  Define
$M_Z$ as the mass in metals in the gas.  Now you assume instantaneous
and well-mixed recycling inside the one zone in the model, and that no
gas or stars enter or leave your zone (this last approximation is what
is meant by {\it closed box}).

Then the changes of these variables over some time ${\rm d}t$ are:
\begin{eqnarray}
{\rm d}M_\ast &=& (1-r) \dot{M_\ast} {\rm d}t \cr
{\rm d}M_g &=& - {\rm d}M_\ast = (1-r) \dot{M_\ast} {\rm d}t \cr
{\rm d}M_Z &=& m_Z \dot{M_\ast} {\rm d}t - Z {\rm d}M_\ast \cr
&=& \left(\frac{m_Z}{1-r} - Z\right) {\rm d}M_\ast
\end{eqnarray}
We can show from these equations that
\begin{equation}
\label{eq:closedz}
Z(t) = - \frac{m_Z}{1-r} \ln f_{\rm gas}.
\end{equation}
$m_Z$ is typically known as the {\it net yield},
and is the metals produced per unit of stars formed. The more common
quoted quantity is the {\it absolute yield}:
\begin{equation}
p = \frac{m_Z}{1-r}
\end{equation}
which is the metals produced per unit star/stellar remnant mass.
Finally, the quantity:
\begin{equation}
y = -\frac{Z(t)}{\ln f_{\rm gas}}
\end{equation}
is actually an observable, sometimes known as the {\it effective
yield}. It is equal to the yield in the closed box case. 

The gas is increasing in metallicity monotonically, so the metallicity
and time map to each other.  This means we can determine the fraction
of stars of a particular metallicity as:
\begin{equation}
M_\ast(<Z(t)) = M_\ast(t) = M_g(0) - M_g(t) = M_g(0)
\left[1-\exp\left(-Z(t)/p\right)\right]
\end{equation}
We can show that as $f_g \rightarrow 0$, the mean metallicity of the
stars $\langle Z\rangle\rightarrow p$.

You can also calculate the fraction of stars with some
factor $\alpha$ fewer metals than the current gas:  
\begin{equation}
  \frac{M_\ast(<\alpha Z)}{M_\ast(< Z)}=
  \frac{1- \exp(-\alpha Z/p)} {1- \exp(-Z/p)} =
  \frac{1-f_g^{\alpha}}{1-f_g}
\end{equation}
For $\alpha \sim 0.3$ and (appropriate to the solar radius) $f_g \sim
0.1$, this fraction is 50\%. So you expect {\it many} low metallicity
stars in the solar neighborhood with a closed box model. However, in
reality the number of low metallicity stars is rather small --- the
metallicity distribution is quite narrow. Because this was determined
early from metallicity distributions of G dwarfs, this issue is known
as the {\it G-dwarf problem}.  There are numerous potential solutions
to this problem, which involve breaking the closed box assumption of
course.

We can consider the differential distribution:
\begin{equation}
\frac{{\rm d}F}{{\rm d}\log\alpha Z} = \alpha \ln 10 \frac{-
 f_g^{\alpha -1 }}{1-f_g}
\end{equation}

\subsection{Extreme infall one-zone model}

Another interesting case can be easily done analytically, one in which
inflowing primordial abundance gas exactly replaces the gas lost due
to star formation. The set of relevant equations is:
\begin{eqnarray}
{\rm d}M_\ast &=& (1-r) \dot{M_\ast} {\rm d}t \cr
{\rm d}M_g &=& 0 \cr
{\rm d}M_Z &=& m_Z \dot{M_\ast} {\rm d}t - Z {\rm d}M_\ast \cr
&=& \left(\frac{m_Z}{1-r} - Z\right) {\rm d}M_\ast
\end{eqnarray}
In this case, the metallicity of the gas  approaches the absolute
yield $p$. The mean metallicity of the stars will remain below the
metallicity of the gas but also approach the yield $p$.      

We can calculate the metallicity distribution of the stars at a given
time $t$:
\begin{equation}
  F=  \frac{M_\ast(<\alpha Z(t))}{M_\ast(<Z(t))}
  = -\frac{M_g}{M_\ast} \ln\left[1- \alpha (1- \exp\left(-M_\ast /
    M_g)\right) \right]
\end{equation}
Taken the logarithmic derivative we find:
\begin{equation}
  \frac{{\rm d}F}{{\rm d} \log\alpha Z}
  = \alpha \ln(10) \frac{1}{s} \frac{1-\exp(-s)}{1-
      \alpha\left(1-\exp(-s)\right) }
\end{equation}
where $s = M_\ast/M_g$. As $s$ increases, we see the distribution
become extremely peaked at the maximum metallicity.

This case is a pretty extreme. We expect over time for the inflowing
gas to be declining.  But it shows how a constant inflow of {\it
pristine} gas can generate a very peaked distribution in stellar
metallicity, more similar to what is observed locally.

\subsection{Outflows}

Another basic process is outflow. Outflows have been observed through
absorption studies, wherein metals in the outflowing gas absorb and
yield clear indicators of the outflow. Outflows can be characterized
by a {\it mass-loading parameter} $\eta$ as the ratio of the outflow
rate to the increase in stellar mass.

If the outflowing gas has the current mean gas metallicity then we can
show:
\begin{equation}
Z(t) = - \frac{1}{1+\eta} \frac{m_Z}{1-r} \ln f_{\rm gas}
\end{equation}
This case thus behaves like a closed box but with a lower effective
yield.

The outflows actually can be metal-enhanced, since much of the metals
are produced by or near the same stars and supernovae that are
producing the wind. This enhancement will reduce the amount of outflow
necessary to reach a particular effective yield.

\subsection{More realistic models}

These illustrative models are not realistic enough to compare with
real data, or to extract the full set of information available in the
observations. Multizone models are necessary because we now recognize
radial migration of stars and winds of gas which may fall back down
onto the galaxy, mixing the enrichment within a single
galaxy. Obviously more realistic models of infall and outflow are
necessary. These two issues are in principle most realistically
addressed in the context of a full hydrodynamic simulations.

Non-instantaneous recycling is an important issue, most notably
because of the differing delays between the $\alpha$-rich enrichment
of core-collapse supernovae and the iron-peak-rich  enrichment of Type
Ia supernovae. The core collapse supernovae occur within a few million
years, but the Type Ia supernovae are spread out over a broad delay
period. 

{\bf discuss Weinberg results}

\subsection{Typical yield values}

\subsection{Observational results}

{\bf mass-metallicity, effective yield}

{\bf alpha and ellipticals}

{\bf within MW}

%\section{Important numbers}

\section{Key References}

\begin{itemize}
  \item Weinberg paper
  \item Pagel
\end{itemize}

\section{Order-of-magnitude Exercises}

\begin{enumerate} 
\item The overall metallicity of baryons in the universe is
 about $Z=0.01$.  What does that imply for the total production of
 stars?
\end{enumerate}   

\section{Analytic Exercises}

\begin{enumerate}
\item Prove Equation (\ref{eq:closedz}), the gas metallicity as a
  function of time for the closed box model, for constant net yield
  $m_Z$ and recycling rate $r$.

\begin{answer}
The change in $Z$ is:
\begin{eqnarray}
  {\rm d}Z &=& {\rm d}\left(\frac{M_Z}{M_g}\right) =
  \frac{1}{M_g} {\rm d}M_Z - \frac{M_Z}{M_g^2} {\rm d}M_g \cr
&=& \frac{1}{M_g} \left(\frac{m_Z}{1-r}  - Z {\rm d}M_\ast -
  \frac{M_Z}{M_g} {\rm d}M_g\right) \cr
&=& \frac{1}{M_g} \left(\frac{m_Z}{1-r} {\rm d}M_\ast - Z {\rm d}M_\ast -
  \frac{M_Z}{M_g} {\rm d}M_g\right) \cr
&=& - \frac{{\rm d}M_g}{M_g} \left(\frac{m_Z}{1-r}\right)
\end{eqnarray}
For constant $m_Z$ and $r$, and if the gas starts out with $Z=0$,
then:
\begin{equation}
Z(t) = - \frac{m_Z}{1-r} \ln\left(\frac{M_g(t)}{M_g(t=0)}\right)
\end{equation}
Since no gas is added or subtracted this can be written as:
\begin{equation}
Z(t) = - \frac{m_Z}{1-r} \ln f_{\rm gas}
\end{equation}
\end{answer}

\item Show that for the closed box model, 
the mean metallicity of the stars $\langle Z\rangle$ approaches the
effectve yield $p$.

\begin{answer}
We write:
\begin{eqnarray}
  \langle Z\rangle &=&
  \frac{1}{M_\ast} \int_0^{M_\ast} {\rm d}M_\ast' Z(M_\ast') \cr
&=&
  \frac{1}{M_\ast} \int_0^{M_\ast} {\rm d}M_\ast' 
  \frac{m_Z}{1-r} \ln \left[\frac{M_\ast'+ M_g}{M_g}\right] 
\end{eqnarray}
We can replace ${\rm d}M_\ast'$ with $-{\rm d}M_g = - {\rm d}f_g
(M_\ast + M_g)$ and find:
\begin{eqnarray}
  \langle Z\rangle
  &=&
  \frac{M_\ast + M_g}{M_\ast} \int_1^{f_g} {\rm d}f_g'
  \frac{m_Z}{1-r} \ln f_g \cr
  &=&
  \frac{M_\ast + M_g}{M_\ast}
  \frac{m_Z}{1-r}
  \left[
    f_g' \ln f_g' - f_g'
    \right]_1^{f_g} \cr
  &=&
  \frac{1}{1-f_g}
  \frac{m_Z}{1-r}
  \left[
    f_g \ln f_g - f_g + 1
    \right] \cr
  &=&
  \frac{m_Z}{1-r}
  \left[
    1 + \frac{f_g \ln f_g}{1-f_g}
    \right]
\end{eqnarray}
As $f_g\rightarrow 0$, $\langle Z\rangle \rightarrow p$. So the mean
metallicity of the stars becomes just the absolute yield.
\end{answer}

\item Show that for the extreme infall model, the gas metallicity and
the mean stellar metallicity approach the absolute yield.

\begin{answer}
The change in $Z$ is:
\begin{eqnarray}
  {\rm d}Z &=& {\rm d}\left(\frac{M_Z}{M_g}\right) =
  \frac{1}{M_g} {\rm d}M_Z \cr
  &=& \frac{1}{M_g} \left(\frac{m_Z}{1-r} - Z\right) {\rm d}M_\ast \cr
  \frac{{\rm d}Z}{{\rm d}M_\ast} &=&
  \frac{1}{M_g} \left(\frac{m_Z}{1-r} - Z\right) 
\end{eqnarray}
If the initial $Z$ is zero, this is solved by:
\begin{equation}
Z = \frac{m_Z}{1-r} \left[1 - \exp(-M_\ast/ M_g)\right]
\end{equation}
because
\begin{eqnarray}
  \frac{{\rm d}Z}{{\rm d}M_\ast} &=&
  \frac{1}{M_g} \frac{m_Z}{1-r} \exp(-M_\ast/ M_g) \cr
  &=&
  \frac{1}{M_g} \frac{m_Z}{1-r} \exp(-M_\ast/ M_g) - \frac{Z}{M_g} +
  \frac{1}{M_g} \frac{m_Z}{1-r} \left[1 - \exp(-M_\ast/ M_g)\right]\cr
  &=&
  \frac{1}{M_g} \frac{m_Z}{1-r} 
  - \frac{Z}{M_g} = \frac{1}{M_g} \left(\frac{m_Z}{1-r} - Z\right) 
\end{eqnarray}
So the gas metallicity approaches the yield.

The mean metallicity of the stars will be below this. We can calculate
it as:
\begin{eqnarray}
  \frac{\langle Z\rangle}{p} &=& \frac{1}{M_\ast} \int_0^{M_\ast}
       {\rm d}M_\ast' \left[ 1 - \exp(-M_\ast' / M_g) \right] \cr
       &=& \frac{1}{M_\ast} \left[M_\ast' + M_g
         \exp(-M_\ast'/M_g\right]_0^{M_\ast} \cr
       &=& \frac{1}{M_\ast} \left[ M_\ast + \exp(-M_\ast/M_g) -
         M_g\right] \cr
       &=& 1 + \frac{M_g}{M_\ast} \left[\exp(-M_\ast/M_g) - 1\right]
\end{eqnarray}
At small $M_\ast/M_g$ this tends towards 0, as it should. At large
$M_\ast/M_g$ it tends toward unity, approaching it from below.
\end{answer}

\item For the unenriched outflow case, show that the gas metallicity
increases in the same manner as for a closed box but with a lower
effective yield.

\begin{answer}
We can write:
\begin{eqnarray}
{\rm d}M_\ast &=& (1-r) \dot{M_\ast} {\rm d}t \cr
{\rm d}M_g &=& - {\rm d}M_\ast - \eta {\rm d}M_\ast \cr
{\rm d}M_Z &=& m_Z \dot{M_\ast} {\rm d}t - Z {\rm d}M_\ast - Z \eta
{\rm d}M_\ast
\cr
&=& \left(\frac{m_Z}{1-r} - (1+\eta) Z\right) {\rm d}M_\ast
\end{eqnarray}
In this case I get to just replace $Z \rightarrow (1+\eta) Z$ in the
solutions for the simple closed box:
\begin{eqnarray}
(1+\eta) Z(t) &=& - \frac{m_Z}{1-r} \ln f_{\rm gas} \cr
Z(t) &=& - \frac{1}{1+\eta} \frac{m_Z}{1-r} \ln f_{\rm gas}
\end{eqnarray}
This means this case behaves just like a closed box, except with a
lower effective yield. 
\end{answer}

\end{enumerate}

\section{Numerics and Data Exercises}

\begin{enumerate}
\item Mass-metallicity in SDSS
\item Effective yield in SDSS
\item Model alpha enrichment vs duration of SF
\end{enumerate}

