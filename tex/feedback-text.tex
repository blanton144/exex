\title{\bf Feedback Effects in Galaxy Formation}

\section{Basics}

The discussion here follows the description found in the textbook of
Cimatti, Fraternali, \& Nipoti.

\subsection{General Concepts in Feedback}

The concept of star formation feedback in galaxy formation is
generally motivated by the need to explain low global star formation
efficiency within high and low mass dark matter halos. For that
reason, generally feedback refers to {\it negative feedback} effects;
that is, processes that inhibit star formation. These effects are
divided into {\it stellar feedback}, which is literally feedback in
the sense that star formation leads to stellar processes that inhibit
further star formation, and {\it AGN feedback}, which refers to AGN
processes that inhibit star formation. In both cases, some combination
of radiative energy and kinetic energy flux couples to gas in the
galaxy to heat it, add turbulence to it, or drive it in large scale
winds to the outskirts of or entirely out of the galaxy.

In the case of winds, the distinction is made between {\it
  momentum-driven} and {\it energy-driven} winds. Physically, these
cases are distinguished by whether the cooling time of the affected
gas is short relative to the dynamical time. If the cooling time is
short, then the energy quickly radiates away, but the momentum
imparted is not radiated away and the wind is momentum driven. If the
cooling time is not short, the thermal energy created by the feedback
helps to drive the wind. The distinction is important to the
dependence on galaxy potential depth of the relationship between a
star formation rate ($\dot M_\ast$) or black hole accretion rate
($\dot M_{\rm BH}$) to a mass outflow rate.

In the case of an energy-driven wind, the energy converted to kinetic
energy per unit time is proportional to the star formation or
black hole accretion rate:
\begin{equation}
\dot K \propto \dot M.
\end{equation}
Because in a time-invariant wind $\dot K = \dot M_{\rm w} v_{\rm w}^2$,
the rate of mass driven to some velocity $v_{\rm w}$ is:
\begin{equation}
\dot M_{\rm w} \propto \dot K v_{\rm w}^{-2} \propto \dot M v_{\rm
  w}^{-2}.
\end{equation}
The escape speed is proportional to the velocity dispersion of the
halo $\sigma$, so if we want to know how much mass is flowing out of
the galaxy (i.e. reaches $v_{\rm w} \sim \sigma$) it will scale as:
\begin{equation}
\dot M_{\rm out} \propto \dot M \sigma^{-2}.
\end{equation}

In the case of a momentum-driven wind, the momentum imparted per unit
time (due to mechanical energy or radiation pressure) will be
proportion to the star formation or black hole accretion rate:
\begin{equation}
\dot p \propto \dot M.
\end{equation}
Because in a time-invariant wind $\dot p = \dot M_{\rm w} v_{\rm w}$,
the rate of mass driven to some velocity $v_{\rm w}$ is:
\begin{equation}
\dot M_{\rm w} \propto \dot p v_{\rm w}^{-1} \propto \dot M v_{\rm
  w}^{-1}.
\end{equation}
The escape speed is proportional to the velocity dispersion of the
halo $\sigma$, so if we want to know how much mass is flowing out of
the galaxy (i.e. reaches $v_{\rm w} \sim \sigma$) it will scale as:
\begin{equation}
\dot M_{\rm out} \propto \dot M \sigma^{-1}.
\end{equation}

In the context of galaxy formation, these different scalings mean that
these two types of wind shape galaxy stellar mass function
differently.

\subsection{Stellar Feedback}

Stellar feedback comes in the form of supernovae and stellar winds,
primarily from massive stars. Individual supernovae generally do not
impart much of their energy to the interstellar medium as kinetic
energy, only a few percent. Stellar feedback generally is thought to
operate effectively through the combination of many massive stars or
supernovae creating {\it superbubbles} from which emanate winds. These
winds are usually taken to be energy driven

For either a single massive star, a group of massive stars, or a group
of supernovae, we can treat the feedback as a steady, very supersonic
wind acting for some period of time. We can calculate what fraction of
that power is converted into kinetic energy of the interstellar
medium. This calculation assumes some constant input of supersonic
wind energy under spherical symmetry. The wind shocks the surrounding
interstellar medium and creates a hot, expanding bubble of gas at
pressure $P_b$, with a shocked shell of interstellar medium gas
surrounding it. The bubble's thermal energy increases due to
mechanical energy being converted to thermal energy, and decreases due
to its expansion:
\begin{equation}
\label{eq:energy-bubble}
\frac{{\rm d}U_b}{{\rm d}t} \approx L_{\rm w} - P_b \frac{{\rm
    d}V}{{\rm d}t}.
\end{equation}
For a nonrelativistic ideal gas, we can write $U_b = (3/2) P_b V$.

The shell's momentum is driven by the pressure of the hot bubble
behind it, and we can write its equation of motion:
\begin{equation}
\label{eq:shell}
\frac{{\rm d}}{{\rm d}t} \left(M_{\rm sh} v_{\rm sh}\right) = 4\pi
r_{\rm sh}^2 P_b
\end{equation}
We can solve for $P_b$, and assuming the interstellar medium is of
constant density $\rho_0$ and is all swept up in the shell, we find:
\begin{equation}
\label{eq:shell-pb}
P_b = \rho_0\left(\frac{1}{3} r_{\rm sh} {\ddot r}_{\rm sh} + {\dot
r}_{\rm sh}^2\right)
\end{equation}
Solving for $r_{\rm sh}$ assuming a power law solution in time we
find:
\begin{equation}
\label{eq:shell-evolution}
r_{\rm sh}
= \left(\frac{125}{154\pi}\right)^{1/5} \left(\frac{L_w}{\rho_0}\right)^{1/5}
t^{3/5}
\end{equation}
We can calculate the kinetic energy of the shell's expansion:
\begin{equation}
\label{eq:shell-kinetic}
K_{\rm sh} = \frac{1}{2} M_{\rm sh} v_{\rm sh}^2 = \frac{15}{77}L_w
t \approx 0.2 L_w t
\end{equation}
This calculation shows that the conversion of wind energy to kinetic
energy is rather efficient for a steady, supersonic wind in a uniform
medium.

Numerically, if we consider massive O-stars, they lose mass at about
$10^{-6}$ $M_\odot$ yr$^{-1}$, in winds that reach thousands of km
s$^{-1}$. Such a wind will produce a mechanical power of:
\begin{equation}
L_{\rm w} = \frac{1}{2} \dot M_{\rm w} v_{\rm w}^2 \sim 10^{36}
\left(\frac{\dot M_{\rm w}}{10^{-6} M_\odot {\rm ~yr}^{-1}}\right)
\left(\frac{v_{\rm w}} {2\times 10^3 {\rm km~s}^{-1}} \right) {\rm ~erg~s}^{-1}.
\end{equation}
Using the equations above, this results over the lifetime of a single
O star (of order million years) a bubble of size $\sim 100$ pc.

For a given cluster of massive stars, the exercises show that a
similar amount of power is produced by supernovae, but over a longer
time (up to the lifetime of 8 $M_\odot$ stars, or about 30 million
years), which means that the feedback is dominated by the combined
effect of these supernovae producing superbubbles.

Per unit star formation the rate of core-collapse supernovae is:
\begin{equation}
R_{\rm SN} \approx 10^{-2} \left(\frac{\rm SFR}{M_\odot {\rm
~yr}^{-1}}\right) {\rm ~yr}^{-1}
\end{equation}
If we relate the mass outflow rate to the kinetic energy input rate by
$\dot K \sim {\dot M}_{\rm out} v_{\rm esc}^2 / 2$, and use an energy for
each supernova of $10^{51}$ erg, then:
\begin{equation}
\dot M_{\rm out} \sim \frac{\eta}{0.1} \left(\frac{{\rm SFR}}{M_\odot
{\rm ~yr}^{-1}}\right) \left(\frac{v_{\rm esc}}{300 {\rm
~km~s}^{-1}}\right)^{-2} M_\odot {\rm ~yr}^{-1}
\end{equation}
where $\eta$ is the fraction conversion of wind power into kinetic
energy (calculated as 0.2 above). 

A momentum-driven model is also possible if the medium around the star
formation is optically thick. This situation would apply mostly to
starbursts. An order-of-magnitude estimate of the momentum-driven
winds yields:
\begin{equation}
\dot M_{\rm out} \sim \left(\frac{{\rm SFR}}{M_\odot
{\rm ~yr}^{-1}}\right) \left(\frac{v_{\rm esc}}{300 {\rm
~km~s}^{-1}}\right)^{-1} M_\odot {\rm ~yr}^{-1}
\end{equation}
which is similar to the energy-driven winds at moderate star formation
rates but much more effective for star bursts
(see \citealt{murray05b}).

For some of the observed wind velocities in star forming galaxies, the
high observed velocities ($500$--$1000$ km s$^{-1}$) and relatively
high inferred mass-loading factors suggest a momentum driven wind.

\subsection{AGN Feedback}

AGN feedback is driven by accretion onto a black hole. The luminosity
of an accreting black hole is limited (in the spherical case) to the
Eddington luminosity $L_{\rm Edd}$. This luminosity can be related to
an Eddington mass accretion rate:
\begin{equation}
{\dot M}_{\rm Edd} = \frac{L_{\rm Edd}} {\epsilon_{\rm rad} c^2},
\end{equation}
where $\epsilon_{\rm rad}$ is the radiative efficiency, which in
theory can be as high as 0.29 for a highly spinning black hole with
very radiatively efficient accretion, but is typically thought to be
$\sim 0.1$ even for AGN in radiatively efficient phases. 

We can estimate whether AGN have sufficient energy to substantially
affect the galaxy by comparing the total energy emitted in the
formation of the black hole to the potential energy of the bulge:
\begin{equation}
\frac{E_{\rm BH}}{E_{\rm bulge}} \sim \frac{\epsilon_{\rm rad} M_{\rm BH}
c^2}{M_{\rm bulge} \sigma^2} \sim 100 \frac{\epsilon_{\rm
rad}}{0.1} \left(\frac{\sigma}{300 {\rm
~km~s}^{-1}}\right)^{-2} \frac{M_{\rm BH}/M_{\rm bulge}}{0.001}
\end{equation}
For a massive galaxy, then, there is amply energy available in
principle to affect the evolution of the bulge region.

It is believed that AGN feedback occurs in both radiatively efficient
and radiatively inefficient AGN phases. The {\it radiative} or {\it
quasar} mode feedback is associated with Eddington ratios of $L/L_{\rm
Edd} > 10^{-2}$. Feedback occurs through photoionization heating and
radiation pressure. The dominant effect on the galaxy as a whole is
thought to be through the radiation pressure, which can create a
galactic wind. The radiation pressure is proportional to the
luminosity and the momentum imparted per unit time to the wind is
dimensionally:
\begin{equation}
p_w \propto {\dot M}_{\rm BH}  v_w
\end{equation}
with the constant of proportionality estimated theoretically at about
0.5. This effect drives a momentum-driven wind. In theoretical models
the mechanical efficiency is:
\begin{equation}
\epsilon_w = \frac{L_w}{{\dot M}_{\rm BH} c^2} \sim 10^{-3}.
\end{equation}

The {\it jet} or {\it radio} mode feedback occurs primarily in the
radiatively inefficient regime (Eddington ratios $<10^{-2}$). The jets
of the AGN interact with the hot gas in the galaxy or surrounding
cluster mechanically. In a few systems, the effects of these
interactions can be seen as X-ray bubbles or cavities. The gas outside
the cavity is at the virial temperature and emits in the X-rays. The
gas inside the cavity is moving relativistically and its primary
emission is radio synchtron. The surrounding pressure can be estimated
from the X-ray data, and the total energy in the bubble is:
\begin{equation}
E = U + W = \frac{1}{\gamma - 1} PV + PV = \frac{\gamma}{\gamma - 1} PV,
\end{equation}
where $\gamma = 4/3$ for a relativistic gas. The kinetic power can be
estimated as $E/t$ where $t$ is some estimate of the age of the
bubble, for example based on its size and the sound speed
(\citealt{fabian12a}). This power is comparable in magnitude and tends
to scale with the X-ray luminosity, indicating that it can balance
cooling. In fact, the kinetic power associated with these bubbles is
thought to couple very efficiently to the interstellar or
intergalactic gas.

One of the reasons that AGN feedback began to be considered seriously
is that it is possible to construct a scaling relation based on the
Eddington limit that looks like $M_{\rm BH}$-$\sigma$ relation of
galaxies. In particular, if you set the total outward force due to
radiation pressure at the Eddington limit, $L_{\rm Edd} / c$, equal to
the total inward force due to gravity on some amount of gas $M_{\rm
gas} = f M_{\rm gal}$:
\begin{equation}
\frac{L_{\rm Edd}}{c} = \frac{4\pi G_{\rm BH} m_p}{\sigma_T}
= \frac{GM_{\rm gal}M_{\rm gas}}{r} = \frac{f
G}{r^2} \left(\frac{2\sigma^2 r}{G}\right)^2 = \frac{4f\sigma^2}{G}
\end{equation}
which implies that $M_{\rm BH} \propto \sigma^4$, relatively similar
to the relation observed. Thus, if the fraction of gas loss $f$ is
constant across galaxy mass, and the typical Eddington ratios are
constant across galaxy mass (or at least that the momentum imparted by
feedback processes scales linear with black hole mass), then the
$M_{\rm BH}$-$\sigma$ relation will follow.

The argument can be made more specific by supposing that the growth of
the surrounding galaxy is in fact Eddington limited by the action of
the quasar. This requires that the quasar radiation couple with the
interstellar gas much better than the Thomson cross-section with the
electrons. In fact, for galaxies with Milky Way-like dust to gas
ratios, the cross-section of dust per proton is about $\sigma_d \sim
1000 \sigma_T$. Thus for dusty neutral gas, the Eddington limit is
about 1000 times larger. Over the course of the lifetime of the galaxy
and black hole, if both grow at their respective Eddington limits (or
some equal fraction thereof) they will end up with $M_{\rm BH}/M_{\rm
gal} \sim \sigma_T / \sigma_d \sim 10^{-3}$ (\citealt{fabian12a}).
This same argument also would lead to $M_{\rm gal} \propto \sigma^4$,
as seen for elliptical galaxies in the Faber-Jackson
relation. 

\section{Commentary}

Much of the literature on feedback is motivated by the desire to get
the theoretical predictions of galaxy formation simulations to agree
with observations. Although some of the arguments are suggestive
regarding AGN feedback and the $M_{\rm BH}$-$\sigma$ relation, it is
still fairly speculative that these processes are related and not a
numerical and dimensional coincidence.

\section{Key References}

\begin{itemize}
  \item Cimatti
  \item
    \href{https://ui.adsabs.harvard.edu/abs/2015ARA%26A..53...51S/abstract}
    {{\it Physical Models of Galaxy Formation in a Cosmological
        Framework} (\citealt{somerville15a})}
\end{itemize}

\section{Order-of-magnitude Exercises}

\begin{enumerate} 
\item Estimate proportionality between the supernova rate and the star
  formation rate.
\item For an O/B association with 100 O and B stars (down to 8
$M_\odot$), what is the typical power output of supernovae?
\item For a single massive with a wind like that described above, in a
typical region of the interstellar medium, after about 100,000 years
what sized region will it have created around it?
\item Estimate of energy driven stellar feedback wind outflow rates
\item Estimate of momentum driven stellar feedback wind outflow rates
\end{enumerate} 

\section{Analytic Exercises}

\begin{enumerate}
\item Sedov
\item Derive Equation \ref{eq:shell-pb} from \ref{eq:shell}. Using
this relationship along with Equation \ref{eq:energy-bubble} show that
Equation \ref{eq:shell-evolution} holds. Finally, show that the
kinetic energy of the shell obeys Equation \ref{eq:shell-kinetic}.
\item Show that about 20\% of the power input into an expanding bubble 
this relationship along with Equation \ref{eq:energy-bubble} show that
Equation \ref{eq:shell-evolution} holds.
\item Energy from SNe (Cimatti S8.7.1)
\end{enumerate}

%\section{Numerics and Data Exercises}
%
%\begin{enumerate}
%\item Dealing with a numerical simulation result
%\end{enumerate}

\bibliographystyle{apj}
\bibliography{exex}  
