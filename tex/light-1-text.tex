\title{\bf Light I}

\section{Basics \& Nomenclature}

In astrophysical parlance, the {\it specific intensity} $I_\nu$ is the
flux of electromagnetic energy across a surface from a particular
direction, per unit area, per unit solid angle, per unit time, per
unit frequency. The specific intensity can also be expressed per unit
wavelength, and denoted in this case $I_\lambda$. For example, units
may be erg s$^{-1}$ cm$^{-2}$ Hz$^{-1}$ arcsec$^{-2}$ or erg s$^{-1}$
cm$^{-2}$ \AA$^{-1}$ arcsec$^{-2}$.  The term {\it surface brightness}
is often also used to denote the specific intensity.

The quantities $I_\nu$ and $I_\lambda$ are related:
\begin{eqnarray}
I_\nu(\nu) |{\rm d}\nu| &=& I_\lambda(\lambda = c/\nu) |{\rm
  d}\lambda| \cr
I_\nu(\nu) &=& \left|\frac{{\rm d}\lambda}{{\rm d}\nu}\right|
I_\lambda = \frac{c}{\nu^2} I_\lambda(\lambda = c/\nu) \cr
I_\lambda(\lambda) &=& \left|\frac{{\rm d}\nu}{{\rm d}\lambda}\right|
I_\nu = \frac{c}{\lambda^2} I_\nu(\nu = c/\lambda).
\end{eqnarray}

A {\it flux density} $f_\nu$ or $f_\lambda$ is the integral of the
specific intensity through the surface integrated over a range of
directions:
\begin{equation}
f_\nu = \int {\rm d}\Omega I_\nu \cos\theta. 
\end{equation}
A special unit for flux density (mostly in use in radio astronomy) is
the Jansky, which is $10^{-23}$ erg s$^{-1}$ cm$^{-2}$
Hz$^{-1}$. Sometimes the flux density as a function of wavelength or
frequency is also called the {\it spectral energy distribution}.

The electromagnetic wave can also be expressed in terms of
photons. The energy of photons correspond to a wavelength of light:
\begin{equation}
E_\nu = h\nu
\end{equation}

If an emitting object is moving along the line-of-sight to the
observer, the photon's wavelengths are shifted by a factor $(1+z)$
where $z$ is the redshift:
\begin{eqnarray}
\lambda_{\rm obs}  &=& (1+z) \lambda_{\rm emit} \cr
\nu_{\rm obs}  &=& \frac{\nu_{\rm emit}}{1+z}
\end{eqnarray}
The specific intensity and the flux is also altered by this
motion. Under both special and general relativity, the photon density
in phase space is conserved. From this it can be calculated that:
\begin{equation}
  \label{eq:sb_dimming}
  I_{\lambda, {\rm obs}} \dd\lambda_{\rm obs} = 
  \frac{I_{\lambda, {\rm emit}} \dd\lambda_{\rm emit}}{(1+z)^4}
\end{equation}
At small velocities, $z \approx v/c$, but at higher velocities
relativistic corrections are important.

A {\it luminosity} is defined as the total energy emitted from some
source per unit time per unit frequency (or wavelength). It has
typical units of erg s$^{-1}$ Hz$^{-1}$ or erg s$^{-1}$
\AA$^{-1}$. For isotropically emitting sources, the total luminosity
is related to the observed flux as:
\begin{equation}
\label{eq:luminosity_distance}
f_\nu = \frac{L_\nu}{4\pi D^2}
\end{equation}
where $D$ is the distance to the source.

In a cosmological context (relevant at 100s of Mpc distance or so
depending on the required precision), there is not a single
well-defined distance to a source. The directly observable quantity is
the redshift $z$, associated with the Universe's expansion. In this
case $D$ must be the {\it luminosity distance}, sometimes denoted
$D_L$, defined to satisfy Equation \ref{eq:luminosity_distance}; its
relation to redshift depends on the cosmological parameters
(see \citealt{hogg99cosm}). The flux dependence on distance in
standard, general relativity-based cosmology is a combination of the
dependence of the angular size on distance times the surface
brightness dimming effect.

An analogous quantity, the {\it angular diameter distance} $D_A(z)$,
can be defined that satisfies the relation for small angles:
\begin{equation}
\theta = \frac{s}{D_A}
\end{equation}
where $s$ is a physical size of an object and $\theta$ is the angular
size of that object when observed at redshift $z$.

The flux density or specific intensity can only be measured at some
finite resolution in wavelength or frequency. In optical astronomical
parlance, a common expression of resolution is:
\begin{equation}
R = \frac{\lambda}{\Delta \lambda}
\end{equation}
where $\Delta \lambda$ is the full-width half maximum (FWHM) of the
line spread function (the Green's function response of the system to a
point source).

Optical and infrared instruments that measure the wavelength
dependence of the flux density through dispersal of the light (with
diffraction gratings or prisms in the optical) are typically known as
spectrographs. Existing instruments range in resolution from $R\sim
20$ to $R>100,000$.  At higher energies, in addition to dispersal
techniques, X-ray detectors and position sensitive proportional
counters also often have intrinsic energy sensitivity.  Radio
receivers separate signals into frequencies electronically.

Imaging instruments with wavelength-dependent sensitivity (either due
to the detector or through a band pass filter) can also measure the
wavelength dependence of the flux density or specific
intensity. Typically, this is performed more coarsely at about $R\sim
5$, though narrow band systems reach $R\sim 50$ or more (typically
though not quite always requiring one exposure per filter).

In the optical and infrared the use of such coarse band pass filters
means that the interpretation of the measurements at high precision
requiring knowing the band pass very well. A full interpretation of
the measurement involves also a model of the flux density. The
observations are usually interpreted as the ratio between the signal
received from the object to the signal that would have been received
by some standard source.

This quantity is termed a {\it maggie} and depends on the band
pass. It can be expressed in terms of a model for $f_\nu$ or
$f_\lambda$ of the object and $g_\nu$ or $g_\lambda$ of the standard
source:
\begin{eqnarray}
\label{eq:maggies}
  \mu_b &=&
  \frac{\int {\rm d}\nu (f_\nu(\nu)/\nu) R_b(\nu)}
       {\int {\rm d}\nu (g_\nu(\nu)/\nu) R_b(\nu)} \cr
       &=&
  \frac{\int {\rm d}\lambda f_\lambda(\lambda) \lambda R_b(\lambda) }
       {\int {\rm d}\lambda g_\lambda(\lambda) \lambda R_b(\lambda)} 
\end{eqnarray}
where $R(\lambda)$ or $R(\nu)$ is the contribution of a photon
entering Earth's atmosphere (or the aperture of a space telescope) to
the output signal of the instrument in band $b$. This formula is the
same for photon-counting and energy-counting devices, though note that
for the latter instrumentalists typically report a definition of
$R(\lambda)$ or $R(\nu)$ which is the contribution of an amount of
energy to the signal rather than that of a photon; therefore to use
Equation (\ref{eq:maggies}) the $R(\lambda)$ or $R(\nu)$ so defined
needs to be transformed into per-photon units.

Typical choices of $g_\nu$ are the spectrum of Vega (which is only
known to a few percent accuracy in the optical and near-infrared) and
the AB system of $g_\nu = 3631$ Jy (the flux density of Vega near 5500
\AA).  All absolute measurements are based on standards whose flux
densities are thought to be known, but only rarely by comparing
directly to Vega (which is too bright) and never to an AB source (they
don't exist). Relative measurements can be calibrated fairly precisely
independently of absolute calibration, though for broad band passes
the colors of the sources need to be accounted for.

The classical astronomical {\it magnitude} is defined as:
\begin{equation}
m = -2.5 \log_{10} \mu
\end{equation}
The bright end of this system corresponds roughly to the original
magnitude system developed by astronomers in the ancient world. That
this remains true is a testament to the near fanatical commitment to
backwards compatibility that characterizes many astronomers.

The luminosity is often expressed as the {\it absolute magnitude},
which is defined as the magnitude the object would have were it at
rest with respect to us, at 10 pc distance. For non-cosmologically
distant objects (within a few 10s of Mpc) this can be expressed as:
\begin{equation}
M = m - 5 \log_{10} \left(\frac{D}{10 {\rm ~pc}}\right) = m - {\rm DM}
\end{equation}
where DM is defined as the distance modulus. 

At cosmological distances, two effects are important. First, $D$ must
be the luminosity distance. Second, the observed band pass corresponds
to a different part of the rest frame spectrum of the object for each
different redshift $z$. This effect is known as the $K$-correction,
defined as:
\begin{equation}
M = m - {\rm DM} - K(z; f_\lambda)
\end{equation}
Basically the $K$-correction accounts for the ratio of the observed to
rest frame flux given the shape of the flux density
$f_\lambda(\lambda)$ (it does not depend on the amplitude). It is
common at low redshifts to $K$-correct from the observed bandpass to
the same band pass in the rest frame. However, at higher redshift it
can be more stable to $K$-correct to a bandpass with a closer
effective wavelength to that observed. \citet{hogg02c} describes the
mathematical definition of the $K$-correction in all such cases. 

\section{Commentary}

While the magnitude system was originally used for convenience in
calibration over a large dynamic range in brightness, it retains its
usefulness because of its encapsulation of the response function
$R(\lambda)$.  Sometimes measurements of $\mu$ are expressed in terms
of $f_{\lambda, {\rm eff}}$ at some choice of $\lambda_{\rm eff}$
(e.g. by just multiplying an AB maggie by 3631 Jy and using some
average effective wavelength of the filter), but for broad band passes
either $f_{\lambda, {\rm eff}}$ or $\lambda_{\rm eff}$ is a strong
function of the actual, often unknown $f_\lambda(\lambda)$. $\mu$ and
$m$ do not suffer from this dependence (though interpreting them
physically still can only be done by accounting for the filter curve
and a model for $f_\lambda(\lambda)$).

$K$-corrections depend on knowing the flux density itself, or at least
its wavelength dependence. To estimate that wavelength dependence
requires having a measurement or model. Thus, the $K$-correction is
naturally an uncertain quantity.

\section{Important numbers}

\begin{itemize}
\item $h\nu \sim 1$ eV for $\lambda \sim 1.2$ $\mu$m.
\item $\nu \sim 300$ THz for $\lambda \sim 1$ $\mu$m.
\item $\nu \sim 30$ GHz for $\lambda \sim 1$ cm.
\item $\nu \sim 1$ GHz for $\lambda \sim 1$ m.
\item $m_{V, {\rm Vega}} \sim 0$ mag.
\item $M_{V, \odot} \sim 5$ mag.
\item $M_{V, {\rm Milky Way}} \sim -20$ mag.
\end{itemize}

\section{Key References}

\begin{itemize}
  \item
    \href{http://adsabs.harvard.edu/abs/2000asqu.book.....C}{
    {\it Allen's Astrophysical Quantities},
      \citet{cox00a}}, Chapter 5
  \item
    \href{http://adsabs.harvard.edu/abs/1999astro.ph..5116H}{
    {\it Distance measures in Cosmology},
      \citet{hogg99cosm}}
  \item
    \href{http://adsabs.harvard.edu/abs/2002astro.ph.10394H}{
    {\it The $K$-correction},
      \citet{hogg02c}}
\end{itemize}

\section{Order-of-magnitude Exercises}

\begin{enumerate} 
\item If Vega is about 8 parsecs away and has $m_V \sim 0$, how far
  away would it be visible through the deepest images from the Hubble
  Space Telescope ($m_V\sim 30$)?

\begin{answer}
  The magnitude difference of 30
    corresponds to a factor of $10^{12}$ in luminosity. Using Equation
    \ref{eq:luminosity_distance}, this translates to $10^6$ in
    distance. So Vega-like stars are visible (in principle) to about 8
    Mpc. In practice, if such stars were in another galaxy, it is very
    difficult to resolve them from the other stars in the galaxy; if
    they were free-floating between galaxies there are still a high
    density of 30th magnitude galaxies causing confusion in the image.
\end{answer}

\item Estimate the number of photons per second that enter your eye
    per second in visible light (4000--7000 \AA) from a star with
  magnitude $\sim 6$ (about the faintest visible at a dark
  site). Assume a nighttime pupil diameter of 5 mm.

\begin{answer}
In this wavelength range, Vega and AB magnitudes are about the same at
the precision necessary here, so we don't have to worry about which
version we are dealing with. So:
\begin{equation}
f_\nu \sim (3631 \mathrm{~Jy}) 10^{-0.4 m} \sim 14 \mathrm{~Jy} = 1.4 \times
10^{-22} \mathrm{~erg} \mathrm{~cm}^{-2} \mathrm{~s}^{-1} \mathrm{~Hz}^{-1}
\end{equation}
The flux density in the visible should be $f_\nu \Delta\nu$ where:
\begin{equation}
\Delta\nu = c \left(\frac{1}{4000 \mathrm{~\AA}} -
\frac{1}{7000 \mathrm{~\AA}}\right) \sim 320 \mathrm{~THz},
\end{equation}
and thus $f \sim 4\times 10^{-8}$ erg cm$^{-2}$ s$^{-1}$. Each photon
has an energy (assuming $\lambda = 5500$ \AA):
\begin{equation}
E = h\nu = (6.62\times
10^{-27} \mathrm{~erg~Hz}^{-1})( \mathrm{550~THz}) \sim 4 \times
10^{-12} \mathrm{~erg}.
\end{equation}
So the flux of photons is:
\begin{equation}
\frac{\dot N}{A} = \frac{f_\nu \Delta\nu}{h\nu} \sim 
10^4 \mathrm{~s}^{-1} \mathrm{~cm}^{-2}.
\end{equation}
If $A \sim \pi r^2 \sim 0.2$ cm$^2$ then $\dot N \sim 2000$
s$^{-1}$.

Note that your retina is a very efficient detector---at low flux
levels it detects pretty much every photon that hits it---and 2000 is
a lot of photons per second. Why are these stars so hard to see?
\end{answer}

\item How much does surface brightness dimming change the magnitudes
  per square arcsecond for a galaxy at redshift $z\sim 1$?

\begin{answer}
The specific intensity integrated over all wavelengths is reduced by
$(1+z)^4$. In magnitudes this is:
\begin{equation}
\Delta m = 2.5 \log_{10} (1+z)^4 = 10 \log_{10} (1+z) \sim
3 \mathrm{~mag}
\end{equation}
\end{answer}
\end{enumerate}   

\section{Analytic Exercises}

\begin{enumerate}
\item For a Gaussian line spread function with a standard deviation
  $\sigma$, what is the FWHM?

\begin{answer}
The FWHM is twice the distance from the peak to the point halfway down
the peak, so it is determined by:
\begin{equation}
\exp\left(- ({\rm FWHM} / 2)^2 / 2 \sigma^2\right) = 0.5.
\end{equation}
Solving for the FWHM yields: 
\begin{equation}
{\rm FWHM} = 2 \sqrt{2\ln 2} \sigma \approx 2.35 \sigma
\end{equation}
\end{answer}
\item Prove Equation \ref{eq:sb_dimming}, based on the fact that
  photon density in phase space is conserved.

\begin{answer}[Author: Trey Jensen]
Beginning with the photon particle distribution, $f(x,p,t)$, we know
that subject only to cosmic expansion, the number of photons will be
conserved,
\begin{equation}
\dd N =f(x,p,t)\dd V \dd^3 p.
\end{equation}
If we rewrite the right hand side in the suggestive manner,
\begin{equation}
f(x,p,t)p^2\dd x \dd A \dd p \dd \Omega,
\end{equation}
where $\dd A$ is the area element, $\dd \Omega$ the solid angle
element of momenta (direction of propagation), then we see this can be
written with the specfic intensity in mind. The specific intensity is
the flux of energy through a surface per area, per solid angle, per
time, per frequency. We can get an infinitesimal number of photons
via,
\begin{equation}
 \dd N =\frac{I_\nu}{E} \dd A \dd \Omega  \dd t \dd \nu.
\end{equation}
Equating these two equations of $\dd N$, and because we have massless
particles, $p=E=h\nu$, then using the fact that $\dd x / \dd t = c$:
\begin{equation}
    I_\nu=f(x,p,t)h^{4} \nu^3 c
\end{equation}
In the FRW metric, frequency $\nu$ scales inversely with $a(t)$ (this
can also be derived from the geodesic equation of a photon). Thus,
inspecting the above equation,
\begin{equation}
    I_\nu\propto a(t)^{3} = \frac{1}{(1+z)^3}.
\end{equation}
Then if we consider the integral of $I_\nu$:
\begin{equation}
    I_{\nu, \rm obs} \dd\nu_{\rm obs}  =
    \frac{1}{(1+z)^4} I_{\nu, \rm emit} \dd\nu_{\rm emit}
\end{equation}
and similar for $I_\lambda$.
\end{answer}

\item Based on Equation \ref{eq:sb_dimming}, how is the angular
  diameter distance related to the luminosity distance? 

\begin{answer}
The angular diameter distance must satisfy:
\begin{equation}
\label{eq:da}
D_A = \frac{s}{\theta}
\end{equation}
Take a uniform surface brightness sphere of radius $s$. The flux
density must satisfy two equalities:
\begin{equation}
f_\nu = \frac{L_\nu}{4\pi D_L^2} = I_\nu \pi \theta^2.
\end{equation}
Using Equation \ref{eq:sb_dimming} and \ref{eq:da}, the second equality
implies:
\begin{equation}
\frac{L_\nu}{4\pi D_L^2} = I_\nu \pi \theta^2 = \frac{I_{\nu, 0} \pi
s^2}{(1+z)^4 D_A^2}.
\end{equation}
Then using $L=4\pi s^2 I_{\nu, 0}$ we find:
\begin{equation}
D_L = (1+z)^2 D_A
\end{equation}
\end{answer}
\end{enumerate}

\section{Numerics and Data Exercises}

\begin{enumerate}
\item Retrieve a spectrum of a star, a quasar, and a galaxy from the
  Sloan Digital Sky Survey. Plot each of them. These spectra are given
  in $f_\lambda$ (per-\AA) units. Convert one them to $f_\nu$
  (per-Hertz) and plot it. Smooth one of them in $f_\lambda$ with a
  Gaussian corresponding to $R\sim 1000$ and plot it.
\item Plot $D_L$ versus $z$ based on the equations found
  in \citet{hogg99cosm}, for a flat $\Lambda$CDM cosmology with
  $\Omega_m = 0.3$ and $H_0 = 70$ km s$^{-1}$ Mpc$^{-1}$. Determine at
  what redshift the difference between the actuall inferred luminosity
  of an object, and the luminosity assuming $D_L = cz / H_0$, would
  reach 1\%.
\item Download the filter curve for the SDSS $g$ and $r$
  bands. Calculate the observed $g$ and $r$ band magnitudes
  corresponding to a galaxy spectrum (say for some galaxy with $z \sim
  0.15$ or greater). Note that this won't necessarily be the same as
  the magnitudes measured from the images, since the spectra are taken
  through 2- or 3-arcsec diameter fibers. Calculate the rest-frame
  $g-r$ color, and also what the $K$-correction would be for galaxies
  with this SED in the $r$-band between about $z\sim 0$ and $z\sim
  0.25$. Download the photometric data for a sample of galaxies
  between about $z\sim 0$ and
  $0.25$. Plot their $g-r$ colors versus redshift, together with the
  predicted colors of the galaxy you have a spectrum of.
\item The \href{http://www.mpia.de/THINGS}{THINGS survey} maps a
  number of nearby galaxies in 21-cm radio emission, mapping a
  hyperfine transition in hydrogen. Download the data on the galaxy
  NGC 2403 and estimate its rotation velocity at the furthest
  measurable points (it is all right to do this by eyeballing rather
  than a detailed fit).
\item Using the ROSAT all sky survey (for example
  at \href{http://www.xray.mpe.mpg.de/cgi-bin/rosat/rosat-survey}{this
  web site}), find an X-ray image of the center of the galaxy NGC
  1068. Compare this to the X-ray image of the center of the galaxy
  NGC 3992. Use the hardest (highest energy) band available.
\end{enumerate}

\bibliographystyle{apj}
\bibliography{exex}  
