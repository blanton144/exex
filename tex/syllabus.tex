\documentclass[11pt, preprint]{aastex}
\usepackage{hyperref}
\usepackage{rotating}
 
\setlength{\footnotesep}{9.6pt}

\newcounter{thefigs}
\newcommand{\fignum}{\arabic{thefigs}}

\newcounter{thetabs}
\newcommand{\tabnum}{\arabic{thetabs}}

\newcounter{address}

\begin{document}

\title{\bf Extragalactic Astrophysics / PHYS-GA 2051 / Fall 2022 / Syllabus }

~

\noindent This course teaches the astrophysics of galaxies and quasars
at the graduate level.

\noindent You can find the course notes at the
\href{http://blanton144.github.io/exex}{course web site}. Please read
the introduction posted on the web site.

\noindent Useful textbooks are {\it Galaxy Evolution} by Cimatti,
Fraternali, \& Nipoti, and {\it Extragalactic Astronomy and
  Cosmology}, by Peter Schneider. A good fraction of my notes are
drawn from those books.

\noindent Class meets Monday and Wednesday at 11:00am in Room 802 of
726 Broadway.

\noindent The classes will proceed as shown on the next page (subject
to revision!).

\noindent The following are the expectations in the course and
classes:

\begin{itemize}

\item {\it Reading}: I expect you to read the provided notes {\it
  before} each class. I am likely to call on you in class to ask 
  specific questions about things I think are especially important.

\item {\it Homework}: Each week I will assign one of the questions in
  the notes we covered. I encourage you to discuss and work on this
  together. If any subset of you would like to submit as a group,
  please ask my permission beforehand.

\item {\it Homework Presentation}: I will ask one of you the following
  week to describe your solution to the class. 

\item {\it Review Paper \& Presentation}: In the first two weeks of
  the course, I will assign you each a topic covering a recent finding
  in extragalactic astrophysics, and you will prepare a short review
  paper and a presentation for the class. The paper should be
  formatted in \LaTeX and be about 5 pages of text plus references and
  (if appropriate) figures. Mid-semester a FULL DRAFT of this paper
  will be due. I expect to give substantial feedback on the draft in
  preparation for the final version due at the semester's end. You
  will each prepare a 10 minute presentation summarizing your review
  paper.

\end{itemize}

\baselineskip 0pt
\begin{table}
\footnotesize
\begin{tabular}{|c||c|c|}
\hline
{\it Sep.~8} & Inventory & \cr
{\it Sep.~12} & Light I \& II & \cr
{\it Sep.~14} & Telescopes \& Atmosphere & \cr
{\it Sep.~19} & Detectors, Images, Spectra & {\bf Exercise \#1 due} \cr
{\it Sep.~21} & Distance Ladder & \cr
{\it Sep.~26} & Cosmology & {\bf Exercise \#2 due} \cr
{\it Sep.~28} & Structure Formation & \cr
{\it Oct.~3} & Galaxy Demographics & {\bf Exercise \#3 due} \cr
{\it Oct.~5} & Galaxy Morphology & \cr
{\it Oct.~11} & Galaxy Scaling Relations & {\bf Exercise \#4 due}\cr
{\it Oct.~12} & Stellar Evolution & --- \cr
{\it Oct.~17} & Stellar Populations &  {\bf Exercise \#5 due} \cr
{\it Oct.~19} & Stellar Populations & \cr
{\it Oct.~24} & Stellar Dynamics & {\bf Exercise \#6 due} \cr
{\it Oct.~26} & Stellar Dynamics &  \cr
{\it Oct.~31} & Stellar Dynamics & {\bf Full paper draft due} \cr
{\it Nov.~2} & ISM \& Dust in Galaxies & \cr
{\it Nov.~7} & ISM \& Dust in Galaxies & {\bf Exercise \#7 due} \cr
{\it Nov.~9} & Gravitational Lensing & \cr
{\it Nov.~14} & Gravitational Lensing &   {\bf Exercise \#8 due} \cr
{\it Nov.~16} & Groups \& Clutsers &   \cr
{\it Nov.~21} & Star Formation in Galaxies &  {\bf Exercise \#9 due} \cr
{\it Nov.~23} & Active Galactic Nuclei  & {\bf May want to reschedule}\cr
{\it Nov.~28} & Quasars & {\bf Exercise \#10 due} \cr
{\it Nov.~30} & High Redshifts & \cr
{\it Dec.~5} & Theory of Galaxy Formation & {\bf Exercise \#11 due} \cr
{\it Dec.~7} & Feedback in Galaxy Formation & \cr
{\it Dec.~12} & Future of Extragalactic Astronomy & \cr
{\it Dec.~19} & --- & {\bf Final paper due} \cr
\hline
\end{tabular}
\end{table}

% Other topics
%  - ly-alpha

\end{document}

