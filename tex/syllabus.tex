\documentclass[11pt, preprint]{aastex}
\usepackage{hyperref}
\usepackage{rotating}
 
\setlength{\footnotesep}{9.6pt}

\newcounter{thefigs}
\newcommand{\fignum}{\arabic{thefigs}}

\newcounter{thetabs}
\newcommand{\tabnum}{\arabic{thetabs}}

\newcounter{address}

% TODO
%
% Links in intro
% Location of class
% Update distance ladder commentary

% Assignments not done:
%  atmosphere OOM and analytic
%  detector OOM #1
%  All analytic in "images"
%  Cosmology OOM and analytic
%  Structure OOM 
%  Structuer analytic

\begin{document}

\title{\bf Extragalactic Astrophysics / PHYS-GA 2051 / Fall 2024 / Syllabus }

~

\noindent This course teaches the astrophysics of galaxies and quasars
at the graduate level.

\noindent You can find the course notes at the
\href{http://blanton144.github.io/exex}{course web site}. Please read
the introduction posted on the web site.

\noindent Useful textbooks are {\it Galaxy Evolution} by Cimatti,
Fraternali, \& Nipoti, and {\it Extragalactic Astronomy and
  Cosmology}, by Peter Schneider. A good fraction of my notes are
drawn from those books.

\noindent Class meets Monday and Wednesday at 11:00am in Room 902 of
726 Broadway.

\noindent The classes will proceed as shown on the next page (subject
to revision!).

\noindent The following are the expectations in the course and
classes:

\begin{itemize}

\item {\it Reading}: I expect you to read the provided notes {\it
  before} each class. I am likely to call on you in class to ask 
  specific questions about things I think are especially important.

\item {\it Homework}: Each week I will assign one of the questions in
  the notes we covered, due each week starting Tuesday, Sept 16. I
  encourage you to discuss and work on this together. If any subset of
  you would like to submit as a group, please ask my permission
  beforehand.

\item {\it Homework Presentation}: On Thursday after each exercise is
  due, I will ask one student to describe their solution to the class;
  you won't have warning but I am not expecting a polished
  presentation.

\item {\it Review Paper \& Presentation}: Before Tuesday, Sept 16, I
  will assign you each a topic covering a recent finding in
  extragalactic astrophysics. You will prepare a short review paper
  and a presentation for the class. The paper should be formatted in
  \LaTeX and be about 8 pages of text and figures plus references and
  (if appropriate) figures. Mid-semester a {\it full draft} of this
  paper will be due---meaning, a version you think is done (spoiler
  alert: I will have comments and questions and requests to more fully
  probe issues so it won't be done). I expect to give substantial
  feedback on the draft in preparation for the final version due at
  the semester's end. You will each prepare a 10 minute presentation
  summarizing your review paper.

\end{itemize}

\noindent It is common now for many people to make use of generative
AI tools (e.g. Co-Pilot) for programming or other tasks. {\it If you
  do so in the homeworks or the review paper, I require you to
  document doing so, providing the prompt and the raw output.} More
generally, just remember that these tools are only useful if you can
determine that they are working correctly, so using them doesn't save
you from checking the results.

\baselineskip 0pt
\begin{table}
\footnotesize
\begin{tabular}{|c||c|c|}
\hline
{\it Sep.~4} & Inventory & \cr
{\it Sep.~9} & Light I \& II & \cr
{\it Sep.~11} & Telescopes \& Atmosphere & \cr
{\it Sep.~16} & Detectors, Images, Spectra & {\bf Exercise \#1 due} \cr
{\it Sep.~18} & Distance Ladder & \cr
{\it Sep.~23} & Cosmology \& Structure & {\bf Exercise \#2 due} \cr
{\it Sep.~25} & Stellar Populations I & \cr
{\it Sep.~30} & Stellar Populations II & {\bf Exercise \#3 due} \cr
{\it Oct.~2} & Stellar Populations III & \cr
{\it Oct.~7} & Galaxy Populations & {\bf Exercise \#4 due}\cr
{\it Oct.~9} & Stellar Dynamics I & \cr
{\it Oct.~14} & Stellar Dynamics II &  {\bf Exercise \#5 due} \cr
{\it Oct.~16} & Stellar Dynamics III & \cr
{\it Oct.~21} & Stellar Dynamics IV & {\bf Exercise \#6 due} \cr
{\it Oct.~23} & Interstellar Medium I &  \cr
{\it Oct.~28} & Interstellar Medium II  & {\bf Exercise \#7 due } \cr
{\it Oct.~30} & Dust & \cr
{\it Nov.~4} & Gravitational Lensing I & {\bf Exercise \#8 due} \cr
{\it Nov.~6} & Gravitational Lensing II & \cr
{\it Nov.~11} & Gravitational Lensing III &  {\bf Full Paper Draft Due} \cr
{\it Nov.~13} & Groups \& Clusters &   \cr
{\it Nov.~18} & Active Galactic Nuclei I &  {\bf Exercise \#9 due} \cr
{\it Nov.~20} & Active Galactic Nuclei II & \cr
{\it Nov.~25} & Lyman-$\alpha$ Forest & {\bf Exercise \#10 due} \cr
{\it Dec.~2} & Theory of Galaxy Formation I & \cr
{\it Dec.~4} & Theory of Galaxy Formation II & {\bf Exercise \#11 due} \cr
{\it Dec.~8} & Theory of Galaxy Formation III & \cr
{\it Dec.~11} & Future of Extragalactic Astronomy & \cr
{\it Dec.~19} & --- & {\bf Final paper due} \cr
\hline
\end{tabular}
\end{table}

% Other topics
%  - ly-alpha

\end{document}

