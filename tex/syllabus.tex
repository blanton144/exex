\documentclass[11pt, preprint]{aastex}
\usepackage{hyperref}
\usepackage{rotating}
 
\setlength{\footnotesep}{9.6pt}

\newcounter{thefigs}
\newcommand{\fignum}{\arabic{thefigs}}

\newcounter{thetabs}
\newcommand{\tabnum}{\arabic{thetabs}}

\newcounter{address}

\begin{document}

\title{\bf Extragalactic Astrophysics / PHYS-GA 2051 / Fall 2018 / Syllabus }

~

\noindent This course teaches the astrophysics of galaxies and quasars
at the graduate level.

\noindent You can find the course notes at the
\href{http://blanton144.github.io/exex}{course web site}. Please read
the introduction posted on the web site.

\noindent A useful textbook is {\it Extragalactic Astronomy and
  Cosmology}, by Peter Schneider. A good fraction of my notes are
drawn from that book.

\noindent Class meets Tuesday and Thursday at 11:00am in Room 1045 of
726 Broadway, according to Albert.

\noindent The classes will proceed as shown on the next page (subject
to revision!).

\noindent There will be two types of assignments in this course:

\begin{itemize}

\item {\it Homework} will be based on exercises described in the
  notes. There are answers that I will make available, but only to
  a small number of the exercises. You will help complete the answers
  (with proper attribution to you of course). Each week I will assign
  one of the questions to each of you in the notes we covered, and you
  will submit an answer in the form of a LaTeX file or Python
  notebook, emailed to me.

\item You will each write a short {\it Review Paper} describing the
  significance of some recent finding in extragalactic
  astrophysics. The paper should be 5--6 pages of text plus references
  and (if appropriate) figures. During the first week or so of the
  course I will assign each of you a topic. Mid-semester a FULL DRAFT
  of this paper will be due. I expect to give substantial feedback on
  the draft in preparation for the final version due at the semester's
  end.

\end{itemize}

\baselineskip 0pt
\begin{table}
\footnotesize
\begin{tabular}{|c||c|c|}
\hline
{\it Sep.~3} & Inventory & \cr
{\it Sep.~8} & Light I \& II & \cr
{\it Sep.~10} & Telescopes \& Atmosphere & \cr
{\it Sep.~15} & Detectors, Images, Spectra & {\bf Exercise \#1 due} \cr
{\it Sep.~17} & Distance Ladder & \cr
{\it Sep.~22} & Cosmology & {\bf Exercise \#2 due} \cr
{\it Sep.~24} & Structure Formation & \cr
{\it Sep.~29} & Galaxy Contents & {\bf Exercise \#3 due} \cr
{\it Oct.~1} & Galaxy Scaling Relations & \cr
{\it Oct.~6} & Stellar Clusters & {\bf Exercise \#4 due}\cr
{\it Oct.~8} & Stellar Evolution & --- \cr
{\it Oct.~13} & Stellar Populations &  {\bf Exercise \#5 due} \cr
{\it Oct.~15} & Stellar Dynamics & \cr
{\it Oct.~20} & Stellar Dynamics & {\bf Exercise \#6 due} \cr
{\it Oct.~22} & Interstellar Medium &  \cr
{\it Oct.~27} & Dust in Galaxies &  {\bf Full paper draft due} \cr
{\it Oct.~29} & Gravitational Lensing & \cr
{\it Nov.~3} & Gravitational Lensing & {\bf Exercise \#7 due} \cr
{\it Nov.~5} & Groups \& Clusters & \cr
{\it Nov.~10} & Mass in Galaxies &   {\bf Exercise \#8 due} \cr
{\it Nov.~12} & Star Formation in Galaxies &   \cr
{\it Nov.~17} & Active Galactic Nuclei &  {\bf Exercise \#9 due} \cr
{\it Nov.~19} & Quasars & \cr
{\it Nov.~24} & High Redshifts & {\bf Exercise \#10 due} \cr
{\it Dec.~1} & Theory of Galaxy Formation& \cr
{\it Dec.~3} & Gas Accretion & {\bf Exercise \#11 due} \cr
{\it Dec.~8} & Chemical Evolution & \cr
{\it Dec.~10} & Feedback & \cr
{\it Dec.~17} & --- & {\bf Final paper due} \cr
\hline
\end{tabular}
\end{table}

% Other topics
%  - ly-alpha

\end{document}

