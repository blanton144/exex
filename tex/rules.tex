\documentclass[11pt, preprint]{aastex}
\usepackage{hyperref}
\usepackage{rotating}
 
\setlength{\footnotesep}{9.6pt}

\newcounter{thefigs}
\newcommand{\fignum}{\arabic{thefigs}}

\newcounter{thetabs}
\newcommand{\tabnum}{\arabic{thetabs}}

\newcounter{address}

\title{\bf Extragalactic Astrophysics / Some Guidelines On Writing}

\section{General Guidance}

Generally: guide the reader through the document. Don't hide the
ball. Make the main points easy to find. When you are done ask
youself: can the reader understand my main point from the abstract?
how well would they get your point if they just read the first
paragraph of each section? or the first sentence of each paragraph? or
just the conclusions? or just looked at the figures?

This classic advice holds for the structure of the paper as a whole:
first, say what you are going to say; then say it; finish by saying
what you said.

Each paragraph should express one idea, and that idea should appear in
the first sentence.

Use the active voice, not the passive voice. The passive voice leads
to ambiguity.

Facts that are not self-evident require a citation. 

Do not use contractions (i.e. say ``do not,'' do not say ``don't'').

Define acronyms exactly once, the first time they are used. For
example: ``We investigate the active galactic nuclei (AGN) in the
local universe, using the AGN sample of \ldots''

\section{Introductions}

The first paragraph of the introduction should introduce what the
paper aims to do.

You do not need to give a review of an entire field in the
introduction. You need to introduce only those things that are
necessary to understand the point of the paper.

The bulk of the introduction should act as a funnel bringing the
reader from the big picture (``we do not know how galaxies form'') to
the narrow question that is the topic of the paper (``we therefore
want to measure the mass-metallicity relation of galaxies at $z\sim
2$'').

The last substantial paragraph of the introduction should outline the
rest of the paper. 

If there are choices of cosmological parameters, magnitude zeropoint
systems, or other general assumptions saturating the paper, list them
at the beginning of the paper.

\section{Figures}

Figures need axis labels, with units listed.

Figures should be intelligible even if converted to black-and-white.

Fonts in figures should not be smaller than the text in the document.

Captions should contain all the information necessary to interpret the
figure---what every line and every type of point is, for example.

Captions do not need to contain the scientific interpretation of the
figure.

If a figure is largely the same as a previous figure but with small
differences, the caption should read: ``Similar to Figure XYZ, with
\ldots'', and then list the differences.

\section{Responding to the referee}

The referee is a volunteer. Be nice to them even if they are cranky to
you, and/or if they are wrong. Thank them for their time, and be
respectful in your language to them.

Respond to {\it every} point that the referee makes, even if you do
not accept their suggestion. 

When the referee does not understand something in the paper, remember
that they probably have read it more carefully than its median reader
will. So consider how you can make the point they did not understand
clearer. 

\end{document}

