\title{\bf Nucleosynthesis}

\section{Big Bang Nucleosynthesis}

We can measure the metallicity and even individual abundances in stars
and in the gas of galaxies. For the Milky Way and some nearby
galaxies, we can measure these quantities for individual stars or
nebulae, through emission or absorption. For more distant galaxies, we
measure these quantities averaged over some or many stars and
individual nebulae. It is possible to make the appropriate
measurements of galaxies at high redshifts, up to $z\sim 2$ in the
case of emission measures and up to $z\sim 6$ in the case of
absorption signatures.

The elements are a signature of the process of
nucleosynthesis. Originally the Universe contained to a good
approximation only hydrogen and helium, and the existence of the rest
of the elements depends on stellar processes of one sort or another.

The initial abundances were determined during Big Bang
nucleosynthesis. At early times, the following reactions maintain
equilibrium abundances between protons and neutrons. 
\begin{eqnarray}
p + e^- &\rightleftharpoons& n + \nu \cr
p + \bar\nu &\rightleftharpoons& n + e^+ \cr
\end{eqnarray}
These reactions depend on having $e^+$ and $e^-$ particles in
abundance. At high temperatures, photons are pair-creating them (and
they are annhilating). As the Universe cools, these reactions fall out
of equilibrium approximately when $kT \sim m_e c^2$; at these
temperatures, the electron-positron pairs will annhilate but not be
replaced by pair-creating photons.

We can estimate the relative number of neutrons and photons at this
stage with the Boltzmann factor:
\begin{equation}
\frac{n_{\rm n}}{n_{\rm p}} \sim \exp\left(-\Delta m_{\rm np} c^2 / kT\right) \sim
\exp\left(-\Delta m_{\rm np} / m_e\right)
\end{equation}
Since $m_{\rm np} \sim m_e$, this ratio is of order unity at
freeze-out. More detailed calculations yield $n_{\rm n}/n_{\rm p} \sim
0.2$.

What happens subsequently is that the neutrons and protons undergo the
following reactions:
\begin{eqnarray}
n + p &\rightarrow& {}^2H + \gamma \cr
{}^2H + {}^2H &\rightarrow& {}^3H + p \cr
{}^3H + {}^2H &\rightarrow& {}^4He + n
\end{eqnarray}
These reactions proceed either until the universe is not dense and hot
enough to sustain them, or until no neutrons are left to fuel them
(since each cycle moves two neutrons into a helium nucleus). The
exercises show that the maximum mass fraction of helium that can
result (based on the starting number of neutrons) is about $Y=0.33$.
If the baryon density was very high this limit would be reached. In
our universe, the baryon density is low and $Y\sim 0.25$.

The amount of other elements produced in this process is tiny. By mass
the deuterium fraction is about $10^{-5}$, and amounts of lithium and
other light elements is even smaller. Deuterium is a key indicator,
because its mass fraction depends on the baryon density; a higher
baryon density produces less deuterium, because the processes above
continue to completion more efficiently. Although deuterium can be
destroyed in stellar processes later, it can be measured in relatively
pristine environments and used as a baryon fraction indicator.

\section{Stellar Nucleosynthesis}

Starting with H and He, stellar processes build up the higher mass
elements. These processes occur in stars during their lifetimes and
are returned to the interstellar medium through stellar winds and
supernovae. Substantial nucleosynthesis also occurs during
supernovae. More exotic phenomena, such as neutron star mergers, can
also create elements (such as $r$-process elements), and interactions
in the interstellar medium with and among cosmic rays can also affect
the abundance distribution. The outlines of the processes involved can
be found in the classic paper of Burbridge, Burbridge, Fowler, \&
Hoyle (1957).

Nuclei are only stable within the isotopic stability band, with
suitable combinations of neutrons and protons.  Coulomb repulsion of
protons makes it necessary to include around the same number (or more)
of neutrons as protons. Typically multiple isotopes of a given element
can be stable. The ratios of the isotopic abundances are set by the
processes that created them.

In the plane of $Z$ and $N$, several different processes can occur.
The first is the set of nuclear fusion reactions, which are
myriad. For example, the $p$-$p$ chain and CNO cycle in stellar
hydrogen burning produce helium.

A second is radioactive decay. Aside from very unstable massive atoms
(e.g. plutonium-239) which can fission dramatically, the dominant
processes are $\beta$-decay and $\alpha$-decay:
\begin{itemize}
\item $\beta^{+}$-decay: $p \rightarrow n + e^+ + \nu_e$, which is
  $+1$ in $N$ and $-1$ in $Z$.
\item $\beta^{-}$-decay: $p \rightarrow n + e^+ + \nu_e$, which is
  $-1$ in $N$ and $+1$ in $Z$.
\item $\alpha$-decay: in which a nuclei emits an $\alpha$ particular,
  which is $-2$ in $N$ and $-2$ in $Z$.
\end{itemize}
A third process important for massive elements is neutron capture.

The binding energy per nucleon along this band increases from H to
$^{56}$Fe, and then declines after that.  This means that building
iron out of H will extract energy, but nucleosynthesis beyond that
cannot extract energy. The maxima in the binding energy are also
related to the stability of those nuclei. Thus, $^4$He is strongly
favored relative to nearby nuclei. Li, Be, and B are easily
destroyed. The $\alpha$-elements $^{12}$C, $^{16}$O, $^{20}$Ne,
$^{24}$Mg, $^{28}$Si, $^{32}$S, $^{36}$Ar and $^{40}$Ca are quite
stable and favored.

All of these features shape the resulting distribution of elements.
Hydrogen ($X=0.7$) and Helium ($Y=0.28$) are the dominant elements. By
mass, the remainder is only about $Z=0.02$ in the Galactic
neighborhood (roughly the solar abundance).

Then there is a gap with Li, Be, and B having very low
abundances. They are easily destroyed in stars (though they are
created by spallation in cosmic rays). Note that there is a serious
problem associated with lithium; there is not enough $^7$Li by a
factor three or so (Fields 2012). D is more common; it is indeed one
of the most powerful indicators that BBN is correct. Its abundance is
also affected by stellar processes; it is very easily destroyed in
stars.

In between C and Fe the $\alpha$-elements dominate. The other elements
here are often referred to as odd-$Z$ elements. All of the elements
here are produced primarily in massive stars in late stages of their
evolution. They are returned to the interstellar medium through
core-collapse supernovae, which eject the parts of the star outside
the core.  The supernova itself also modifies the elemental
distribution, as we will see in a second. {\bf need figure 5.9}

The Fe group elements show a broad peak. This peak shape corresponds
roughly to thermodynamic equilibrium at $10^9$ K or so. These
conditions can occur in the shock emerging from a core-collapse
supernova. However, it can also occur when a white dwarf detonates in
a Type Ia supernova. In either case, a substantial fraction of the Fe
results from producing unstable $^{56}$Ni which decays into $^{56}$Fe.

Finally, the elements beyond the Fe peak are formed mostly through
neutron capture. There are two basic mechanisms under which this
happens; under ``slow'' neutron flux or a ``rapid'' neutron flux.  The
$s$-process is found in AGB stars. The $r$-process is found in
supernovae and neutron star mergers. $r$ and $s$-process are
characteristically different because of the set of isotopes that each
nucleus passes through.

Under a slow neutron flux, the nucleus has time to return to the
valley of stability, and so comes up the valley slowly. With a rapid
flux, the nucleus hugs the neutron-rich edge of the valley; basically
staying near where the neutron capture rate equals the $\beta$-decay
rate. This leads to characteristic differences in $r$-process and
$s$-process abundances, and even to certain elements being uniquely
associated with one or the other. Particularly dramatically, there are
certain magic numbers of neutrons for which nuclei are particularly
resistant to neutron capture. $s$-process hits those magic numbers at
higher-$Z$, where the nuclei are close to stable; there is a peak in
the elemental distribution there. $r$-process hits those at lower $Z$,
and then they decay up to the vallay; this creates a peak slightly
lower in $Z$ than the $s$-process. There are three main magic numbers
that are important, $N=50$, 82, and 126, leading to pairs of $r$ and
$s$ peaks at $A=80$, 90, at $A=130$, 138, and at $A=195$, 208.

For the Sun, its abundances seem to reflect a mix of $\alpha$ and
Fe-peak elements that reflects contributions from both Type Ia
supernovae and core collapse SNe.

\section{Galaxy observations}

From the point of view of measurements of external galaxies,
$\alpha$-elements and Fe are the main observable quantities in the
stars. Rarer elements, particular $r$- and $s$-process, are hard to
find in galaxy spectra; high resolution spectra are required for
stars, but galaxies are intrinsically blurred so you cannot obtain
true high resolution spectra.

The $[\alpha/Fe]$ abundance is especially informative. Core-collapse
supernovae yield $\alpha$-elements preferentially, and occur promptly
after star formation. SN Ia yield Fe-peak elements preferentially, and
occur with a range of delay times that can extend billions of years
after star formation. Therefore, {\it $\alpha$-rich} stellar
populations are presumed to have resulted from a short burst of star
formation, long enough for core collapse supernova to have enriched
the star forming gas but not long enough for Type Ia supernovae to
have enriched the star forming gas much. In contrast, {\it solar
  abundance} stellar populations likely reult from more extended star
formation history.

For elliptical galaxies, the $\alpha$-enrichment increases with
luminosity. The metallicity of elliptical galaxies decreases as a
function of galactocentric radius, but the $\alpha$ abundance relative
to iron remains constant. 

Within the Milky Way, the distribution of elements can be more
accurately traced. In the Milky Way disk, the $\alpha$-enrichment
increases as position $Z$ above the Galactic plane increases, and as
galactocentric distance decreases; the metallicity decreases at the
same time.

Modern observations of stars can trace 10--30 elements for hundreds of
thousands of stars. One goal for these studies is known as {\it
  chemical tagging}: to use these detailed observations to connect
different stars to the same nucleosynthetic history and thus
presumably the same series of star formation events.

\begin{itemize}
  \item {\it BBFH}
  \item {\it Pagel book}
\end{itemize}

\section{Order-of-magnitude Exercises}

\begin{enumerate} 
\item Show that if $n_n/n_p \sim 0.2$ at the onset of Big Bang
  nucleosynthesis, that the maximum helium mass fraction in the
  universe that can be produced is $Y\sim 0.33$.
\item How much star formation needs to have occurred to produce the
  amount of metals?
\item What is the balance between Ia and CC that goes into solar
  abundance ratios?
\end{enumerate} 

\section{Numerics and Data Exercises}

\begin{enumerate}
\item Elliptical galaxy abundances vs mass
\item Elliptical galaxy abundances vs position
\item MW abundances vs position
\item Helium abundances at low metallicity
\item Deuterium observations at high redshift
\end{enumerate}

\end{document}
