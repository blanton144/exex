\title{\bf Stellar Population Synthesis}

\section{Basics \& Nomenclature}

The theory of stellar structure and evolution predicts how the
luminosity, temperature, and other key parameters defining stars
change over time, and how a population of such stars will change over
time. The theory of stellar atmospheres and observations of stellar
spectra, then tie the stellar parameters to observable spectra and
colors of the stellar population. This overall project of {\it stellar
  population synthesis} connects a star formation history to its
observable results, and can in principle allow observations to be used
to infer quantities such as the total stellar mass, mean stellar age,
metallicity, and other more detailed parameters.

The ingredients of a stellar population synthesis model start with the
stellar initial mass function. The IMF can in principle range from
Salpeter (very ``bottom-heavy'') through to a very ``bottom light''
distribution. There is little evidence that the IMF is constant in all
environments, and some evidence that it does differ in different
environments.

Stellar structure and evolution models yield stellar parameters like
luminosity and effective surface temperature of the population as a
function of time. These results are usually expressed as a series of
{\it isochrones} showing the position of stars in the population as a
function of time. Uncertainties in these models arise. Even on the
main sequence, three-dimensional effects on stellar evolution, such as
convection, rotation, and stellar binarity, are at best approximately
included in isochrone modeling.  Post main sequence phases suffer from
this problem even more greatly. Mass-loss due to winds is uncertain
and affects the final fate of the stars.  Modeling of
thermally-pulsating asymptotic giant branch (TP-AGB) stars is very
uncertain (partly due to winds but also modeling uncertainties and a
paucity of individual stars for calibration). The TP-AGB stars cause
particular uncertainty for NIR observations of intermediate age
populations. The dependence of the temperatures of horizontal branch
stars on stellar parameters is also uncertain (again likely due to
winds).

Stellar atmosphere theory and stellar libraries yield spectra as a
function of these parameters. The theoretical models can be computed
for any stellar parameters in a uniform fashion, but do not always
include all the relevant atomic and molecular lines, or correct
consideration of microturbulent velocities, convection, departures
from local thermodynamic equilibrium, or the spherical
geometry. Empirical libraries, on the other hand, tend to be
heterogeneous in their observational properties, suffer from
spectrophotometric calibration issues, and sparsely cover the
necessary space of temperature, surface gravity, metallicity, and
abundance space.  Again, the major uncertainties arise for rare
populations, and particularly post main sequence populations, but also
metal poor and/or $\alpha$-rich populations.

A population with a single age and abundance is known as a {\it simple
stellar population} (SSP). Its spectrum can be expressed as:
\begin{equation}
f(t, Z) \int_{m_{\rm low}}^{m_{\rm high}(t)} \dd M \Phi(M) f_{\rm
star}(t, M, Z)
\end{equation}
We often need to model a more complicated star formation and chemical
evolution history (sometimes using the post main sequence evolution in
the models to self-consistently predict the chemical enrichment of the
ISM). Any such history may be written as a sum of SSPs, and the
resulting spectrum is the corresponding sum of the individual SSP
spectra.

In many galaxies, the effects of dust on the spectrum cannot be
ignored. Dust is mixed into the interstellar medium gas and it
attenuates the light in a wavelength dependent manner, preferentially
scattering blue light and transmitting red light. In modeling emergent
spectra from galaxies, it appears necessary to account for the dust's
distribution, which is preferentially around star forming, gas-rich
regions. 

Broad trends from stellar population synthesis are:
\begin{itemize}
\item Stellar populations start blue, with spectra that do not
have features useful for measuring metallicity.
\item Stellar populations evolve redward after the cessation of star
formation, during which time their Balmer absorption features become
weaker (on time scales of a few 100 million years) and their 4000 \AA\
break grows stronger (on about a billion year time scale).
\item When they are red, their spectra are similar to $K$ stars, since
they are dominated by $K$ giants; these spectra are
metallicity-sensitive, but with a degeneracy with age.
\item At early times, the populations have low mass to light ratios;
as the stellar population ages, it fades and the mass to light ratio
increases up to higher than solar values.
\item Low mass stars contribute a large fraction of the mass but a
small fraction of the light, which means that considerable mass can
reside in old stellar populations (if there is a younger population to
mask them) or in low mass stars in the IMF (if there are massive stars
to mask them).
\end{itemize}

Stellar population synthesis methods are commonly used to infer
galactic star formation histories and other physical properties.
In broad strokes, the galactic properties are parametrized, and then
parameters are varied to fit some set of observable quantities. There
is considerable variation among investigators as to the quantities
used and the parametrizations.

The {\it stellar mass} is almost always a quantity of interest. It is
defined as the mass currently in stars and stellar remnants. Of the
total mass in stars that have formed, only of order half the mass will
remain as stars or remnants after a few billion years, and about a
quarter of that is in the (extremely low luminosity) remnants.

The most ambitious approaches attempt constraints on the full star
formation history and fit to a detailed observed spectrum. A proper
analysis must marginalize over stellar population uncertainties and
dust effects and over spectrophotometric calibration uncertainties. In
addition, the presence of interstellar and AGN emission lines needs to
be accounted for, either by excluding their wavelengths from the fit
or including the lines explicitly.

Other approaches simplify the problem, usually in an attempt to
regularize the model space in ways thought to be reasonable, or in an
attempt to mitigate observational uncertainties (without performing an
explicit marginalization).

The classic {\it Lick index} analysis technique, commonly used for
elliptical galaxies since the 1980s, employs both approaches. Lick
indices are defined for 21 features between 4000--6500 \AA. They are a
form of {\it equivalent width}. Equivalent widths measure:
\begin{equation}
{\rm EW} = \frac{f_{\rm feature}}{f_{\lambda, {\rm continuum}}},
\end{equation}
the total absorption or emission of a feature ($f_{\rm feature}$, in
units of flux) relative to some local definition of the continuum upon
which the feature is imprinted ($f_{\rm continuum}$, in units of flux
density). Usually, the flux density in question is in $f_\lambda$
units and the equivalent width is expressed in units of Angstroms
(thus the terminology ``width,'' which does not in all cases imply a
measure of actual width in wavelength space). Often, the continuum
level is measured using side bands slightly blueward and redward of
the feature.

Equivalent widths, and specifically Lick indices, have some attractive
qualities. Because they are relatively local measurements in
wavelength, they are insensitive to dust and to spectrophotometric
error. In the case of Lick indices, the index measurements were
performed on stars with well-calibrated properties, allowing the
corresponding measurements in galaxies to be tied directly to the
stellar models.

However, equivalent widths involve defining a continuum, which can be
ambiguous in regions with other features.  In addition, for Lick
indices specifically, the stellar observations were taken at
relatively low resolution, which needs to be accounted for in studying
low velocity dispersion galaxies and higher resolution spectra. 

The Lick indices were chosen by hand for their sensitivity to specific
atomic features. They are in the blue regime (between 4000 and 5300
Angstroms), due to the technology available at the time of their
original definition.  The Lick index analysis proceeds by measuring a
set of indices associated with the Balmer absorption features
(measuring the fraction of young stars), Fe features (which arise in a
complex suite of lines), and the Mg $b$ feature. The Balmer and Fe
features require some correction for emission line contamination. The
indices are then compared to a simple stellar population model
characterized by its age, metallicity, and $\alpha$-enhancement. Since
galaxies do not have simple stellar populations, the resulting
parameters at best express some weighted average of the underlying
star formation history; in the case of the standard Lick analysis, the
ages are always smaller than the mass-weighted age estimate.  Other
uncertainties are known to affect the existing Lick analyses, to do
with the stellar population modeling accuracy; most significantly, old
stellar populations contain an uncertain number of blue horizontal
branch stars, which confuse the interpretation of the Balmer
absorption features.

Between full spectral analysis and Lick index analysis lies a large
space of possible analyses that has been explored by many
investigators. Some important facts revealed by these analyses are as
follows:
\begin{itemize}
\item Age and metallicity both have similar effects on spectral
features, both broad band and for individual lines. Older ages lead to
a greater dominance of the red giant branch, and thus redder colors
and stronger absorption features. High metallicity leads to lower
effective temperatures of stars, and also strong absorption at fixed
temperature, both effects also producing redder colors and stronger
absorption features.
\item Stellar mass-to-light ratios depend to a surprising degree just
on the color of the stellar population; greater age, greater
metallicity, and greater dust all affect the broad band flux by making
it redder and fainter (relative to mass), in similar proportions to
each other.
\item A definition $D_n(4000)$ for the 4000 \AA break is informative
as to stellar population age (though it carries metallicity dependence
as well).  
\end{itemize}

Beyond metallicity, stellar population analyses can be sensitive to
individual abundances as well. Most commonly, these are expressed in
term of [$\alpha$/Fe], the $\alpha$-element abundance relative to
iron. In contrast to observations of the interstellar medium, typical
observations of stellar populations are most sensitive to Mg rather
than the more abundant oxygen, due to the lack of atomic oxygen lines
in stars in the optical. In the Milky Way, it is known that oxygen and
magnesium do not, however, always vary together, so Mg as a proxy for
other $\alpha$ elements is not perfect. 

A special case for the use of stellar populations is the inference of
recent star formation, as directly traced by O and B
stars. Ultraviolet emission can be used to trace the star formation
rate, preferably far UV ($\sim$ 1500 \AA), though near near UV ($\sim$
2500 \AA) or even $u$-band is occasionally used. Such measures need to
be dust-corrected. In the UV, this is often done using the FUV to NUV
ratio (since in principle this ratio is fixed for hot stars). If
optical imaging is available, one can estimate reddening using stellar
population analysis of the optical bands. If infrared imaging is
available, the IR flux directly traces the extincted light, modulo the
geometric anisotropy of the UV extinction. In this last case, a
weighted sum of the UV and infrared luminosity can yield a reliable
star formation rate estimate. The mid-IR is thought to be more
reliable in this regard, since the colder dust in the far-IR can be
heater by old stellar populations.

With spectra, star formation averaged over longer time scales is
possible. For example, star formation over a few hundred million year
time scales is traced by the Balmer lines.  A very special cases
arises for K$+$A, sometimes called E$+$A, or post-starburst,
galaxies. Such galaxies show few traces of star formation on million
year time scales (as traced usually by H$\alpha$ interstellar
emission), but do show strong star formation on a few hundred million
year timescales (as traced by Balmer absorption). These galaxies
likely experienced a large burst of star formation a few hundred
million years ago, which is now over.

Interpretation of star formation rates, especially but not only at low
values, requires care.  Old stellar populations do emit some UV light,
depending on their population of horizontal branch stars; in practice,
measuring specific star formation rates below about $10^{-11}$
yr$^{-1}$ is not possible.  Both ultraviolet and near-infrared
emission can be powered by AGN as well, which near the centers of
galaxies can be a confusing factor as well.

There are a handful of cases for which stellar population histories
can be derived directly from resolved stellar populations. Stellar
clusters in the Milky Way are one case. To a certain extent, the
population of stars with parallaxes in the Milky Way can also be used
in this way. Some nearby galaxies (M31, M33, and others) have resolved
stellar populations from space, and these cases can be studied from
the isochrones. Since they also have broad band and spectroscopic
measurements similar to more distant galaxies, they can provide a
calibration between the two approaches.

\section{Commentary}

In principle, stellar population analysis requires marginalization
over all of the uncertain parameters that characterize the stellar
population modeling. These uncertainties are:
\begin{itemize}
\item Stellar IMF
\item TP-AGB phases
\item Blue stragglers
\item Extreme horizontal branch stars
\item Effects of convection on evolution
\item Mass-loss through winds
\item Effects of binarity on evolution
\item Stellar atmospheres models \& libraries
\item Dust and dust geometry
\end{itemize}

\section{Key References}

\begin{itemize}
  \item
    {\it Old stellar populations. 5: Absorption feature indices for
    the complete {L}ick/{IDS} sample of stars}, \citet{worthey94}
  \item
    {\it Modeling the Panchromatic Spectral Energy Distributions of Galaxies}, \citet{conroy13a}
\end{itemize}

\section{Important numbers}

\begin{itemize}
\item $M_{\odot} = 1.989 \times 10^{30} {\rm ~kg} $
\item $R_{\odot} = 6.955 \times 10^{8} {\rm ~m} $
\item $T_{\odot}{\rm (surface)} = 5500 {\rm ~K} $
\item $T_{\odot}{\rm (core)} = 1.5 \times 10^7 {\rm ~K} $
\item $L_{\odot} = 3.828 \times 10^{33} {\rm ~erg} {\rm ~s}^{-1}$
\end{itemize}

\section{Order-of-magnitude Exercises}

\begin{enumerate} 
\item Estimate the ratio between star formation rate and FUV flux
    around $\lambda \sim 1200$ \AA.
    Assume that the luminosity in the FUV band of an O-star
    is $\sim 10^{34}$ erg s$^{-1}$ \AA$^{-1}$ at that wavelength (a
    very approximate number for the O2 star BI 237 in the Large
    Magellanic Cloud, according to the atlas of \citealt{smith12a}).
\item Estimate the difference in mass-to-light ratio in the optical
    between a 200 million year old and 10 billion year old stellar
    population. Assume that the solar mass stars (which dominate the
    flux of the older population) spend about 10\% of their time at
    the tip of the red giant branch and that the distribution of stars
    along the red giant branch is constant per unit magnitude.

\begin{answer}
At 200 Myr in age, the stars in stellar population stars are mostly in
main sequence. The luminosity is dominated by late-type B stars of
about 5 $M_\odot$, whereas the mass is dominated by low mass
stars. The mass-to-light ratio may be calculated as:
\begin{equation}
\frac{M}{L} = \frac{\int_{0.08M_\odot}^{5 M_\odot} {\rm d}M M \Phi(M)}
{\int_{0.08M_\odot}^{5 M_\odot} {\rm d}M L(M) \Phi(M)}
\end{equation}
where $\Phi(M)$ is the initial stellar mass function. If we assume
$(L/L_\odot) = (M/M_\odot)^4$ and $\Phi\propto M^{-2.35}$, then in
solar units:
\begin{equation}
\frac{M}{L} = \frac{\int_{0.08}^{5} {\rm d}x x^{-1.35}}
{\int_{0.08}^{5} {\rm d}x x^{1.65}}
\end{equation}
which evaluates to $M/L\sim 0.2$. 

Now, for the 10 Gyr stellar population, solar mass stars are just
entering the red giant branch. The luminosity of solar mass stars in
the red giant phase is $L_{\rm RGB} \sim 10^{3}L_{\odot}$. Solar mass
stars spend 10\% of the main sequence lifetime on the RGB, meaning
that the set of stars there will be those that leave the main sequence
within about 10\% of the total lifetime (1 Gyr). Let us call the range
of masses this corresponds to $\Delta M(t_\odot)$. We can write the
mass to light ratio as:
\begin{equation}
\frac{M}{L} = \frac{\int_{0.08 M\odot}^{M_\odot} {\rm
d}M M \Phi(M)}{\Phi(M_\odot) \Delta M(t_\odot) \langle L_{\rm
RGB}\rangle}
\end{equation}
We can determine $\Delta_M(t_\odot)$ as follows, by relating the range
in time to a range in mass:
\begin{equation}
\Delta M (t) \sim \Delta t \left|\frac{{\rm d}M}{{\rm d}t}\right|
\end{equation}
and differentiating the relationship
\begin{equation}
\left(\frac{t}{t_\odot}\right) = \left(\frac{M}{M_\odot}\right)^3
\end{equation}
to show
\begin{equation}
\left|\frac{{\rm d}M}{{\rm d}t}\right|
= \frac{M_\odot}{3 t_\odot} \left(\frac{M}{M_\odot}\right)^4 
\end{equation}
and so:
\begin{equation}
\Delta M (t) \sim \frac{1}{3} M_\odot \frac{\Delta t}{t_\odot} \left(\frac{M}{M_\odot}\right)^4 
\end{equation}
Then we can calculate:
\begin{equation}
\langle L_{\rm RGB} \rangle =
\frac{\int_{L_\odot}^{10^3 L_\odot} {\rm d}L N(L) L}
{\int_{L_\odot}^{10^3 L_\odot} {\rm d}L N(L)}
\end{equation}
where $N(L)\propto 1/L$ (constant per unit magnitude), so:
\begin{equation}
\langle L_{\rm RGB} \rangle =
\frac{\int_{L_\odot}^{10^3 L_\odot} {\rm d}L}
{\int_{L_\odot}^{10^3 L_\odot} {\rm d}L L^{-1}} \sim \frac{1000}{3\ln
10} L_\odot \sim 140 L_\odot
\end{equation}
Plugging in and using solar units:
\begin{equation}
\frac{M}{L} = \frac{\int_{0.08}^{1} {\rm
d}x x^{-1.35}}{(1/3) (140) (0.1)} \sim 1
\end{equation}
In fact, this is an underestimate; for old populations the
RGB tends to be more heavily populated per unit magnitude near its
base, which increases the mass-to-light ratio by a factor of a few.

An important point is that although the old population is dominated by
solar mass stars, and also has a mass-to-light ratio similar to solar
values, this is the consequence of the balance between the numerator,
dominated by the lower stellar masses, and the denominator, dominated
by the post-main sequence lifetime and luminosity distribution. Thus
the rough agreement in mass-to-light ratio is a coincidence.
\end{answer}
\item Using a Salpeter initial mass function, and assuming that
$L\propto M^4$ determines both the luminosity and lifetimes of the
stars, calculate how the stellar luminosity varies as a function of
time expected at early times (e.g. $\sim 100$ Myr), before the red
giant branch population of stars becomes important. Do not worry about
the absolute value of the luminosity, just its time dependence.

\begin{answer}[Author: Nicholas Faucher]
The lifetime of a star with mass $m$ can be estimated by comparing to
the lifetime of the Sun, $\tau_{\odot} \approx 10^{10} \, yr$, as
$\tau \approx \frac{m}{L} \, \tau_{\odot}$, where $m$ and $L$ are the
mass and luminosity of a star in solar units. Assuming a mass-to-light
ratio of $L \propto m^4$, we can rewrite this equation in terms of
mass only:
\begin{equation}
\tau \approx \frac{1}{m^3} \, \tau_{\odot}
\end{equation}
Solving this equation for $m$ sets the upper limit on the masses of
stars in a population at time $\tau$. Assuming a Salpeter IMF, which
asserts that the number of stars in a stellar population with mass
between $m$ and $m+dm$ is proportional to $m^{-2.35}$, we can
determine the time dependence of a stellar population's luminosity. To
do this, we also assume that each star has constant luminosity until
the end of its lifetime, at which point it has no luminosity. Then the
total luminosity is just the integral of the stars
\begin{equation}
L(t) \propto \int_{0.08}^{\big( \frac{10^{10}}{t} \big) ^{\frac{1}{3}}} m^{-2.35} \, m^4 \, dm
\end{equation}
The term corresponding to the lower limit on the integral is
negligibly small ($\sim 10^{-4}$) and so we get
\begin{equation}
L(t) \propto t^{-0.883}
\end{equation}
This dependence should hold at early times, when post-main sequence
stars are not dominating the luminosity. As the population ages past a
billion years, the luminosity is dominated by red giant branch stars
and this dependence changes (slowing down generally).
\end{answer}

\item What is difference between the mass-to-light ratios of two
10-billion-year-old stellar populations, one with a Salpeter IMF and
one with an IMF which is Salpeter above $M=0.7M_\odot$ but has
$\Phi(M) \propto$ constant at lower masses?
\end{enumerate} 

%\section{Analytic Exercises}
%
%\begin{enumerate}
%\end{enumerate}

\section{Numerics and Data Exercises}

\begin{enumerate}
\item There is
a \href{https://python-fsps.readthedocs.io/en/latest/installation/}{Python
interface} to the Flexible Stellar Population Synthesis
(FSPS; \citealt{conroy09a}) software, which is one of several
libraries that exist for this purpose. Install this package and (using
the scant available documentation!) run it to produce spectra for
dust-free populations with a range of ages and
metallicities. Specifically, do 25 models, with five metallicities and
five ages spaced logarithmically between $Z/Z_\odot = 0.01$ and
$Z/Z_\odot = 2$, and ${\rm age} = 10^8$ years and ${\rm age} =
10^{10}$ years. Then create the same set of models with a modest
amount of dust reddening (say around $A_V \sim 0.5$). Plot the spectra
for comparison with each other in the UV through near-IR
range. Warning: this software is incredibly slow.
\item For the same set of models as the first question, use FSPS to
project the spectra onto the $u$ and $r$ filters. Then plot the
mass-to-light ratio in the $r$-band as a function of $u-r$ color for
all models.  Also plot $g-r$ versus age. In each plot, distinguish
between the no-dust and dust cases.
\item For the same set of models as the first question, use $D_n$4000,
and plot it versus age, again distinguishing between the no-dust and
dust cases.
\item For the same set of models as the first question, calculate
the Lick Mg $b$ and the $\langle$Fe$\rangle$ indices (the latter being
the mean of Fe5270 and Fe5335). Plot the mapping of this set
of models into the Lick index space as two grids (one for each dust
case). 
\end{enumerate}

\bibliographystyle{apj}
\bibliography{exex}  
