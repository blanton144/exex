\title{\bf Stellar Population Synthesis}

\section{Basics \& Nomenclature}

The theory of stellar structure and evolution predicts how the
luminosity, temperature, and other key parameters defining stars
change over time, and how a population of such stars will change over
time. The theory of stellar atmospheres and observations of stellar
spectra, then tie the stellar parameters to observable spectra and
colors of the stellar population. This overall project of {\it stellar
  population synthesis} connects a star formation history to its
observable results, and can in principle allow observations to be used
to infer quantities such as the total stellar mass, mean stellar age,
metallicity, and other more detailed parameters.

The ingredients of a stellar population synthesis model start with the
stellar initial mass function. The IMF can in principle range from
Salpeter (very ``bottom-heavy'') through to a very ``bottom light''
distribution. There is little evidence that the IMF is constant in all
environments, and some evidence that it does differ in different
environments.

Stellar structure and evolution models yield stellar parameters like
luminosity and effective surface temperature of the population as a
function of time. These results are usually expressed as a series of
{\it isochrones} showing the position of stars in the population as a
function of time. Uncertainties in these models arise particularly for
post main sequence phases, such as thermally-pulsating asymptotic
giant branch stars or the temperature distribution of the horizontal
branch population.

Stellar atmospheres and stellar libraries yield spectra as a function
of these parameters. Again, the major uncertainties arise for rare
populations, and particularly post main sequence populations, but also
metal poor and/or $\alpha$-rich populations.

To produce a spectrum one assumes a star formation history and a
chemical evolution history (sometimes using the post main sequence
evolution in the models to self-consistently predict the chemical
enrichment of the ISM). Each age and metallicity corresponds to a {\it
  simple stellar population}.  The the resulting spectrum is just a
sum of the simple stellar population models, over the distribution of
ages and metallicities.

In many galaxies, the effects of dust on the spectrum cannot be
ignored. Dust is mixed into the interstellar medium gas and it
attenuates the light in a wavelength dependent manner, preferentially
scattering blue light and transmitting red light. In modeling emergent
spectra from galaxies, it appears necessary to account for the dust's
distribution, which is preferentially around star forming, gas-rich
regions. 

Broad trends from stellar population synthesis are:
\begin{itemize}
\item Stellar populations start blue, with spectra that do not
have features useful for measuring metallicity.
\item Stellar populations evolve redward after the cessation of star
formation, during which time their Balmer absorption features become
weaker (on time scales of a few 100 million years) and their 4000 \AA\
break grows stronger (on about a billion year time scale).
\item When they are red, their spectra are similar to $K$ stars, since
they are dominated by $K$ giants; these spectra are
metallicity-sensitive.
\item At early times, the populations have low mass to light ratios;
as the stellar population ages, it fades and the mass to light ratio
increases up to higher than solar values.
\end{itemize}

Stellar population synthesis methods are commonly used to infer
galactic star formation histories and other physical properties.
In broad strokes, the galactic properties are parametrized, and then
parameters are varied to fit some set of observable quantities. There
is considerable variation among investigators as to the quantities
used and the parametrizations. 

The most ambitious approaches attempt constraints on the full star
formation history and fit to a detailed observed spectrum. A proper
analysis must marginalize over stellar population uncertainties and
dust effects and over spectrophotometric calibration uncertainties. In
addition, the presence of interstellar and AGN emission lines needs to
be accounted for, either by excluding their wavelengths from the fit
or including the lines explicitly.

Other approaches simplify the problem, usually in an attempt to
regularize the model space in ways thought to be reasonable, or in an
attempt to mitigate observational uncertainties (without performing an
explicit marginalization).

The classic {\it Lick index} analysis technique, commonly used for
elliptical galaxies, employs both approaches. Lick indices are a form
of {\it equivalent width}. Equivalent widths measure:
\begin{equation}
{\rm EW} = \frac{f_{\rm feature}}{f_{\lambda, {\rm continuum}}},
\end{equation}
the total absorption or emission of a feature ($f_{\rm feature}$, in
units of flux) relative to some local definition of the continuum upon
which the feature is imprinted ($f_{\rm continuum}$, in units of flux
density). Usually, the flux density in question is in $f_\lambda$
units and the equivalent width is expressed in units of Angstroms
(thus the terminology ``width,'' which does not in all cases imply a
measure of actual width in wavelength space). Often, the continuum
level is measured using side bands slightly blueward and redward of
the feature.

Equivalent widths, and specifically Lick indices, have some attractive
qualities. Because they are relatively local measurements in
wavelength, they are insensitive to dust and to spectrophotometric
error. In the case of Lick indices, the index measurements were
performed on stars with well-calibrated properties, allowing the
corresponding measurements in galaxies to be tied directly to the
stellar models.

However, equivalent widths involve defining a continuum, which can be
ambiguous in regions with other features.  In addition, for Lick
indices specifically, the stellar observations were taken at
relatively low resolution, which needs to be accounted for in studying
low velocity dispersion galaxies and higher resolution spectra. 

The Lick index analysis proceeds by measuring a set of indices
associated with the Balmer absorption features (measuring the fraction
of young stars), Fe features (which arise in a complex suite of
lines), and Mg features (which are more distinct. The Balmer and Fe
features require some correction for emission line contamination. The
indices are then compared to a simple stellar population model
characterized by its age, metallicity, and $\alpha$-enhancement. Since
galaxies do not have simple stellar populations, the resulting
parameters at best express some weighted average of the underlying
star formation history; in the case of the standard analysis, they are
said to {\it light-weighted} estimates, which tend to weight the
average towards younger ages.

\section{Commentary}

In principle, to do so precisely requires marginalization over all of
the uncertain parameters that characterize the stellar population
modeling. This margin

\section{Key References}

\begin{itemize}
  \item
    {\it Conroy}
  \item
    {\it Kennicutt on star formation}
\end{itemize}

\citet{gunn06a}

\section{Important numbers}

\begin{itemize}
\item $M_{\odot} = 1.989 \times 10^{30} {\rm ~kg} $
\item $R_{\odot} = 6.955 \times 10^{8} {\rm ~m} $
\item $T_{\odot}{\rm (surface)} = 5500 {\rm ~K} $
\item $T_{\odot}{\rm (core)} = 1.5 \ times 10^7 {\rm ~K} $
\item $L_{\odot} = 3.828 \times 10^{33} {\rm ~erg} {\rm ~s}^{-1}$
\end{itemize}

\section{Order-of-magnitude Exercises}

\begin{enumerate} 
\item Argue why higher mass stars produce more of their energy through
    the CNO cycle than lower mass stars do.
\item Detailed stellar evolution calculations predict a main sequence
    lifetime for the Sun of 10 billion years. What fraction of the
    total hydrogen in the Sun needs to be converted to helium to
    provide this lifetime?
\end{enumerate} 

\section{Analytic Exercises}

\begin{enumerate}
\item Under Salpeter, calculate luminosity as a function of time
\end{enumerate}

\section{Numerics and Data Exercises}

\begin{enumerate}
\item HR diagram for an open cluster
\item HR diagram for a globular cluster 
\item HR diagram locally
\item Compare K spectrum to E spectrum
\item Compare O/B spectrum to blue galaxy
\end{enumerate}

\bibliographystyle{apj}
\bibliography{exex}  
