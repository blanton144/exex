\title{\bf Groups and Clusters}

\section{Basics}

The largest gravitationally bound structures in the universe are
massive dark matter halos hosting numerous galaxies. Large mass halos
($M > 10^{14} M_\odot$) are referred to as {\it clusters}, whereas
smaller halos are referred to as {\it groups}. The virial radii of
clusters are around 1--2 Mpc, and the virial velocities are $>700$ km
s$^{-1}$ and can be up to several thousand km s$^{-1}$. The gas in
these systems reaches temperatures of $10^8$ K; it is typically very
diffuse, with $n\sim 10^{-3}$ cm$^{-3}$. The main contents of groups
and clusters are galaxies (about 0.03 of the group mass), hot X-ray
emitting gas (about 0.15 of the group mass), and dark matter (the
remainder).

\subsection{Searches for clusters}

The first large catalog of clusters was created in the early 1950s by
Abell by examining galaxy counts on Palomar plates. More recent
methods of finding clusters are:
\begin{itemize}
\item Photometric surveys, using photometric redshifts and/or color
criteria to mitigate projection effects and find clusters with higher
efficiency. Cluster galaxies tend to be red sequence galaxies, a fact
that cluster-finders often rely upon.
\item Redshift surveys, using more precise three-dimensional redshift
information to select clusters. 
\item X-ray surveys, relying on thermal bremstrahlung emission from
hot intergalactic gas in the clusters.
\item Sunyaev-Zeldovich effect searches, relying on distortions of
background cosmic microwave background light by the hot gas in
foreground clusters. Because it is absorption, the SZ effect is
dependent on the integrated gas density and is nearly independent of
redshift (the redshift dependence comes in only through the angular
resolution of the CMB observations).
\end{itemize}
In principle, one can search for clusters in weak lensing maps as
well, though to date this method is not practical.

\subsection{Sizes of clusters}


\subsection{Galaxies in clusters and groups}

Clusters contain numerous galaxies. They often have a dominant
central, most luminous galaxy, which is termed a {\it brightest
cluster galaxy (BCG)}. Often but not always, BCGs can be classified as
{\it cD} galaxies; this terminology refers to the existence of an
extended stellar halo, usually with an $r^{-2}$ radial profile. cD
does not stand for ``central dominant.'' 

The galaxy population varies with environment, and the larger groups
and clusters preferentially contain more massive galaxies and (even at
fixed galaxy mass) the red sequence population of galaxies. It is
convenient to understand this trend in terms of {\it central} galaxies
and {\it satellite} galaxies inside of the cluster host halos. Central
galaxy masses are roughly monotonically related to the halo masses,
though their color distribution does not appear strongly related to
halo mass. Satellite galaxy masses also increase with host halo mass
(since they are generally smaller than the central galaxy), and in
addition the larger the host halo the more prominent the red galaxy
population.

In general, the scaling relations for galaxies in clusters are very
similar to those of isolated galaxies, with some
differences. Elliptical galaxies lie on a slightly different
fundamental plane and have slightly different stellar populations at a
fixed mass. Both effects are consistent with a slightly older
population in clusters. Whether this is because star formation is
truncated earlier in these populations or because it started earlier
is not clear.

Spiral galaxies have HI disks which are somewhat truncated relative to
spiral galaxies in the field, an effect known as {\it HI
deficiency}. This truncation is likely due to ram pressure stripping
in the clusters.

For many clusters, the cD galaxy halo contains an enormous number of
intergalactic stars. These halos are very low surface brightness but
can contain around 50\% of the total stellar mass of the cluster. They
have been studied in some detail for Coma and Virgo, and in both cases
exhibit streams and other tidal structures. They appear to be similar
in age and metallicity to the galactic stellar populations. These
facts appear consistent with the intergalactic population having been
tidally stripped from satellite galaxies, many of which may have been
completely disrupted.

These trends exist at least down to scales of groups that contain 2--3
$L_\ast$ galaxies. Below that scale environmental effects continue to
matter for low luminosity satellites of individual $L_\ast$
galaxies. A notable feature of such low luminosity galaxies is that in
the stellar mass range $10^7$--$10^9$ only satellite galaxies ever
become red galaxies (dE galaxies).

\subsection{Gas in clusters}

The dominant baryonic component of clusters is not stars in galaxies,
but is hot, diffuse gas, visible in brehmstrahhlung. The total
emission is of order $L_X \sim 10^{43}$--$10^{45}$ erg s$^{-1}$ and
the gas is $10^7$--$10^8$ K.

The emission per unit volume has a form similar to:
\begin{equation}
\epsilon \propto n^2 \left(T\right)^{-1/2} \exp\left(-h\nu / kT\right)
\end{equation}
The wavelength dependence allows determination of the temperature from
an X-ray spectrum.  However, in practice the above approximation is
not sufficient and a temperature determination requires a detailed
model of plasma emission, such as CHIANTI or MEKAL. 

With a projected temperature and emission profile, the gas mass and
dynamical mass can be derived. The simplest version of this approach
is the {\it $\beta$ model}, which is spherically symmetric and
isothermal. Although it is not precise enough for modern work, it
illustrates the general principles.  As for static isothermal
collisionless systems, isothermal hydrodynamic systems can be
approximated by the King model:
\begin{equation}
\rho \propto \left[1+\left(\frac{r}{r_c}\right)\right]^{-3\beta/2},
\end{equation}
where in detail:
\begin{equation}
\beta = \frac{\mu m_p \sigma_v^2}{kT}
\end{equation}
The resulting emission as projected on the sky becomes:
\begin{equation}
I(R) \propto \left[1+\left(\frac{R}{r_c}\right)\right]^{-3\beta + 1/2}
\end{equation}
With $T$ determined from the spectra, fitting $\beta$ the overall
amplitude constraints $\rho_g(r)$ and therefore the total gas mass.

In reality, galaxy cluster gas temperatures vary throughout the
cluster. Many clusters are {\it cool core} clusters, whose inner parts
are roughly a factor two cooler than the outer parts. These cool
clusters are typically more regular and tend to have a clear brightest
cluster galaxy. Other clusters appear to be more disturbed, due to
mergers and infalls, and have a less regular temperature pattern. They
correspondingly tend to lack regularity or a clear brightest cluster
galaxy.

One expects cooling near cluster centers theoretically due to the
higher density at the center. The cooling time for the gas and the
mass rate of cooling at the centers of many clusters is extremely
high. These clusters are referred to as {\it cooling flow} clusters,
though there is not direct evidence for any inward flow of gas.
Nevertheless, high resolution X-ray spectra of atomic lines in these
clusters, which reveal the ionization states and therefore gas
temperature distribution, are not consistent with much gas cooling
below 3 keV. The mass cooling rate does correlate with BCG star
formation signatures, but those star formation rates are only a few
percent of the cooling rates except in rare cases.

Highly ionized species of iron and other elements (C, N, O, Ne, Mg,
Si, S, Ar, Ca, Cr, Mn, and Ni) create detectable atomic lines in the
intracluster gas. Combined with plasma emission models, these lines
constrain the abundance distributions of elements. For clusters of
galaxies, abundances range from 0.1 to 1 solar. Metallicities
generally decrease with distance from the cluster center. The gas was
enriched from a combination of in situ enrichment from the galaxies
and intergalactic stars and from stripping of previously enriched
gas. 

The overall cluster X-ray luminosity is a function of mass and
temperature, with $L_X\propto T^3$ and $L_X\propto M^2$. The nature of
these scaling relations suggests that at lower mass the cluster gas
has an entropy floor. The entropy is usually quantified by the {\it
entropy index}:
\begin{equation}
K\equiv \frac{P}{\rho^\gamma} = \frac{kT}{\mu m_p \rho^{2/3}}
\end{equation}
where the second equality is for a monoatomic, nonrelativistic, ideal
gas. The entropy per unity mass $s$ can be written in that case as:
\begin{equation}
\label{eq:entropy}
s = \frac{k}{\mu m_p} \ln K^{3/2} + {\rm ~constant}
\end{equation}
The entropy floor is usually interpreted as the result of a heating
source prior to falling into the cluster.

Cluster gas can also be detected through the {\it thermal
Sunyaev-Zeldovich effect}, which is the inverse Compton scattering of
cosmic microwave background photons on the electrons in the cluster
gas. This process results in a slight increase in energies of the
photon distribution, and this spectral distortion is observable. Above
$\nu \sim 218$ GHz (near the CMB spectrum peak), this yields an
increase in flux, and below that frequency it yields a decrease in
flux. Motion of the cluster with respect to the CMB will also yield an
additional spectral distortion, called the {\it kinetic
Sunyaev-Zeldovich effect} which tends to be at least an order of
magnitude smaller and must be observed at $\nu \sim 218$ GHz.

Gas at lower temperatures exists in groups and individual galaxies as
well, but is too cold to be detected in the X-ray regime. The
extreme-UV emission from this gas is difficult to observe.

\subsection{Dark matter in clusters}

The dark matter in clusters can be measured through the dynamics of
the galaxies, the luminosity and temperature of the X-ray gas, and
through weak and strong lensing.

In the case of galaxies, the virial theorem can be used; the
gravitational radius must be estimated from the galaxies assuming they
are faithful tracers of the full mass distribution.

In the case of the X-ray gas, the modeling approach emission can be
used. With a temperature and density model, spherically symmetric
hydrostatics implies that:
\begin{equation}
M(<r) = - \frac{kTr^2}{G\mu m_p} \left[\frac{\dd \ln \rho_g}{\dd r}
+ \frac{\dd T}{\dd r}\right]
\end{equation}

\section{Commentary}

The largest halos form galaxies with a low stellar-to-halo mass ratio,
a fact which is one of the major goals of galaxy formation theory to
explain. This problem must be closely related to the ``cooling flow''
problem---the lack of cold gas and star formation in the centers of
clusters for which the cooling times are short---and the entropy floor
in lower mass clusters.

\section{Important numbers}

\begin{itemize}
\item $k(T / 10^{6}{\rm ~K}) \approx 0.86 {\rm ~keV}$
\end{itemize}

\section{Key References}

\begin{itemize}
\item {\it Formation of Galaxy
Clusters, \href{https://ui.adsabs.harvard.edu/abs/2012ARA%26A..50..353K/abstract}{\citet{kravtsov12a}}}
\item {\it X-ray Properties of Groups of
Galaxies, \href{https://ui.adsabs.harvard.edu/abs/2000ARA%26A..38..289M/abstract}{\citet{mulchaey00a}}}
\end{itemize}

\section{Order-of-magnitude Exercises}

\begin{enumerate} 
\item Estimate the virial velocities to expect for a galaxy cluster
    with $M\sim 10^{14}$ $M_\odot$ and $R\sim 1$ Mpc.
\item Estimate the gas temperature that should be required for
hydrostatic equilibrium in such a cluster.
\item For a cluster with a total mass of $10^{14}$ $M_\odot$, estimate
    the star formation that would be associated with a cooling time of
    $10^{10}$ Gyr.
\item Calculate the cross-section for interaction of a photon
    traveling through a massive cluster; i.e. for the inverse Compton
    scattering that causes the Sunyaev-Zeldovich effect.
\end{enumerate}   

\section{Analytic Exercises}

\begin{enumerate}
\item Estimate how viable ram pressure stripping will be for a galaxy
    disk. Assume a thin disk of gas coplanar with the disk of stars,
    with surface densities such that $\Sigma_{\rm gas} \ll\Sigma_\ast$.
    \begin{enumerate}
    \item Approximate the disks as infinitely thin, and calculate the
    acceleration that will result for a mass displaced from the center
    of the stellar disk. Using that acceleration, what pressure will
    the gas feel towards the disk plane if it is displaced?
    \item Through dimensional analysis, estimate the form the ram
    pressure due to motion through the intracluster medium.
    \item For a spiral galaxy like the Milky Way, compare
    quantitatively the restoring pressure due to gravitation to the
    ram pressure.
    \end{enumerate}
\item In the Sunyaev-Zeldovich process, the scattering yields a change
    in frequency $\Delta\nu/\nu \approx kT/m_ec^2$. Show that at
    $h\nu\ll kT$, the change in brightness temperature is:
    \begin{equation}
   \frac{\Delta T_b}{T_b} \equiv -2y =
    - \frac{2k\sigma_T}{m_ec^2} \int {\rm d}x n_e T_e,
    \end{equation}
    where $x$ is the coordinate along the line of sight.
\item Prove Equation \ref{eq:entropy}.
\end{enumerate}

\section{Numerics and Data Exercises}

\begin{enumerate}
\item Look at specific groups
\item Find groups in a sample
\item SZ in Planck
\item X-ray fluxes in groups
\item Morphological segregation
\end{enumerate}

\bibliographystyle{apj}
\bibliography{exex}  
