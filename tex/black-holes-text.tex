\title{Black Holes in Galaxies}

\section{Basics}

Astronomical evidence now suggests that every galaxy in the Universe
with a mass $M\sim M_\ast$ hosts a supermassive ($\sim
10^7$--$10^{10}$ $M_\odot$) black hole at its center. While early
evidence for the existence of such black holes originally came in the
1960s through their manifestation as extremely luminous quasars,
nowadays there is direct evidence for extremely massive objects at the
centers of galaxies.  There are a number of ways to estimate the
masses of these black holes, using the orbital velocities of stars and
gas surrounding the black hole as a function of distance. These masses
correlate well with global properties of the galaxies, such as the
total luminosity or velocity dispersion of the galaxies; apparently
these correlations are especially good with relation to the properties
of the galactic bulges.

\subsection{Sphere of influence}

Supermassive black hole masses are measured dynamically. However,
these black holes are surrounded by the host galaxy, and therefore we
need to define their {\it sphere of influence} within which the black
hole dominates the dynamics. {\bf give relation}

\subsection{The Milky Way's Black Hole}

At the center of the Milky Way exists a dense stellar cluster as well
as neutral, molecular and hot X-ray-emitting gas. The gas emits radio
continuum emission, and there exists a very compact radio source known
as Sgr A$\ast$. VLBI observations have constrained its size to be
roughly a few light-minutes.  Sgr A$\ast$ is the location of a massive
($\sim 4\times 10^6$ $M_\odot$) black hole.  The study of this black
hole and its environment is reviewed by \citet{genzel10a}.

The first spectroscopic observations of individual stars in the
central region were in the 1990s, with near-infrared speckle imaging
and (later) adaptive optics based imaging and spectroscopy that
definitively characterized the central black hole through study of
individual orbits.  The population of stars within 1 arcsec (or about
0.04 pc) consists primarily of a cluster of old stars ($>1$ Gyr) with
apparently random orbits, and a much smaller group of young massive
stars that appear to be mostly in one disk, with a second distinct
disk possibly existing. The orbital times are 10s of years.

\subsection{Stellar dynamics signatures}

For other galaxies the observations are less detailed. High resolution
imaging and spectroscopy have allowed central stellar velocity
dispersion and rotation maps of galaxies {\bf within ??}. These
observations, showing central spikes in velocity dispersion {\bf
right??} point to the common existence of central point masses that we
associate with black holes. 

\subsection{Gas dynamic signatures}

Ionized gas dynamics near the centers of galaxies also reveal rapid
rotation near the galactic centers for some of these galaxies. For
example, M87 shows these signatures.

In some cases, the gas near the centers of the galaxies are {\it
masers} (the microwave versions of {\it lasers}). Continuum radio
emission from the central AGN provides the pumping mechanism. In at
least one case (NGC 4258), the masing gas is in a convenient disk-like
geometry that allows its angular and Doppler rotational velocity to be
measured, allowing measurement of the mass and distance to the black
hole.

\subsection{Reverberation mapping}

Reverberation mapping provides a third measurement of black hole
mass. In this case we measure the continuum and line emission of
quasars over time. The continuum varies over week-to-month long time
scales. The broad line emission is powered by the continuum but with a
delay of months (depending on the line, since different ionization
stages have different ionization energies) because of the size of the
broad line emitting region. Measurement of the delay determines the
size of the region emitting for any given line, and the Doppler width
of the line then allows the use of the virial theorem to determine the
mass. A dimensionless factor associated with the geometry still must
be determined. 

\subsection{$M$--$\sigma$ relation}

Measurements of the black hole masses can be correlated with the
galaxy stellar mass and other properties. $M_{\rm BH}$ and the stellar
mass are generally correlated, with a fair amount of scatter. A
smaller scatter is obtained between $M_{\rm BH}$ and the bulge
velocity dispersion (the center of the galaxy, but outside the black
holes sphere of influence).  {\bf need more detail here}

\section{Commentary}

The significance of the $M$-$\sigma$ relation is unclear. It is not
particularly surprising that the black hole mass correlates with the
galaxy mass, because with with its central location the black hole can
grow as a result of the accretion and merging processes that also
drive the galaxy growth. However, the specific scaling of the black
hole mass must tell us something about how these processes are
linked. Furthermore, the improved correlation with bulge properties
may indicate that the black hole growth is related mostly to the
processes that grow the bulge specifically.

\section{Key References}

\begin{itemize}
  \item
    {\it Netzer}
  \item
    {\it AGN Book}
\end{itemize}

\citet{gunn06a}

\section{Order-of-magnitude Exercises}

\begin{enumerate} 
\item Calculate the sizes of the spheres of influence for 
\item Ionization energies of BLR lines
\end{enumerate} 

\section{Analytic Exercises}

\begin{enumerate}
\item Sphere of influence
\end{enumerate}

\section{Numerics and Data Exercises}

\begin{enumerate}
\item Effect of resolution on central velocity dispersion measurements
\end{enumerate}

\bibliographystyle{apj}
\bibliography{exex}  
