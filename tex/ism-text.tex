\title{\bf Overview of the Interstellar Medium}

\section{Basics \& Nomenclature}

The {\it interstellar medium} is the medium between stars in
galaxies. We will include the {\it circumgalactic medium} in our
discussion, though usually that refers to diffuse gas which may be
bound to the galaxy but is not intermixed with its main population of
stars. Here we concentrate on the overall picture and the processes at
play in determining the conditions of the gas.

The contents of the interstellar medium are: gas (ions, atoms,
molecules); dust (small solid particles, mostly $< 1\mu$m; cosmic rays
(ions and electrons, distinguished from the ``gas'' by its nonthermal
energy distribution extending to as high as $10^{21}$ eV); radiation
(from many sources); and dark matter.  The interstellar medium is also
threaded by magnetic fields.

The gas content can be divided into several classes, from hottest to
coolest: 
\begin{ditemize}
\item Coronal gas at $10^{5}$--$10^6$ K and typical density of
$n \sim 0.005$ cm$^{-3}$, attributed to shockheating by supernova
blast waves, and highly collisionally ionized. Observable through
synchrotron in the radio, and UV and X-ray emission.
\item HII gas at $10^4$ K, with densities range from $0.3$
(in the diffuse medium) to $10^4$ cm$^{-3}$ (in HII regions).
Photoionized primarily by nearby hot stars. Observable through optical
emission lines and thermal radio continuum.
\item HI gas at a range of temperatures (100 K to 5000 K) and
densities (30 to 0.5 cm$^{-3}$). Observable through embedded metal
absorption lines and the 21~cm line.
\item H$_2$ at 10 K to 50 K and in diffuse or dense regions with
densities ranging from 100 to $10^6$ cm$^{-3}$. Observable through CO
rotational transitions, dust emission, and metal absorption lines.
\end{ditemize}

The overall cycle of the interstellar medium is thought to be as
follows. Gas flows into dark matter halos. Relative to the dark
matter, gas can efficiently radiate energy and can fall deeper into
the potential well. It loses angular momentum less efficiently and
forms a rotating disk. The disk tends have a large fraction of neutral
gas; as described below, the temperature at 5000 K of neutral gas is
stable. Within the disk and perhaps at previous stages, perturbations
in the gas can become unstable to cooling and can form molecular
clouds. A small fraction of the mass in such clouds (around 1\%) will
form into stars. Hot, massive O and B stars, and later on evolved
stars, can ionize the gas around them. Stellar winds, supernovae, and
stellar mergers can cause gas to be returned to the interstellar
medium, along with nucleosynthetic products of stellar processes.
Blast waves from supernovae and perhaps winds from active galactic
nuclei can remove gas from the galactic system, either temporarily in
a fountain-like structure or permanently.

The multiphase interstellar medium is far from thermodynamic
equilibrium, and is prevented from achieving equilibrium primarily
through stellar processes, which produce ultraviolet radiation and
input kinetic energy in a non-uniform fashion and on time scales much
shorter than any equilibration time for the interstellar medium as a
whole. Thus, the thermal state of the gas is determined by local
conditions and a number of relevant heating and cooling processes. The
relevant processes depend on the temperature and density of the gas as
well as the radiation field.

\subsection{Coronal gas}

Diffuse hot gas, such as coronal gas (or gas in a dark matter halo
surrounding a galaxy or cluster), can often be approximated with
collisional ionization equilibrium. In this case the effect of
ionizing radiation is ignored and the ionization fraction is only a
function of temperature. Such gas experiences cooling due to
collisional excitation followed by radiative deexcitation, and at low
densities (below the critical density for the ions) collisional
deexcitation may be ignored. At temperatures above $10^4$ K, the
ionized electrons of H dominate collisions. In this case cooling can
be quantified by the cooling function:
\begin{equation}
\Lambda = n_e n_H f_{\rm c} (T; \vec{A})
\end{equation}
where $\vec{A}$ indicates the elemental abundances relative to H. At
high temperatures ($10^{6.5}$--$10^{7.5}$ depending on abundance),
this cooling is dominated by bremstrahlung and scales as $f_{\rm
c} \propto T^{1/2}$. For low metallicity gas there are peaks at $\sim
2\times 10^4$ K and $\sim 10^5$ K associated collisional excitation
within H and He. For solar abundance gas, the addition of O and C
cooling lead to a broad peak around $\sim 2 \times 10^5$ K and Fe and
Ne dominate between $10^6$ and $10^7$ K.

\subsection{Ionized gas}

HII gas is ionized gas at around $10^4$ K typically found around stars
hot enough to emit photons blueward of the Lyman limit (912 \AA, or
13.6 eV). The illustrative case of an HII region is a {\it Stromgren
sphere}, which results from a spherically symmetric, uniform cloud of
gas being ionized by such a star. If Q is the number of photons
emitted by the star, and assuming Case B recombination (optically
thick to photons slightly more energetic that the Lyman limit), then
equilibrium between ionization and recombination dictates:
\begin{equation}
Q = \frac{4\pi}{3} R_{S}^3 \alpha_B n(H+) n_e = 
\frac{4\pi}{3} R_{S}^3 \alpha_B n_e^2 
\end{equation}
where the recombination coefficient can be approximated as $\alpha_B =
(2.56 \times 10^{-13}) (T/10^4 {\rm ~K})^{-0.83}$ cm$^3$ s$^{-1}$. The
exercises below will show that these regions have sizes of a few
parsecs and sharp boundaries.

The HII gas is usually primarily heated by photoionization. In
ionization equilibrium the number density of ionizations equals the
recombination rate $\alpha_B n_e n_H$. If the mean energy per
photoelectron is $\phi kT_\ast$, where $T_\ast$ is the effective
temperature of the star then the heating rate is:
\begin{equation}
\Gamma = \alpha_B n_e n_H \phi k T_\ast
\end{equation}
An exercise below demonstrates that $\phi$ is of order unity. The
dependence of cooling on gas temperature is contained in $\alpha_B$,
which declines almost linearly with $T$. 

As long as metals are present in the gas, HII regions are primarily
cooled by collisional excitation of ions followed by radiative
decay. The primary ions responsible are OIII, SIII, and to a lesser
extent OII, NeII, and NII. Within the relevant range of gas
temperatures for HII regions, collisionally excited cooling increases
with $T$, since higher temperatures lead to stronger collisions and
more excited ions. Cooling is an increasing function of metallicity
due to the greater number of ions. Cooling is a decreasing function of
density due to increased collisional deexcitation.

The decreasing heating and increasing cooling as a function of gas
temperature lead to an equilibrium temperature at around 8,000 K for
$n \sim 4000$ cm$^{-3}$, $T_\ast \sim 35,000$ K, and solar
abundances. The equilibrium temperature can range between 5,000 and
15,000 K depending on metallicity, and is an increasing function of
density. This temperature dependence and the resulting signatures in
the collisionally excited lines are an important indicator of
metallicity.

Similar regions are found around evolved stars whose hot stellar cores
are exposed, on their path to becoming white dwarfs. Because of the
smaller flux of these stars, these {\it planetary nebulae} are smaller
than HII regions.

Ionized gas in HII regions is typically detected through recombination
lines of hydrogen and collisionally excited lines.

Diffuse ionized H also exists throughout disk galaxies like the Milky
Way in a warm phase. In the Milky Way its distribution can be
constrained from the dispersion measures to pulsars, and it is
distributed throughout the disk and has a vertical structure more
extended than the HI disk. In other galaxies it can be observed
through its collisionally excited line emission.

\subsection{Neutral gas}

Diffuse neutral gas in the disk exists in multiple phases. In many
galaxies, including the Milky Way, it extends considerably further
than the stars, up to several times further out in radius.  Heating
and cooling of this gas are complex processes. Heating is dominated by
cosmic ray ionization and photoelectric heating by dust (which
requires lower frequency photons than photoelectric heating by
H). Cooling is dominated by collisional excitation lines [CII] 158
$\mu$m and [OI] 63 $\mu$m. [CII] becomes important above 100 K, and
the cooling rate rises slowly until $10^4$ K, when Lyman-$\alpha$
cooling rapidly becomes important. Under typical conditions, a stable
thermal equilibrium is available at both 100--200 K (where [CII] 158
$\mu$m has a stronger flux) and at 5,000--6,000 K (where [OI] 63
$\mu$m has a stronger flux).

The neutral gas can best be traced through the 21 cm emission due to
the hyperfine transition of H, between misaligned and aligned spins of
the electron and proton. The energy of the system is lower by an
amount equivalent to a 21 cm photon when the magnetic moments are
aligned, and thus when the spins are mis-aligned. The Einstein A
coefficient for the radiative transition from aligned to mis-aligned
is $2.85\times 10^{-15}$ s$^{-1}$. In the exercises, we will show that
under interstellar conditions without self-shielding these transitions
lead to a 21-cm line luminosity directly proportional to the hydrogen
mass of:
\begin{equation}
\label{eq:hi}
L = \left[ \frac{3}{4} \frac{A hc} {\lambda m_p} \right] M_H
\end{equation}
Commonly the gas is sufficiently dense that self-shielding is
important; this effect depends on the geometry and conditions of the
gas but can easily be 20--30\% overall for a galaxy.


\subsection{Molecular gas}

Molecular hydrogren populates the densest regions of the interstellar
medium. H$_2$ can form in pristine gas but only through indirect
mechanisms. In galactic environments, molecular hydrogren is formed
through dust grain catalysis; two H atoms bound to the grain surface
react to form H$_2$, and the released energy (4.5 eV) detaches
them. H$_2$ is susceptible to photodissociation, through excitation to
rotation-vibration levels that can decay into a dissociated state. Due
to this effect, H$_2$ is present in the neutral medium in very small
amounts. Molecular regions remain molecular due to self-shielding;
deep in the cloud there is a large optical depth for the photons
exciting to the rotation-vibration levels. H$_2$ emission is very
weak, so molecular clouds are typically studied through the emission
of other molecules, principally through radio-wavelength lines of CO
calibrated relative to H$_2$ through virial masses of local clouds
(\citealt{bolatto13a}).

The rotational energy states for a diatomic molecule are:
\begin{equation}
E_{\rm rot} = \frac{J(J+1) \hbar^2}{2I}
\end{equation}
The permitted transitions are $\Delta J = \pm 1$. The $J\rightarrow
J-1$ transition releases $\hbar^2J/I$ amount of energy. This leads to
a ladder of transitions equally spaced in frequency. The most commonly
observed ones for CO are $J1\rightarrow 0$ (2.6 mm) and $J2\rightarrow
1$ (1.3 mm). In the inner regions of molecular clouds these lines are
optically thick for the dominant isotopes of C and O, and detailed
study of the interior density requires looking at rarer isotopes
involving ${}^{13}$C (about $1/65$ fraction) or ${}^{18}$O (about
$1/500$ fraction).

Molecular regions are the where star formation occurs, through further
cooling and collapse of protostellar objects. These star forming
regions lead to the dissociation and ionizing of the gas surrounding
them (creating HII regions), as well as to supernovae which can shock
heat the interstellar medium to create coronal phase gas.

\subsection{Dust}

Dust comprises of order 1\% of the gas by mass. In the diffuse
interstellar medium the gas can be shown to be depleted in a number of
abundant elements, specifically C (about 30--40\% depletion), and Mg,
Si, and Fe (about 90\% depletion). How depleted oxygen is is unclear.
There is a tendency for higher depletion of elements with a high
condensation temperature (approximately $T_c>1000$ K). These depleted
elements likely form the interstellar dust, whose presence was first
detected through the extinction and reddening it causes. Analysis of
the extinction leads the conclusion that the size of dust particles
range from 0.01 to a few tenths $\mu$m. Warm dust emission in the
infrared reveals the presence of small dust grains. Mid-infrared
emission lines indicate the presence of polyaromatic hydrocarbons. A
broad absorption band at 2250 \AA may also be due to aromatic carbon
of some variety. A large number of diffuse interstellar bands are
detectable whose origin is unknown.

\subsection{Cosmic Rays}

Cosmic ray particles are present in the gas, with a steeply declining
spectrum in energy ($E^{-2.6}$). They are thought to be accelerated
around shock fronts through the first-order Fermi process. Most of the
particles are protons, but they come in many elements with a vaguely
solar abundance ratios. They are overabundant in Li, B, and Be,
probably because these can be formed through spallation of higher mass
cosmic rays against protons, and in Mg, Si, and Fe, perhaps because
dust plays some role in sourcing the cosmic ray distribution.

\section{Commentary}

The complexity of the interstellar medium and its physics are apparent
even in this brief discussion. A course in extragalactic astrophysics
can hardly do it justice given the phenomenology we need to cover
about galaxies. 

However, much of that phenomenology, and many of the theoretical
predictions of galaxy evoution, are derived from an understanding of
the interstellar medium. Particularly as observations of galaxies
reach ever higher redshifts --- and thus physical conditions with a
smaller likelihood of a well-studied local analog --- we all do well
to keep this dependence in mind.

\section{Key References}

\begin{itemize}
  \item
    \href{http://}
    {\it Physics of the Interstellar and Intergalactic Medium,
      \citet{draine07a}}
\item \citet{bolatto13a}
\item \citet{kalberla09a}
\end{itemize}

\section{Order-of-magnitude Exercises}

\begin{enumerate} 
\item Estimate the cooling time of hot gas.
\begin{answer}
The cooling time of hot gas (assuming constant density) can be
estimated as $3nkT / 2 \Lambda$. At solar abundance, $n=0.005$
cm$^{-3}$, and $T=10^6$ K this is $\sim 2 \times 10^7$ yr; it varies
roughly in proportion to metallicity and approximately as $T^{1.7}$.
\end{answer}
\item Estimate the temperatures at which cooling due to [CII]
158 $\mu$m and [OI] 63 $\mu$m should start to become important.
\item Estimate the optical depth due to dust through the warm-phase neutral
medium, for 100 pc and for 8 kpc.
\begin{answer}[Author: Nicholas Faucher]
The warm-phase neutral medium consists mainly of neutral Hydrogen
atoms at a temperature around 5,000 K and number density $0.5 \, {\rm
~cm}^{-3}$. In order to estimate the optical depth due to dust within
this medium, we must find the number of density of dust, which
comprises around 1\% of the mass. Multiplying the number density of
Hydrogen by its mass gives the mass density of the gas:
\begin{equation}
\rho \approx (0.5 \, {\rm ~cm}^{-3}) \, (1.674\times10^{-24} \, {\rm
~g}) \approx 7.85\times10^{-25} \frac{{\rm ~g}}{{\rm ~cm}^3}
\end{equation}
From this we can estimate the mass density of the dust to be
$\rho_{\rm dust} \sim 7.85\times 10^{-27} {\rm ~g}{\rm ~cm}^{-3}$. To
find the number density of dust, we need to estimate the mass of the
dust grains. Assuming spherical dust grains with diameter $0.1 \, \mu
m$ with density similar to that of ${\rm SiO}_2$ ($\sim 3 {\rm ~g}{\rm
~cm}^3$), we find a mass of $M_{\rm dust} \sim 1.57\times10^{-15} \,
{\rm ~g}$. We can now estimate the number density of such dust grains:
\begin{equation}
n_{\rm dust} \sim \frac{\rho_{\rm dust}}{M_{\rm dust}} \sim 5\times
10^{-12} \, {\rm ~cm}^{-3}
\end{equation}
Now we can estimate the optical looking through a given
distance $d$:
\begin{equation}
\tau = n \sigma_n d \sim (5 \times 10^{-12} \, {\rm cm}^{-3})
\left[\pi (5\times10^{-6} \, {\rm ~cm})^2\right] d \sim
4\times 10^{-22} \left(\frac{d}{\rm cm}\right) \sim
10^{-3} \left(\frac{d}{\rm pc}\right)
\end{equation}
Therefore $\tau \sim 0.1$ for 100 pc and $\tau\sim 8$ 
for 8 kpc. Thus, we see that looking up out of the plane there will be
a modest amount of extinction; looking towards the center of the
Galaxy there will be many, many magnitudes of extinction!
\end{answer}
\item We often see little reddening of the central star of an HII
region. Assuming the extinction is less than 0.1 mag, what would that
say about the dust fraction in HII regions relative to its presence in
the neutral and molecular gas?
\item Dust emits a large proportion of its light around 100 $\mu$m in
an approximately thermal distribution. What temperature is the dust?
\end{enumerate}   

\section{Analytic Exercises}

\begin{enumerate}
\item Derive Equation \ref{eq:hi} for the luminosity of the HI 21-cm
line, a hyperfine transition between $n=1$ ground states of neutral
hydrogen, the upper one with spins of the electron and proton aligned
and the lower one with them misaligned.

\begin{answer}
We proceed first by showing that the population levels of the
hyperfine states are set entirely by the statistical weights of the
states, and not by the details of the gas temperature and
density. This is because collisions are far more common than radiative
transitions, and the energy of collision is far greater than the
hyperfine transition energy.

Interstellar gas conditions vary, but it is sufficient for our
purposes here to assume $n\sim 1$ cm$^{-3}$ and $T\sim 10^4$ K.  The
rate of collisions between neutral atoms is:
\begin{equation}
C = v \sigma n \sim 3\times 10^{-10} {\rm ~s}^{-1}
\end{equation}
where $\sigma \sim 10^{-16}$ cm$^{2}$ and $v\sim \sqrt{kT/m_p}$. This
rate $C\gg A$.

The 21-cm line corresponds to $\Delta E = hc / \lambda = $. Each
collision correponds to $E_c \sim kT =$. The Boltzmann factor
$\exp(-\Delta E/kT)$ is therefore extremely close to unity for these
two states.

Together, these results imply that at any time the population of the
two hyperfine states are simply their statistical weight. Spin-aligned
states have a total angular momentum number $l=1$, so the upper state
has $g=3$ ($m=-1$, 0, 1); spin-misaligned states have $l=0$, so the
lower state has $g=0$ ($m=0$). Thus $3/4$ of the hydrogren is in the
upper state. 

Putting this together we find that the luminosity is the number of
hydrogen atoms $N_H$, times the fraction in the upper state $3/4$,
times the rate each upper state decays at, times the energy it emits
when it does:
\begin{equation}
\label{eq:hi}
L = \frac{3}{4} N_H \Delta E A =  \left[ \frac{3}{4} \frac{A hc}
{\lambda m_p} \right] M_H. 
\end{equation}
\end{answer}
\end{enumerate}

\section{Numerics and Data Exercises}

\begin{enumerate}
\item Install and use {\tt ChiantiPy} to calculate the total radiative
losses for optically thin, hot gas (i.e. coronal gas) as a function of
temperature from $T \sim 10^4$ to $10^8$ K.
\item Calculating $\phi$
%For the process
%\begin{equation}
%X^{+r} + h\nu \rightarrow X^{+r +1} + e^{-1} + {\rm ~kinetic energy},
%\end{equation}
%the heating rate per unit volume is:
%\begin{equation}
%\Gamma = n\left(X^{+r}\right)  \int_{\nu_0}^\infty {\rm
%d}\nu \sigma_{\rm pi} c \left(\frac{u_n}{h_\nu} \right)
%(h\nu - h \nu_0)
%\end{equation}
%where $h\nu_0$ is the ionization energy. The number of ionizations
%per unit volume can be calculated similarly.
\item depletion onto dust
\item Use the \citet{schlegel98a} dust maps (available for example
through the {\tt dustmaps} Python package, to find the dust extinction
for two regions on the sky: $(l, b) = (180^\circ, 10^\circ)$ and
$(180^\circ, 70\circ)$. Find GALEX, SDSS, and 2MASS fluxes and Gaia
parallaxes for stars in those directions. Based on the high Galactic
latitude direction, find a color cut in $J-K$ which selects red giant
stars. Using the UV-optical-infrared colors of those stars in the two
directions, estimate the dust extinction curve.
\item Virial estimate of H2 masses
\item HII map exploration; dust, CO, HI emission
\end{enumerate}`

\bibliographystyle{apj}
\bibliography{exex}  
