\title{\bf Spectra}

\section{Basics \& Nomenclature}

Spectra of objects can be determined in a number of ways, through
refraction, diffraction, or through energy-sensitive devices (most
commonly in the X-rays). Here we concentrate on the issues most
relevant to diffraction grating spectra, though many of these issues
are also relevant in other contexts.

Typically, spectra are obtained by putting a dispersive element in the
collimated beam. Although prisms can and have been used at the
objective pupil, more typically the dispersing element is at a pupil
beyond the focus. In addition to providing a smaller area for the
dispersing element to cover, this allows the focal plane to be used to
remove light not associated with the object or object of
interest. Slit spectrographs put a slit (or many slits) in this focal
plane, recollimate the diverging beam, disperse it, and then refocus
it on a new focal plane. Fiber spectrographs use optical fibers in the
focal plane and then align the fiber outputs in a convenient manner
with respect to the detectors.

Diffraction gratings consist of optical elements ruled in some manner
with a set of closely spaced apertures. At any given wavelength, light
coming through the grating will be diffracted. The majority of energy
passes straight through in the 0th order mode, but diffraction causes
additional modes at angles perpendicular to the ruling. The location
of the 1st and greater order modes depends on wavelength, meaning the
higher-order modes yield spectra, with larger dispersion but a smaller
fraction of the incoming energy at higher order modes. Traditional
diffraction gratings were usually ruled metal or mirrors, often mated
to a prism (a configuration known as a {\it grism}) to reorient the
desired mode along a convenient path. Modern gratings are usually
volume-phase holographic gratings, created within a substrate through
holographic techniques. With greater control over the grating
geometry, these gratings typically result in optics with higher
throughput by avoiding occlusion along the geometric optics light
path.

Spectra can be described as having a characteristic resolution. The
raw spectrum is usually detected in a two-dimensional image, and the
spectrum appears along a somewhat curved line, known as the {\it
trace}, traversing the image along the dispersion direction. The
resolution parallel to the dispersion direction is known as the {\it
line spread function}, whereas the resolution perpendicular to the
dispersion direction is known as the {\it point spread function}, in
analogy to an image.

For slit spectra, the point spread function is determined largely by
the seeing and telescope PSF; the actual profile of the spectrum
perpendicular to the trace is the image of the object convolved with
the PSF. For fiber spectra, the light is usually sufficiently
scrambled that the width of the PSF is determined by the size of the
fiber, as reimaged by the camera.

The line spread function is charactered by the resolution $R= \lambda
/ \Delta \lambda$, where $\Delta\lambda$ is usually the FWHM of the
LSF (but not always). It is determined by the size of the fiber and
the dispersion power of the grating. For exposures more than a few
minutes, any flexure of the spectrograph over the course of the
exposure may also cause blurring in the LSF (or PSF).

The raw spectra need to be extracted and calibrated. The {\it
extraction} involves inferring a signal as a function of position on
the trace, to create a one-dimensional spectrum. The position in
pixels needs to be converted to wavelength; the relationship between
the two is usually inferred by injecting a signal into the
spectrograph, often sourced by an {\it arc lamp} which emits light at
discrete lines associated with ionized inert gases (He, Ne, Ar,
etc).

The signal needs to be converted to flux units. If we assume the
detector is bias-subtracted as if it were an image, the signal in the
trace can be modeled as:
\begin{equation}
{\rm DN}(x) = \left[f(\lambda(x)) + f_{\rm
sky}(\lambda(x))\right] \times F(x) \times T(\lambda)
\end{equation}
The quantity $F(x)$, the {\it flat}, characterizes the effects of the
detector response and is typically determined by injecting light from
a flat-field lamp, which has a broad spectrum; in fact, only
small-scale features are usually retained from the flat-field for
reasons that will soon be clear.

The quantity $T(\lambda)$ characterizes the throughput as a function
of wavelength, and is affected by the atmospheric transmission, how
much of the light enters the spectroscopic aperture at any given
wavelength, and the throughput of the optics, as well as any remaining
dependence of efficiency on wavelength on scales larger than those
probed by the flat. It must be determined in part by observing
standard stars whose true spectra are assumed, either through
additional slits or fibers at the same time as the objects of
interest, or through the same slits or fibers at slightly different
times. Either approach requires accounting for the resulting slight
differences in the observations. 

The sky signal $f_{\rm sky}(\lambda)$ also needs to be removed. In the
case of slit observations, the sky can be estimated from the outer
parts of the slit, which contain little object flux. In the case of
fiber observations, the sky needs to be estimated from other nearby
fibers. Because the fiber varies with position and time on the sky,
this can be uncertain.

\section{Commentary}

In ground-based spectroscopy it is extremely important to pay
attention to the chromatic atmospheric refraction. This effect can
lead to different amounts of light getting through fiber or slit as a
function of wavelength, because the image in the telescope focal plane
is a function of wavelength. These differences lead to both a
variation in throughput and potential loss of signal-to-noise, and
also may make the spectrum hard to calibrate (if the standards used
are not under the same conditions). Slit spectra are often arranged so
that the slit is parallel to the parallactic direction to minimize
this effect, but there is no such mitigation for fibers. Atmospheric
dispersion correctors consisting of crossed prisms can be used to
reduce this effect, but they need to act in a pupil so are large and
expensive.

In practice, the difficulty in sky subtraction is almost entirely due
to the LSF modeling. Typically the issue arises because there are
strong lines that need to be subtracted, and so the LSF model needs to
be very good to be accurate enough. In addition, the sky can have a
different LSF than the object, since there are slit-filling
issues or due to imperfections in the extraction method.

\section{Key References}

\begin{itemize}
  \item
    {\it Design and Construction of Large Telescopes},
      \citet{bely03a}
  \item
    {\it Astrophysical Techniques}, \citet{kitchin09a}
\end{itemize}


\section{Order-of-magnitude Exercises}

\begin{enumerate} 
\item Assuming the centroid accuracy is related to $R$ and $S/N$ ratio
    in a similar way for optical spectra of single emission lines as
    for point sources in an image, for an unresolved line with
    $S/N \sim 10$ in its total flux, how high $R$ do you have to be to
    determine a Doppler velocity to 10 km s$^{-1}$ precision?  How
    would a redshift determination using a full optical spectrum
    (i.e. using many lines) change the precision of the determination
    qualitatively? How does dependence of the velocity precision on
    $R$ differ qualitatively for absorption lines?

\begin{answer}[Author: Matthew Daunt]
    Due to the Doppler shift, we can predict the radial velocity of an
    object using the observed wavelength and observed wavelength,
    via
\begin{equation}
    v_r = \frac{\lambda_{obs} -\lambda_0}{\lambda_0} c
\end{equation}
    Resolution is defined as
    \begin{equation}
        R = \frac{\lambda}{\Delta \lambda}
    \end{equation}
    where $\Delta \lambda$ is the full width half max (FWHM). We
    can express the resolution in velocity units therefore as
    \begin{equation}
    \Delta v = \frac{\lambda}{R } \frac{c}{\lambda} = \frac{c}{R}
\end{equation}
    The precision in the radial velocity should be better when the FWHM is
    smaller and the $S/N$ is higher. The scaling is the same as for
    positions in an image, such that
    \begin{equation}
    \sigma \sim {\rm FWHM~}/ (S/N) = \frac{c}{R} \frac{1}{S/N}
    \end{equation}
    For $S/N = 10$, this equation is therefore:
    \begin{equation}
    \sigma \sim \frac{30,000 {\rm ~km} {\rm ~s}^{-1}}{R}
    \end{equation}
    which therefore requires $R\sim 3,000$.  If you have more than one
    emission line, you can use the $S/N$ in each of the, and the
    combined precision will be better than just using one. The case of
    absorption lines will differ because the noise in the continuum
    will contribute, and a larger width of the line will degrade the
    precision further than for an emission line.
\end{answer}
\end{enumerate} 


% \section{Analytic Exercises}

% \begin{enumerate}
% \item Assuming that the 
% \end{enumerate}

\section{Numerics and Data Exercises}

\begin{enumerate}
\item Find an SDSS spectrum of a galaxy and sky fiber from the same
    plate of the BOSS survey. Plot the sky and the object spectrum.
\item Find an arc spectrum from the BOSS survey, and a corresponding sky
spectrum taken during the same calibration-exposure sequence. You will
need to use
the \href{https://data.sdss.org/datamodel/files/SPECTRO_REDUX/RUN2D/PLATE4/spPlan2d.html}{\tt
spPlan2d} files to figure out which
\href{https://data.sdss.org/datamodel/files/BOSS_SPECTRO_REDUX/RUN2D/PLATE4/spCFrame.html}{\tt
spCframe} files to use. Find at least one pair of sky and arc lines,
each of which is bright and isolated from other lines. Estimate the
FWHM of the LSF from each and compare them.
\end{enumerate}

\bibliographystyle{apj}
\bibliography{exex}  
