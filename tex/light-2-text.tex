\title{\bf Light II: emission and propagation}

\section{Basics \& Nomenclature}

In thermal equilibrium, a photon distribution has the Planck spectrum,
with a volume energy density per unit frequency as follows:
\begin{equation}
u_\nu = \frac{8\pi \nu^2}{c^3} \frac{h\nu}{\exp(-h\nu/kT)-1} 
\end{equation}
Typical units are erg Hz$^{-1}$ cm$^{-3}$.
The peak of $u_\nu$ or $u_\lambda$ is simply related to $T$ according
to Wien's Law:
\begin{equation}
\label{eq:wiens}
h\nu_{\rm max} = 2.8 kT \quad\quad
\lambda_{\rm max} T = 2.9 {\rm ~mm} {\rm K} 
\end{equation}
This leads to a specific intensity of this radiation field:
\begin{equation}
I_\nu = \frac{u_\nu c}{4\pi}.
\end{equation}
The flux density through a flat surface is then:
\begin{equation}
f_\nu = \pi I_\nu 
\end{equation}
and the total flux is (integrating over $\nu$):
\begin{equation}
f = \frac{2\pi^5 k^4}{15 c^2 h^3} T^4 = \sigma T^4
\end{equation}
where this question uses the definition of the Stefan-Boltzmann
constant $\sigma$.

At low frequency this is a power-law distribution, the Rayleigh-Jeans
tail:
\begin{equation}
u_\nu \approx \frac{8\pi k T \nu^2 }{c^3}
\end{equation}
and the specific intensity is:
\begin{equation}
I_\nu \approx \frac{2 \nu^2 }{c^2}
\end{equation}

Under many conditions, the photon distribution is not in thermal
equilibrium, and therefore differs from the Planck spectrum. Such
spectra reveal details about the specific physical interactions the
photons are undergoing that yield important clues about the conditions
of the emitting material.

{\it Lines}, or sharp features in the spectrum, can be created due to
discrete energy level differences between atomic or molecular
states. {\it Emission lines} occur when photons are released (and
escape the medium) from a downwards transition. {\it Absorption lines}
occur when photons coming toward the observer incite upwards
transitions in intervening material. We usually speak of these lines
as separate from the {\it continuum} spectrum, so that emission is in
addition to the continuum and absorption is absorbing the continuum;
however what we mean by ``continuum'' varies somewhat depending on
context.  The interpretation of these features of spectra comprise a
large portion of optical astrophysics, and in this section we discuss
only a couple of examples to introduce nomenclature.

Lines can be quantified in several ways:
\begin{itemize}
\item Both emission and absorption lines have some intrinsic width, which can
be expressed as a full width half maximum (FWHM) or
otherwise. Physically, this width can come from a combination of the
intrinsic transition width, pressure-induced width, and Doppler
velocity width (due to thermal or other motions).
\item Emission lines have a total flux or luminosity that can be associated
with them by subtracting an estimate of the continuum and integrating
over wavelength or frequency (e.g., with units erg s$^{-1}$ for
luminosity). There is an equivalent quantity for absorption lines (the
flux or luminosity of the continuum that is absorbed) but this is
rarely referred to.
\item Both emission and absorption lines  have an {\it equivalent
width} (EW), which is usually expressed in units of wavelength, and is
the flux emitted or absorbed divided by a continuum flux density
estimate (e.g. $f_\lambda$) at the location of the line. Sometimes we
use the convention that positive EW indicates absorption and negative
indicates emission; sometimes the opposite.
\end{itemize}

An illustrative example is the atomic transition sequences of
Hydrogen. These transitions are determined by the well-known Bohr
sequence:
\begin{equation}
E_n = - \frac{e^4m_e}{2\hbar^2} \frac{1}{n^2}
\end{equation}
The transitions between these states and important in stellar
atmospheres (typically in absorption) and in interstellar medium
emission (typically in emission). The lowest state is $E_0 = - 13.6$
eV, which corresponds to a photon of 912 \AA; higher frequency photons
will ionize H. The transitions between states are classified
according to the lower state as follows:

\begin{table}[htp]
\caption{
\label{table:hydrogen} Wavelengths (vacuum $\AA$) of hydrogen transitions between $n$ and $m$}
\begin{tabular}{llllll}
\hline\hline
Series & Lower state ($n$) & $\alpha$ ($m = n+1$) & $\beta$ ($m=n+2$)
& $\gamma$ ($m=n+3$) & $\delta$ ($m=n+4$) \\
\hline
Lyman (Ly) & 1 & 1216 & 1026	& 973	& 950 \\
Balmer (H) & 2 & 6564 & 4861	& 4340	& 4102 \\
Paschen (Pa) & 3 & 18756 & 12822	& 10941	& 10052 \\
\hline
\end{tabular}
\end{table}

Another important process, especially in the radio and X-ray domain,
is {\it bremsstrahlung}, the radiation due to the acceleration of
charges. An astrophysical plasma emits {\it thermal bremsstrahlung}
due to Coulomb accelerations of the electrons against each
other. Under most conditions, at wavelengths less than about 1 m
(frequencies greater than 1 GHz), this emission occurs under optically
thin conditions. At low enough frequencies this process is optically
thick and thus thermal (since $h\nu \ll kT$ in practice,
$f_\nu\propto \nu^2$). At higher frequency but still at $h\nu < kT$
(typically in the cm-radio regime), the emission is optically thin. In
this regime, bremsstrahlung has a very flat spectrum, and
$f_\nu\propto \nu^{-0.1}$. Above $h\nu> kT$, the Boltzmann cutoff
leads to $f_\nu\propto \exp(-h\nu/kT)$. The calculation of thermal
bremsstrahlung is somewhat complex.

In plasmas with significant magnetic fields, electrons spiraling
around the magnetic fields yield synchrotron radiation due to their
acceleration. The distribution of electron energies determines the
shape of the resulting spectrum, which can often be approximated as a
power law $f_\nu \propto \nu^{-\alpha}$, where $\alpha$ can range from
0 to over 2.

A final highly significant effect in the propagation of light across
space is due to the effect of interstellar dust. Interstellar dust
typically consists of silica grains and some carbonaceous grains, plus
a small admixture of organic molecules like polycyclic aromatic
hydrocarbons (\citealt{weingartner01a, draine03a}). The grains are
typically less than a few tenths of a micron in size. Because of this,
radio and infrared frequencies are not affected by dust very much. In
general, the amount of extinction is wavelength dependent, with bluer
frequencies experiencing more absorption and scattering; the
dependence varies depending on the nature of the dust but is very
approximately $A(\lambda) \propto \lambda^{-1}$, where $A(\lambda)$ is
in units of magnitudes. In the ultraviolet through near infrared this
causes {\it reddening} of the light. Meanwhile, at the very highest
frequencies (X-ray and $\gamma$-ray) the photons pass through the dust
(in fact, extreme UV and X-ray radiation can destroy dust). The Milky
Way dust imposes a screen of extinction and reddening that must be
corrected for all extragalactic observations, and for most
observations within the Galaxy (\citealt{schlegel98a}).

As light from cosmic sources comes toward us, it can be modified in
other ways than the ones we just discussed, for example through
refraction in astrophysical plasma, by intervening radiation fields,
or through other processes. 

{\bf need to add HI to this discussion}
  
% \section{Commentary}

\section{Key References}

\begin{itemize}
  \item
    \href{http://adsabs.harvard.edu/abs/2000asqu.book.....C}{
    {\it Allen's Astrophysical Quantities},
      \citet{cox00a}}, Chapter 5
\end{itemize}

\section{Order-of-magnitude Exercises}

\begin{enumerate} 
\item As you can see when looking outside during the day, the Sun is
    neither very blue nor very red. Assuming it emits approximately as
    a blackbody, estimate the temperature of its surface.

\begin{answer}
If we assume its peak output is near the middle of our eye's
sensitivity range, around $\lambda \sim 5500$ \AA, we can use Wien's
Law to show that the temperature of the emitting surface is:
\begin{equation}
T \sim \frac{2.9 {\rm ~mm~K}}{5.5 \times 10^{-4} {\rm ~mm}} \sim 5300
{\rm ~K},
\end{equation}
which is surprisingly close to the right answer.
\end{answer}

\item Estimate the approximate temperature of a radiation field that
    will provide a substantial flux of photons to ionize hydrogen.

\begin{answer}[Author: David Mykytyn]
The energy required to ionize a hydrogen atom
is 13.6 eV. This corresponds to a photon of
wavelength $\lambda = $ 912 \AA. Now, from Wien's Law, we
can estimate the temperature of a gas whose emission would peak at
this wavelength.
\begin{eqnarray} 
\lambda_\textrm{max}T&=& 2.9\mathrm{~mm~K}\\
T&\approx& 32000 \mathrm{~K}
\end{eqnarray}
Below this temperature, the number of photons available to ionize
hydrogen will rapidly decline due to the exponential tail of the
Planck energy distribution.
\end{answer}

\item If you have an optical spectrograph with $R\sim 4000$, for what
line-of-sight velocity dispersion is the intrinsic width of the line
equal to the width due to the resolution?  We will learn later that
depending on signal-to-noise ratio, velocities much smaller than the
resolution are hard to measure.

\item Galaxy clusters emit thermal bremstrahlung at energies $\nu > 1$
keV. What is the temperature necessary to do this? 

\item The center of the Milky Way is very heavily extincted: by about
    30 magnitudes in the $V$ band. Approximately how much is that in
    the near-infrared $K$ band?

\begin{answer}
The effective wavelength of $V$ is 5500 {\AA}, and of $K$ is about 2.2
  $\mu$m, or about 4 times longer. If the extinction curve scales as
  $\lambda^{-1}$, then the magnitude difference will be around 7.5
  magnitudes. This fact means that observations of the Galactic
  center, completely impossible in the optical, are possible in the
  infrared.  In fact, the extinction curve is somewhat steeper so the
  benefit is even larger than we estimate here.
\end{answer}

\end{enumerate}   

\section{Analytic Exercises}

\begin{enumerate}
\item Prove Wien's law, Equation \ref{eq:wiens}.

\begin{answer}[Author: Matthew Daunt]
We begin with the Planck spectrum:
    \begin{align*}
        u(\nu) & = \frac{8 \pi h}{c^3} \frac{\nu^3}{\exp[h \nu / k T] - 1}
    \end{align*}
    To find the max of the energy density with respect to $\nu$, I will differentiate with respect to $\nu$, and then set equal to 0 and solve for $\nu$.
    \begin{align*}
        \frac{d u}{d \nu} & = \frac{8 \pi h}{c^3}
    \left( \frac{3 \nu^2}{\exp[h \nu / k T] - 1} - \frac{\nu^3}{(\exp[h \nu
    / k T] - 1)^2}\exp[h \nu/k T]\left(\frac{h}{k T}\right) \right) \\
        \frac{d u}{d \nu} & = \frac{8 \pi h}{c^3} \frac{3 \nu^2 (\exp[h \nu / k T] - 1) - \nu^3 \frac{h}{k T} \exp[h \nu/k T]}{(\exp[h \nu / k T] - 1)^2} \\
        0 & = \frac{8 \pi h}{c^3} \frac{3 \nu^2 (\exp[h \nu / k T] - 1) - \nu^3 \frac{h}{k T} \exp[h \nu/k T]}{(\exp[h \nu / k T] - 1)^2} \\
        0 & =  3 \nu^2 (\exp[h \nu / k T] - 1) - \nu^3 \frac{h}{k T} \exp[h \nu/k T] \\
        0 & =  3 (\exp[h \nu / k T] - 1) - \nu \frac{h}{k T} \exp[h \nu/k T]
    \end{align*}
    Now let $x = \frac{h \nu}{k T}$.
    \begin{align*}
        0 & =  3 (\exp[x] - 1) - x \exp[x]
    \end{align*}
    This root-finding problem cannot be further reduced
    analytically. Newton's method as
    implemented by {\tt scipy} in Python can be applied, which
    converges to $x=2.82$. Thus:
    \begin{align*}
        2.8 k T & = h \nu
    \end{align*}
\end{answer}

\item The Bohr sequence is a consequence of Schroedinger's equation
  applied to the electron. However, the original Bohr model was based
  on the classical picture of an electron orbiting the proton due to
  Coulomb attraction. In this context, the electron is actually
  orbiting the center of mass of the proton-electron system, but this
  two-body problem can be reduced to a central force problem, with the
  true electron mass being replaced by the {\it reduced mass}. Given
  this understanding, how should the spectrum of deuterium compare to
  the spectrum of hydrogen?
\end{enumerate}

\section{Numerics and Data Exercises}

\begin{enumerate}
\item Download an optical spectrum of an A star. Identify 
  all Balmer absorption lines that are apparent in that spectrum.
\item Download an optical spectrum of a star forming galaxy. Identify
all Balmer emission lines that are apparent in the spectrum. Zooming
in on H$\alpha$ or $H\beta$, visually compare the Balmer absorption
(in the stellar continuum) to the emission.
\item Download an optical spectrum of a luminous quasar at redshift $z\sim
2.5$. Identify the Ly$\alpha$ emission line and estimate its full
width half maximum in \AA. Assuming the width is dominated by Doppler
motions, what is that width in km s$^{-1}$. You should also be able to
see absorption lines blueward of Ly$\alpha$; this is Ly$\alpha$
absorption by intervening gas that is therefore less redshifted.
\item \citet{schlegel98a} published an estimate of Galactic dust
extinction in our galaxy. Download four stars with similar
temperatures and metallicities over a range of Galactic reddening
values. Utilize the extinction model of \citet{cardelli89a} and try to
explain the broad-band differences between stellar spectra. Can you
explain any discrepancies with the SFD prediction?
\item Access the dust map of \citet{schlegel98a} and plot the $E(B-V)$
    reddening as a function of Galactic Coordinates. 
\item Find the SDSS optical spectra and images for the two galaxies
UGC 10227 (a typical-looking disk galaxy observed at high inclination)
and MCG -01-53-020 (a typical-looking disk galaxy observed at low
inclination). A major difference in observing galaxies at these
inclinations is the resulting amount of dust extinction. For a
standard reddening law, how much extinction due you need to explain
the first galaxy spectrum as a reddened version of the second?
% \item PAH emission
% \item Brehmstrahlung
% \item Synchrotron
\end{enumerate}

\bibliographystyle{apj}
\bibliography{exex}  
