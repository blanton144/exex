\title{\bf Light II: emission and propagation}

\section{Basics \& Nomenclature}

In thermal equilibrium, a photon distribution has the Planck spectrum,
with a volume energy density per unit frequency as follows:
\begin{equation}
u_\nu = \frac{8\pi \nu^2}{c^3} \frac{h\nu}{\exp(-h\nu/kT)-1} 
\end{equation}
Typical units are erg Hz$^{-1}$ cm$^{-3}$. At low frequency this is a
power-law distribution, the Rayleigh-Jeans tail:
\begin{equation}
u_\nu \approx  \frac{8\pi k T \nu^2 }{c^3}
\end{equation}
The peak of $u_\nu$ or $u_\lambda$ is simply related to $T$:
\begin{equation}
h\nu_{\rm max} = 2.8 kT \quad\quad
\lambda_{\rm max} T = 2.9 {\rm ~mm} {\rm K} 
\end{equation}
This leads to a specific intensity of this radiation field:
\begin{equation}
I_\nu = \frac{u_\nu c}{4\pi} = \frac{2\nu^2}{c^2}
\end{equation}
The flux density through a flat surface is then:
\begin{equation}
f_\nu = \pi I_\nu 
\end{equation}
and the total flux is:
\begin{equation}
f = \frac{2\pi^5 k^4}{15 c^2 h^3} T^4 = \sigma T^4
\end{equation}
where this question uses the definition of the Stefan-Boltzmann
constant $\sigma$.

Under many conditions, the photon distribution is not in thermal
equilibrium, and therefore differs from the Planck spectrum. Such
spectra reveal details about the specific physical interactions the
photons are undergoing that yield important clues about the conditions
of the emitting material.

{\it Lines}, or sharp features in the spectrum, can be created due to
discrete energy level differences between atomic or molecular
states. {\it Emission lines} occur when photons are released (and
escape the medium) from a downwards transition. {\it Absorption lines}
occur when photons coming toward the observer incite upwards
transitions in intervening material. We usually speak of these lines
as separate from the {\it continuum} spectrum, so that emission is in
addition to the continuum and absorption is absorbing the continuum;
however what we mean by ``continuum'' varies somewhat depending on
context.  The interpretation of these features of spectra comprise a
large portion of optical astrophysics, and in this section we discuss
only a couple of examples to introduce nomenclature.

Lines can be quantified in several ways:
\begin{itemize}
\item Both emission and absorption lines have some intrinsic width, which can
be expressed as a full width half maximum (FWHM) or
otherwise. Physically, this width can come from a combination of the
intrinsic transition width, pressure-induced width, and Doppler
velocity width (due to thermal or other motions).
\item Emission lines have a total flux or luminosity that can be associated
with them by subtracting an estimate of the continuum and integrating
over wavelength or frequency (e.g., with units erg s$^{-1}$ for
luminosity). There is an equivalent quantity for absorption lines (the
flux or luminosity of the continuum that is absorbed) but this is
rarely referred to.
\item Both emission and absorption lines  have an {\it equivalent
width} (EW), which is usually expressed in units of wavelength, and is
the flux emitted or absorbed divided by a continuum flux density
estimate (e.g. $f_\lambda$) at the location of the line. Sometimes we
use the convention that positive EW indicates absorption and negative
indicates emission; sometimes the opposite.
\end{itemize}

An illustrative example is the atomic transition sequences of
Hydrogen. These transitions are determined by the well-known Bohr
sequence:
\begin{equation}
E_n = - \frac{e^4m_e}{2\hbar^2} \frac{1}{n^2}
\end{equation}
The transitions between these states and important in stellar
atmospheres (typically in absorption) and in interstellar medium
emission (typically in emission). The lowest state is $E_0 = - 13.6$
eV, which corresponds to a photon of 912 \AA; higher frequency photons
will ionize H. The transitions between states are classified
according to the lower state as follows:

\begin{table}[htp]
\caption{
\label{table:hydrogen} Wavelengths (vacuum $\AA$) of hydrogen transitions between $n$ and $m$}
\begin{tabular}{llllll}
\hline\hline
Series & Lower state ($n$) & $\alpha$ ($m = n+1$) & $\beta$ ($m=n+2$)
& $\gamma$ ($m=n+3$) & $\delta$ ($m=n+4$) \\
\hline
Lyman (Ly) & 0 & 1216 & 1026	& 973	& 950 \\
Balmer (H) & 1 & 1216 & 1026	& 973	& 950 \\
Paschen (Pa) & 2 & 1216 & 1026	& 973	& 950 \\
\hline
\end{tabular}
\end{table}

Another important process, especially in the radio and X-ray domain,
is {\it bremsstrahlung}, the radiation due to the acceleration of
charges. An astrophysical plasma emits {\it thermal bremsstrahlung}
due to Coulomb accelerations of the electrons against each
other. Under most conditions, at wavelengths less than about 1 m
(frequencies greater than 1 GHz), this emission occurs under optically
thin conditions. At low enough frequencies this process is optically
thick and thus thermal (since $h\nu \ll kT$ in practice,
$f_\nu\propto \nu^2$). At higher frequency but still at $h\nu < kT$
(typically in the cm-radio regime), the emission is optically
thin. Free-free absorption at these wavelengths leads to
$f_\nu\propto \nu^{-0.1}$. Above $h\nu> kT$, the Boltzmann cutoff
leads to $f_\nu\propto \exp(-h\nu/kT)$. The calculation of thermal
bremsstrahlung is somewhat complex.

In plasmas with significant magnetic fields, electrons spiraling
around the magnetic fields yield synchrotron radiation due to their
acceleration. The distribution of electron energies determines the
shape of the resulting spectrum, which can often be approximated as a
power law $f_\nu \propto \nu^{-\alpha}$, where $\alpha$ can range from
0 to over 2.

A final highly significant effect in the propagation of light across
space is due to the effect of interstellar dust. Interstellar dust
typically consists of silica grains and some carbonaceous grains, plus
a small admixture of organic molecules like polycyclic aromatic
hydrocarbons. The grains are typically less than a few tenths of a
micron in size. Because of this, radio and infrared frequencies are
not affected by dust very much. In general, the amount of extinction
is wavelength dependent, with bluer frequencies experiencing more
absorption and scattering; the dependence varies depending on the
nature of the dust but is very approximately $\sim \lambda^{-1}$. In
the ultraviolet through near infrared this causes ``reddening'' of the
light. Meanwhile, at the very highest frequencies (X-ray and
$\gamma$-ray) the photons pass through the dust (in fact, extreme UV
and X-ray radiation can destroy dust).

As light from cosmic sources comes toward us, it can be scattered or
absorbed in a number of ways other than atomic and molecular
transitions, for example by interstellar dust, by plasma, by
intervening radiation fields, or through other processes. We will
discuss these as appropriate later.
  
% \section{Commentary}

\section{Key References}

\begin{itemize}
  \item
    \href{http://adsabs.harvard.edu/abs/2000asqu.book.....C}{
    {\it Allen's Astrophysical Quantities},
      \citet{cox00a}}, Chapter 5
\end{itemize}

\section{Order-of-magnitude Exercises}

\begin{enumerate} 
\item As you can see when looking outside during the day, the Sun is
    neither very blue nor very red. Assuming it emits approximately as
    a blackbody, estimate the temperature of its surface.

\item Estimate the approximate temperature of a radiation field that
    will provide a substantial flux of photons to ionize hydrogen.

\item If you have a spectrograph with $R\sim 4000$, for what
line-of-sight velocity dispersion is the intrinsic width of the line
equal to the width due to the resolution? [We will learn later that
depending on signal-to-noise ratio, velocities much smaller than the
resolution are hard to measure.]

\item Galaxy clusters emit thermal bremstrahlung at energies $\nu > 1$
keV. What is the temperature necessary to do this? 

\item The center of the Milky Way is very heavily extincted: by about
    30 magnitudes in the $V$ band. Approximately how much is that in
    the near-infrared $K$ band?

  \begin{answer}
  The magnitude difference of 30
    corresponds to a factor of $10^{12}$ in luminosity. Using Equation
    \ref{eq:luminosity_distance}, this translates to $10^6$ in
    distance. So Vega-like stars are visible (in principle) to about 8
    Mpc.  
  \end{answer}

\item Estimate the number of photons per second that enter your eye
    per second in visible light (4000--7000 \AA) from a star with
  magnitude $\sim 6$ (about the faintest visible at a dark
  site). Assume a nighttime pupil diameter of 5 mm.

\begin{answer}
In this wavelength range, Vega and AB magnitudes are about the same at
the precision necessary here, so we don't have to worry about which
version we are dealing with. So:
\begin{equation}
f_\nu \sim (3631 \mathrm{~Jy}) 10^{-0.4 m} \sim 14 \mathrm{~Jy} = 1.4 \times
10^{-22} \mathrm{~erg} \mathrm{~cm}^{-2} \mathrm{~s}^{-1} \mathrm{~Hz}^{-1}
\end{equation}
The flux density in the visible should be $f_\nu \Delta\nu$ where:
\begin{equation}
\Delta\nu = c \left(\frac{1}{4000 \mathrm{~\AA}} -
\frac{1}{7000 \mathrm{~\AA}}\right) \sim 320 \mathrm{~THz}
\end{equation}
And thus $f \sim 4\times 10^{-8}$ erg cm$^{-2}$ s$^{-1}$. Each photon
has an energy (assuming $\lambda = 5500$ \AA):
\begin{equation}
E = h\nu = (6.62\times
10^{-27} \mathrm{~erg~Hz}^{-1})( \mathrm{550~THz}) \sim 4 \times
10^{-12} \mathrm{~erg}
\end{equation}
So the flux of photons is:
\begin{equation}
\frac{\dot N}{A} = \frac{f_\nu \Delta\nu}{h\nu} \sim 
10^4 \mathrm{~s}^{-1} \mathrm{~cm}^{-2}
\end{equation}
If $A \sim \pi r^2 \sim 0.2$ cm$^2$ then $\dot N \sim 2000$ s$^{-1}$. 
\end{answer}

\item How much does surface brightness dimming change the magnitudes
  per square arcsecond for a galaxy at redshift $z\sim 1$?

\begin{answer}
The specific intensity is reduced by $(1+z)^4$. In magnitudes this is:
\begin{equation}
\Delta m = 2.5 \log_{10} (1+z)^4 = 10 \log_{10} (1+z) \sim
3 \mathrm{~mag}
\end{equation}
\end{answer}
\end{enumerate}   

\section{Analytic Exercises}

\begin{enumerate}
\item For a Gaussian line spread function with a standard deviation
  $\sigma$, what is the FWHM?
  \begin{answer}
The FWHM is determined by:
\begin{equation}
\Delta\lambda = 2  \ln(0.5)
\end{equation}
\end{answer}
\item Prove Equation \ref{eq:sb_dimming}, based on the fact that
  photon density in phase space is conserved.
\item Based on Equation \ref{eq:sb_dimming}, how is the angular
  diameter distance related to the luminosity distance? 
\end{enumerate}

\section{Numerics and Data Exercises}

\begin{enumerate}
\item Retrieve a spectrum of a star, a quasar, and a galaxy from the
  Sloan Digital Sky Survey. Plot each of them. These spectra are given
  in $f_\lambda$ (per-\AA) units. Convert one them to $f_\nu$
  (per-Hertz) and plot it. Smooth one of them in $f_\lambda$ with a
  Gaussian corresponding to $R\sim 100$ and plot it.
\item Plot $D_L$ versus $z$ based on the equations found
  in \citet{hogg99cosm}, for a flat $\Lambda$CDM cosmology with
  $\Omega_m = 0.3$ and $H_0 = 70$ km s$^{-1}$ Mpc$^{-1}$. Determine
  where the difference in inferred luminosity of an object would reach
  1\%. 
\item Download the filter curve for the SDSS $g$ and $r$
  bands. Calculate the observed $g$ and $r$ band magnitudes
  corresponding to a galaxy spectrum (say for some galaxy with
  $z<0.1$). Note that this won't necessarily be the same as the
  magnitudes measured from the images, since the spectra are taken
  through 2- or 3-arcsec diameter fibers. Calculate the rest-frame
  $g-r$ color, and also what the $K$-correction would be for galaxies
  with this SED in the $r$-band between about $z\sim 0$ and $z\sim
  0.25$. Download a sample of galaxies between about $z\sim 0$ and
  $0.25$. Plot their $g-r$ colors versus redshift, together with the
  predicted colors of the galaxy you have a spectrum of.
\item \todo{Would be nice to have radio, X-ray, other examples}
\end{enumerate}

\bibliographystyle{apj}
\bibliography{exex}  
