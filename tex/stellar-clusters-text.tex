\title{\bf Stellar Clusters}

\section{Basics \& Nomenclature}

Stars form in dense regions of molecular clouds, typically in galactic
disks. The regions of formation typically form numerous stars across a
large range of masses. If the region is dense enough, the resulting
star cluster becomes gravitationally bound and is long-lived. The {\it
  globular clusters} found in the Milky Way and other galaxies are the
canonical examples of this case. If not, the star cluster will
disperse within a few hundred million years. The canonical example of
this case are the {\it open clusters}. Here we review the
observational properties of these cases.

Globular clusters were first found in the Milky Way. There are
currently about 150 known. The brightest ones can be glimpsed by eye
in dark sites; a number of Messier objects are globular clusters. They
have total absolute magnitudes between about $M_V \sim -4$ to $-12$
and estimated total stellar masses of $10^4$--$10^6$ $M_\odot$ with a
roughly log-normal luminosity function. They are a few pc in
half-light size.  Most of them have constant density {\it cores} with
a power law surface brightness distribution ($r^{-0.75}$), but for
about 20\% this power law continues through to the center and thus
they have a {\it cusp}.

Globulars tend to not have any net rotation of their stars, and have a
typical internal stellar velocity dispersion of 5--10 km s$^{-1}$. As
will become apparent in the discussion of stellar evolution, they
appear to have nearly single-age stellar populations. In the Milky
Way, the majority of them are old, often $\sim 10$--$12$
Gyr. Chemically, they are relatively metal-poor in general.

In the Milky Way are two classes of globular, the halo population and
the disk population (and sometimes astronomers refer to the bulge
population). The halo population extends to larger distances (out past
10 kpc, and a handful to 50 kpc) and is more metal poor. The disk
population is closer ($< 10$ kpc) and is an oblate distribution and is
more metal rich. 

Most other galaxies have globular clusters. We define the specific
frequency of globular clusters as:
\begin{equation}
S_N = N_{\rm GC} 10^{0.4(M_V + 15)},
\end{equation}
where estimates of $S_N$ are of order unity, but $S_N$ seems to be an
increasing function of galaxy mass. There are indications that the
globular cluster metallicity increases with galaxy mass. The bimodal
distribution of metallicity seen in the Milky Way is also seen in
other systems (sometimes traced by color). In general, extragalactic
globular cluster systems are better studied for elliptical galaxies
rather than spiral galaxies, because of the smoother background
provided by the former for detection of clusters.

Globulars are thought to form in major bursts of star formation within
galaxies, not from individual dark matter halos collapsing at high
redshift, because of their lack of dark matter. Young massive clusters
have been observed in local merging galaxies (\citealt{ashman92a}),
and these clusters are thought to be similar to the birthplace of
globulars. This implies that in star forming galaxies as a whole,
globular clusters presumably exist with a range of ages (not just very
old).  In the Milky Way and in other galaxies, it is likely that the
luminosity and size distribution of globulars is strongly shaped by
internal gravitational effects and gravitational interaction with the
Milky Way, and that the remaining globulars are only a fraction of the
original population.

Open clusters are smaller and more diffuse. The brightest ones are
also visible to the naked eye, for example the Pleiades. They range
from a few tens of millions to around a billion years old. Like
globulars they appear to be close to single stellar populations. Even
over their short lifetimes, internal gravitational effects have
strongly affected the distribution of stars, in general causing {\it
mass segregation} and other effects. Open clusters are almost by
definition unbound or loosely bound. We believe that the majority and
perhaps all stars formed in clusters, most of which were open clusters
that dissipated soon after their star formation ended
(\citealt{lada03a}).

\section{Commentary}

\section{Key References}

\begin{itemize}
  \item
    {\it The Formation of Globular Clusters in Merging and Interacting
    Galaxies}, \citet{ashman92a}
  \item
    \href{http://physwww.mcmaster.ca/~harris/mwgc.dat}{\it The Harris globular cluster catalog}, \citet{harris96a}
  \item
    {\it Extragalactic Globular Clusters and Galaxy
    Formation}, \citet{brodie06a}
\end{itemize}

\section{Order-of-magnitude Exercises}

\begin{enumerate} 
\item Typical time of dissolution of star cluster
\item What fraction of the stellar mass in the Milky Way is in
    globular clusters?
\item How many open clusters would be necessary to explain the current
stellar mass in the MW disk?
\end{enumerate} 

\section{Numerics and Data Exercises}

\begin{enumerate}
\item HR diagram for an open cluster
\item HR diagram for a globular cluster 
\item Open clusters in MCs
\item Extragalactic GC systems
\item YMC systems
\end{enumerate}

\bibliographystyle{apj}
\bibliography{exex}  
