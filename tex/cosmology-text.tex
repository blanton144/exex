\title{\bf Cosmology}

\section{Basics \& Nomenclature}

The universe is expanding. What that means is that at some point all
of the known universe was previously in a tiny space. Since that time
it has expanded dramatically. This expansion can be seen today in the
fact that more distant galaxies are moving faster away from us, the
{\it Hubble Law}. The density of the universe appears homogeneous and
the expansion appears isotropic.

Our best evidence today suggests that at the earliest times we can
hope to constraint the Universe was expanding exponentially with time,
during a phase called {\it inflation}. Inflation explains how widely
separated regions of the universe have similar density, by supposing
that they were close enough to come to thermal equilibrium at
extremely early times, but during inflation were ripped apart.
Inflation also explains the small fluctuations in density that form
structure in the universe, as quantum fluctuations. After inflation,
the expansion of the universe was controlled by the amount of matter
and radiation within it.

At this phase the universe was very hot. At some point protons,
neutrons, electrons, and photons formed a baryon-photon fluid tightly
bound held in equilibrium by the electromagnetic and weak
forces. During big bang nucleosynthesis, the protons and the neutrons
fell out of equilibrium, and deuterium, ${}^3$He, ${}^4$He formed,
along with some Li. Later, at the recombination redshift of $z\sim
1100$, the electrons became bound to the hydrogen atoms; the
temperature of the gas at that redshift was $T\sim 3000$ K. Since that
time, at some point between $z\sim 7$ and 20, the first stars and
quasars ionized the vast majority of the gas again.

To understand the expansion itself, consider the gravitational
dynamics of a universe with just matter (no radiation), which is a
good enough approximation at $z<30$. The general relativistic version
of this picture can be derived from the Einstein equations. This
approach is necessary to understand how light travels through the
resulting curved space-time of the universe.

However, to just understand the dynamics of the expansion a
straightforward Newtonian approach suffices.  Assume a homogeneous,
expanding universe, and pick some center.  Consider some particle a
distance $r$ from the center today. Since the universe has to remain
homogeneous, it doesn't matter which direction it is in.  The particle
will be moving radially from the center. In fact, at one point it must
have been at radius zero. Let us express its distance as a function of
time as $a(t)r$ where $a(t_0 = {\rm now}) = 1$. Newton's laws mean
that the gravitational force is just due to the mass interior to $r$;
alternatively, the motion of the % particle is driven by the energy
equation, where the potential is given by the Keplerian potential due
to the mass interior:
\begin{equation}
\label{eq:energy}
E = \frac{1}{2} m v^2 + m \phi(a(t) r) = \frac{1}{2} m v^2 -
\frac{GM(<a(t) r)m}{a(t) r} - \frac{m\Lambda}{6} a^2(t) r^2 =
\mathrm{~constant}
\end{equation}
The $\Lambda$ accounts for the cosmological constant (in general
relativity, introducing $\Lambda$ retains some essential properties of
the theory, which we do not demonstrate here). The equations here are
only valid when the universe is matter-dominated; the earlier phase
when the universe was radiation-dominated requires some relativity to
understand. But this simplistic approach demonstrates the differences
between $E<0$ (closed universe), $E=0$ (``flat'' universe), and $E>0$
(open universe).

We define $H=v/a(t)r$. Note that $v={\dot a}(t) r$, so the {\it Hubble
  parameter} $H(z)$ is:
\begin{equation}
  H(t) = \frac{\dot a}{a}
\end{equation}
At redshift zero, the Hubble parameter is equal to the Hubble
Constant.

We define also:
\begin{eqnarray}
\Omega_k &=& \frac{2Ea^2(t) r^2}{mH^2} \cr
\Omega_m &=& \frac{8\pi G\rho}{3H^2}\cr
\Omega_\Lambda &=& \frac{\Lambda}{3H^2} 
\end{eqnarray}
$\Omega_m$ is the scaled matter density, $\Omega_\Lambda$ is the
contribution of the matter density, and for reasons that become clear
in the general relativistic picture, $\Omega_k$ is the term related to
the curvature of space-time.

Under these definitions:
\begin{equation}
\Omega_k + \Omega_m + \Omega_\Lambda = 1
\end{equation}
at all times.
  
The parameters $\Omega_{m0}$ and $\Omega_{\Lambda 0}$ are the present
time values of the matter density and the cosmological constant, and
they are roughly $0.25$--$0.30$ and $0.70$--$0.75$ respectively.  The
curvature $\Omega_k$ is consistent with zero within $0.02$ or so.
The Hubble parameter today is $H_0 \sim 65$--$75$ km/s/Mpc. 

For the majority of galaxies whose distances we have an estimate of,
it comes from the Doppler shift inference of velocity and the Hubble
Law. Two common choices of units are as follows:
\begin{equation}
r {\rm ~in~}h^{-1}{\rm ~Mpc} {\rm ~where~} h = \frac{H_0}{100 {\rm
    ~km/s/Mpc}} 
\end{equation}
and 
\begin{equation}
r {\rm ~in~}h_{70}^{-1}{\rm ~Mpc} {\rm ~where~} h_{70} = \frac{H_0}{70
  {\rm ~km/s/Mpc}}
\end{equation}
The latter is becoming far more standard these days but the former
still abounds.

In homogeneous expansion, the universe just scales overall by the
factor $a(t)$. We can define a {\it comoving coordinate system} that
expands with the universe --- $r$ as used above is the radius in that
coordinate system. In contrast {\it physical} units express a fixed
size.

In the general relativistic picture, the metric for this expanding
universe is:
\begin{equation}
{\rm d}s^2 = c^2 {\rm d}t^2 - a^2(t) \left[ \frac{{\rm d}q^2}{1-Kq^2}
  + q^2\left({\rm d}\theta^2+ \sin^2\theta{\rm d}\phi^2\right) \right] 
\end{equation}
This is the {\it Friemann-Lemaitre-Robertson-Walker metric}.  $K$
determines the curvature of space and can be $K=1$, $-1$, or $0$.
$K=0$ corresponds to the $E=0$ and is the ``flat'' case, because the
spatial term is Euclidean (though spacetime is not flat even in this
case).  The FLRW metric is necessary for determining light paths
through the universe.

At low redshift, the expansion velocity translates into a
nonrelativistic Doppler redshift:
\begin{equation} 
\frac{\lambda_o}{\lambda_e} = 1+z  \approx 1 + \frac{v}{c} \approx 1
+ \frac{H_0 d}{c}
\end{equation} 
and so
\begin{equation}
d \approx \frac{cz}{H_0}
\end{equation}
and sometimes the cosmological redshift is expressed in terms of
$v\approx cz$. At high redshift (e.g. above unity), the relationship
between $d$ and $z$ is ambiguous. 

Within the FLRW metric, there is a relationship between the expansion
factor $a(t)$ and the redshift that that time is observed at today:
\begin{equation}
a(t) = \frac{1}{1+z(t)}
\end{equation}
Heuristically, the photons are stretched by the same factor that the
universe has expanded. The exercises derive this more rigorously. 

The FLRW metric allows us to relate intrinsic luminosities and sizes
to observed fluxes and angular sizes of objects at a given
redshift. The focusing theorem in general relativity states that
matter makes light converge. For a flat, matter-dominated universe:
\begin{equation}
D_C= 
\frac{c}{H_0} \int_0^z \frac{{\rm d}z}{E(z)}
= 
\frac{c}{H_0} \int_0^z \frac{{\rm d}z}{\sqrt{\Omega_{\Lambda0} +
    \Omega_{m0}(1+z)^3}},
\end{equation}
where here we use the expression for the Hubble parameter as a
function of redshift $H(z) = H_0 E(z)$.  This expression can be
derived from the energy equation. In a flat universe, the luminosity
distance is simply:
\begin{equation}
D_L = D_C (1+z)
\end{equation}
and the angular diameter distance is:
\begin{equation}
D_A = \frac{D_C}{1+z}
\end{equation}
In non-flat universes things are a bit more complicated
mathematically.

The universe has perturbations and is not quite homogeneous, inducing
motions with respect to the Hubble flow, called {\it peculiar
velocities}.  The observed redshift is due to both effects combined:
\begin{equation}
1+z_{\rm obs} = (1+z_{\rm cosmo})(1+z_{\rm pec})
\end{equation}
Peculiar velocities are typically a few hundred km/s, and are at most
a few thousand, or $\Delta z \sim 0.01$ at the most, but we map the
universe out to much larger redshift. So usually it is adequate to
express this expression as follows:
\begin{equation}
z_{\rm obs} = z_{\rm cosmo}+z_{\rm pec} + z_{\rm cosmo}z_{\rm pec}
\approx z_{\rm cosmo} + \frac{v_{\rm pec}}{c} (1+z_{\rm cosmo})
\end{equation}

The universe does not only contain matter, but contains radiation and
neutrinos and potentially other ingredients.  The density of
non-relativistic matter scales as $a^{-3}$ but due to redshift the
energy density of relativistic particles scales as
$a^{-4}$. Therefore, sufficiently far into the past the radiation
dominates. This is called the epoch of {\it matter-radiation
equality}.

The cosmic microwave background is observed today as a 2.7 K bath of
photons with a nearly perfect blackbody spectrum.  There is also a
bath of neutrinos surrounding us, of roughly the same
density. Together these relativistic particles have $\Omega_{r0} \sim
4\times 10^{-5}$ $h^{-2}$.  Therefore matter-radiation equality occurs
at a redshift of order $\Omega_{m0}/\Omega_{r0} \sim 10^4$. Note that
the relativistic contribution of radiation still needs to be accounted
for in equations for the expansion at redshifts down to about $z\sim
30$.

\section{Key References}

\begin{itemize}
  \item
    \href{http://adsabs.harvard.edu/abs/1999astro.ph..5116H}{
    {\it Distance measures in Cosmology},
      \citet{hogg99cosm}}
\end{itemize}

\section{Order-of-magnitude Exercises}

\begin{enumerate} 
\item What does the mean density of the universe correspond to in
    particles per cm$^3$?
\item Approximately, why does recombination occur at $z\sim 1000$?
\item How many ionizing photons per cubic Mpc are necessary to
reionize the Universe?
\end{enumerate} 

\section{Analytic Exercises}

\begin{enumerate}
\item Demonstrate that:
\begin{equation}
\Omega_k + \Omega_m + \Omega_\Lambda = 1
\end{equation}
\begin{answer}
We start by rearranging Equation (\ref{eq:energy}) expressing the
energy considerations:
\begin{eqnarray}
\frac{E}{m} &=& \frac{v^2}{2}-\frac{GM}{r} - \frac{\Lambda}{6}a^2(t)r^2 \cr
1 &=& \frac{2E}{mv^2} + \frac{2GM}{a(t) rv^2} -
\frac{\Lambda}{3}\frac{a^2(t) r^2}{v^2} 
\end{eqnarray}
Now use $H=v/a(t)r$. Note that $v={\dot a}(t) r$, so:
\begin{equation}
H(t) = \frac{\dot a}{a}
\end{equation}
This is the time-dependent Hubble parameter. Also let us express the
mass $M$ in terms of the density: $M=4\pi r^3\rho/3$. Then:
\begin{eqnarray}
\frac{2Ea^2(t) r^2}{mH^2} + \frac{8\pi
  G\rho}{3H^2}+\frac{\Lambda}{3H^2} &=& 1 \cr 
\Omega_k + \Omega_m + \Omega_\Lambda &=& 1
\end{eqnarray}
Note that the $E=0$ case is just the $\Omega_k=0$ case, where the
spatial curvature is zero. It can be useful to recast this in yet
another way:
\begin{eqnarray}
\frac{2Ea^2(t_0) r^2}{mH_0^2} + \frac{8\pi
  G\rho(t_0)}{3H_0^2}+\frac{\Lambda}{3H^2} &=& 1 \cr 
\end{eqnarray}
\end{answer}
\item Derive the expansion $a(t)$ as a function of $t$ in the matter
dominated case and the radiation-dominated case.
\item 
In the local universe, clearly the redshift $z$ can be interpreted as
a recession velocity $v$.  However, on non-local scales in general
relativity, there is no unambiguous definition of relative
velocity. Instead, the redshift $z$ needs to be interpreted as
resulting from the expansion of the universe over the path of the
photon.  How does the expansion scale $a(t)$ of the universe at the
time of emission $t$ relate to the redshift $z$?

\begin{answer}
The photons travel along null geodesics. If photons reach an observer,
then they have been traveling radially in the FRW coordinate system
centered on the observer (other photons are going other directions,
but they don't reach the observer). So:
\begin{equation}
{\rm d}s^2 = c^2 {\rm d}t^2 - a^2(t) \frac{{\rm d}q^2}{1-Kq^2} = 0
\end{equation}
and therefore:
\begin{equation}
\frac{{\rm d}t^2}{a^2(t)} = \frac{{\rm d}q^2}{c(1-Kq^2)} 
\end{equation}
Now conceive of a wave peak leaving some radius $q$ at some time
$t_e$, and the next one leaving at $t_e+\delta t_e$.  Integrate the
square root of the above equation over the path taken by the first:
\begin{equation}
\int_{t_e}^{t_o} \frac{{\rm d}t'}{a(t')} = 
\int_q^0 {\rm d}q' 
\frac{1}{c\sqrt{1-Kq^2}} 
\end{equation}
Note that for both wave peaks, the right hand side is the same,
because they leave from the same radius $q$. Therefore:
\begin{equation}
\int_{t_e}^{t_o} \frac{{\rm d}t'}{a(t')} = 
\int_{t_e+\delta t_e}^{t_o+\delta t_o} \frac{{\rm d}t'}{a(t')} = 
\int_{t_e}^{t_o} \frac{{\rm d}t'}{a(t')} + \frac{\delta t_o}{a(t_o)} -
\frac{\delta t_e}{a(t_e)}
\end{equation}
where the last equation is just from the way integrals are defined.
Then clearly:
\begin{equation}
\frac{\delta t_e}{\delta t_o} = \frac{a(t_e)}{a(t_o)}
\end{equation}
This means that the period of the wave is increasing, so the
wavelength is increasing and the frequency is decreasing. Assuming we
observe at $t_o=t_0$, we can write:
\begin{equation}
a(t_e) = \frac{\nu_o}{\nu_e} = \frac{\lambda_e}{\lambda_o} =
\frac{1}{1+z} 
\end{equation}
where the last equality comes from the definition of redshift $z$:
\begin{equation}
\frac{\lambda_o}{\lambda_e} = 1+z 
\end{equation}
You can think of this as the fact that the photons are being stretched
in precisely the same way that the universe is being expanded.
\end{answer}
\item If we measure a peak rotation velocity $v_{\rm  max}$ for a
galaxy at redshift $z$, what is the inferred rotation velocity in that
galaxy's rest frame?
\item The blackbody radiation leftover after recombination is redshifted over
time. Assuming that the radiation is only affected by the redshift,
show how the energy density and temperature of this radiation depends
on redshift.
\end{enumerate}

\section{Numerics and Data Exercises}

\begin{enumerate}
\item Plot $\Omega_m(z)$ and $\Omega_\Lambda(z)$ between $z=0$ and 5,
for a flat Universe.
\end{enumerate}

\bibliographystyle{apj}
\bibliography{exex}  
