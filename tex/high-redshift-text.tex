\title{\bf High redshifts}

\section{Basics}

{\it High redshift} is a relative term that always implies $z>0.2$ but
in the modern era more commonly signifies $z>1$, at which epochs the
galaxy and quasar populations are much different, revealing the
statistical properties the local population would have had much
earlier in time.

\subsection{Detection of high redshift galaxies}

The study of high redshifts in the optical and infrared requires large
aperture telescopes ($> 4$ meter). Infrared observations are required
to obtain the same rest-frame measurements as exist at low
redshift. Space-based observations or adaptive optics are required to
resolve galaxy structures well. A particularly troublesome effect is
the $(1+z)^{-4}$ scaling of the observed surface brightness, which
leads to difficulties in detecting resolved galaxies above background
noise, and to a very steep decline in total flux for both resolved and
unresolved galaxies.

Determining redshifts is most reliably performed with spectra. At high
redshift, obtaining large numbers of spectra becomes increasingly
expensive for the reasons stated above. Between about $z\sim$1--2,
galaxies have few identifiable features in the optical, leading to the
{\it redshift desert}, mitigated with near infrared spectra. The

The largest available samples of high redshift galaxies tend to be
from {\it photometric redshifts}. As the spectral energy distribution
of galaxies $K$-corrects with redshift, its colors change. If fluxes
are measured in enough bandpasses, both the intrinsic color of the
galaxy and the redshift can be inferred. With high precision
photometry over 4--5 optical bandpasses, redshift errors $\delta
z/(1+z) \sim 0.02$ are achievable, although some fraction of large
outliers do persist.

Although precise photometric redshifts involve using the full spectral
energy distribution shape, at some wavelengths strong breaks in the
flux are particularly useful. For red galaxies, the 4000 \AA break is
particularly informative, as it redshifts through the $g$-band and to
redder and redder bands. For galaxies with younger stellar
populations, the somewhat bluer Balmer break is more prominent (and
this can cause degeneracies between models of intrinsically blue
galaxies at one redshift and models of somewhat redder galaxies at
redshifts $\delta z / (1+z) \sim 0.1$ closer). For all galaxies, a
break at 912 \AA is prominent since essentially no flux can be
transmitted blueward due to the high photoionization cross-section of
neutral hydrogen inside and around the galaxies. Galaxies identified
as high redshift with this break are known as {\it Lyman break}
galaxies.

At redshifts $z>2.5$ there are substantial numbers of Ly-$\alpha$
emitting galaxies, which can be selected with narrow band imaging. A
given narrow band observation is sensitive to a particular range of
redshifts. 

Because high angular resolution space-based imaging currently only
covers at most a few tens of degrees of the sky, there are a handful
of fields that have been investigated in detail by many groups. In
particular, the Great Observatories Origins Deep Surveys (GOODS),
Galaxy Evolution from Morphology and SEDS (GEMS), the Extended Groth
Strip (EGS), and the COSMOS field.

\subsection{Demographics of high redshift galaxies}

The luminosity and stellar mass functions can be measured out to 


%\section{Commentary}

\section{Key References}

\begin{itemize}
  \item
    {\it Blanton \& Moustakas}
  \item
    {\it Graham paper on Sersic}
\end{itemize}

\citet{gunn06a}

\section{Order-of-magnitude Exercises}

\begin{enumerate} 
\item Using the numbers in the text, estimate $\Omega_b$ and
    $\Omega_m$. 
\item Estimate the stellar mass surface density for a Milky Way like
disk galaxy.
\end{enumerate} 

\section{Analytic Exercises}

\begin{enumerate}
\item Petrosian quantities for Sersic galaxies
\item $\kappa$ space
\item Tully-Fisher
\end{enumerate}

\section{Numerics and Data Exercises}

\begin{enumerate}
\item Lookback times at high redshift
\end{enumerate}

\bibliographystyle{apj}
\bibliography{exex}  
