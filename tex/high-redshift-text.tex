\title{\bf High redshifts}

\section{Basics}

{\it High redshift} is a relative term that always implies $z>0.2$ but
in the modern era more commonly signifies $z>1$, at which epochs the
galaxy and quasar populations are much different, revealing the
statistical properties the local population would have had much
earlier in time.

\subsection{Detection of high redshift galaxies}

The study of high redshifts in the optical and infrared requires large
aperture telescopes ($> 4$ meter). Infrared observations are required
to obtain the same rest-frame measurements as exist at low
redshift. Space-based observations or adaptive optics are required to
resolve galaxy structures well. A particularly troublesome effect is
the $(1+z)^{-4}$ scaling of the observed surface brightness, which
leads to difficulties in detecting resolved galaxies above background
noise, and to a very steep decline in total flux for both resolved and
unresolved galaxies.

Determining redshifts is most reliably performed with spectra. At high
redshift, obtaining large numbers of spectra becomes increasingly
expensive for the reasons stated above. Between about $z\sim$1--2,
galaxies have few identifiable features in the optical, leading to the
{\it redshift desert}, mitigated with near infrared spectra. The

The largest available samples of high redshift galaxies tend to be
from {\it photometric redshifts}. As the spectral energy distribution
of galaxies $K$-corrects with redshift, its colors change. If fluxes
are measured in enough bandpasses, both the intrinsic color of the
galaxy and the redshift can be inferred. With high precision
photometry over 4--5 optical bandpasses, redshift errors $\delta
z/(1+z) \sim 0.02$ are achievable, although some fraction of large
outliers do persist.

Although precise photometric redshifts involve using the full spectral
energy distribution shape, at some wavelengths strong breaks in the
flux are particularly useful. For red galaxies, the 4000 \AA break is
particularly informative, as it redshifts through the $g$-band and to
redder and redder bands. For galaxies with younger stellar
populations, the somewhat bluer Balmer break is more prominent (and
this can cause degeneracies between models of intrinsically blue
galaxies at one redshift and models of somewhat redder galaxies at
redshifts $\delta z / (1+z) \sim 0.1$ closer). For all galaxies, a
break at 912 \AA is prominent since essentially no flux can be
transmitted blueward due to the high photoionization cross-section of
neutral hydrogen inside and around the galaxies. Galaxies identified
as high redshift with this break are known as {\it Lyman break}
galaxies.

At redshifts $z>2.5$ there are substantial numbers of Ly-$\alpha$
emitting galaxies, which can be selected with narrow band imaging. A
given narrow band observation is sensitive to a particular range of
redshifts. 

Because high angular resolution space-based imaging currently only
covers at most a few tens of degrees of the sky, there are a handful
of fields that have been investigated in detail by many groups. In
particular, the Great Observatories Origins Deep Surveys (GOODS),
Galaxy Evolution from Morphology and SEDS (GEMS), the Extended Groth
Strip (EGS), and the COSMOS field.

\subsection{Demographics of high redshift galaxies}

The optical luminosity functions and stellar mass functions can be
measured out to $z\sim 1$--3. Beyond that redshift, only the most
highly star forming galaxies are easily visible, typically in the
rest-frame ultraviolet or at sub-mm wavelengths.

At all of these redshifts, the star forming main sequence is well
established. It may evolve in shape but mostly after $z\sim 2$ it
declines (\citealt{whitaker14a}).

Between $z\sim 3$ and $z\sim 1$, the inferred build up of total
stellar mass density is rapid, increasing by around a factor of around
5--10. At these redshifts, the red sequence and blue star forming
sequence of galaxies are already established. This growth is
accompanied by a growth in the fraction of red galaxies
(e.g. \citealt{mortlock15a}). The growth in the fraction of red
galaxies occurs at all stellar masses, but at the highest stellar
masses results in red galaxies dominating by $z\sim 1$. During this
same time period, the star formation rate density is at its highest,
experiencing a long plateau.

Between $z\sim 1$ and $z\sim 0$, the inferred build up of total
stellar mass density appears to nearly stall. The growth in the
fraction of red galaxies continues, again at all stellar masses. The
total star formation rate density declines sharply during this period,
falling by a factor of ten.

The highly star forming galaxies at $z\sim 3$ are unusual relative to
typical galaxies at low redshift. They are often photometrically and
kinematically irregular starburst galaxies. Meanwhile, the red
galaxies at these redshifts tend to be very compact relative to
similarly massive galaxies today; these galaxies probably become lower
density over time through minor mergers.

\section{Commentary}

Measurement and detection of galaxies at high redshift is still
extremely challenging. The surface brightness incompleteness effects
may be altering our view of high redshift in a manner not sufficiently
understood to correct for. Furthermore, inferences of total integrated
properties of the population like stellar mass density and star
formation rate often depend on extrapolation of increasing function to
low stellar mass or star formation ratios.

\section{Key References}

\begin{itemize}
  \item
    {\it Madau \& Dickinson}
  \item
    {\it CANDELS}
\end{itemize}

\citet{gunn06a}

%\section{Order-of-magnitude Exercises}
%
%\begin{enumerate} 
%\item Estimate the stellar mass surface density for a Milky Way like
%disk galaxy.
%\end{enumerate} 

\section{Analytic Exercises}

\begin{enumerate}
\item Growth by merging and density
\end{enumerate}

\section{Numerics and Data Exercises}

\begin{enumerate}
\item Lookback times at high redshift
\item Photometric redshifts
\end{enumerate}

\bibliographystyle{apj}
\bibliography{exex}  
