\title{\bf Stellar Dynamics}

\section{Basics \& Nomenclature}

Stellar dynamics is almost entirely collisionless, due to the low
number density of stars relative to their radii. It is therefore
governed by the collisionless Boltzmann equation (sometimes called the
Vlasov equation) acting under gravity.

In the continuum limit, we can express the distribution function of
stars in phase space as $f(\vec{x}, \vec{v}, t)$, in units of per
length-cubed per unit velocity-cubed. If we define the phase space
vector $\vec{w} = \{\vec{x}, \vec{v}\}$ then we can write $f(\vec{w},
t)$. The distribution in phase space can be arbitrarily
complicated. It will not be thermalized as in a gas or fluid, or obey
any particular equation of state.

The continuum limit will be violated most rapidly by two-body
interactions. We find in the exercises that the time for an $N$-body
system to ``relax'' due to this effect is:
\begin{equation}
t_{\rm relax} \sim \frac{0.1 N}{\ln N} t_{\rm cross}
\end{equation}
where $t_{\rm cross}$ is the crossing time of the system. Globular
clusters have relaxation times short compared to their ages.
Galaxies, over most of their extent, have relaxation times high
compared to their ages.

In the continuum limit, the system obeys the collisionless Boltzmann
equation under just gravity:
\begin{equation}
 \frac{\partial f}{\partial t} + \vec{v}\cdot\vec{\nabla} f -
\vec{\nabla}\Phi\cdot\frac{\partial f}{\partial \vec{v}} = 0
\end{equation}
It can be shown that this equation obeys a special case of Liouville's
Theorem:
\begin{equation}
\frac{{\rm d}f}{{\rm d}t} = 0
\end{equation}
where in this case the substantive derivative is:
\begin{equation}
\frac{\rm d}{{\rm d}t} = \frac{\partial}{\partial t}
+ \sum_\alpha \frac{\partial}{\partial w_\alpha}
\end{equation}

\section{Jeans Equations}

The first few moments collisionless Boltzmann equation are instructive
and can be useful. These equations are known as the Jeans Equations.

The zeroth moment yields the equation of continuity:
\begin{equation}
\frac{\partial n}{\partial t}
+ \vec{\nabla} \cdot\left(n \left\langle\vec{v}\right\rangle\right) =
0
\end{equation}
where $n(\vec{x})$ is the mean density per unit volume and
$\langle\rangle$ indicates a density weighted-mean over all
velocities.

The first moment yields something akin to Euler's equations:
\begin{equation}
n \frac{\partial \langle\vec{v}\rangle}{\partial t}
 + n \langle \vec{v} \rangle \cdot
 \vec{\nabla} \langle\vec{v}\rangle
  = -n \vec{\nabla}\Phi(\vec{x}, t) - \vec{\nabla} \cdot(n
 {\mathbf{\sigma^2}})
\end{equation}
where ${\mathbf{\sigma^2}}$ is the tensor second moment of the velocity
field, analogous to pressure. Each moment of the collisional Boltzman
equation involves terms of higher order in this fashion; whereas in a
collisional fluid the system would close with an equation of state, in
a collisionless system the equations never close. 

In steady state, where the density is not changing with time anywhere
and therefore the mean velocity is zero everywhere, we find:
\begin{equation}
 \vec{\nabla} \cdot(n
 {\mathbf{\sigma^2}}) = -n \vec{\nabla}\Phi(\vec{x}, t)
\end{equation}
Under spherical symmetry in configuration space, this can be
rewritten:
\begin{equation}
\frac{\partial(n \sigma_{rr}^2)}{\partial r}
+ \frac{n}{r} \left[2 \sigma_{rr}^2
- \left(\sigma_{\theta\theta}^2 + \sigma_{\phi\phi}^2\right)\right] =
- n \frac{\partial \Phi}{\partial r}
\end{equation}
Although the spherical symmetry in configuration space means that $f$
does not depend on $\theta$ or $\phi$, it can clearly depend on
$v_\theta$ and $v_\phi$. Thus, $\sigma_{rr}^2$ does not have to equal
$\sigma_{\theta\theta}^2$ pr $\sigma_{\phi\phi}^@$. The orbit
distribution can be anisotropic, and the degree anisotropy affects the
radial distribution $n(r)$. 

\section{Virial Theorem}

The virial equations establish the relationship between kinetic
and potential energy in collisionless gravitating systems. They are
obtained as a further moment of the Jeans Equation. Specifically, one
takes the first moment of position over the analog of Euler's
equation. For a time-independent system, in the center-of-mass frame,
we can establish the {\it tensor virial theorem}:
\begin{equation}
2 K_{jk} + W_{jk} = 0
\end{equation}
where the internal {\it kinetic energy tensor} is:
\begin{equation}
K_{jk} = \frac{1}{2} \int \dd{}^3\vec{x}^3 \rho \sigma_{jk}^2
\end{equation}
(where $\rho$ is the mass density, so for particles of equal mass $m$,
$\rho = nm$). The {\it potential energy tensor} is:
\begin{equation}
W_{jk} = - \frac{G}{2} \int \dd{}^3\vec{x}'
\dd{}^3\vec{x} \rho(\vec{x}') \rho(\vec{x}) \frac{(x_j' - x_j)(x_k' -
x_k)}{\left| \vec{x}' - \vec{x}\right|^3}
\end{equation}

The trace of the tensor virial theorem yields the {\it scalar virial
theorem}:
\begin{equation}
2K + W = 0
\end{equation}
where $K$ is the total kinetic energy in the center of mass frame, and
$W$ is the total potential energy. 

\section{Jeans Theorem}

Jeans Theorem yields an important tool for modeling equilibrium
self-gravitating systems. These systems can be described by the set of
orbits of the particles comprising them. The density field resulting
from this distribution of orbits generates a potential. To remain in
equilibrium, the orbit distribution needs to be stable in that
potential. Jeans Theorem yields a way of generating orbit
distributions that are self-consistent in this sense that they are in
equilibrium.

We showed earlier that $f$ is conserved along orbits in phase space:
\begin{equation}
\frac{\dd{f}}{\dd{t}} = 0
\end{equation}
Further, if $\Phi(\vec{x})$ is time-independent, there are six {\it
constants of motion} $C(\vec{x}, \vec{v}, t)$ conserved along the
orbit (there must be six because the orbit is fully defined by
$\vec{w}(t=0)$).

The {\it integrals of motion} are related. These are functions only of
phase space position that are conserved along orbits:
\begin{equation}
\frac{\dd{I(\vec{x}, \vec{v})}}{\dd{t}} = 0 
\end{equation}
Each is also a constant of the motion; so there are at most six of
them. The existence of these integrals of motion implies:
\begin{equation}
f(\vec{x}, \vec{v}) = f\left(I_1(\vec{x}, \vec{v}),
I_2(\vec{x}, \vec{v}), \ldots, I_6(\vec{x}, \vec{v})\right)
\end{equation}
If this were not the case, then $f$ would be an independent integral
of the motion itself!

An integral of motion that always exists is total energy. It is
conserved for each particle along its orbit. Under specific
symmetries, other useful integrals of motion exist. For example, in
spherical symmetry, the angular momentum $\vec{J}$ is conserved; note
that only its amplitude is physically significant in spherical
symmetry however. Therefore under spherical symmetry all equilibrium
distribution functions can be written as $f(E, J)$. 

A specific case of interest is the {\it isothermal sphere}. This
distribution results from the choice $f\propto \exp(-E/\sigma^2)$. The
resulting $f$ can be shown to have a velocity distribution that has a
Gaussian width $\sigma$ in each dimension. Generically, at large
radius $\rho \propto r^{-2}$ for an isothermal sphere. At these radii
the circular velocity $v_c = \sqrt{2}\sigma$. The case in which
$r^{-2}$ at all radii is known as the {\it singular isothermal
sphere}.

\section{Chandrasekhar Dynamical Friction}

Collisionless, gravitating, dynamical systems exhibit an effect known
as {\it dynamical friction} that converts ``bulk'' kinetic energy into
``internal'' kinetic energy, even in systems with long two-body
relaxation times. This effect is calculated in the exercises below,
where it is shown that for a mass $M$ moving through a system with
density $\rho$ with an isothermal distribution function of velocity
distribution $\sigma$ there is a drag force:
\begin{equation}
\frac{\dd{\vec{v}_M}}{\dd{t}} = - \frac{4\pi \ln\Lambda
G^2M\rho}{v_M^3} \left[\erf X
-  \frac{2X}{\sqrt{\pi}} \exp\left(-X^2\right) \right] \vec{v}_M
\end{equation}
where $X = v_M /\sqrt{2} \sigma$ and $\Lambda \sim M_{\rm total} / M$.
For an initial circular orbit of radius $r_i$, this drag leads to a
dynamical friction time scale:
\begin{equation}
t_{f} = \frac{2.6 \times
10^{11} \mathrm{~yr}}{\ln \Lambda} \left[\frac{r_i}{2\mathrm{~kpc}}\right]^2
\left[\frac{v_c}{250\mathrm{~km~s}^{-1}} \right]
\left[\frac{10^6 M_\odot}{M}\right]
\end{equation}



\section{Commentary}

The fact that collisionless systems have a non-trivial phase space is
of enormous significance. It provides another way that each objects'
history may be encoded in its dynamics. It also means that the
properties of a system have a full six-dimensional structure to their
description. This complexifies accurate predictions of $N$-body
gravitating systems when $N$ cannot be achieved computationally.

The virial theorem is often spoken of in the casual terms that
$v^2 \sim GM/r$ for the characteristic, $v$, $M$ and $r$ for the
system. While this relation follows from dimensional analysis alone,
the virial theorem goes further and is a precise relationship. It is
clearest to think of the virial theorem establishing the relationship
between $K$ and $U$ for a bound, equilibrium system.  However, as
described in the problems, one can create definitions of
``characteristic'' for $v$, $M$, and $r$ for which the equation $v^2 =
GM/r$ holds strictly.

\section{Important numbers}

\section{Key References}

\begin{itemize}
  \item
    \href{http://adsabs.harvard.edu/abs/2000asqu.book.....C}{
    {\it Binney \& Tremaine}
      \citet{cox00a}}, Chapter 5
\end{itemize}

\section{Order-of-magnitude Exercises}

\begin{enumerate} 
\item What is the typical relaxation time for globular clusters?
    Galaxies? Clusters of galaxies?
\end{enumerate}   

\section{Analytic Exercises}

\begin{enumerate}
\item Show that Liouville's Theorem follows from the collisionless
Boltzmann equation. 
\item Verify the expressions for the Jeans Equations. 
\item Verify the expressions for the Virial relation.
\item Define the characteristic v, M, R for virial theorem
\item Plummer model
\item Isothermal sphere model
\item Lowered isothermal sphere model
\item Tidal radius
\item Dynamical friction
\end{enumerate}

\section{Numerics and Data Exercises}

\begin{enumerate}

\item $\Phi$, $n$ and $\beta$
\item Dynamics estimates from Gaia
\item Globular cluster radial profiles.
\item Map Palomar 5 tidal tail
\end{enumerate}

\bibliographystyle{apj}
\bibliography{exex}  
