\title{\bf Structure Formation}

\section{Basics \& Nomenclature}

Within the overall cosmological growth, fluctuations grow through
gravitation. Inflationary theory predicts that these fluctuations
originate from quantum fluctuations frozen in as progressively larger
scales become causally disconnected in inflation. Cosmic microwave
background observations of temperature fluctuations find that at
recombination ($z\sim 1100$) the density fluctuations were
fractionally of order $10^{-5}$. These fluctuations grow first
linearly and then nonlinearly to form bound structures known as {\it
  dark matter haloes}. It is within these bound structures that
galaxies form.

At lower redshifts, we can define the matter fluctuations around the
homogenous density $\rho_0$:
\begin{equation}
\frac{\rho}{\rho_0} = 1+ \delta
\end{equation}
When it is considered in configuration space, $\delta$ is often
filtered on some scale $\gg 1$ kpc.  However, we often quantify the
two point statistics of this field using the power spectrum $P(k)$ of
$\delta$ and a corresponding correlation function $\xi(r)$. The
inflationary $\Lambda$CDM prediction for $P(k)$ is that during the era
of linear gravitational growth, on large scales (low $k$) its power
law slope is $n\sim 1$ and on small scales (high $k$) its power law
slope is $n\sim -3$. The turnover scale is associated with the horizon
size at matter-radiation equality, for reasons explored in the
exercises. We can characterize the overall second-order amplitude
fluctuations on any scale as:
\begin{equation}
\Delta(k) \sim k^3 P(k)
\end{equation}
which makes it clear that the strongest fluctuations are on the
smallest scales, a characteristic known as {\it hierarchical
clustering}. As shown below, $\Delta(k)$ will undergo a linear growth
phase at early times.  When $\Delta(k) \sim 1$, fluctuations on that
scale go nonlinear, and in general the growth rate of fluctuations
accelerates. Because smaller scales clearly go nonlinear first, this
process leads to a nonlinear power spectrum flatter than the linear
spectrum.

The overall amplitude is often quantified by $\sigma_8$, which is the
standard deviation of fluctuations in 8 $h^{-1}$ Mpc radius spheres,
which can also be expressed as an integral of $P(k)$. When the
equivalent quantity $\sigma_{8,g}$ for galaxies is measured in the
galaxy distribution, this quantity is expressed as the observed level
of fluctuations, consequently includes the nonlinear effects present
in the real universe. When $\sigma_8$ of the matter is inferred from
cosmological observations (the cosmic microwave background, or
gravitational lensing, or redshift space distortions) it is usually
defined as the primordial $\sigma_8$ linearly evolved to $z=0$ or the
redshift in question.

In galaxy surveys, $\delta$ is not directly observable, but the
overdensity $\delta_g$ of some particular class of galaxies can be. On
large scales, where $\delta\ll 1$, often it is sufficient to
approximate the relationship between the two with a {\it linear, local
galaxy bias}:
\begin{equation}
\delta_g(\vec{x}) \approx b \delta(\vec{x})
\end{equation}
On small scales this relationship cannot remain linear and in general
cannot be local either. Bias can alternatively be defined as
$\sigma_{8,g} / \sigma_8$ (or equivalent statistical quantities on
larger scales). In general the halo occupation distribution model is a
more accurate description of the relationship between galaxies and
matter, but the concept of galaxy bias as defined here is still
useful, especially on linear scales.

To understand the linear growth, we start with the equations of motion
for a pressureless, gravitating fluid:
\begin{eqnarray}
\frac{\mathrm{D}\vec{v}}{\mathrm{D}t} &=& - \vec\nabla\phi
\mathrm{\quad (Euler's~equation)}\cr
\frac{\mathrm{D}\rho}{\mathrm{D}t} &=& - \rho \vec\nabla\cdot \vec{v}
\mathrm{\quad (Continuity~equation)}\cr
\nabla^2\phi &=& 4\pi G \rho 
\mathrm{\quad (Poisson's~equation)}
\end{eqnarray}
where the convective derivative is:
\begin{equation}
\frac{\mathrm{D}}{\mathrm{D}t} = \frac{\partial}{\partial t} +
\vec{v}\cdot\vec{\nabla}
\end{equation}

For the $\Omega_m = 1$ case, we can show that:
\begin{equation}
a(t) = \left(\frac{t}{t_0}\right)^{2/3}
\end{equation} 
which if we use to construct a homogeneous solution to the above
equations, we can perturb the density around the homogenous density
$\rho_0(a) \propto a^{-3}$:
\begin{equation}
\frac{\rho}{\rho_0} = 1+ \delta
\end{equation}
and find the continuity equation holds for peculiar velocities:
\begin{equation}
\label{eq:continuity}
\frac{\mathrm{d}\delta}{\mathrm{d}t} =  
- \vec\nabla\cdot\vec{v}_p
\end{equation}
In the perturbed quantities we find:
\begin{eqnarray}
\frac{\mathrm{d}\vec{v}_p}{\mathrm{d}t} &=&
- \vec\nabla(\delta\phi)
- H(t) \vec{v}_p \cr
\nabla^2\left(\delta\phi\right)&=& 4\pi G \rho_0 \delta \cr
\frac{\mathrm{d}}{\mathrm{d}t} \left[ \vec\nabla\cdot\vec{v}_p \right]
&=& 
\vec\nabla\cdot
\left[\frac{\mathrm{d}\vec{v}_p}{\mathrm{d}t} \right]
+ H(t) \frac{\mathrm{d}\delta}{\mathrm{d}t} 
\end{eqnarray}
which can be combined into a second-order equation for the density:
\begin{equation}
\frac{\mathrm{d}^2\delta}{\mathrm{d}t^2} + 2 \frac{\dot a}{a} 
\frac{\mathrm{d}\delta}{\mathrm{d}t} - 4 \pi G \rho_0 \delta = 0 
\end{equation}
This linear set of equations is separable, so that whatever spatial
pattern exists simply changes in amplitude over time:
\begin{equation}
\delta(x, t) = \delta(x, t_0) \frac{D(t)}{D(t_0)}
\end{equation}
  The general solution is:
\begin{equation}
D(t) = A t^{-1} + B t^{2/3}
\end{equation}
The first mode is decaying, and thus not important to the growth
of structure.  The second mode is the one that contributes to the
growth of structure.

This set of solutions is appropriate for the zero-energy, or ``flat''
Universe, without a cosmological constant. At early times (but after
recombination), while deceleration dominates the dynamics, it is a
very good description of the Universe.  However, at later times it
becomes less good. In particular, in our Universe, which appears to be
accelerating, the growth is slowed down considerably by the
acceleration. 

The continuity equation (\ref{eq:continuity}) and linear growth imply
a relationship between the peculiar velocity field and the growth
rate:
\begin{equation}
\vec\nabla\cdot\vec{v}_p = - a \delta(\vec{x}) \frac{\dot
D(t)}{D(t_0)}
\end{equation}
where:
\begin{equation}
f = \frac{\dot D}{D} \approx \Omega_m^{0.6}
\end{equation}
This peculiar velocity field distorts redshift-based maps of the
universe in a specific way on large scales, that can be measured to
constrain $f$. Since $\delta$ is not directly observable, the directly
observable quantity on linear scales is $\beta = f/b$. Since the
fluctuations in the galaxy sample can be observed, we can recast
$\beta = f \sigma_8 / \sigma_{8,g}$ and the observable is
$\beta \sigma_{8,g}$, from which we infer $f\sigma_8$.

As small scales go nonlinear, gravitationally bound objects will
form. This process can be approximated in the spherical case. If we
situate our coordinate system on the center of a spherical system with
a constant overdensity $\bar\delta > 0$ and size $R$, the system can
be considered completely analogous to a universe with matter density
of $\Omega_m(1+\bar\delta)$. Therefore, if this quantity is greater
than unity, than the sphere will expand for some time, then turn
around at $t=t_{\rm TA}$, and then collapse on itself; this process
can be followed exactly. It can be shown that the mean density of the
sphere at turn-around is about 5.5 times the mean density of the
universe, and collapse occurs in twice the turn-around time. The
virial theorem and energy conservation lead to a typical overdensity
of the collapsed object within its virial radius of $18\pi^2 \approx
178$. Meanwhile, the linearly extrapolated overdensity at that time is
only about $\delta_{\rm linear}\approx 1.7$.

The mass spectrum of collapsed halos can be predicted approximately
using {\it excursion set theory}, or the {\it Press-Schechter}
approach. Imagine a patch of mass $M$ at early times; it will have
some specific radius $R$ depending on the mean density. We can predict
when it will collapse to a virialized object when $\delta_{\rm
linear} \approx 1.7$ within radius $R$. At any given time, we can ask
what fraction of the universe's volume, when smoothed on radius $R$,
has $\delta_{\rm linear} > 1.7$. For simplicity, we will smooth by a
top-hat in $k$-space (in configuration space this is smothing by the
first order spherical Bessel function $j_1$). Calculating this
fraction tells us for any mass (that is, smoothing scale), what
fraction of the volume ends up in dark matter halos greater than that
mass. This function can be differentiated to yield the halo mass
function:
\begin{equation}
\Phi(M) \dd M
= \frac{1}{\sqrt{2\pi}} \frac{\bar \rho}{M} \frac{\delta_c}{\sigma^2(M)} \left[
- \frac{\dd \sigma^2}{\dd
M}\right] \exp\left[-\frac{\delta_c^2}{2\sigma^2}\right] \dd M
\end{equation}
For $P(k)\propto k^n$, one can show:
\begin{equation}
\Phi(M) = \frac{\bar\rho}{2\pi
M} \left(\frac{M}{M_\ast}\right)^{(n+3)/6}
\left(\frac{n+3}{3}\right)
\exp\left[-\frac{1}{2}\left(\frac{M}{M_\ast}\right)^{(n+3)/3}\right] \frac{\dd
M}{M}
\end{equation}
Where the nonlinear mass $M_\ast$ is defined by the relation:
\begin{equation}
\sigma^2 = \left(\frac{M}{M_\ast}\right)^{-(n+3)/3} \delta_c^2.
\end{equation}
Because $n>-3$ always, as $\sigma^2$ grows with time, the nonlinear
mass scale grows.  In the standard cosmology, at low small scales (and
thus low masses) $n$ slowly approaches $-3$ from above and
$\Phi(M) \propto M^{-2 + \epsilon}$, where $\epsilon = (n+3) /3$, and
thus is almost divergent.

The detailed prediction of nonlinear growth and collapse to dark
matter halos requires the use of simulations. Because the dark matter
is collisionless, fluid simulations are not sufficient. The universal
approach is to model the dark matter statistically using a large
number of collisionless particles; the $N$-body approximation. An {\it
N-body simulation} is understood to model only the dark matter. These
simulations use some variant of particle-mesh techniques on large
scales, often with an adaptive component on small scales that may use
direct calculations of mutual forms. They invariably employ some
softening length that is reported as the resolution. {\it
Hydrodynamic} simulations include baryonic fluids in the modeling, and
often their cooling and collapse to stellar systems. They may also
include feedback of supernovae, winds, and active galactic nuclei on
the fluid; this {\it subgrid physics} is typically parameterized in a
simple way.

\section{Commentary}


\section{Key References}

\begin{itemize}
  \item
    {\it Les Houches}
\end{itemize}

\citet{gunn06a}

\section{Order-of-magnitude Exercises}

\begin{enumerate} 
\item a
\end{enumerate} 

\section{Analytic Exercises}

\begin{enumerate}
\item Origin of turnover in $P(k)$
\item Linear growth solutions
\item Spherical collapse solutions
\item Press-Schechter
\item Bias from PS
\end{enumerate}

\section{Numerics and Data Exercises}

\begin{enumerate}
\item $P(k)$ estimates from CAMB, etc
\item Simple PM
\end{enumerate}

\bibliographystyle{apj}
\bibliography{exex}  
