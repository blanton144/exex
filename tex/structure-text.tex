\title{\bf Structure Formation}

\section{Basics \& Nomenclature}

Within the overall cosmological growth, fluctuations grow through
gravitation. Inflationary theory predicts that these fluctuations
originate from quantum fluctuations frozen in as progressively larger
scales become causally disconnected in inflation. Cosmic microwave
background observations of temperature fluctuations find that at
recombination ($z\sim 1100$) the density fluctuations were
fractionally of order $10^{-5}$. These fluctuations grow first
linearly and then nonlinearly to form bound structures known as {\it
  dark matter haloes}. It is within these bound structures that
galaxies form.

At lower redshifts, we can define the matter fluctuations around the
homogenous density $\rho_0$:
\begin{equation}
\frac{\rho}{\rho_0} = 1+ \delta
\end{equation}
When it is considered in configuration space, $\delta$ is often
filtered on some scale $\gg 1$ kpc.  However, we often quantify the
two point statistics of this field using the power spectrum $P(k)$ of
$\delta$ and a corresponding correlation function $\xi(r)$.

The power spectrum is defined as:
\begin{equation}
\left\langle \tilde\delta(\vec{k}) \tilde\delta(\vec{k}') \right\rangle
= (2\pi)^3 \delta_D\left(\vec{k} - \vec{k}'\right) P(k).
\end{equation}
In plain language, the power spectrum is the variance in the
amplitudes of the Fourier mode amplitudes as a function of wavenumber
$k$. The Fourier transform of the power spectrum is the correlation
function:
\begin{equation}
\left\langle\delta\left(\vec{x}\right) \delta\left(\vec{x}
+ \vec{r} \right) \right\rangle = \xi(r).
\end{equation}
In plain language, the correlation function is the excess probability
of finding a pair of galaxies with separation $r$, above the
probability for a spatially uniform Poisson distribution with the same
number density of galaxies.

The inflationary $\Lambda$CDM prediction for $P(k)$ is that during the
era of linear gravitational growth, on large scales (low $k$) its
power law slope is $n\sim 1$ and on small scales (high $k$) its power
law slope is $n\sim -3$ (e.g. \citealt{bardeen86a}; Appendix G). The
turnover scale is associated with the horizon size at matter-radiation
equality, for reasons explored in the exercises. We can characterize
the overall second-order amplitude fluctuations on any scale as:
\begin{equation}
\Delta(k) \sim k^3 P(k)
\end{equation}
which makes it clear that the strongest fluctuations are on the
smallest scales, a characteristic known as {\it hierarchical
clustering}. As shown below, $\Delta(k)$ will undergo a linear growth
phase at early times.  When $\Delta(k) \sim 1$, fluctuations on that
scale go nonlinear, and in general the growth rate of fluctuations
accelerates. Because smaller scales clearly go nonlinear first, this
process leads to a nonlinear power spectrum flatter than the linear
spectrum.

The overall amplitude is often quantified by $\sigma_8$, which is the
standard deviation of fluctuations in 8 $h^{-1}$ Mpc radius spheres,
which can also be expressed as an integral of $P(k)$. When the
equivalent quantity $\sigma_{8,g}$ for galaxies is measured in the
galaxy distribution, this quantity is expressed as the observed level
of fluctuations, and consequently includes the nonlinear effects
present in the real universe. When $\sigma_8$ of the matter is
inferred from cosmological observations (the cosmic microwave
background, or gravitational lensing, or redshift space distortions)
it is usually defined as the primordial $\sigma_8$ linearly evolved to
$z=0$ or the redshift in question.

In galaxy surveys, $\delta$ is not directly observable, but the
overdensity $\delta_g$ of some particular class of galaxies can be. On
large scales, where $\delta\ll 1$, often it is sufficient to
approximate the relationship between the two with a {\it linear, local
galaxy bias}:
\begin{equation}
\delta_g(\vec{x}) \approx b \delta(\vec{x})
\end{equation}
On small scales this relationship cannot remain linear and in general
cannot be local either. Bias can alternatively be defined as
$\sigma_{8,g} / \sigma_8$ (or equivalent statistical quantities on
larger scales). In general the halo occupation distribution model is a
more accurate description of the relationship between galaxies and
matter, but the concept of galaxy bias as defined here is still
useful, especially on linear scales.

To understand the linear growth, we start with the equations of motion
for a pressureless, gravitating fluid:
\begin{eqnarray}
\label{eq:homogeneous}
\frac{\mathrm{D}\vec{v}}{\mathrm{D}t} &=& - \vec\nabla\phi
\mathrm{\quad (Euler's~equation)}\cr
\frac{\mathrm{D}\rho}{\mathrm{D}t} &=& - \rho \vec\nabla\cdot \vec{v}
\mathrm{\quad (Continuity~equation)}\cr
\nabla^2\phi &=& 4\pi G \rho 
\mathrm{\quad (Poisson's~equation)}
\end{eqnarray}
where the convective derivative is:
\begin{equation}
\frac{\mathrm{D}}{\mathrm{D}t} = \frac{\partial}{\partial t} +
\vec{v}\cdot\vec{\nabla}
\end{equation}
Here and below $\nabla$ refers to a spatial derivative in physical
units (not comoving units).

For the $\Omega_m = 1$ case, we can show that:
\begin{equation}
a(t) = \left(\frac{t}{t_0}\right)^{2/3}
\end{equation} 
which if we use to construct a homogeneous solution to the above
equations, we can perturb the density around the homogenous density
$\rho_0(a) \propto a^{-3}$:
\begin{equation}
\frac{\rho}{\rho_0} = 1+ \delta
\end{equation}
We will also make use of the time derivative with respect to a
comoving observer, which is related to the convective derivative by:
\begin{equation}
\frac{\mathrm{D}}{\mathrm{D}t} = \frac{\mathrm{d}}{\mathrm{d} t} +
\vec{v}_p\cdot\vec{\nabla}
\end{equation}
Here we choose to keep the peculiar velocity in physical units, and
the derivative $\nabla$ in physics units (not comoving units).  We can
show the continuity equation holds for peculiar velocities:
\begin{equation}
\label{eq:continuity}
\frac{\mathrm{d}\delta}{\mathrm{d}t} =  
- \vec\nabla\cdot\vec{v}_p
\end{equation}
In the perturbed quantities we find to linear order:
\begin{eqnarray}
\frac{\mathrm{d}\vec{v}_p}{\mathrm{d}t} &=&
- \vec\nabla(\delta\phi)
- H(t) \vec{v}_p \cr
\nabla^2\left(\delta\phi\right)&=& 4\pi G \rho_0 \delta
\end{eqnarray}
Remembering that the spatial derivatives are in physical, not comoving
units, we can write:
\begin{equation}
\frac{\mathrm{d}}{\mathrm{d}t} \left[ \vec\nabla\cdot\vec{v}_p \right]
= 
\vec\nabla\cdot
\left[\frac{\mathrm{d}\vec{v}_p}{\mathrm{d}t} \right]
+ H(t) \frac{\mathrm{d}\delta}{\mathrm{d}t} 
\end{equation}
Then one can take a time derivative with respect to a comoving
observer to Equation \ref{eq:continuity}, and with substitutions this
leads to a second-order equation for the density:
\begin{equation}
\label{eq:secondorder}
\frac{\mathrm{d}^2\delta}{\mathrm{d}t^2} + 2 \frac{\dot a}{a} 
\frac{\mathrm{d}\delta}{\mathrm{d}t} - 4 \pi G \rho_0 \delta = 0 
\end{equation}
This linear set of equations is separable, so that whatever spatial
pattern exists simply changes in amplitude over time:
\begin{equation}
\delta(x, t) = \delta(x, t_0) \frac{D(t)}{D(t_0)},
\end{equation}
where often the convention is $D(t_0) = 1$.
  The general solution is:
\begin{equation}
\label{eq:lineargrowth}
D(t) = A t^{-1} + B t^{2/3}
\end{equation}
The first mode is decaying, and thus not important to the growth
of structure.  The second mode is the one that contributes to the
growth of structure.

This set of solutions is appropriate for the zero-energy, or ``flat''
Universe, without a cosmological constant, when matter density (rather
than radiation) dominates. At early times (but after matter-radiation
equality), while deceleration dominates the dynamics, it is a good
description of the Universe.  However, at later times it becomes less
good. In particular, in our Universe, which appears to be
accelerating, the growth is slowed down considerably by the
acceleration.

The continuity equation (\ref{eq:continuity}) and linear growth imply
a relationship between the peculiar velocity field and the growth
rate. If we convert the spatial derivative to comoving units we find:
\begin{equation}
\frac{1}{a} \vec\nabla_c\cdot\vec{v}_p = - \delta(\vec{x}, t_0) \dot
D(t),
\end{equation}
(taking $D(t_0) = 1$), which we can rewrite as:
\begin{equation}
\vec\nabla_c\cdot\vec{v}_p = - a \delta(\vec{x}, t_0) \dot  
D(t) = - a \delta(\vec{x}, t_0) H f
\end{equation}
where the {\it growth rate} is:
\begin{equation}
f = \frac{\dd \ln D}{\dd \ln a}
\end{equation}
The choice to express this result in terms of the comoving spatial
derivative of the physical peculiar velocity is strange but
conventional.

This peculiar velocity field distorts redshift-based maps of the
universe in a specific way on large scales, that can be measured to
constrain $f$. Since $\delta$ is not directly observable, the directly
observable quantity on linear scales is $\beta = f/b$. Since the
fluctuations in the galaxy sample can be observed, we can recast
$\beta = f \sigma_8 / \sigma_{8,g}$ and the observable is
$\beta \sigma_{8,g}$, from which we infer $f\sigma_8$.

As small scales go nonlinear, gravitationally bound objects will
form. This process can be approximated in the spherical case. If we
situate our coordinate system on the center of a spherical system with
a constant overdensity $\bar\delta > 0$ and size $R$, the system can
be considered completely analogous to a universe with matter density
of $\Omega_m(1+\bar\delta)$. Therefore, if this quantity is greater
than unity, than the sphere will expand for some time, then turn
around at $t=t_{\rm TA}$, and then collapse on itself; this process
can be followed exactly. It can be shown that the mean density of the
sphere at turn-around is about 5.5 times the mean density of the
universe, and collapse occurs in twice the turn-around time. The
virial theorem and energy conservation lead to a typical overdensity
of the collapsed object within its virial radius of $18\pi^2 \approx
178$. Meanwhile, the linearly extrapolated overdensity at that time is
only about $\delta_{\rm linear}\approx 1.7$.

The mass spectrum of collapsed halos can be predicted approximately
using {\it excursion set theory}, or the {\it Press-Schechter}
approach (\citealt{press74a, bond91a, lacey93a}). Imagine a patch of
mass $M$ at early times; it will have some specific radius $R$
depending on the mean density. We can predict when it will collapse to
a virialized object when $\delta_{\rm linear} \approx 1.686$ within
radius $R$. At any given time, we can ask what fraction of the
universe's volume, when smoothed on radius $R$, has $\delta_{\rm
linear} > 1.686$. For simplicity, we will smooth by a top-hat in
$k$-space (in configuration space this is smothing by the first order
spherical Bessel function $j_1$). Calculating this fraction tells us
for any mass (that is, smoothing scale), what fraction of the volume
ends up in dark matter halos greater than that mass. This function can
be differentiated to yield the halo mass function:
\begin{equation}
\Phi(M) \dd M
= \frac{1}{\sqrt{2\pi}} \frac{\bar \rho}{M} \frac{\delta_c}{\sigma^2(M)} \left[
- \frac{\dd \sigma^2}{\dd
M}\right] \exp\left[-\frac{\delta_c^2}{2\sigma^2}\right] \dd M
\end{equation}
For $P(k)\propto k^n$, one can show:
\begin{equation}
\Phi(M) = \frac{\bar\rho}{2\pi
M} \left(\frac{M}{M_\ast}\right)^{(n+3)/6}
\left(\frac{n+3}{3}\right)
\exp\left[-\frac{1}{2}\left(\frac{M}{M_\ast}\right)^{(n+3)/3}\right] \frac{\dd
M}{M}
\end{equation}
Where the nonlinear mass $M_\ast$ is defined by the relation:
\begin{equation}
\sigma^2 = \left(\frac{M}{M_\ast}\right)^{-(n+3)/3} \delta_c^2.
\end{equation}
Because $n>-3$ always, as $\sigma^2$ grows with time, the nonlinear
mass scale grows.  In the standard cosmology, at low small scales (and
thus low masses) $n$ slowly approaches $-3$ from above and
$\Phi(M) \propto M^{-2 + \epsilon}$, where $\epsilon = (n+3) /3$, and
thus is almost divergent.

The detailed prediction of nonlinear growth and collapse to dark
matter halos requires the use of simulations. Because the dark matter
is collisionless, fluid simulations are not sufficient. The universal
approach is to model the dark matter statistically using a large
number of collisionless particles; the $N$-body approximation. An {\it
N-body simulation} is understood to model only the dark matter. These
simulations use some variant of particle-mesh techniques on large
scales, often with an adaptive component on small scales that may use
direct calculations of mutual forms. They invariably employ some
softening length that is reported as the resolution.  {\it
Hydrodynamic} simulations include baryonic fluids in the modeling, and
often their cooling and collapse to stellar systems. They may also
include feedback of supernovae, winds, and active galactic nuclei on
the fluid; this {\it subgrid physics} is typically parameterized in a
simple way.

An important insight from N-body simulations is how halos grow through
accretion of smaller companion halos. These accreted halos often
survive for long periods of time, and are therefore distinct clumps
known as {\it subhalos} within each halo. The centers of halos and
subhalos are the locations where galaxies form.

\section{Commentary}


\section{Key References}

\begin{itemize}
  \item
    {\it Physics Foundations of Cosmology},
    \citet{mukhanov05a}
  \item
    {\it The large-scale structure of the
    universe}, \citet{peebles80a}
  \item
    {\it Formation and Evolution of Galaxies: Les Houches
    Lectures}, \citet{white94a} 
\end{itemize}

\section{Order-of-magnitude Exercises}

\begin{enumerate} 
\item At approximately what redshift does structure growth start to
    slow down for a Universe with $\Omega_m = 0.3$, $\Omega_\Lambda = 0.7$?
\end{enumerate} 

\section{Analytic Exercises}

\begin{enumerate}
\item We can understand the turnover in $P(k)$ very simply once we
understand how growth proceeds for $k$ modes inside and outside the
particle horizon. Here we give the picture in the ``synchronous
gauge''; the heuristic narrative outside the horizon depends on gauge,
but the observables do not (\citealt{ma95a}). The following results in
this case regarding density perturbation growth allow us to
characterize the tranfer function $T(k)$ which modifies the density
field modes.  During radiation domination, inside the horizon the
density only grows logarithmically (because the Jeans scale is nearly
the horizon size for a relativistic fluid) and outside the horizon the
density grows as $\delta\propto a^2$. During matter domination, at all
scales $\delta\propto a$.
\begin{enumerate}
\item For a mode $k$ that enters the horizon before matter dominates,
how does the time it spends outside the horizon scale with $k$? 
\item How does the factor growth experienced outside the horizon scale
with $k$?
\item How is an initial $P(k) \propto k$ modified after the universe
has entered the matter-dominated phase?
\end{enumerate}
\item Starting from the continuity equation in Equation
   \ref{eq:homogeneous}, assuming a flat matter dominated universe
    ($\Omega_m = 1$), and keeping only first-order terms, derive
    Equation \ref{eq:continuity}.
\item Starting from Equation \ref{eq:homogeneous}, and assuming a flat
    matter dominated universe ($\Omega_m = 1$), derive
    Equation \ref{eq:secondorder}.
\item Show that Equation \ref{eq:lineargrowth} solves
Equation \ref{eq:secondorder}.
\item Consider a spherical region with mean overdensity $\bar\delta
>0$, within an expanding universe with no cosmological constant. As
 long as there is no {\it shell crossing} --- that is, material at one
 radius does not catch up to material at another radius --- the
 equations governing the radius of this sphere over time are
 \begin{equation}
\frac{\dd^2 R}{\dd t^2} = - \frac{GM(<r)}{R^2} = - \frac{4\pi
 G}{3} \bar\rho(1+\bar\delta) R
 \end{equation}
\begin{enumerate}
\item In terms of $\Omega_m$ at the present time, what is the
condition that the spherical region will collapse on itself?

\begin{answer}
This region will evolve the same way as a universe with
matter density $\Omega_m(1+\bar\delta)$. The condition for a closed
universe then implies $\Omega_m(1+\bar\delta)>1$ or
\begin{equation}
\bar\delta > \frac{1}{\Omega} -1.
\end{equation}
\end{answer}

\item Demonstrate that the solutions to the above equation can be
expressed as:
\begin{eqnarray}
\label{eq:sphericalcollapse}
\frac{R}{R_{\rm max}} &=& \frac{1}{2}\left(1 - \cos\eta\right), \cr
\frac{t}{t_{\rm max}} &=& \frac{1}{\pi}\left(\eta - \sin\eta\right)
\end{eqnarray}
where at time $t_{\rm max}$ the sphere reaches its maximum radius of
expansion $R_{\rm max}$, before collapsing. 

\item Show that at time $t_{\rm max}$, the density of the sphere
relative to the mean density of the universe will be $\rho_{\rm max}
/ \bar\rho(t_{\rm max}) = 9 \pi^2 / 16 \approx 5.5$.

\item The collapse of the sphere will proceed in reverse, and will
therefore take $t_{\rm max}$ to do so. However, upon full collapse
shell-crossing will occur, because the collisionless dark matter will
pass through the origin and oscillate around it. This process can be
modeled (\citealt{bertschinger85a, lithwick11a}) to derive the detailed
structure of the resulting halo mass profile, but the virial theorem
($U=-2K$) can tell us about its overall size. Show that the final
characteristic radius of the resulting {\it virialized} halo is
$R_{\rm vir} = R_{\rm max} / 2$.
\item Show that the mean overdensity within the resulting halo  is
$\delta_{\rm vir} = 18\pi^2 \approx 178$. 
\item By linearizing the Equations \ref{eq:sphericalcollapse}, show
that the linearly extrapolated overdensity at the time of collapse is
$\delta_{\rm lin}(2 t_{\rm max}) \approx 1.686$.
\end{enumerate}
\item The Press-Schechter or excursion set estimate of the halo mass
function can be calculated from the statistics of Gaussian random
fields. We can ask what fraction of the volume in the nearly-uniform
early universe ends up in halos of a given mass.  Consider the density
field linearly-evolved to some redshift $z$.
\begin{enumerate}
\item If we smooth the density field on some characteristic scale $R$,
the smoothed density field will relate to the the statistics of halos
of what mass $M$?
\item If the smoothing is performed as a top-hat function in
$k$-space, what does that smoothing corrrespond to in real space?
\item In terms of the power spectrum, what is the variance
$\sigma^2(M)$?
\item Assume that locations above some linearly-evolved
overdensity $\delta_c \sim 1.686$ on scale $R$ or larger have in fact
collapsed into halos of the corresponding mass $M$ or larger. What
fraction $F(>M)$ of the volume has done so (express in terms of $\delta_c$ and
$\sigma(M)$)?
\item Derive from $F(>M)$ the mass function of halos $\Phi(M)$. 
\item Assume $P(k)\propto k^n$. Define the nonlinear mass $M_\ast$:
\begin{equation}
\sigma^2 = \left(\frac{M}{M_\ast}\right)^{-(n+3)/3}.
\end{equation}
Write $\Phi(M)$  in terms of $M_\ast$, $\bar\rho$, and $n$. What
happens as $n\rightarrow -3$, as it does at small scales?
\end{enumerate}
\item Argue why the excursion set approach should lead to the
prediction that dense regions on large scales should have more and
larger dark matter halos.
\end{enumerate}

\section{Numerics and Data Exercises}

\begin{enumerate}
\item Download and install CAMB, the standard code to calculate the
power spectrum. Plot the linear $P(k)$, for $\Omega_m= 0.1$, $0.3$ and
1 (assume $h=0.7$). Examine the dependence on baryon density by
doubling it from the standard value; the wiggles you see getting
stronger are due to the baryon acoustic oscillation.
\end{enumerate}

\bibliographystyle{apj}
\bibliography{exex}  
