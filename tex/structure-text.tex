\title{\bf Structure Formation}

\section{Basics \& Nomenclature}

Within the overall cosmological growth, fluctuations grow through
gravitation. Inflationary theory predicts that these fluctuations
originate from quantum fluctuations frozen in as progressively larger
scales become causally disconnected in inflation. Cosmic microwave
background observations of temperature fluctuations find that at
recombination ($z\sim 1100$) the density fluctuations were
fractionally of order $10^{-5}$. These fluctuations grow first
linearly and then nonlinearly to form bound structures known as {\it
  dark matter haloes}. It is within these bound structures that
galaxies form.

At lower redshifts, we can define the matter fluctuations around the
homogenous density $\rho_0$:
\begin{equation}
\frac{\rho}{\rho_0} = 1+ \delta
\end{equation}
When it is considered in configuration space, $\delta$ is often
filtered on some scale $\gg 1$ kpc.  However, we often quantify the
two point statistics of this field using the power spectrum $P(k)$ of
$\delta$ and a corresponding correlation function $\xi(r)$.

The power spectrum is defined as:
\begin{equation}
\left\langle \tilde\delta(\vec{k}) \tilde\delta(\vec{k}') \right\rangle
= (2\pi)^3 \delta_D\left(\vec{k} - \vec{k}'\right) P(k).
\end{equation}
In plain language, the power spectrum is the variance in the
amplitudes of the Fourier mode amplitudes as a function of wavenumber
$k$. The Fourier transform of the power spectrum is the correlation
function:
\begin{equation}
\left\langle\delta\left(\vec{x}\right) \delta\left(\vec{x}
+ \vec{r} \right) \right\rangle = \xi(r).
\end{equation}
In plain language, the correlation function is the excess probability
of finding a pair of galaxies with separation $r$, above the
probability for a spatially uniform Poisson distribution with the same
number density of galaxies.

The inflationary $\Lambda$CDM prediction for $P(k)$ is that during the
era of linear gravitational growth, on large scales (low $k$) its
power law slope is $n\sim 1$ and on small scales (high $k$) its power
law slope is $n\sim -3$ (e.g. \citealt{bardeen86a}; Appendix G). The
turnover scale is associated with the horizon size at matter-radiation
equality, for reasons explored in the exercises. We can characterize
the overall second-order amplitude fluctuations on any scale as:
\begin{equation}
\Delta(k) \sim k^3 P(k)
\end{equation}
which makes it clear that the strongest fluctuations are on the
smallest scales, a characteristic known as {\it hierarchical
clustering}. As shown below, $\Delta(k)$ will undergo a linear growth
phase at early times.  When $\Delta(k) \sim 1$, fluctuations on that
scale go nonlinear, and in general the growth rate of fluctuations
accelerates. Because smaller scales clearly go nonlinear first, this
process leads to a nonlinear power spectrum flatter than the linear
spectrum.

The overall amplitude is often quantified by $\sigma_8$, which is the
standard deviation of fluctuations in 8 $h^{-1}$ Mpc radius spheres,
which can also be expressed as an integral of $P(k)$. When the
equivalent quantity $\sigma_{8,g}$ for galaxies is measured in the
galaxy distribution, this quantity is expressed as the observed level
of fluctuations, and consequently includes the nonlinear effects
present in the real universe. When $\sigma_8$ of the matter is
inferred from cosmological observations (the cosmic microwave
background, or gravitational lensing, or redshift space distortions)
it is usually defined as the primordial $\sigma_8$ linearly evolved to
$z=0$ or the redshift in question.

In galaxy surveys, $\delta$ is not directly observable, but the
overdensity $\delta_g$ of some particular class of galaxies can be. On
large scales, where $\delta\ll 1$, often it is sufficient to
approximate the relationship between the two with a {\it linear, local
galaxy bias}:
\begin{equation}
\delta_g(\vec{x}) \approx b \delta(\vec{x})
\end{equation}
On small scales this relationship cannot remain linear and in general
cannot be local either. Bias can alternatively be defined as
$\sigma_{8,g} / \sigma_8$ (or equivalent statistical quantities on
larger scales). In general the halo occupation distribution model is a
more accurate description of the relationship between galaxies and
matter, but the concept of galaxy bias as defined here is still
useful, especially on linear scales.

To understand the linear growth, we start with the equations of motion
for a pressureless, gravitating fluid:
\begin{eqnarray}
\label{eq:homogeneous}
\frac{\mathrm{D}\vec{v}}{\mathrm{D}t} &=& - \vec\nabla\phi
\mathrm{\quad (Euler's~equation)}\cr
\frac{\mathrm{D}\rho}{\mathrm{D}t} &=& - \rho \vec\nabla\cdot \vec{v}
\mathrm{\quad (Continuity~equation)}\cr
\nabla^2\phi &=& 4\pi G \rho 
\mathrm{\quad (Poisson's~equation)}
\end{eqnarray}
where the convective derivative is:
\begin{equation}
\frac{\mathrm{D}}{\mathrm{D}t} = \frac{\partial}{\partial t} +
\vec{v}\cdot\vec{\nabla}
\end{equation}
Here and below $\nabla$ refers to a spatial derivative in physical
units (not comoving units).

For the $\Omega_m = 1$ case, we can show that:
\begin{equation}
a(t) = \left(\frac{t}{t_0}\right)^{2/3}
\end{equation} 
which if we use to construct a homogeneous solution to the above
equations, we can perturb the density around the homogenous density
$\rho_0(a) \propto a^{-3}$:
\begin{equation}
\frac{\rho}{\rho_0} = 1+ \delta
\end{equation}
We will also make use of the time derivative with respect to a
comoving observer, which is related to the convective derivative by:
\begin{equation}
\frac{\mathrm{D}}{\mathrm{D}t} = \frac{\mathrm{d}}{\mathrm{d} t} +
\vec{v}_p\cdot\vec{\nabla}
\end{equation}
Here we choose to keep the peculiar velocity in physical units, and
the derivative $\nabla$ in physics units (not comoving units).  We can
show the continuity equation holds for peculiar velocities:
\begin{equation}
\label{eq:continuity}
\frac{\mathrm{d}\delta}{\mathrm{d}t} =  
- \vec\nabla\cdot\vec{v}_p
\end{equation}
In the perturbed quantities we find to linear order:
\begin{eqnarray}
\frac{\mathrm{d}\vec{v}_p}{\mathrm{d}t} &=&
- \vec\nabla(\delta\phi)
- H(t) \vec{v}_p \cr
\nabla^2\left(\delta\phi\right)&=& 4\pi G \rho_0 \delta
\end{eqnarray}
Remembering that the spatial derivatives are in physical, not comoving
units, we can write:
\begin{equation}
\frac{\mathrm{d}}{\mathrm{d}t} \left[ \vec\nabla\cdot\vec{v}_p \right]
= 
\vec\nabla\cdot
\left[\frac{\mathrm{d}\vec{v}_p}{\mathrm{d}t} \right]
+ H(t) \frac{\mathrm{d}\delta}{\mathrm{d}t} 
\end{equation}
Then one can take a time derivative with respect to a comoving
observer to Equation \ref{eq:continuity}, and with substitutions this
leads to a second-order equation for the density:
\begin{equation}
\label{eq:secondorder}
\frac{\mathrm{d}^2\delta}{\mathrm{d}t^2} + 2 \frac{\dot a}{a} 
\frac{\mathrm{d}\delta}{\mathrm{d}t} - 4 \pi G \rho_0 \delta = 0 
\end{equation}
This linear set of equations is separable, so that whatever spatial
pattern exists simply changes in amplitude over time:
\begin{equation}
\delta(x, t) = \delta(x, t_0) \frac{D(t)}{D(t_0)},
\end{equation}
where often the convention is $D(t_0) = 1$.
  The general solution is:
\begin{equation}
\label{eq:lineargrowth}
D(t) = A t^{-1} + B t^{2/3}
\end{equation}
The first mode is decaying, and thus not important to the growth
of structure.  The second mode is the one that contributes to the
growth of structure.

This set of solutions is appropriate for the zero-energy, or ``flat''
Universe, without a cosmological constant, when matter density (rather
than radiation) dominates. At early times (but after matter-radiation
equality), while deceleration dominates the dynamics, it is a good
description of the Universe.  However, at later times it becomes less
good. In particular, in our Universe, which appears to be
accelerating, the growth is slowed down considerably by the
acceleration.

The continuity equation (\ref{eq:continuity}) and linear growth imply
a relationship between the peculiar velocity field and the growth
rate. If we convert the spatial derivative to comoving units we find:
\begin{equation}
\frac{1}{a} \vec\nabla_c\cdot\vec{v}_p = - \delta(\vec{x}, t_0) \dot
D(t),
\end{equation}
(taking $D(t_0) = 1$), which we can rewrite as:
\begin{equation}
\vec\nabla_c\cdot\vec{v}_p = - a \delta(\vec{x}, t_0) \dot  
D(t) = - a \delta(\vec{x}, t_0) H f
\end{equation}
where the {\it growth rate} is:
\begin{equation}
f = \frac{\dd \ln D}{\dd \ln a}
\end{equation}
The choice to express this result in terms of the comoving spatial
derivative of the physical peculiar velocity is strange but
conventional.

This peculiar velocity field distorts redshift-based maps of the
universe in a specific way on large scales, that can be measured to
constrain $f$. Since $\delta$ is not directly observable, the directly
observable quantity on linear scales is $\beta = f/b$. Since the
fluctuations in the galaxy sample can be observed, we can recast
$\beta = f \sigma_8 / \sigma_{8,g}$ and the observable is
$\beta \sigma_{8,g}$, from which we infer $f\sigma_8$.

As small scales go nonlinear, gravitationally bound objects will
form. This process can be approximated in the spherical case. If we
situate our coordinate system on the center of a spherical system with
a constant overdensity $\bar\delta > 0$ and size $R$, the system can
be considered completely analogous to a universe with matter density
of $\Omega_m(1+\bar\delta)$. Therefore, if this quantity is greater
than unity, than the sphere will expand for some time, then turn
around at $t=t_{\rm TA}$, and then collapse on itself; this process
can be followed exactly. It can be shown that the mean density of the
sphere at turn-around is about 5.5 times the mean density of the
universe, and collapse occurs in twice the turn-around time. The
virial theorem and energy conservation lead to a typical overdensity
of the collapsed object within its virial radius of $\delta_{\rm vir}
= 18\pi^2 \approx 178$. Meanwhile, the linearly extrapolated
overdensity at that time is only about $\delta_{\rm linear}\approx
1.7$.

The mass spectrum of collapsed halos can be predicted approximately
using {\it excursion set theory}, or the {\it Press-Schechter}
approach (\citealt{press74a, bond91a, lacey93a}). Imagine a patch of
mass $M$ at early times; it will have some specific radius $R$
depending on the mean density. We can predict when it will collapse to
a virialized object when $\delta_{\rm linear} \approx 1.686$ within
radius $R$. At any given time, we can ask what fraction of the
universe's volume, when smoothed on radius $R$, has $\delta_{\rm
linear} > 1.686$. For simplicity, we will smooth by a top-hat in
$k$-space (in configuration space this is smothing by the first order
spherical Bessel function $j_1$). Calculating this fraction tells us
for any mass (that is, smoothing scale), what fraction of the volume
ends up in dark matter halos greater than that mass. This function can
be differentiated to yield the halo mass function:
\begin{equation}
\Phi(M) \dd M
= \frac{1}{\sqrt{2\pi}} \frac{\bar \rho}{M} \frac{\delta_c}{\sigma^3(M)} \left[
- \frac{\dd \sigma^2}{\dd
M}\right] \exp\left[-\frac{\delta_c^2}{2\sigma^2}\right] \dd M
\end{equation}
For $P(k)\propto k^n$, one can show:
\begin{equation}
\Phi(M) \dd{M} = \frac{\bar\rho}{\sqrt{2\pi}
M} \left(\frac{M}{M_\ast}\right)^{(n+3)/6}
\left(\frac{n+3}{3}\right)
\exp\left[-\frac{1}{2}\left(\frac{M}{M_\ast}\right)^{(n+3)/3}\right] \frac{\dd
M}{M}
\end{equation}
Where the nonlinear mass $M_\ast$ is defined by the relation:
\begin{equation}
\sigma^2 = \left(\frac{M}{M_\ast}\right)^{-(n+3)/3} \delta_c^2.
\end{equation}
Because $n>-3$ always, as $\sigma^2$ grows with time, the nonlinear
mass scale grows.  In the standard cosmology, at low small scales (and
thus low masses) $n$ slowly approaches $-3$ from above and
$\Phi(M) \propto M^{-2 + \epsilon}$, where $\epsilon = (n+3) /3$, and
thus is almost divergent.

The detailed prediction of nonlinear growth and collapse to dark
matter halos requires the use of simulations. Because the dark matter
is collisionless, fluid simulations are not sufficient. The universal
approach is to model the dark matter statistically using a large
number of collisionless particles; the $N$-body approximation. An {\it
N-body simulation} is understood to model only the dark matter. These
simulations use some variant of particle-mesh techniques on large
scales, often with an adaptive component on small scales that may use
direct calculations of mutual forms. They invariably employ some
softening length that is reported as the resolution.  {\it
Hydrodynamic} simulations include baryonic fluids in the modeling, and
often their cooling and collapse to stellar systems. They may also
include feedback of supernovae, winds, and active galactic nuclei on
the fluid; this {\it subgrid physics} is typically parameterized in a
simple way.

One important insight from N-body simulations is how halos grow through
accretion of smaller companion halos. These accreted halos often
survive for long periods of time, and are therefore distinct clumps
known as {\it subhalos} within each halo. The centers of halos and
subhalos are the locations where galaxies form (\citealt{wechsler18a}).

The simulations also clarify the internal structure of
halos. Generally their radial profiles can be modeled as:
\begin{equation}
\rho(r) = \frac{\rho_s}{\frac{r}{r_s}\left(1
+ \frac{r}{r_s}\right)^2},
\end{equation}
the ``NFW'' profile of \citet{navarro97a}. In the context of a
specific profile, we define $R_{\rm vir}$ and $M_{\rm vir}$ based on
the radius and enclosed mass within which the mean overdensity is
$\delta_{\rm vir}$ as predicted by the spherical collapse model. Thus:
\begin{equation}
M_{\rm vir} = \frac{4\pi}{3} R_{\rm vir}^3 \bar\rho \left(1+ \delta_{\rm
vir}\right)
\end{equation}
The halo concentration can be characterized by:
\begin{equation}
c_{\rm vir} = \frac{R_{\rm vir}}{r_s}
\end{equation}
The structure of a halo is fully defined by $M_{\rm vir}$ and $c_{\rm
vir}$. In simulations, typically $c_{\rm vir} \sim 5$--$30$
(\citealt{bullock01b; wang20a}), declining with increasing halo mass
and with increasing redshift. Any given choice of halo structure
yields a particular maximum circular velocity $v_{\rm max}$; the
dependence on mass is such that:
\begin{equation}
M_{\rm vir} \propto v_{\rm max}^{3.4}.
\end{equation}

\section{Commentary}


\section{Key References}

\begin{itemize}
  \item
    {\it Physics Foundations of Cosmology},
    \citet{mukhanov05a}
  \item
    {\it The large-scale structure of the
    universe}, \citet{peebles80a}
  \item
    {\it Formation and Evolution of Galaxies: Les Houches
    Lectures}, \citet{white94a} 
\end{itemize}

\section{Order-of-magnitude Exercises}

\begin{enumerate} 
\item At approximately what redshift does structure growth start to
    slow down for a Universe with $\Omega_m = 0.3$, $\Omega_\Lambda = 0.7$?
\end{enumerate} 

\section{Analytic Exercises}

\begin{enumerate}
\item We can understand the turnover in $P(k)$ very simply once we
understand how growth proceeds for $k$ modes inside and outside the
particle horizon. Here we give the picture in the ``synchronous
gauge''; the heuristic narrative outside the horizon depends on gauge,
but the observables do not (\citealt{ma95a}). The following results in
this case regarding density perturbation growth allow us to
characterize the tranfer function $T(k)$ which modifies the density
field modes.  During radiation domination, inside the horizon the
density only grows logarithmically (because the Jeans scale is nearly
the horizon size for a relativistic fluid) and outside the horizon the
density grows as $\delta\propto a^2$. During matter domination, at all
scales $\delta\propto a$.
\begin{enumerate}
\item For a mode $k$ that enters the horizon before matter dominates,
how does the time it spends outside the horizon scale with $k$? 
\item How does the factor growth experienced outside the horizon scale
with $k$?
\item How is an initial $P(k) \propto k$ modified after the universe
has entered the matter-dominated phase?
\end{enumerate}
\item Starting from the continuity equation in Equation
   \ref{eq:homogeneous}, assuming a flat matter dominated universe
    ($\Omega_m = 1$), and keeping only first-order terms, derive
    Equation \ref{eq:continuity}.

%\begin{answer}
%\begin{eqnarray}
%\frac{{\rm D}\vec{v}}{{\rm D} t} =
%\left[ \frac{\dd{}}{\dd{t}} + \vec{v}_p \cdot \vec{\nabla} \right]
%\left[\rho_0 \left(1+\delta\right)\right] &=&
%- \rho_0 (1+\delta) \vec{\nabla}\cdot\vec{v}_0
%- \rho_0 (1+\delta) \vec{\nabla}\cdot\vec{v}_p \cr
%
%\frac{\dd{\rho_0}}{\dd{t}}
%\frac{\dd{}}{\dd{t}}\left(\rho_0 \delta\right)
%+ \vec{v}_p \cdot \vec{\nabla} \right]
%\left[\rho_0 \left(1+\delta\right)\right] &=&
%- \rho_0 (1+\delta) \vec{\nabla}\cdot\vec{v}_0
%- \rho_0 (1+\delta) \vec{\nabla}\cdot\vec{v}_p
%\end{eqnarray}
%\end{answer}


\item Starting from Equation \ref{eq:homogeneous}, and assuming a flat
    matter dominated universe ($\Omega_m = 1$), derive
    Equation \ref{eq:secondorder}.

\begin{answer}[Author: Kate Storey-Fisher]
Equations \ref{eq:homogeneous} are the equations of motion for a
pressureless, gravitating fluid (Euler's equation, the continuity
equation, and Poisson's equation).

We can perturb the density around the homogeneous density $\rho_0$.
If $\Omega_m = 1$, conservation of energy gives us $\rho_0
\propto a^{-3}$. We also perturb the other quantities, giving:
\begin{eqnarray}
\rho  &\rightarrow& (1+\delta)\rho_0 \\
\vec{v} &\rightarrow& \vec{v}_0 + \vec{v}_p \\
\phi  &\rightarrow& \phi_0 + \delta\phi
\end{eqnarray}

For Poisson's equation we find:
\begin{eqnarray}
\nabla^2\phi &=& 4\pi G \rho  \cr
\nabla^2\left(\phi_0 + \delta\phi\right) &=& 4\pi
G \rho_0\left(1+\delta\right)  \cr
\nabla^2\left(\delta\phi\right) &=& 4\pi G \rho_0 \delta
\end{eqnarray}

For Euler's equation we find:
\begin{eqnarray}
\frac{{\rm D}\vec{v}}{{\rm D} t} =
\left[ \frac{\dd{}}{\dd{t}} + \vec{v}_p \cdot \vec{\nabla} \right]
\left[\vec{v}_0 + \vec{v}_p\right] &=& - \vec{\nabla}\left(\phi_0
+ \delta\phi\right) \cr
\frac{\dd{\vec{v}_0}}{\dd{t}} + \vec{v}_p \cdot \vec{\nabla}\vec{v}_0
+ \frac{\dd{\vec{v}_p}}{\dd{t}} + \vec{v}_p \cdot \vec{\nabla}\vec{v}_p
&=& - \vec{\nabla}\phi_0 - \vec{\nabla}\left(\delta\phi_0\right)
\end{eqnarray}
The first terms on both sides cancel because they solve the
homogeneous equations (for which ${\rm D}/{\rm D}t
= \dd{}/\dd{t}$). The last term on the left hand side is
second-order. Rearranging, we are left with
\begin{eqnarray}
\frac{\dd{\vec{v}_p}}{\dd{t}}
&=& - \vec{v}_p \cdot \vec{\nabla}\vec{v}_0
- \vec{\nabla}\left(\delta\phi_0\right)
\end{eqnarray}
Remembering that $\vec{v}_0 = ({\dot a}/a)\vec{r} = H\vec{r}$, where
$\vec{r}$ is in physical units, we have
\begin{eqnarray}
\label{eq:dvdt}
\frac{\dd{\vec{v}_p}}{\dd{t}}
&=&
- H \vec{v}_p
- \vec{\nabla}\left(\delta\phi_0\right)
\end{eqnarray}
where the first term on the right-hand side is the Hubble drag term.

We now take the divergence of Equation \ref{eq:dvdt}, and get
\begin{eqnarray}
\nabla \cdot \frac{\mathrm{d}\vec{v}_p}{\mathrm{d} t} &=& -\vec{\nabla}^2(\delta\phi) - H(t)\nabla \cdot \vec{v}_p \\  
 &=& -4\pi G \rho_0 \delta + H(t) \frac{\mathrm{d}\delta}{\mathrm{d} t}  \label{eq:4piG}
\end{eqnarray}
where in the second line we have substituted the continuity equation
and the Poisson equation.

Taking the time derivative of the continuity equation, we find:
\begin{equation}
\frac{\mathrm{d}^2\delta}{\mathrm{d} t^2}
= \frac{\mathrm{d}}{\mathrm{d}
t}\left(-\vec{\nabla} \cdot \vec{v}_p\right) \label{eq:d2delta}
\end{equation}

Because the spatial derivatives are in physical units, and using the
continuity equation again:
\begin{equation}
\frac{\mathrm{d}^2\delta}{\mathrm{d} t^2} =
- \frac{\mathrm{d}}{\mathrm{d}t} \left(\vec{\nabla} \cdot \vec{v}_p\right)
- = \vec{\nabla} \cdot \frac{\mathrm{d}\vec{v}_p}{\mathrm{d} t} - H(t) \frac{\mathrm{d}\delta}{\mathrm{d} t} 
\end{equation}

And now substitute Equation \ref{eq:4piG} into the RHS of the above equation:
\begin{equation}
\frac{\mathrm{d}^2\delta}{\mathrm{d} t^2} = -\Bigg(-4\pi G \rho_0 \delta + H(t) \frac{\mathrm{d}\delta}{\mathrm{d} t}\Bigg) - H(t) \frac{\mathrm{d}\delta}{\mathrm{d} t} 
\end{equation}
Rearranging,
\begin{equation}
\frac{\mathrm{d}^2\delta}{\mathrm{d} t^2} +
2H(t) \frac{\mathrm{d}\delta}{\mathrm{d} t} - 4\pi G \rho_0 \delta =
0,
\end{equation}
which is the equation for linear growth of the overdensity field.
\end{answer}

\item Show that Equation \ref{eq:lineargrowth} solves
Equation \ref{eq:secondorder}.
\item Consider a spherical region with mean overdensity $\bar\delta
>0$, within an expanding universe with no cosmological constant. As
 long as there is no {\it shell crossing} --- that is, material at one
 radius does not catch up to material at another radius --- the
 equations governing the radius of this sphere over time are
 \begin{equation}
\frac{\dd^2 R}{\dd t^2} = - \frac{GM(<r)}{R^2} = - \frac{4\pi
 G}{3} \bar\rho(1+\bar\delta) R
 \end{equation}
\begin{enumerate}
\item In terms of $\Omega_m$ at the present time, what is the
condition that the spherical region will collapse on itself?

\begin{answer}
This region will evolve the same way as a universe with
matter density $\Omega_m(1+\bar\delta)$. The condition for a closed
universe then implies $\Omega_m(1+\bar\delta)>1$ or
\begin{equation}
\bar\delta > \frac{1}{\Omega} -1.
\end{equation}
\end{answer}

\item Demonstrate that the solutions to the above equation can be
expressed as:
\begin{eqnarray}
\label{eq:sphericalcollapse}
\frac{R}{R_{\rm max}} &=& \frac{1}{2}\left(1 - \cos\eta\right), \cr
\frac{t}{t_{\rm max}} &=& \frac{1}{\pi}\left(\eta - \sin\eta\right)
\end{eqnarray}
where at time $t_{\rm max}$ the sphere reaches its maximum radius of
expansion $R_{\rm max}$, before collapsing. 

\item Show that at time $t_{\rm max}$, the density of the sphere
relative to the mean density of the universe will be $\rho_{\rm max}
/ \bar\rho(t_{\rm max}) = 9 \pi^2 / 16 \approx 5.5$.

\item The collapse of the sphere will proceed in reverse, and will
therefore take $t_{\rm max}$ to do so. However, upon full collapse
shell-crossing will occur, because the collisionless dark matter will
pass through the origin and oscillate around it. This process can be
modeled (\citealt{bertschinger85a, lithwick11a}) to derive the detailed
structure of the resulting halo mass profile, but the virial theorem
($U=-2K$) can tell us about its overall size. Show that the final
characteristic radius of the resulting {\it virialized} halo is
$R_{\rm vir} = R_{\rm max} / 2$.
\item Show that the mean overdensity within the resulting halo  is
$\delta_{\rm vir} = 18\pi^2 \approx 178$. 
\item By linearizing the Equations \ref{eq:sphericalcollapse}, show
that the linearly extrapolated overdensity at the time of collapse is
$\delta_{\rm lin}(2 t_{\rm max}) \approx 1.686$.
\end{enumerate}
\item The Press-Schechter or excursion set estimate of the halo mass
function can be calculated from the statistics of Gaussian random
fields. We can ask what fraction of the volume in the nearly-uniform
early universe ends up in halos of a given mass.  Consider the density
field linearly-evolved to some redshift $z$.
\begin{enumerate}
\item If we smooth the density field on some characteristic scale $R$,
the smoothed density field will relate to the the statistics of halos
of what mass $M$?
\begin{answer}[Author: Trey Jensen]
    Given that we have a nearly uniform universe with density $\bar{\rho}$, we know that the scale will be directly related to the mass via 
    \begin{equation}
        M=\bar{\rho} \frac{4}{3}\pi R^3.
    \end{equation}
\end{answer}
\item If the smoothing is performed as a top-hat function in
$k$-space, what does that smoothing corrrespond to in real space?
\begin{answer}[Author: Trey Jensen]
    The Fourier transform of a top-hat in three dimensions is a first order spherical Bessel function:
\begin{equation}
\tilde \delta = \frac{\sin k R}{kR}
\end{equation}
\end{answer}
\item In terms of the power spectrum, how does the variance
$\sigma^2(M)$ scale with $M$?
\begin{answer}[Author: Trey Jensen]
The variance as a function of the wavenumber $k$ is approximately
$\sigma^2(k) \propto k^3 P(k)$. Using $k=2\pi/R$ to associate $k$ and
$R$, and from the previous part knowing the relation of mass to
radius, then, we find:
\begin{equation}
    \sigma^2(k) \sim P(k)
    \left(\frac{32\pi^4\bar{\rho}}{3M}\right)
\end{equation}
\end{answer}
\item Assume that locations above some linearly-evolved
overdensity $\delta_c \sim 1.686$ on scale $R$ or larger have in fact
collapsed into halos of the corresponding mass $M$ or larger. What
fraction $F(>M)$ of the volume has done so (express in terms of $\delta_c$ and
$\sigma(M)$)?
\begin{answer}[Author: Trey Jensen]
    Because we have a Gaussian random field, and we know the standard
    deviation of density from above, then the fraction of volume above
    this critical density on scale $M$ is the probability of this
    Gaussian random field being above this value, that is, we simply
    integrate the the PDF, giving the complementary error function.
    which is the complementary error function. However, we also need
    to account for the regions which are below the critical overdensity on
    scale $M$, but above it on some larger scale. Since the $\delta$
    as a function of increasing smoothing wavenumber $k$ (decreasing
    mass) is a random walk, for all points that cross the critical
    overdensity at some wavenumber smaller than $k$ that are still above
    the critical overdensity at $k$, there is another point that takes the
    equal and opposite path after crossing the critical overdensity, and
    is below the critical overdensity at scale $k$. This leads to an
    extra factor of two so:
    \begin{equation}
        F(>M)
    = \frac{2}{\sqrt{2\pi}\sigma}\int^\infty_{\delta_c}\dd{M} \delta \exp\left(-\frac{\delta^2}{2\sigma^2}\right)
    \end{equation}
    We leave the expression in this form rather than in terms of the
    error function, because it makes the calculation below easier.
\end{answer}

\item Derive from $F(>M)$ the mass function of halos $\Phi(M)$. 
\begin{answer}
    The mass function is simply the derivative of $F(>M)$,
    converted to a number density with the factor $\bar{\rho}/M$:
    \begin{equation}
    \Phi(M) =
        \frac{\bar{\rho}}{M} \frac{\dd{}}{\dd{M}} F(>M)
    \end{equation}
    If we rewrite $F(>M)$ with a change of variables $x=\delta/\sigma$:
    \begin{equation}
    F(>M) = \frac{2}{\sqrt{2\pi}} \int_{\delta_c/\sigma}^\infty \dd{x}
                          \exp\left(-\frac{x^2}{2}\right)
    \end{equation}
    then it is apparent that:
    \begin{equation}
    \frac{\partial F}{\partial M}
    = \frac{2}{\sqrt{2\pi}} \exp\left(-\frac{\delta_c^2}{2\sigma^2}\right)
    \frac{\partial}{\partial M}\left(\frac{\delta_c}{\sigma}\right).
    \end{equation}
    We can then convert this to a slightly different form:
    \begin{equation}
    \Phi(M) = \frac{\bar{\rho}}{M}
    \frac{1}{\sqrt{2\pi}} \frac{\delta_c}{\sigma^3}
    \exp\left(-\frac{\delta_c^2}{\sigma^2}\right)
    \left[ - \frac{\partial \sigma^2}{\partial M}\right]
    \end{equation}
    to match the text.
\end{answer}

\item Assume $P(k)\propto k^n$. Define the nonlinear mass $M_\ast$:
\begin{equation}
\sigma^2 = \delta_c^2 \left(\frac{M}{M_\ast}\right)^{-(n+3)/3}.
\end{equation}
Write $\Phi(M)$  in terms of $M_\ast$, $\bar\rho$, and $n$. What
happens as $n\rightarrow -3$, as it does at small scales?
\begin{answer}
First let us evaluate:
\begin{equation}
\frac{\dd \sigma^2}{\dd M}
= - \delta_c^2 \frac{1}{M_\ast} \frac{n+3}{3}
\left(\frac{M}{M_\ast}\right)^{-(n+6)/3}
\end{equation}
Then we can plug in
\begin{eqnarray}
\Phi(M) &=& \frac{\bar{\rho}}{M}
\frac{1}{\sqrt{2\pi}}
\frac{1}{M_\ast} \frac{n+3}{3}
\left(\frac{M}{M_\ast}\right)^{-(n+6)/3}
\frac{\delta_c^3}{\sigma^3}
\exp\left(-\frac{\delta_c^2}{\sigma^2}\right) \cr
&=& \frac{\bar{\rho}}{M}
\frac{1}{\sqrt{2\pi}}
\frac{1}{M_\ast} \frac{n+3}{3}
\left(\frac{M}{M_\ast}\right)^{-(n+6)/3}
\left(\frac{M}{M_\ast}\right)^{(n+3)/2}
\exp\left(-\frac{\delta_c^2}{\sigma^2}\right) \cr
&=& \frac{\bar{\rho}}{M}
\frac{1}{\sqrt{2\pi}}
\frac{1}{M_\ast} \frac{n+3}{3}
\left(\frac{M}{M_\ast}\right)^{(n-3)/6}
\exp\left(-\frac{\delta_c^2}{\sigma^2}\right) \cr
&=& \frac{\bar{\rho}}{M}
\frac{1}{\sqrt{2\pi}}
\frac{1}{M}
\frac{n+3}{3}
\left(\frac{M}{M_\ast}\right)^{(n+3)/6}
\exp\left[-\left(\frac{M}{M_\ast}\right)^{(n+3)/3}\right]
\end{eqnarray}
where the last term matches the corresponding equation in the text.
When $n\rightarrow -3$ from above, and we consider $M\ll M_\ast$, this
mass function becomes close to but slightly shallower then $M^{-2}$,
which would be the divergent function.

It is worth pausing here and asking what it would mean if $n<-3$ ---
would that lead to a divergent mass function? Obviously that cannot
reflect reality, but what happens to the excursion set argument?  What
happens is that $n<-3$ implies that the smallest scales do not
collapse first, because $k^3P(k)$ is not monotonically
increasing. This means that the hierarchical collapse ansatz  that
underlies the excursion set picture fails and it is not applicable in
this case. 
\end{answer}


\end{enumerate}

\item Argue {\it qualitatively} why the excursion set approach should
lead to the prediction that dense regions on large scales should have
more and larger dark matter halos; this argument was formalized
originally by \citet{mo96a}.

\end{enumerate}

\section{Numerics and Data Exercises}

\begin{enumerate}
\item Download and install CAMB, the standard code to calculate the
power spectrum. Plot the linear $P(k)$, for $\Omega_m= 0.1$, $0.3$ and
1 (assume $h=0.7$). Examine the dependence on baryon density by
doubling it from the standard value; the wiggles you see getting
stronger are due to the baryon acoustic oscillation.
\end{enumerate}

\bibliographystyle{apj}
\bibliography{exex}  
