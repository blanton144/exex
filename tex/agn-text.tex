\title{\bf Active Galactic Nuclei \& Quasars}

\section{Basics}

Every massive galaxy contains supermassive black hole ($M> 10^5$
$M_\odot$) at its center. In cases that gas is accreting onto the
black hole, these can emit copious luminosity and are referred to as
{\it active galactic nuclei}. Their total luminosities can be from
$10^{40}$ to $10^{47}$ erg s$^{-1}$, in the brighter cases outshining
the rest of the galaxy. An accretion disk a few hundred Schwarschild
radii in size emits a thermal {\it continuum} in the UV and optical,
as well as in the X-rays. The accretion can power an outflowing {\it
  jet} visible in synchroton in the radio. Surrounding the accretion
disk is a {\it broad line region} a few pc in size, consisting of
dense clouds ($n>10^4$ cm$^{-3}$); the Doppler broadening of these
clouds orbiting the center of the galaxy is 1000s of km
s$^{-1}$. Outside that region is a {\it narrow line region} a few
hundred pc in size, also consisting of clouds but of lower density,
and with a Doppler broadening of 100s of km s$^{-1}$. In between is a
dusty region, possibly toroidal in shape, but in any case causing some
the broad line region to be heavily extincted along at least some
lines of sight. This picture, known as the {\it unification model} is
designed to explain the various observational classes of active
galactic nuclei, as explained below.

There is an array of nomenclature associated with quasars. Below
luminosities of around $10^{44}$ erg s$^{-1}$ they are known as {\it
  Seyfert galaxies}. If broad line emission is visible, they are known
as {\it Type 1}, and if it is not they are known as {\it Type 2};
there are also intermediate types occasionally defined. Depending on
their radio flux, Seyfers and quasars can be {\it radio-loud} or not.

\subsection{Black hole accretion}

Active galactic nuclei are powered by accretion. Matter accreting
loses energy (and necessarily angular momentum) through some form of
viscosity. This orbital energy loss heats the gas and the energy is
emitted radiatively. This process can proceed until the material
reaches the innermost stable circular orbit, at $3r_S$, at which point
the material does not need to lose energy to fall directly into the
black hole. We quantify the efficiency of this process as a fraction
$\epsilon$ of the rest mass energy ($mc^2$). The luminosity of the
active galactic nucleus is then $\epsilon \dot M$, if $\dot M$ is the
mass accretion rate. As shown in the exercises, the maximum efficiency
of this process for a non-spinning black hole is about $\epsilon \sim
0.06$; it may be up to about 0.29 for a spinning black hole.

Regardless of the mass accretion rate, the luminosity is limited by
the fact that if the luminosity is too high, outward radiation
pressure on the surrounding material will prevent inward flow due to
gravitational accelerations. The exercises derive this {\it Eddington
  limit}:
\begin{equation}
L_{\rm Edd} = \frac{4\pi G c m_p}{\sigma_T} \approx 1.26 \times
10^{38} \left(\frac{M_{\rm BH}}{M_\odot}\right) \mathrm{~erg~s}^{-1}
\end{equation}
This limit implies that to reach $L\sim 10^{44}$ erg s$^{-1}$ requires
a black hole with a mass at least $10^6$ $m_\odot$. Although this
limit only strictly applies in spherical symmetry, estimates of black
hole masses imply luminosities of $0.01$--$0.1 L_{\rm Edd}$. Sometimes
the Eddington limit is expressed as a mass accretion limit assuming
some $\epsilon$, but of course for low $\epsilon$ the mass accretion
is not prevented from being very high.

If accreting gas has angular momentum, it will naturally form a disk
of gas. The gas will experience viscosity. If the disk is Keplerian,
this viscosity will naturally transport angular moment outward. The
loss of angular momentum and the conversion of energy to heat allows
the accretion. If the disk is optically thick, and geometrically thin,
it can be shown that the temperature profile will be:
\begin{equation}
  T(r) = \left(\frac{3 c^6}{64\pi G^2 \sigma_{SB}}\right)^{1/4}
    \left(\frac{r}{r_S}\right)^{-3/4} {\dot m}^{1/4} M^{-1/2}
\end{equation}
The global resulting spectrum is the sum of the blackbody spectrum
from all the radii. Generally, the spectrum will be hotter for lower
mass black holes, or for more rapidly accreting black holes.

However, a thin disk is not the only conceivable outcome for the
accretion. An alternative possibility is that the disk (or some part
of it) does not efficiently radiate the energy dissipated, in which
case the disk can thicken. The resulting {\it advection dominated
  accretion flow} transports the thermal energy produced directly into
the black hole through accretion, and thus is less efficient at
turning gravitational energy into radiation. Theoretical models favor
the creation of jets in this scenario, collimated by the inner edge of
the thick disk formed. 

X-rays arise from the accretion disk. In AGN, the thermal continuum
only provides substantial X-rays in the soft regime ($<2$ keV). The
higher energy X-rays are thought to arise from a coronal layer
surrounding a thin accretion disk, similar to the solar corona. This
hot corona can in principle be as hot as the virial
temperature. Through inverse Compton scattering, it converts the disk
continuum radiation into hard X-rays, extending in a power law
spectrum $f_\nu \propto \nu^{-2}$ up to 100 keV.

\subsection{Jet}

In many systems, the accretion drives a jet of ionized gas outwards,
in some cases hundreds of kpc in length.  The jets often end in lobed
structures.  This jet is visible in the radio through synchrotron
radiation with a power law spectrum. The radiating electrons are
highly relativistic. The synchrotron radiation can be visible at
higher frequencies, up through the optical. X-ray emission can be
visible as well, which may be related to synchrotron processes as
well. The radio spectra have a variety of shapes depending on the
level of self-absorption.

The jets are typically moving at a high fraction of the speed of
light, based on observed motions of features in the jets when resolved
in the radio and optical. This causes the radiation to be beamed
strongly along the jet axis; this effect on the surface brightness
causes the jet toward us to be detectable more often than the jet away
from us.

The beaming can cause the inferred luminosity of the source to be
extremely high if it is observed along the beam. This phenomenon is
known as a {\it blazar}. The beamed synchrotron emission can overwhelm
the ultraviolet-optical line and continuum emission, resulting in a
pure power law continuum in the optical. This phenomenon is referred
to as a {\it BL Lac} after the archetypal case. These sources are
typically highly variable. If some line emission remains, the galaxy
is known as an {\it optically violent variable (OVV)} star.

\subsection{Broad line region}

Many quasars and AGN exhibit broad emission lines {\bf what fraction}
with Doppler widths of order $10^4$ km s$^{-1}$. These velocities
imply a distance of a few hundred Schwarzschild radii from the central
black hole, which translates into {\bf what} for a $10^8$ $M_\odot$
black hole. Forbidden lines are not seen in these regions, which
implies a high electron density, and from the presence of
semiforbidden lines we infer $n_e \sim 10^{10}$ cm$^{-3}$. The
ionization stages present imply temperatures of $2\times 10^4$ K. With
the density, temperature, and volume implied by these numbers, total
emission from the gas then can be used to infer the filling factor,
which is at the very most $f\sim 0.1$. From the total actual
luminosity compared to the ionizing continuum flux, it can be inferred
that the solid angle covered by the clouds is about 10\%. These
considerations lead to the picture of the broad line region as a large
number of dense clouds surrounding the central black hole and its
accretion disk.

\subsection{Narrow line region}

Essentially all quasars and AGN exhibit narrow optical emission lines,
with widths of hundreds of km s$^{-1}$, indicating a region of a few
hundred pc. The narrow line region can be resolved in many cases, and
often has a biconical shape. The emitting lines include forbidden
lines, with line ratios indicating densities of $n\sim 10^4$ cm$^{-3}$
and temperatures of around 15,000 K. Like the broad line region, the
filling factor can be shown to be low ($\sim 0.01$) by comparing the
density, temperature, and volume to the total emission.

Narrow forbidden lines are also excited in gas ionized by
star-formation, hot evolved stars, and (more rarely) shocks. Emission
line ratios can be used to distinguish the nature of the ionization
source. The classic set of ratios are those published by Baldwin,
Phillips, \& Terlevich (BPT), the most commonly used of which is
considering [NII]/H$\alpha$ versus [OIII]/H$\beta$. The choices of
emission line ratios are partly motivated by the desire to combine
lines near the same wavelength, whose spectrophotometric and internal
reddening correction errors will partly or mostly cancel. Seyfert
galaxies have relatively high [NII]/H$\alpha$ and very high
[OIII]/H$\beta$. As shown in the exercises, the [OIII]/H$\beta$ ratio
is a good indicator of the ionization parameter.

A separate class of galaxies in the BPT diagram are low ionization
galaxies, sometimes called LINERs, which have high [NII]/H$\alpha$ but
lower [OIII]/H$\beta$. Although some of these galaxies have active
galactic nuclei, their emission is considerably more extended than
that of the Seyferts and appears inconsistent with being ionized by a
central source. 

\subsection{Dusty torus}

Many galaxies exhibit narrow line AGN emission but no broad line or
continuum emission in the optical and ultraviolet. However, the narrow
line region must be ionized by some continuum. Furthermore, some
galaxies have strong nuclear infrared emission. We believe that in
these cases, the broad line and continuum regions are obscured by dust
along our line of sight, which is in some anisotropic distribution
such that along other lines of sight they would be visible. Usually
this is described as a torus, but that specific geometry isn't
necessary. To explain the observations, the dusty region should be
larger that the broad line region, but smaller than the narrow line
region. The most direct evidence for this hypothesis is the
observation of broad lines in the polarization spectrum of narrow line
AGN. The broad line region light is being scattered and polarized by
gas in the narrow line region. These observations demonstrate that the
dusty region does not obscure in all directions.

\subsection{Variability}

Quasars and active galactic nuclei are variable on time scales of a
few weeks and above. Typically, the variability is anticorrelated with
luminosity, with larger amplitude variability for less luminous
objects. The variability in the continuum, which is likely due to
accretion disk activity, typically occurs on shorter time scales for
higher wavelengths, likely reflecting the dynamical time scale of the
location of the emission.  The line emission variability can be shown
to lag the continuum, and the delay can be used to determine the
radius of emission of the lines, and combined with their Doppler
shifts, an estimate of the black hole mass; this technique is known as
{\it reverberation mapping}. Occasionally even on a year-to-decade
time scale active galactic nuclei and quasars have been known to turn
off rapidly; the continuum and broad line emission disappears, but the
narrow line emission (which extends for hundreds of lightyears)
remains. This phenomenon is known as {\it changing look}.

From the demographics and clustering of optically-bright quasars it is
known that they are not in that state all of the time. From these
considerations, {\it duty cycle} of quasars is estimated to be about
{\bf 10 percent??}. Thus over long time scales (likely tens of
millions of years) all quasars must turn on and off.

\subsection{Luminosity function and evolution}


\section{Commentary}

\section{Key References}

\begin{itemize}
  \item
    {\it Netzer}
  \item
    {\it AGN Book}
\end{itemize}

\citet{gunn06a}

\section{Order-of-magnitude Exercises}

\begin{enumerate} 
\item Compare maximum efficiency to stellar
\item Estimate efficiency of accretion
\item Estimate hottest gas in continuum
\item What is virial tempereature near disk
\item Doppler velocities to sizes
\item Critical density of CIII] and OIII
\item Estimate filling factor of BLR, NLR
\item Estimate duty cycle
\end{enumerate} 

\section{Analytic Exercises}

\begin{enumerate}
\item Eddington luminosity
\item Temperature profile of accretion disk
\item Relationship between energy distribution and synchrotron
\item Superluminal motion
\item Relativistic Beaming
\item Polarization
\end{enumerate}

\section{Numerics and Data Exercises}

\begin{enumerate}
\item Quasar SEDs
\item Quasar variability
\item Densities from line ratios
\item Temperatures from line ratios
\item OIII/Hbeta and ionization parameter
\item resolved spectroscopy of AGN
\end{enumerate}

\bibliographystyle{apj}
\bibliography{exex}  
