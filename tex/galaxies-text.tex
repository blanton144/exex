\title{\bf Galaxies}

\section{Basics \& Nomenclature}

The basic phenomenology of galaxies is necessary to understand before
embarking on their detailed study. In the broadest terms, galaxies are
gravitationally bound collections of stars, gas, and dark
matter. Their masses range from extremely small, a few $10^6$
$M_\odot$ in dark matter, up to around $10^{13}$ $M_\odot$ in dark
matter. Baryonic mass in the form of stars and cold gas is only about
5\% of the dark matter mass, smaller than the cosmic mean
fraction. The extent of stars and gas ranges from a few tenths of a
kpc to 10s of kpc; the extent of the dark matter appears to be much
larger. Galaxies the size of the Milky Way have number densities such
that their mean separation is about 5 Mpc.

Theoretically, it appears that galaxies form near the centers of dark
matter halos and subhalos that form as dark matter collapses
gravitationally from the initial density perturbations. To zeroth
order galaxy phenomenology can be explained by the relationship of
galaxies to their host halos and subhalos.

In the optical, a critical quantity characterizing a galaxy is its
luminosity. The luminosity is typically measured from light in an
image associated with the galaxy, either by summing light with a
circular or elliptical aperture or by fitting the amplitude of a
model. Aperture fluxes can be defined in multiple ways. The most
robust is the Petrosian definition used by the Sloan Digital Sky
Survey. Most aperture fluxes need to be corrected for the effect of
the point spread function, especially at small galaxy sizes. At faint
fluxes, aperture fluxes defined to contain most of the expected light
tend to be noisy.

Model fluxes are higher signal-to-noise and can be constructed to
account for the point spread function and any inherent ellipticity in
the model. Inaccuracy in the model can on the other hand result in
systematic errors in the flux.

The {\it luminosity function} $\Phi(L)$ of galaxies characterize their
luminosity distribution.  A simple approximation to this function is
the Schechter function:
\begin{equation}
\Phi(L) \dd{L} = \left(\frac{\phi_0}{L_\ast}\right)
\left(\frac{L}{L_\ast}\right)^\alpha \exp\left(L / L_\ast\right)
\dd{L}
\end{equation}
where $L_\ast$ is similar to the Milky Way luminosity and $\alpha \sim
-1.3$ is the {\it faint end slope}. A better approximation is a broken
power law at the faint end, which gets steeper at lower
luminosities. The luminosity function depends on band pass, because
the colors of galaxies depend on luminosity. A similar function
characterizes the stellar mass function.

The {\it stellar mass function} $\Phi(M_\ast)$ characterizes their
distribution in total stellar mass. Estimates of stellar mass depend
on broad-band observations and/or spectroscopic observations combined
with models of the underlying stellar population. Broadly speaking,
the stellar mass is proportional to the light multiplied by a
mass-to-light ratio. The mass-to-light ratio depends on the age,
metallicity, dust content, and other properties of the stellar
population, with the trend in all cases for the redder populations to
have higher mass-to-light ratios. Stellar mass is defined as the
current stellar mass in objects above the hydrogen burning limit,
including remnants such as white dwarfs and neutron stars, but not
including any mass returned through winds or supernovae to the
interstellar medium.

As a function of redshift, the luminosity function and stellar mass
function change. The stellar mass function in general needs to grow
with cosmic time, since star formation proceeds within galaxies over
time. The luminosity function however tends to decrease with cosmic
time, since the rate of star formation decreases and the mass-to-light
ratio tends to grow.

The spectral energy distributions of galaxies depend on their
luminosity. The simplest characterization of these spectral energy
distributions are from the broad-band color. At the highest
luminosities, low redshift galaxies tend to be uniformly red,
indicating an old stellar population. At the lowest luminosities, they
tend to be primarily blue, indicating a young stellar popation, except
in those cases where they are satellites of a larger galaxy. At
intermediate luminosities (around $L_\ast$) both populations exist and
the color distribution is said to be {\it bimodal}.

The internal structure of galaxies also varies. Very broadly speaking,
the stellar components of galaxies are:
\begin{itemize}
 \item {\it disks}: an extended, flattened, rotating distribution of
 stars and (usually) gas. Disks often have spiral structure through
 them, with star formation concentrated on the spiral arms.
 \item {\it bulges}: generally puffy, velocity dispersion supported
  collections of stars near the centers of galaxies, thought to arise
  through the destruction of merging galaxies. 
 \item {\it pseudobulges}: generally flatter, more exponential
   collections of stars near the centers of galaxies, thought to arise
   through internal processes within disks. 
 \item {\it bars}: elongated structures near the centers of galaxies,
   thought to arise through dynamical instabilities of bulges or
   pseudobulges. 
\end{itemize}

Disks tend to have an exponential radial profile when azimuthally
averaged:
\begin{equation}
I(r) = I_0 \exp\left(- r / r_e\right)
\end{equation}
where for the Milky Way $r_e \sim 3$ kpc. When observed at an {\it
inclination} that is $i<90^\circ$, the two-dimensional image is
projected into an ellipse on the sky that can be characterized by an
{\it axis ratio}, which if the disk were transparent and infinitely
thin would have a straightforward relationship with the inclination.

Bulges have a range range of radial profiles that can be characterized
by the \Sersic\ profile:
\begin{equation}
I(r) = I_0 \exp\left[- \left(r / r_e\right)^{1/n}\right]
\end{equation}
where $n \sim 1--4$ but can be larger than that. $n=1$ is the same as
an exponential profile, and $n=4$ is a special profile known as the
{\it de Vaucouleurs} profile. Bulges are typically ellipsoidal or
triaxial systems, seen at some angle, so that their axis ratios do not
simply translate into an inclination.

The size of a galaxy is often quantified by the {\it half-light
radius} or {\it effective radius}, which is the radius containing 50\%
of the estimated total light within a circular or elliptical
annulus. Another common measure of size is $D_{25}$, which is the
radius of the isophote as which the surface brightness in $B$ becomes
25 mag arcsec$^{-2}$; this definition is not suitable at high redshift
because surface brightness is not conserved.

A related measurement is the surface brightness. For example, the {\it
half-light surface brightness} is the mean surface brightness within
the half-light radius. The {\it central surface brightness} is the
surface brightness at the center, only measurable if the center is
smooth relative to the resolution of the image.

Internal structure has traditionally been characterized by a measure
of morphology. Definitions of galaxy morphology vary among
investigators, but the broad classifications of elliptical (or E),
lenticular (or S0), or spiral (ranging from Sa through Sd) are common.

Elliptical galaxies are red, puffy, and generally velocity dispersion
supported. They lie on the red sequence in color space (but S0 and Sa
galaxies are there as well). They are very much like bulges
unencumbered by disks. They contain small amounts of cold gas,
typically about 1\% of the stellar mass, and also warm and hot
gas. Their \Sersic\ indices $n$ vary from around 1 at low luminosities
({\it dwarf ellipticals}, or {\it dE} galaxies) up to 4 or more for
higher luminosities ({\it giant ellipticals}, sometimes {\it gE}
galaxies).  There is a rare class of low luminosity elliptical with
$n\sim 4$, known as {\it compact elliptical}, or {\it cE}, galaxies.

Lenticular galaxies are red but disk like, though not as thin as true
disks. They can be hard to distinguish from ellipticals when seen face
on, but commonly have bars or rings, and their stellar distribution
commonly has a distinct dropoff in its outer parts (i.e. the
exponential profile effectively stops). They also have little gas. 

Disk galaxies can be blue or red, but generally consist of thin disks
of gas and stars (a few 100 pc thick). The gas is typically a mix of
ionized, neutral, and molecular gas, the latter occurring in molecular
clouds. Much ionized gas surrounds star forming regions that are often
embedded in the molecular clouds, while warm ionized gas does exist
elsewhere in the disks.

The centers of disk galaxies can consist of a range of interesting
structures, including bars, bulges, and pseudobulges. High luminosity
galaxies tend to have large bulges, and low luminosity galaxies weak
ones. Spirals have a morphology classication that ranges from Sa to
Sd, but there are different choices in different catalogs as to what
the classification indicates. They generally are meant to capture the
pitch angle of the spiral arms, their organizational structure, and
the prominence of the bulge.  The Reference Catalog classifications of
de Vaucouleurs tend to emphasize the bulge, whereas the Revised
Shapley-Ames Catalog of Sandage and Tammann depends more strongly on
the spiral arms.

A common quantitative measure of morphology is the bulge-to-total
ratio $B/T$, which is the result of fitting a bulge and disk model to
a galaxy. By definition, this quantity ranges from zero to one, and
pure disk galaxies lie at zero and elliptical galaxies lie at one. A
model independent, related quantity is the {\it concentration},
defined by $r_{90}/r_{50}$, the 90\% to 50\% radius ratio.

The dark matter content of galaxies is revealed by their dynamics and
in some cases by gravitational lensing. Elliptical galaxies are
supported by a combination of rotation and velocity dispersion;
interpreting the dynamical masses of elliptical galaxies is complex
but achievable. The best evidence to date suggests that massive
elliptical galaxies have high mass-to-light ratios, which indicates
either a non-standard IMF that changes with stellar mass, or the
presence of dark matter.  Disk galaxies are primarily supported by
rotation. Particularly in their outer parts, the gas dynamics is
relatively straightforward to interpret and unambiguous signatures of
dark matter are present.

Two major scaling relations quantify the relationship between
luminosity and dynamics: The {\it fundamental plane} of giant
elliptical galaxies, the relationship between size, luminosity, and
velocity dispersion, which takes the approximate form:
\begin{equation}
r_e \propto \sigma^\alpha I_e^{-\beta}
\end{equation}
which implies a plane in the three-dimensional space of the logarithms
of these quantities. The exercises below explore the implications of
this relationship.

The {\it Tully-Fisher law} disk galaxies near $L_\ast$ relates their
luminosities and their peak circular velocities, as measured with
optical or radio observations. Since the colors of disk galaxies
depend on luminosity, the slope of this relationship depends on the
band that the luminosity is measured in. In the $I$-band, the
relationship is equivalent to $L\propto v^{3.1}$. At low luminosities,
the velocity is independent of $L$ (but is dependent on the total
baryonic mass). 

Luminous galaxies all seem to have a supermassive black holes at their
centers, with evidence for the black holes from active galactic
nucleus emission (called {\it Seyfert galaxies}) but also from
dynamical signatures through stellar dynamics, reverberation mapping,
and central masers.

\todo{spectra?}

\section{Commentary}

The description here relies on our best understanding of underlying
physical properties, but these properties are imperfectly measured.

Unusual galaxies exist: ellipticals with gas disks, polar ring
galaxies, etc. central star bursts. ULSBs. UCDs.

\section{Key References}

\begin{itemize}
  \item
    {\it Blanton \& Moustakas}
  \item
    {\it Graham paper on Sersic}
\end{itemize}

\citet{gunn06a}

\section{Order-of-magnitude Exercises}

\begin{enumerate} 
\item Using the numbers in the text, estimate $\Omega_b$ and
    $\Omega_m$. 
\item Estimate the stellar mass surface density for a Milky Way like
disk galaxy.
\end{enumerate} 

\section{Analytic Exercises}

\begin{enumerate}
\item Petrosian quantities for Sersic galaxies
\item $\kappa$ space
\item Tully-Fisher
\end{enumerate}

\section{Numerics and Data Exercises}

\begin{enumerate}
\item Luminosity function vs Press-Schechter
\item Looking at galaxy images across color magnitude space
\item Spectra and colors
\item Measurement of bulge and disk 
\item Axis ratios of disk galaxies and elliptical galaxies
\item Dust reddening vs. axis ratio
\item Velocity dispersion
\item Rotation curves
\item Tully Fisher
\item FP
\end{enumerate}

\bibliographystyle{apj}
\bibliography{exex}  
