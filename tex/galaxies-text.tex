\title{\bf Galaxies}

\section{Basics \& Nomenclature}

The basic phenomenology of galaxies is necessary to understand before
embarking on their detailed study. In the broadest terms, galaxies are
gravitationally bound collections of stars, gas, and dark
matter. Their masses range from extremely small, a few $10^6$
$M_\odot$ in dark matter, up to around $10^{13}$ $M_\odot$ in dark
matter. Baryonic mass in the form of stars and cold gas is only about
5\% of the dark matter mass, smaller than the cosmic mean
fraction. The extent of stars and gas ranges from a few tenths of a
kpc to 10s of kpc; the extent of the dark matter appears to be much
larger. Galaxies the size of the Milky Way have number densities such
that their mean separation is about 5 Mpc.

Theoretically, it appears that galaxies form near the centers of dark
matter halos and subhalos that form as dark matter collapses
gravitationally from the initial density perturbations. To zeroth
order galaxy phenomenology can be explained by the relationship of
galaxies to their host halos and subhalos.

In the optical, a critical quantity characterizing a galaxy is its
luminosity. The luminosity is typically measured from light in an
image associated with the galaxy, either by summing light with a
circular or elliptical aperture or by fitting the amplitude of a
model. Aperture fluxes can be defined in multiple ways. The most
robust is the Petrosian definition used by the Sloan Digital Sky
Survey. Most aperture fluxes need to be corrected for the effect of
the point spread function, especially at small galaxy sizes. At faint
fluxes, aperture fluxes defined to contain most of the expected light
tend to be noisy.

Model fluxes are higher signal-to-noise and can be constructed to
account for the point spread function and any inherent ellipticity in
the model. Inaccuracy in the model can on the other hand result in
systematic errors in the flux.

The {\it luminosity function} $\Phi(L)$ of galaxies characterize their
luminosity distribution.  A simple approximation to this function is
the Schechter function:
\begin{equation}
\Phi(L) \dd{L} = \left(\frac{\phi_0}{L_\ast}\right)
\left(\frac{L}{L_\ast}\right)^\alpha \exp\left(L / L_\ast\right)
\dd{L}
\end{equation}
where $L_\ast$ is similar to the Milky Way luminosity and $\alpha \sim
-1.3$ is the {\it faint end slope}. A better approximation is a broken
power law at the faint end, which gets steeper at lower luminosities
(e.g. \citealt{blanton04b}). The luminosity function depends on band
pass, because the colors of galaxies depend on luminosity. A similar
function characterizes the stellar mass function.

Estimates of the luminosity function are based on counting galaxies as
a function of luminosity. However, for realistic catalogs we do not
have complete samples and all estimates must be corrected for
incompleteness. The most common sort of incompleteness is to have a
{\it flux-limited} sample, so that underluminous galaxies cannot be
seen over a large volume. However, for color-selected surveys or
surveys that probe high redshift the incompleteness can take more
complicated forms. The completeness is also a function of galaxy
surface brightness. The simplest method to correct for incompleteness
is the {\it $1/V_{\rm max}$ method}. If for each galaxy we can
calculate the maximum volume over which it could have still entered
the sample, then we can estimate the luminosity function at some $L$
as:
\begin{equation}
\Phi(L) \dd{L} = \sum_i \frac{1}{V_{\rm max}}
\end{equation}
for all galaxies $i$ in a range $\dd{L}$ around $L$. However, this
technique assumes we have a sample of luminosity $L$ galaxies spanning
the full range of properties that might affect their selection. It
also tends to be sensitive to the presence of large scale structure in
the sample. Note that under some conditions, we can construct {\it
volume-limited} samples which are statistically complete over some
range of luminosities and some range of distances, which can be
useful.

The {\it stellar mass function} $\Phi(M_\ast)$ characterizes their
distribution in total stellar mass. Estimates of stellar mass depend
on broad-band observations and/or spectroscopic observations combined
with models of the underlying stellar population. Broadly speaking,
the stellar mass is proportional to the light multiplied by a
mass-to-light ratio. The mass-to-light ratio depends on the age,
metallicity, dust content, and other properties of the stellar
population, with the trend in all cases for the redder populations to
have higher mass-to-light ratios. For example, \citet{bell00a} find
\begin{equation}
\label{eq:belldejong}
\log_{10} \left(\frac{M}{L}\right) \approx -0.3 + 1.1 (g-r)
\end{equation}
for the stellar mass-to-light ratio in solar units. Stellar mass is
defined as the current stellar mass in objects above the hydrogen
burning limit, including remnants such as white dwarfs and neutron
stars, but not including any mass returned through winds or supernovae
to the interstellar medium.

As a function of redshift, the luminosity function and stellar mass
function change. The stellar mass function in general needs to grow
with cosmic time, since star formation proceeds within galaxies over
time. The luminosity function however tends to decrease with cosmic
time, since the rate of star formation decreases and the mass-to-light
ratio tends to grow.

The spectral energy distributions of galaxies depend on their
luminosity. The simplest characterization of these spectral energy
distributions are from the broad-band color. At the highest
luminosities, low redshift galaxies tend to be uniformly red,
indicating an old stellar population. At the lowest luminosities, they
tend to be primarily blue, indicating a young stellar popation, except
in those cases where they are satellites of a larger galaxy. At
intermediate luminosities (around $L_\ast$) both populations exist and
the color distribution is said to be {\it bimodal}. The broad-band
colors reflect the underlying spectrum, whose detailed shape can be
interpreted in terms of the stellar and gas content of the galaxies.

The internal structure of galaxies also varies. Very broadly speaking,
the stellar components of galaxies are:
\begin{itemize}
 \item {\it disks}: an extended, flattened, rotating distribution of
 stars and (usually) gas. Disks often have spiral structure through
 them, with star formation concentrated on the spiral arms.
 \item {\it bulges}: generally puffy, velocity dispersion supported
  collections of stars near the centers of galaxies, thought to arise
  through the destruction of merging galaxies. 
 \item {\it pseudobulges}: generally flatter, more exponential
   collections of stars near the centers of galaxies, thought to arise
   through internal processes within disks. 
 \item {\it bars}: elongated structures near the centers of galaxies,
   thought to arise through dynamical instabilities of bulges or
   pseudobulges. 
 \item {\it stellar halos}: extended distributions of stars, with a
 power law density profile.
\end{itemize}
The disks and bulges dominate the stellar mass budget integrated over
all galaxies, with the other components each comprising a few percent
of the total (e.g., \citealt{gadotti09a}).

Disks tend to have an exponential radial profile when azimuthally
averaged:
\begin{equation}
I(r) = I_0 \exp\left(- r / r_e\right)
\end{equation}
where for the Milky Way $r_e \sim 3$ kpc. When observed at an {\it
inclination} that is $i<90^\circ$, the two-dimensional image is
projected into an ellipse on the sky that can be characterized by an
{\it axis ratio}, which if the disk were transparent and infinitely
thin would have a straightforward relationship with the inclination.

Bulges have a range range of radial profiles that can be characterized
by the \Sersic\ profile:
\begin{equation}
I(r) = I_0 \exp\left[- \left(r / r_e\right)^{1/n}\right]
\end{equation}
where $n \sim 1--4$ but can be larger than that. $n=1$ is the same as
an exponential profile, and $n=4$ is a special profile known as the
{\it de Vaucouleurs} profile. Bulges are typically ellipsoidal or
triaxial systems, seen at some angle, so that their axis ratios do not
simply translate into an inclination.

The size of a galaxy is often quantified by the {\it half-light
radius} or {\it effective radius}, which is the radius containing 50\%
of the estimated total light within a circular or elliptical
annulus. Another common measure of size is $D_{25}$, which is the
diameter of the isophote as which the surface brightness in $B$
becomes 25 mag arcsec$^{-2}$; this definition is not suitable at high
redshift because surface brightness is not conserved.

A related measurement is the surface brightness. For example, the {\it
half-light surface brightness} is the mean surface brightness within
the half-light radius. The {\it central surface brightness} is the
surface brightness at the center, only measurable if the center is
smooth relative to the resolution of the image.

Internal structure has traditionally been characterized by a measure
of morphology. Definitions of galaxy morphology vary among
investigators, but the broad classifications of elliptical (or E),
lenticular (or S0), or spiral (ranging from Sa through Sd) are common.

Elliptical galaxies are red, puffy, and generally velocity dispersion
supported. They lie on the red sequence in color space (but S0 and Sa
galaxies are there as well). They are very much like bulges
unencumbered by disks. They contain small amounts of cold gas,
typically about 1\% of the stellar mass, and also warm and hot
gas. Their \Sersic\ indices $n$ vary from around 1 at low luminosities
({\it dwarf ellipticals}, or {\it dE} galaxies) up to 4 or more for
higher luminosities ({\it giant ellipticals}, sometimes {\it gE}
galaxies).  There is a rare class of low luminosity elliptical with
$n\sim 4$, known as {\it compact elliptical}, or {\it cE}, galaxies.

Lenticular galaxies are red but disk like, though not as thin as true
disks. They can be hard to distinguish from ellipticals when seen face
on, but commonly have bars or rings, and their stellar distribution
commonly has a distinct dropoff in its outer parts (i.e. the
exponential profile effectively stops). They also have little gas. 

Disk galaxies can be blue or red, but generally consist of thin disks
of gas and stars (a few 100 pc thick). The gas is typically a mix of
ionized, neutral, and molecular gas, the latter occurring in molecular
clouds. Much ionized gas surrounds star forming regions that are often
embedded in the molecular clouds, while warm ionized gas does exist
elsewhere in the disks.

The centers of disk galaxies can consist of a range of interesting
structures, including bars, bulges, and pseudobulges. High luminosity
galaxies tend to have large bulges, and low luminosity galaxies weak
ones. Spirals have a morphology classication that ranges from Sa to
Sd, but there are different choices in different catalogs as to what
the classification indicates. They generally are meant to capture the
pitch angle of the spiral arms, their organizational structure, and
the prominence of the bulge.  The Reference Catalog classifications of
de Vaucouleurs tend to emphasize the bulge, whereas the Revised
Shapley-Ames Catalog of Sandage and Tammann depends more strongly on
the spiral arms.

A common quantitative measure of morphology is the bulge-to-total
ratio $B/T$, which is the result of fitting a bulge and disk model to
a galaxy. By definition, this quantity ranges from zero to one, and
pure disk galaxies lie at zero and elliptical galaxies lie at one. A
model independent, related quantity is the {\it concentration},
defined by $r_{90}/r_{50}$, the 90\% to 50\% radius ratio.

The dark matter content of galaxies is revealed by their dynamics and
in some cases by gravitational lensing. Elliptical galaxies are
supported by a combination of rotation and velocity dispersion;
interpreting the dynamical masses of elliptical galaxies is complex
but achievable. The best evidence to date suggests that massive
elliptical galaxies have high mass-to-light ratios, which indicates
either a non-standard IMF that changes with stellar mass, or the
presence of dark matter.  Disk galaxies are primarily supported by
rotation. Particularly in their outer parts, the gas dynamics is
relatively straightforward to interpret and unambiguous signatures of
dark matter are present.

Two major scaling relations quantify the relationship between
luminosity and dynamics: The {\it fundamental plane} of giant
elliptical galaxies, the relationship between size, luminosity, and
velocity dispersion, which takes the approximate form:
\begin{equation}
\label{eq:fundamentalplane}
r_e \propto \sigma^\alpha I_e^{-\beta}
\end{equation}
which implies a plane in the three-dimensional space of the logarithms
of these quantities. The exercises below explore the implications of
this relationship (\citealt{bender92a}).

The {\it Tully-Fisher law} disk galaxies near $L_\ast$ relates their
luminosities and their peak circular velocities, as measured with
optical or radio observations (\citealt{tully77a}). Since the colors
of disk galaxies depend on luminosity, the slope of this relationship
depends on the band that the luminosity is measured in. In the
$I$-band, the relationship is equivalent to $L\propto v^{3.1}$, and
has a scatter of 0.2--0.4 dex (depending on the sample used). At low
luminosities, the velocity is independent of $L$ (but is dependent on
the total baryonic mass).

Luminous galaxies all seem to have a supermassive black holes at their
centers, with evidence for the black holes from active galactic
nucleus emission (called {\it Seyfert galaxies}) but also from
dynamical signatures through stellar dynamics, reverberation mapping,
and central masers.

\section{Commentary}

The description here relies on our best understanding of underlying
physical properties, but these properties are imperfectly measured,
especially with regards to inferred stellar populations. In the
stellar population notes, the uncertainties are more fully explored.

Unusual galaxies exist in the universe: there are ellipticals with gas
disks of with typically small but sometimes large masses, polar ring
galaxies with gas disks or rings perpendicular to the stellar disk,
galaxies with strong central star bursts, and other unusual cases. Of
growing interest are the populations of very low surface surface
brightness galaxies (not detectable by large surveys like SDSS) and
ultra-compact dwarf galaxies (indistinguishable from stars by large
surveys); it is unclear yet how common these cases are overall
relative to more ``typical'' galaxies.

\section{Key References}

\begin{itemize}
\item {\it Physical Properties and Environments of Nearby
Galaxies}, \citet{blanton09a}
\item {\it A Concise Reference to (Projected) S{\'e}rsic
Quantities}, \citet{Graham:2005p2523}
\end{itemize}

\section{Order-of-magnitude Exercises}

\begin{enumerate} 
\item Using the numbers in the text, estimate $\Omega_b$ and
    $\Omega_m$. 
\item Estimate the stellar mass surface density for a Milky Way-like
disk galaxy.
\end{enumerate} 

\section{Analytic Exercises}

\begin{enumerate}
\item The fundamental plane (Equation \ref{eq:fundamentalplane}) can
be reexpressed in quantities with a clearer physical
interpretation. The usual form is called $\kappa$-space
(\citealt{bender92a}): 
\begin{eqnarray}
\kappa_1 &\equiv& \left[ \log_{10} \sigma^2 + \log_{10} r_e \right]
/ \sqrt{2} \cr
\kappa_2 &\equiv& \left[ \log_{10} \sigma^2 + 2 \log_{10} I_e
- \log_{10} r_e \right] / \sqrt{6} \cr
\kappa_3 &\equiv& \left[ \log_{10} \sigma^2 - \log_{10} I_e
- \log_{10} r_e \right] / \sqrt{3}
\end{eqnarray}
where $\sigma$ is in km s$^{-1}$, $r_e$ is in kpc, and $I_e$ is in
$L_\odot$ kpc$^{-2}$. 
\begin{enumerate}
\item Show that $\kappa_1$ is proportional to the logarithm of the
virial mass.
\item Show that $\kappa_3$ is proportional to the logarithm of the
virial mass to stellar light ratio.
\item \cite{bender92a} and others find:
\begin{equation}
\kappa_3 \approx 0.15 \kappa_1 + 0.36.
\end{equation}
How does the mass-to-light ratio depend on virial mass (or
alternatively, on luminosity)?
\item Elliptical galaxies fall along the red sequence, which depends
on absolute magnitude as:
\begin{equation}
g-r \approx {\rm constant} - 0.03 M_r
\end{equation}
This trend implies that the stellar mass to light ratio is a function
of luminosity. Assuming that Equation \ref{eq:belldejong} is constant as a
function of luminosity, and a constant virial-to-stellar mass ratio,
what would this trend imply for the virial mass-to-light ratio
dependence on luminosity? Comment.
\end{enumerate}
\item The Tully-Fisher relation can be approximated under similarly
simple assumptions. Assume (a) a constant dynamical-to-stellar mass ratio,
(b) that the stellar mass-to-light ratio scales with the colors as in
Equation \ref{eq:belldejong}, with a color dependence on absolute
magnitude of: 
\begin{equation}
g-r \approx {\rm constant} - 0.06 M_r
\end{equation}
(again from \citealt{blanton03d}), and (c) that the observed
dependence of surface brightness on luminosity for exponential-profile
galaxies found in \citet{blanton03d} of $I\propto L^{1/4}$ in the
$i$-band (see Figure 14 and Table 1). Under these conditions and using
the virial relation, how does luminosity depend on circular velocity?

\begin{answer}
The \citet{bell00a} dependence of stellar mass-to-light ratio color
can be combined with the dependence of color on absolute magnitude:
\begin{eqnarray}
\log_{10} \left(\frac{M_\ast}{L}\right) &\approx& -0.3 + 1.1 (g-r) \cr
&\approx& \mathrm{constant} - 0.066 M_r  \cr
\end{eqnarray}
or equivalently:
\begin{eqnarray}
\left(\frac{M_\ast}{L}\right) &\propto& L^{(-0.066)(-2.5)} \cr
\left(\frac{M_\ast}{L}\right) &\propto& L^{0.17} \cr
M_\ast &\propto& L^{1.17}
\end{eqnarray}
The stellar radius (the scale out to which the circular velocity is
probed by an optical rotation curve) is related to the surface
brightness and luminosity as:
\begin{equation}
L \propto I_\ast r_\ast^2 \propto L^{1/4} r_\ast^2,
\end{equation}
where we have (somewhat shakily) used $i$-band relationship between
$I$ and $L$.  Rearranging we find:
\begin{equation}
r_\ast \propto L^{3/8}
\end{equation}
If the dynamical-to-stellar mass ratio is constant then the virial
relation gives:
\begin{equation}
M_\ast \propto L^{1.17} \propto v^2 r_\ast \propto v^2 L^{3/8}
\end{equation}
or equivalently:
\begin{equation}
L \propto v^{2.5}
\end{equation}
This obviously is a rough approximation because the relationship
between the dynamical and stellar mass depends a lot on the radius at
which you define it.
\end{answer}
\end{enumerate}

\section{Numerics and Data Exercises}

\begin{enumerate}
\item Download a catalog SDSS Legacy Survey galaxies in the Main
Sample between redshifts $0.01 < z < 0.2$. Estimate the shape of the
luminosity function in the $r$-band using the $1/V_{\rm max}$
method. In calculating $V_{\rm max}$ and luminosities, ignore both
$K$-corrections and evolution of the population; a precise estimate
would require estimating both, but in the $r$-band it is approximately
the case that these effects cancel. Also account only for the redshift
selection effects, and for the angular selection effects simply assume
an effective area of 7500 square degrees.
\item Using a Schechter function approximation to the luminosity
function in the optical (e.g. from the $r$-band as found
in \citet{blanton03c}) compare to the predicted mass function of halos
(e.g. from Appendix C of \citealt{tinker08a}). Specifically, find and
plot the (zero-scatter) relationship between $L$ and $M_{\rm halo}$
that satisfies the abundance-matching requirement:
\begin{equation}
\Phi\left(>L(M_{\rm halo})\right) = \Phi\left(>M_{\rm halo}\right).
\end{equation}
\item Download the catalog of galaxies from
the \href{http://nsatlas.org}{NASA Sloan Atlas} between redshifts
$0.01 < z <0.05$. Plot their absolute magnitude vs. color. Select
several galaxies along the red and blue sequences, and download and
show their color images. For the same galaxies, download from the SDSS
database their spectra. Zoom in on the H$\alpha$ region and the
4000 \AA region. Comment on the major differences between the blue and
red galaxies.
\item With the NASA Sloan Atlas, for red and blue galaxies separately, plot the
distribution of their axis ratios $b/a$. Show the color images for
several high and low axis ratio cases. Choose galaxies (red or blue)
with $n<2$ and plot their color vs. axis ratio --- can you explain the
correlation that you see?
\item For five red sequence galaxies from the NASA Sloan Atlas with
between about $10^9$--$10^{11}$ $M_\odot$, download their
spectra. Choose bright galaxies (e.g. $m_r < 16$) to maximize
signal-to-noise ratio. Plot the spectra around the region of the Mg
$b$ signature, which is an important absorption line in red giant
stars (which dominate the light of red galaxies with old stellar
populations). Normalize the spectra so you can compare the widths of
the absorption line. If you assume the line is intrinsically narrow,
can you determine the velocity dispersion of the stars in the galaxy?
\item Install the {\tt sdss-marvin} package to get MaNGA data within
Python. Search for MaNGA IFU observations of several nearly edge-on
galaxies on the blue sequence between $10^9$--$10^{11}$
$M_\odot$. Plot the rotation curve for each one just by plotting the
velocity as a function of position along the major axis. How does the
maximum rotation velocity depend on stellar mass?
\end{enumerate}

\bibliographystyle{apj}
\bibliography{exex}  
