\title{\bf Interstellar Medium}

\section{Basics \& Nomenclature}

The {\it interstellar medium} is the medium between stars in
galaxies. Here we will include the {\it circumgalactic medium} in our
discussion, though usually that refers to diffuse gas which may be
bound to the galaxy but is not intermixed with its main population of
stars.

The contents of the interstellar medium are: gas (ions, atoms,
molecules); dust (small solid particles, mostly $< 1\mu$m; cosmic rays
(ions and electrons, distinguished from the ``gas'' by its nonthermal
energy distribution extending to as high as $10^{21}$ eV); radiation
(from many sources); and dark matter.  The interstellar medium is also
threaded by magnetic fields.

[phases of ISM] 

[signatures in emission]

[signatures in absorption]

% \section{Commentary}

\section{Key References}

\begin{itemize}
  \item
    \href{http://adsabs.harvard.edu/abs/2000asqu.book.....C}{
    {\it Allen's Astrophysical Quantities},
      \citet{cox00a}}, Chapter 5
\end{itemize}

\section{Order-of-magnitude Exercises}

\begin{enumerate} 
\item As you can see when looking outside during the day, the Sun is
    neither very blue nor very red. Assuming it emits approximately as
    a blackbody, estimate the temperature of its surface.

\item Estimate the approximate temperature of a radiation field that
    will provide a substantial flux of photons to ionize hydrogen.

\item If you have a spectrograph with $R\sim 4000$, for what
line-of-sight velocity dispersion is the intrinsic width of the line
equal to the width due to the resolution? [We will learn later that
depending on signal-to-noise ratio, velocities much smaller than the
resolution are hard to measure.]

\item Galaxy clusters emit thermal bremstrahlung at energies $\nu > 1$
keV. What is the temperature necessary to do this? 

\item The center of the Milky Way is very heavily extincted: by about
    30 magnitudes in the $V$ band. Approximately how much is that in
    the near-infrared $K$ band?

  \ifanswers \textcolor{blue}{ The magnitude difference of 30
    corresponds to a factor of $10^{12}$ in luminosity. Using Equation
    \ref{eq:luminosity_distance}, this translates to $10^6$ in
    distance. So Vega-like stars are visible (in principle) to about 8
    Mpc.  }
  \fi

\item Estimate the number of photons per second that enter your eye
    per second in visible light (4000--7000 \AA) from a star with
  magnitude $\sim 6$ (about the faintest visible at a dark
  site). Assume a nighttime pupil diameter of 5 mm.

  \ifanswers
\textcolor{blue}{
In this wavelength range, Vega and AB magnitudes are about the same at
the precision necessary here, so we don't have to worry about which
version we are dealing with. So:
\begin{equation}
f_\nu \sim (3631 \mathrm{~Jy}) 10^{-0.4 m} \sim 14 \mathrm{~Jy} = 1.4 \times
10^{-22} \mathrm{~erg} \mathrm{~cm}^{-2} \mathrm{~s}^{-1} \mathrm{~Hz}^{-1}
\end{equation}
The flux density in the visible should be $f_\nu \Delta\nu$ where:
\begin{equation}
\Delta\nu = c \left(\frac{1}{4000 \mathrm{~\AA}} -
\frac{1}{7000 \mathrm{~\AA}}\right) \sim 320 \mathrm{~THz}
\end{equation}
And thus $f \sim 4\times 10^{-8}$ erg cm$^{-2}$ s$^{-1}$. Each photon
has an energy (assuming $\lambda = 5500$ \AA):
\begin{equation}
E = h\nu = (6.62\times
10^{-27} \mathrm{~erg~Hz}^{-1})( \mathrm{550~THz}) \sim 4 \times
10^{-12} \mathrm{~erg}
\end{equation}
So the flux of photons is:
\begin{equation}
\frac{\dot N}{A} = \frac{f_\nu \Delta\nu}{h\nu} \sim 
10^4 \mathrm{~s}^{-1} \mathrm{~cm}^{-2}
\end{equation}
If $A \sim \pi r^2 \sim 0.2$ cm$^2$ then $\dot N \sim 2000$ s$^{-1}$. 
}
 \fi

\item How much does surface brightness dimming change the magnitudes
  per square arcsecond for a galaxy at redshift $z\sim 1$?

  \ifanswers
\textcolor{blue}{
The specific intensity is reduced by $(1+z)^4$. In magnitudes this is:
\begin{equation}
\Delta m = 2.5 \log_{10} (1+z)^4 = 10 \log_{10} (1+z) \sim
3 \mathrm{~mag}
\end{equation}
}
\fi
\end{enumerate}   

\section{Analytic Exercises}

\begin{enumerate}
\item For a Gaussian line spread function with a standard deviation
  $\sigma$, what is the FWHM?
  \ifanswers
\textcolor{blue}{
The FWHM is determined by:
\begin{equation}
\Delta\lambda = 2  \ln(0.5)
\end{equation}
}
\fi
\item Prove Equation \ref{eq:sb_dimming}, based on the fact that
  photon density in phase space is conserved.
\item Based on Equation \ref{eq:sb_dimming}, how is the angular
  diameter distance related to the luminosity distance? 
\end{enumerate}

\section{Numerics and Data Exercises}

\begin{enumerate}
\item Retrieve a spectrum of a star, a quasar, and a galaxy from the
  Sloan Digital Sky Survey. Plot each of them. These spectra are given
  in $f_\lambda$ (per-\AA) units. Convert one them to $f_\nu$
  (per-Hertz) and plot it. Smooth one of them in $f_\lambda$ with a
  Gaussian corresponding to $R\sim 100$ and plot it.
\item Plot $D_L$ versus $z$ based on the equations found
  in \citet{hogg99cosm}, for a flat $\Lambda$CDM cosmology with
  $\Omega_m = 0.3$ and $H_0 = 70$ km s$^{-1}$ Mpc$^{-1}$. Determine
  where the difference in inferred luminosity of an object would reach
  1\%. 
\item Download the filter curve for the SDSS $g$ and $r$
  bands. Calculate the observed $g$ and $r$ band magnitudes
  corresponding to a galaxy spectrum (say for some galaxy with
  $z<0.1$). Note that this won't necessarily be the same as the
  magnitudes measured from the images, since the spectra are taken
  through 2- or 3-arcsec diameter fibers. Calculate the rest-frame
  $g-r$ color, and also what the $K$-correction would be for galaxies
  with this SED in the $r$-band between about $z\sim 0$ and $z\sim
  0.25$. Download a sample of galaxies between about $z\sim 0$ and
  $0.25$. Plot their $g-r$ colors versus redshift, together with the
  predicted colors of the galaxy you have a spectrum of.
\item \todo{Would be nice to have radio, X-ray, other examples}
\end{enumerate}

\bibliographystyle{apj}
\bibliography{exex}  
