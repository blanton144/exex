\title{\bf Star formation}

\section{Basics}

Star formation is a fundamental process in forming galaxies. It
proceeds through the cooling and gravitational collapse of gas within
gas {\it clouds}. Based on molecular gas measurements, the mass
function of clouds has a slope of about $\alpha\sim 1.5$--2
($\dd{N}/\dd{M} = M^{-\alpha}$) between $10^3$ and $10^6$ $M_\odot$.
During the collapse process, the gas becomes cold enough for molecules
to form, and stars form always within molecular clouds. Individual
stars form in a range of masses. Light from the most massive forming
stars tends to disassociate and then ionize the surrounding gas
cloud. Dust in the cloud absorbs and is heated by ultraviolet light
from the stars. The {\it star clusters} that result seem to have a
similar mass distribution to the gas clouds (though it is thought that
clusters form within isolated {\it clumps} within the clouds).

\subsection{Initial mass function}

The stars within the clusters form over a large range of masses. The
initial mass function estimated within the Milky Way appears to peak
between 0.1 and 0.3 $M_\odot$, and at higher masses decline with a
slope close to $-2.35$ (the {\it Salpeter} slope). This initial mass
function may depend on environment in various ways.  Estimates of star
formation rate integrate the mass of stars extending down to the
hydrogen burning limit ($0.08$ $M_\odot$) even though the star
formation indicators are only directly sensitive to much more massive
stars.

\subsection{Color-magnitude diagram}

For systems whose stars can be resolved from each other, we can
estimate the star formation rate by modeling the color-magnitude
diagram, sometimes using all the stars or sometimes only the most
massive O-type or Wolf-Rayet stars. This method can estimate the star
formation rate over the past 100 million years or so. Systems like the
Orion Nebula Cluster can be studied in this way, and with Hubble
Space Telescope observations nearby galaxies can be mapped in their
star formation rate.

\subsection{Ultraviolet light}

More useful for most galaxies is the integrated ultraviolet continuum
light, which (depending on the IMF) traces stars of a few solar
masses, and consequently star formation over the past 100 million
years.

However, dust attenuates the ultraviolet light strongly. In principle,
the slope of the ultraviolet continuum constrains the dust
attenuation. Because the dust re-emits in the infrared, this leads to
a relationship between the ultraviolet spectral slope $\beta$ and the
ratio of the infrared to ultraviolet light
(``IRX''; \citealt{overzier11a, grasha13a}). However, in practice
there appears to be too much scatter in the dust attenuation as a
function of reddening for this correction to be sufficient.

\subsection{Emission lines}

The most massive young stars ($>15$ $M_\odot$) produce large ionized
{\it HII regions} surrounding them. Recombination in these regions
produces a sequence of Balmer lines (and also higher wavelength
Paschen and Brackett lines) with line ratios that are only a weak
function of the gas conditions. To a good approximation, there is one
Balmer line produced for every ionized photon. Therefore, Balmer lines
can be used to determine the total ionizing flux, which is related to
the number of massive stars, which due to their short lifetimes is
related to the star formation rate within the past 10 million
years. H$\alpha$ is the usual Balmer line of choice as the brightest
and reddest, to minimize dust extinction. The Paschen and Brackett
lines are more robust but less accessible with ground-based
instruments.

The H$\alpha$ emission needs to be corrected for dust as well.
Typically, we use the {\it Balmer decrement}, the ratio of the Balmer
lines compared to the (unextincted) theoretical expectation. With an
assumed extinction curve, the Balmer decrement yields an extinction
correction. For low attenuation levels (up to about 1 magnitude at
H$\alpha$), this method is sufficient but it becomes uncertain when
there is large small scale variation in the extinction, which is
typical at high extinction.

\subsection{Infrared light}

Dust extinction of the ultraviolet light leads to reemission at
infrared wavelengths. This emission is a mix of lines from polycyclic
aromatic hydrocarbons (PAHs) and thermal emission from dust grains at
a mix of temperatures. PAHs are typically only $\sim 1$\% of the dust
map but dominate the emission between 5--20 $\mu$m. Longward of 20
$\mu$m thermal emission from dust grains dominates, though note there
are some atomic lines from the gas.

The total infrared luminosity plus the total ultraviolet light traces
the total star formation, in principle. However, evolved stars older
than a few hundred million years can heat the dust as well. This
heating typically occurs more diffusely throughout the galaxy and is
less concentrated in star forming regions; it can contribute to up to
half the infrared signal for low specific star formation rate
systems.

A number of calibrations have been performed that are useful if one
only has infrared observations or only PAH measurements (e.g. for
higher redshift systems).

\subsection{Radio continuum}

The radio continuum of galaxies correlates with other measures of star
formation. There are two components, a thermal free-free component
which at cm wavelengths is in the optically thin limit so relatively
flat, and a synchrotron component with a steeper spectrum. The
free-free component should correlate with the ionizing luminosity that
heats the plasma. The synchrotron component correlates with other star
formation indicators for sufficiently luminous galaxies, but the
physical reasons behind this correlation are unknown.

The synchrotron component dominates, so high enough signal-to-noise
ratio multifrequency data is necessary to measure the free-free
emission. Most radio measurements of star formation are based on 20 cm
(1.4 GHz) continuum measurements of the synchrotron.

\subsection{Kennicutt-Schmidt Law}

The Kennicutt-Schmidt law relates the surface density of atomic and
molecular gas to the surface density of star formation, averaged over
the galactic disk. Above $\Sigma_{\rm gas} = 10$ $M_\odot$ pc$^{-2}$,
galaxies obey $\Sigma_{\rm SF} \propto \Sigma_{\rm gas}^n$ with $n\sim
1.4$, with a scatter of about 0.3 dex. At lower densities, there is
less star formation and more scatter than the power law predicts. Note
that star formation surface density is linearly related to the total
dense molecular gas (as traced by HCN for example). When comparing
the star formation locally within galaxies, similar trends are seen,
with a similar threshold density.

The physical causes of the Kennicutt-Schmidt law are unclear. The
scaling can be derived if the star formation rate is driven by large
scale instabilities in the gas, leading the rate to be proportional to
the free fall time $t_{\rm ff}$ in the disk, with the depletion time
$M_\ast/\dot M_\ast \sim 100 t_{\rm ff}$. However, there are many
intermediate processes between the large scale gravitational collapse
of the gas and the production of stars. The Kennicutt-Schmidt law
could be explained by an appropriate variation of the fraction of
molecular gas as a function of density and a constant conversion rate
of molecular gas into stars.

The cause of the threshold at 10 $M_\odot$ pc$^{-2}$ is also not
clear. If star formation is initiated by large scale instabilities,
then the {\it Toomre instability criterion} must be satisfied
(\citealt{toomre64a}). A thin gas disk is unstable to large scale
perturbations when:
\begin{equation}
Q_{\rm gas} = \frac{\sigma_g \kappa}{\pi G \Sigma_{\rm gas}} < 1
\end{equation}
where $\kappa$ is the epicyclic frequency of the disk, and $\sigma_g$
is the velocity dispersion of the gas. For higher $Q$, the Coriolis
forces in the disk prevent large scale collapse even above the Jeans
mass. In the $Q$ value in disks tends to correlate with many other
parameters and it is difficult to disentangle whether the observed
threshold is due to the Toomre criterion or a different one
(\citealt{leroy08a}).

\section{Key References}

\begin{itemize}
  \item \citet{kennicutt12a}
\end{itemize}

\section{Order-of-magnitude Exercises}

\begin{enumerate} 
\item Assuming that the observables are unchanged, what is the
difference in inferred star formation rate between assuming a Salpeter
function all the way to the hydrogen burning limit, and one that turns
over to a constant below 0.5 $M_\odot$?
\end{enumerate}   

\section{Analytic Exercises}

\begin{enumerate}
\item Derive KS law
\item Toomre criterion
\end{enumerate}

\section{Numerics and Data Exercises}

\begin{enumerate}
\item SP models and UV star formation rate
\item Comparing SFRs for a single galaxy from UV, Halpha, IR.
\item Balmer decrement
\item Strongest star formers
\item XUV disks
\end{enumerate}

\bibliographystyle{apj}
\bibliography{exex}  
