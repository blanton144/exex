\title{\bf Optical Emission Line Spectra}

\section{Basics \& Nomenclature}

Optical (and near ultraviolet and near infrared) emission lines of
galaxies are predominantly associated with ionized gas. The emission
lines are the result typically of {\it recombination radiation} or
{\it collisionally excitation}. The physical conditions of ionized gas
in galaxies varies by large orders of magnitude in density, from 0.3
to $10^4$ cm$^{-3}$ but typically has temperatures around $10^4$ K.

In astronomical spectroscopy, the ionization state is indicated by
Roman numerals. XI indicates neutral ``X,'' XII indicates
singly-ionized ``X,'' XIII indicates doubly ionized ``X,'' etc. 

The ionization state of gas is determined by the balance among effects
such as photoionization, collisional ionization, and recombination. In
the interstellar medium, the radiation field is generally far from
blackbody and statistical mechanical rules like the law of mass action
that would otherwise govern the ionization state do not hold.

\subsection{Photoionization}

[photoionization]

[other sources of ionization: collisional, etc]

\subsection{Recombination radiation}

[cross section]

In ionized gas, hydrogen and other elements have electrons recombining
at some rate that balances the ionization rate. Typically the
recombination is to a high energy bound state. The electron will
thereafter decay into successively lower states. For hydrogen, two
limiting cases exist:
\begin{itemize}
\item Case A: optically thin to ionizing photons. The recombination
  rate is the sum of the recombination rates to each energy level
  indexed by quantum states $n$ and $l$:
  \begin{equation}
    \alpha_A(T) =
    \sum_{n=1}^{\infty} 
    \sum_{l=0}^{n-1} \alpha_{nl}(T)
  \end{equation}
\item Case B: optically thick to photons at energies just above the H
  ionization limit (13.6 eV). The recombinations to $n=1$ do not
  count, because they just result in an ionizing photon that is
  immediately recaptured:
  \begin{equation}
    \alpha_B(T) = \alpha_A(T) - \alpha_{1s} (T)
  \end{equation}
\end{itemize}
For most cases within the interstellar medium, Case B is appropriate
because enough neutral H is available, even in HII regions. In such
cases, the system is optically thick to all Lyman-series photons.

The Case A recombination spectrum can be calculated by assuming the
levels are populated by recombination as above, and then using the
decay probabilities from each state. The Case B recombination spectrum
is calculated the same way, but just taking all transitions to $n=1$
out of the picture. The temperature enters the calculation weakly due
to its effect on the recombination coefficients $\alpha$. The density
enters the calculation even more weakly (until $n>10^6$ cm$^{-3}$),
because collisions affect the high $n$ levels.

In HII regions, where Case B holds, Lyman lines have a special
fate. They undergo {\it resonant scattering}; every Lyman line emitted
is very quickly reabsorbed. Since they undergo many scatterings, and
also can decay to lower levels between scatterings, eventually they
result in a Lyman-$\alpha$ photon. In HII regions, they can escape
only through a rare two-photon decay to a continuum (when the state is
2s) or by being Doppler scattered into a wing of the line. 

Helium also contributes recombination radiation. For very hard
photoionization sources, He III will predominate. This is a
hydrogen-like system with an ionization energy four times as
large. The emission lines are shifted in energy by that much, and the
overall pattern of transitions in gas at temperature $T$ is the same
as hydrogen for a gas with temperature $T/4$.

[Other elements]

\subsection{Collisional excitation}

\subsection{Measurements of emission lines}

\section{Key References}

\begin{itemize}
  \item
    \href{http://}
    {\it Physics of the Interstellar and Intergalactic Medium,
      \citet{draine07a}}
\end{itemize}

\section{Order-of-magnitude Exercises}

\begin{enumerate} 
\item Dependence on temperature and resulting metallicity dependence
  of Halpha 
\end{enumerate}   

\section{Analytic Exercises}

\begin{itemize}
\item alpha calculation

\item Ly-alpha scattering
\end{itemize}

\section{Numerics and Data Exercises}

\begin{enumerate}
\item Use of Balmer decrement for dust
\item Running MAPPINGS or other
\end{enumerate}`

\bibliographystyle{apj}
\bibliography{exex}  
