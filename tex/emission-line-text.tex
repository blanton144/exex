\title{\bf Optical Emission Line Spectra}

\section{Basics \& Nomenclature}

Optical (and near ultraviolet and near infrared) emission lines of
galaxies are predominantly associated with ionized gas. The emission
lines are the result typically of {\it recombination radiation} or
{\it collisionally excitation}. The physical conditions of ionized gas
in galaxies varies by large orders of magnitude in density, from 0.3
to $10^4$ cm$^{-3}$ (up to $10^{10}$ cm$^{-3}$ or more in quasars) but
typically has temperatures around $10^4$ K.

In astronomical spectroscopy, the ionization state is indicated by
Roman numerals. XI indicates neutral ``X,'' XII indicates
singly-ionized ``X,'' XIII indicates doubly ionized ``X,'' etc.

The electronic states can be classified by their spin, orbital, and
total angular momentum, and is typically labeled ${}^{2S+1}L_J$, where $S$
is the total spin, $L=S, P, D, F, G, \ldots$ is the total orbital
angular momentum (corresponding to $L=0,1,\ldots$), and $J$ is the
total sum. So ${}^3P_0$ indicates an atom with $S=1$, $P=1$, and
$J=0$.

A critical number for any radiative transition is the transition
probability $A$, sometimes known as the {\it Einstein A coefficient},
and typically expressed in units of s$^{-1}$.

\subsection{Ionization processes}

The ionization state of gas is determined by the balance among effects
such as photoionization, collisional ionization, and recombination. In
the interstellar medium, the radiation field is generally far from
blackbody and statistical mechanical rules like the law of mass action
that would otherwise govern the ionization state do not hold.

In many applications in extragalactic astrophysics we are concerned
with photoionized gas, in which photons of sufficient energy liberate
electrons. The cross-section for photoionization is maximum at the
ionization energy; it declines as $E^{-3}$ (or equivalently
$\nu^{-3}$) at higher energy, so photoionization for most ionizing
spectra is dominated by photons near the threshold.

For hydrogen in the electronic ground state, the {\it ionization
potential} is 13.6 eV, corresponding to a photon of wavelength
912 \AA. The ionization of other atoms is strongly affected by their
ionization energy. Helium has an ionization potential of 24.6 eV, and
HeII (He$^+$) has an ionization potential of 54.4 eV. For single
ionization, carbon, nitrogen, and oxygen have similar ionization
potentials to hydrogen, whereas neon has a slightly deeper potential,
and silicon and sulfur have slightly shallower potentials. For more
highly ionized species, the ionization potential becomes deeper in
general. 

The electron temperatures in the interstellar medium ($\sim 10^4$ K)
usually correspond to energies $kT$ far below these ionization
potentials (which require temperatures $\sim 10^5$ K). Therefore
collisional ionization is usually not important. However, in shock
heated gas (for example in supernova remnants) or in some regions of
active galactic nuclei these high temperatures can be achieved.

\subsection{Recombination}

In ionized gas at equilibrium, hydrogen and other elements have
electrons recombining at some rate that balances the ionization rate.

The recombination rate for hydrogen can be written:
\begin{equation}
R = n_e n_p \alpha(T)
\end{equation}
The recombination coefficient $\alpha_{nl}$ (usually in units
cm$^{3}$ s$^{-1}$) for a hydrogen atom to some energy level $n$ and
orbital angular momentum state $l$ can be expressed in terms of the
cross-section to recombination $\sigma_{nl}$:
\begin{equation}
\label{eq:alpha}
\alpha_{nl}(T) = \int_0^\infty \dd{u}\, u f(u) \sigma_{nl}(u)
\end{equation}
where $f(u)$ is the Maxwell-Boltzmann distribution. Approximately,
$\sigma \propto u^{-2}$, so $\alpha\propto \langle 1/u\rangle \propto
T^{-1/2}$.

Typically the recombination is to a high energy bound state. The
electron will thereafter decay into successively lower states. For
hydrogen, two limiting cases exist:
\begin{itemize}
\item Case A: optically thin to ionizing photons. The recombination
  rate is the sum of the recombination rates to each energy level
  indexed by quantum states $n$ and $l$:
  \begin{equation}
    \alpha_A(T) =
    \sum_{n=1}^{\infty} 
    \sum_{l=0}^{n-1} \alpha_{nl}(T)
  \end{equation}
\item Case B: optically thick to photons at energies just above the H
  ionization limit (13.6 eV). The recombinations to $n=1$ do not
  count, because they just result in an ionizing photon that is
  immediately recaptured:
  \begin{equation}
    \alpha_B(T) = \alpha_A(T) - \alpha_{1s} (T)
  \end{equation}
\end{itemize}
For most cases within the interstellar medium, Case B is appropriate
because enough neutral H is available, even in HII regions. In such
cases, the system is optically thick to all Lyman-series photons.

The Case A recombination spectrum can be calculated by assuming the
levels are populated by recombination as above, and then using the
decay probabilities from each state. The Case B recombination spectrum
is calculated the same way, but just taking all transitions to $n=1$
out of the picture. The temperature enters the calculation weakly due
to its effect on the recombination coefficients $\alpha$. The density
enters the calculation even more weakly (until $n>10^6$ cm$^{-3}$),
because collisions affect the high $n$ levels.

In HII regions, where Case B holds, Lyman lines have a special
fate. They undergo {\it resonant scattering}; every Lyman line emitted
is very quickly reabsorbed. Since they undergo many scatterings, and
also can decay to lower levels between scatterings, eventually they
result in a Lyman-$\alpha$ photon. In HII regions, they can escape
only through a rare two-photon decay to a continuum (when the state is
2s) or by being Doppler scattered into a wing of the line. 

Helium also contributes recombination radiation. For very hard
photoionization sources, He III will predominate. This is a
hydrogen-like system with an ionization energy four times as
large. The emission lines are shifted in energy by that much, and the
overall pattern of transitions in gas at temperature $T$ is the same
as hydrogen for a gas with temperature $T/4$.

\subsection{Collisional excitation}

Heavier elements than hydrogen tend to be incompletely ionized in gas
nebulae. For such elements, the remaining electrons can be excited by
collisions, and then decay radiatively. The excitation potential for
He$+$ is too deep for most nebular temperatures, but O$+$, O$++$,
N$+$, and many other species have excited states within reach of
collisions in $10^4$ K gas. Meanwhile, these excited states can have
short enough radiative decay times that at the density of nebular gas
they can decay radiatively prior to collisional deexcitation. The
resulting lines are called {\it collisionally excited lines}. Since
under terrestrial conditions collisional deexcitation so greatly
dominates, they are also called {\it forbidden lines}.

Among the collisionally excited lines, depending on the radiative
decay times, there are both {\it strong lines} that are bright and
easy to measure, and other lines sometimes referred to as {\it auroral
lines} (particularly [OIII] 4363) that tend to be fainter.

\subsection{Measurements of emission lines}

We can measure recombination lines and collisionally excited lines in
spectra of galaxies.  These measurements can constrain properties of
the ionized gas, such as its temperature, density, ionization source,
and metallicity.

A critical issue for measurement is determination and subtraction of
the {\it continuum} underneath the line, which is primarily due to
stars in the galaxies. The stellar continuum is not smooth, which
means that subtracting a low order fit near the emission line location
is not sufficient. Modern measurements of emission lines use stellar
population models to fit the continuum (most often, excluding the
wavelengths very close to any emission lines from the fit), or
sometimes empirically based models.  This process is especially
important for Balmer lines, where there is underlying Balmer
absorption in the stellar continuum, but affects all lines at some
level.

Once the continuum is subtracted, the line can be measured in multiple
ways, either by integrating over some fixed wavelength aperture or
fitting a Gaussian or other profile to the line. Because lines can
vary in their width due to different Doppler shifts, usually the
latter route is taken to avoid wavelength aperture effects and to
maximize signal-to-noise ratio. The flux of the line can be then
inferred from the fit.

The Doppler widths are typically 10--20 km s$^{-1}$ locally within
star forming regions, a few hundred km s$^{-1}$ in narrow line regions
of AGN, and a few thousaid km s$^{-1}$ in broad line regions. For
spectra that integrate over substantial parts of galaxies, the line
widths may be larger because of large scale motions such as rotation.

Emission line strengths can be quantified also by their equivalent
width, which is their flux divided by the local flux density of the
continuum. The equivalent width is handy in cases that the
spectrophotometry is not well understood, but fundamentally it is a
combination of the emission line properties and the stellar continuum
properties so can be more difficult to interpret in terms of physical
parameters.

\subsection{Temperature diagnostics}

A standard use for emission line measurements is to determine the
electron temperature within the ionized gas. The temperature is
interesting because it is set by the physical conditions of the nebula
and it also is an important parameter to know to interpret the line
emission in terms of metallicity and ionization source.

Emission lines from OIII, NII, NeIII, and SIII can used to determine
temperature, because they have excited states that are low enough to
be well populated in $10^4$ K gas but well enough separated to have
significantly different Boltzmann factors, and in addition decay
through channels that release optical emission lines. The first
excited electronic state ${}^{1}D_2$ can decay to several nearly
degenerate ground state levels ${}^3P_0$, ${}^3P_1$, or ${}^3P_2$; the
transition to the first requires a quadrupole transition so has a slow
rate. The other two correspond to the [OIII] 4959 and 5007 lines and
the [NII] 6583 and 6548 lines.

A second excited electronic state ${}^1S_0$ can decay to the ground
states ([OIII] 2321 or [NII] 3063) but in the optical also emits the
so-called auroral line to ${}^1D_2$ ([OIII] 4363 or [NII] 5755). These
lines are weak both because the higher state has a larger Boltzmann
factor and because the decay probability to the intermediate state is
lower. 

At low enough densities ($n<10^5$ cm$^{-3}$), the populations are not
in equilibrium with the electrons; every collision is followed by a
radiative transition.  The ratio of the collisional excitation rates
to the ${}^{1}D_2$ and ${}^1S_0$ populations is equal to:
\begin{equation}
\frac{\Upsilon({}^3P, {}^{1}D_2)}
{\Upsilon({}^3P, {}^1S_0)} \exp\left(\Delta E / kT \right)
\end{equation}
where $\Delta E$ is the energy difference between the ${}^{1}D_2$ and
${}^1S_0$ levels, and $\Upsilon$ is the collision strength (which is
itself a function of temperature). Therefore, if the ratio of the
auroral lines and the strong lines can be measured, and the atomic
data on transition probabilities and collision strengths is known,
then the temperature can be measured.

At higher densities, collisional deexcitation tends to lower the
 populations in ${}^{1}D_2$ and weaken the pair of strong lines. The
 collision rate scales as $n_e/T^{1/2}$ and it therefore turns out
 that the weakening of the line ratio is proportional to:
 \begin{equation}
\frac{1}{1 + \alpha n_e / T^{1/2}}
 \end{equation} where $\alpha \sim 10^{-5}$--$10^{-3}$ cm$^{3}$
K$^{1/2}$, depending on the species.

\subsection{Density diagnostics}

Like temperature, electron density within the ionized gas can be
determined from line ratios and knowledge of it helps determine the
gas metallicity and ionization source. The line ratios of interest are
between lines with nearly the same excitation energy, but different
radiative transition probabilities or collisional deexcitation rates.

At low density, every collisional excitation is followed by a
radiative transition, so the fluxes of the lines are proportional to
the excitation rates (often just the statistical weights of the
levels).

At high density, the population levels are in thermal equilibrium so
that their relative abundance is just their statistical weight
(because their energies are nearly identical). The line strength
ratios then are just the statistical weights times the transition
probabilities. 

The transition between the two regimes is defined by the {\it critical
density}, at which the radiative transition rates are equal to the
collisional transition rates. At $T\sim 10^4$ K these critical
densities are $\sim 10^3$--$10^5$ for many of the species of
interest. The radiative transition rates are constant, whereas the
collisional rates have a dependence of $n_e T^{-1/2}$, so the
line ratio dependence on density shifts to higher density as
temperature increases. 

The relevant lines for this measurement are usually [OII] 3729 and
3726 or SII 6731 and 6716 (from the ${}^2D$ levels of each
singly-ionized atom).

\subsection{Ionization spectrum indicators}

The ionization spectrum can also be constrained with line ratios. In
the context of galactic observations, indicators exist for both the
total flux of ionizing photons, and the hardness of the ionizing
spectrum.

The ionizing flux can be quantified by the ionization parameter,
usually quantified as:
\begin{equation}
U = \frac{Q_{H}}{n_H c},
\end{equation}
where $Q_H$ is the flux (for example, in units cm$^{-2}$ s$^{-1}$) of
hydrogen-ionizing photons and $n_H$ is the number density of hydrogen
atoms. Sometimes the quantity $q=U c$ is used instead. In addition,
there is some ambiguity to the definition in the context of
models. For plane-parallel photoionization models, it is evaluated at
the point that the ionization flux impinges on the gas, but for
spherical models an inner radius of the region must be defined.

An important indicator of ionizing parameter is the [OIII] line. A
higher flux of photons will allow the gas to maintain a larger
population of doubly-ionized oxygen atoms. The line ratio [OIII]/[OII]
is typically used, though it is also somewhat metallicity
dependence. The [OIII]/[SII] also can be used, though it too has some
sensitivity to metallicity.  .

The hardness of the radiation field can be measured, somewhat
counterintuitively, by the low-ionization lines of NII and
SII. Because the photoionization cross-section declines strongly with
frequency above the ionization threshold, the mean free path for these
photons are longer. This mean free path sets the size of the partially
ionized portion at the edge of the ionized region. These regions
contain considerable OI, NII, and SII, and therefore hard radiation
fields tend to emit these lines strongly.

\subsection{BPT diagrams}

\subsection{Metallicity diagnostics for star-forming galaxies}

\section{Commentary}

An entire course may be given on the astrophysics of nebulae. This
description is extremely simplified. It ignores the detailed
derivation of the equilibrium populations as a function of density and
temperature in the idealized cases, as well the effects of realistic
issues such as fluctuations in density and temperature in the ionized
gas. It also ignores many diagnostics, particularly those outside the
optical.

\section{Key References}

\begin{itemize}
  \item
    \href{http://}
    {\it Physics of the Interstellar and Intergalactic Medium,
      \citet{draine07a}}
  \item
    \href{http://}
    {\it Osterbrock
      \citet{osterbrock06a}}
  \item
    \href{http://}
    {\it Blanc
      \citet{blanc15a}}
\end{itemize}

\section{Order-of-magnitude Exercises}

\begin{enumerate} 
\item Dependence on temperature and resulting metallicity dependence
  of Halpha 
\item Using typical mass-to-light ratios in the $i$-band, and star
formation rate normalizations for H$\alpha$, approximately relate a
H$\alpha$ equivalent width to the equivalent specific star formation
rate.
\end{enumerate}   

\section{Analytic Exercises}

\begin{itemize}
\item Show explicitly from Equation \ref{eq:alpha} that the
recombination coefficients should scale
as $\alpha\propto T^{-1/2}$.
\item alpha calculation
\item Calculating line ratios vs. temperature
\item Ly-alpha scattering
\end{itemize}

\section{Numerics and Data Exercises}

\begin{enumerate}
\item Use of Balmer decrement for dust
\item Running MAPPINGS or other
\end{enumerate}`

\bibliographystyle{apj}
\bibliography{exex}  
