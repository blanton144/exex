\title{\bf Inventory}

\section{Basics}

We begin with a basic inventory of what our subject is, and the
relevant length and time scales.

The Sun is a not-untypical star in the Milky Way galaxy. It is a hot
ball of gas, primarily consisting of H, with about 25\% ${}^4$He and
about 2\% higher mass elements, known as {\it metals}. It produces
light through nuclear fusion of H to ${}^4$He in its core. Its mass
$M_\odot = 2\times 10^{33}$ g and its radius is $R_\odot = 0.7 \times
10^6$ kc. The distance from the Earth is $1.4\times 10^8$ km = 1 {\it
  Astronomical Unit} (AU). The nearest other stars are very far away,
typically {\it parsecs} ($\approx 2 \times 10^5$ AU, or about 3.3
lightyears) away.

\citet{blandhawthorn16a} describes the Milky Way galaxy. The Sun
orbits the Galaxy in a roughly circular orbit at about 8.5 kpc from
the center, with a velocity of $\sim 220$ km s$^{-1}$, along with most
stars in the Milky Way's disk. The Milky Way visible by eye in the
night sky is resolved into many billions of stars, with a total {\it
stellar mass} of about $3\times 10^{10}$ $M_\odot$. Like many
galaxies, the Milky Way has a {\it thin disk} (a few 100 pc) thick, a
{\it thick disk} (about 1 kpc thick), a {\it bulge} and a {\it bar} in
its center, and a {\it stellar halo} that extends out to about 100
kpc. It has a moderately large black hole ($4\times 10^6$ $M_\odot$)
in its center, a relatively small {\it supermassive black hole}. The
disk has neutral and molecular gas as well as dust, and outside the
disk and surrounding the Milky Way is a diffuse halo of gas. The
molecular gas regions are forming young stars only a few million years
old. The oldest stars in the galaxy appear to be about 10 billion
years old. Orbiting the Milky Way are about 150 globular clusters,
which are roughly spherical, very old, bound stellar systems.

The Milky Way is filled with dust consisting of particles of up to a
fraction of a micron in size. This dust reddens and extincts light
passing through it. Using infrared satellite observations of 100
$\mu$m emission from this cold dust, we have estimates of the amount
of dust in front of any object outside of our galaxy
(\citealt{schlegel98a}). All extragalactic observations must be
corrected for this effect. Near the Galactic Plane, the dust
extinction is extremely high, causing the {\it Zone of Avoidance} for
galaxy maps. The dust greatly hinders study of the Milky Way disk in
the optical, making objects fainter and very difficult to correct.

The galaxy exists within the deep potential well of a {\it dark
  matter} halo, which is only detectable today through its
gravitational influence. The rotation velocity stays close to flat at
220 km s$^{-1}$ out to at least 20 kpc. The total dark matter mass
appears to be about $10^{12}$ $M_\odot$ and it extends past 100
kpc. Numerous lines of evidence suggest that the dark matter is not
{\it baryonic} (i.e. not ordinary standard model particles) and
interacts with baryonic matter primarily gravitationally.

The nearest galaxies to the Milky Way are its {\it dwarf galaxy}
satellites (\citealt{mcconnachie12a}), the largest of which are the
Large and Small Magellanic Clouds (LMC and SMC). These two satellites
are visible from the Southern Hemisphere with the naked eye. They are
about 10\% and 1\% of the Milky Way's luminosity and are about 50 kpc
away, and therefore within the Milky Way's dark matter halo. Many
dozens of other satellites are known.

The nearest galaxy of comparable size to the Milky Way is Messier 31,
also known as M31 or Andromeda. It is about 800 kpc away, and moving
toward the Milky Way, indicating the two are part of a bound group
that will eventually merge. M31 is somewhat more luminous then the
Milky Way and differs in a number of important details, most obviously
having a larger bulge relative to its disk.

There are nearly a billion other detected galaxies, and the census of
galaxies suggests there are hundreds of billions total in the
observable universe. They have a mean separation of a few Mpc, but are
not uniformly distributed. Instead they exist in dense clusters,
connected by filaments and walls, with void regions in between. These
large scale structure form from initial primordial density
fluctuations through gravitational growth. The galaxies have a range
of luminosities, with a characteristic exponential cutoff in number
density at high luminosity, called $L_\ast$ (a bit brighter than the
Milky Way's luminosity) and a power law distribution of luminosity
below that.

These galaxies come in a number of varieties. \citet{hubble36a}
classified galaxies with comparable luminosities to the Milky Way from
{\it early-type} or {\it elliptical} galaxies, through {\it late-type}
or {\it spiral} galaxies.  Elliptical galaxies old, red, and
puffy. Spiral galaxies are younger, bluer, and have cold thin
disks. An apparently intermediate variety of {\it lenticular} or {\it
S0} galaxies have disks like spiral galaxies, but are puffier and do
not have spiral structure. There are {\it irregular} galaxies of
various types. Dwarf galaxies tend to deviate from Hubble's system in
detail, as do distant galaxies observed as they were when the universe
was younger.

The dynamics of stellar systems, revealed through Doppler shifts or
proper motions, reveals their matter density. A characteristc mass can
be derived through the {\it virial relation}, which dimensionally is:
\begin{equation}
v^2 \sim \frac{GM}{R}
\end{equation}
for characteristic velocity $v$, mass $M$, and radius $R$. Stellar
dynamics theory yields a {\it virial theorem}, which define more
precisely what we mean by these characteristic quanties. The virial
theorem can be quantified as $U = -2 K$, where $U$ is the total
potential energy and $K$ is the total kinetic energy.

As one looks at galaxies of greater and greater distance, one finds
that they are receding with a velocity $v= H_0d$, where $H_0 \sim 70 $
km s$^{-1}$ Mpc$^{-1}$ (\citealt{freedman10a}). The line-of-sight
velocities are determined through their Doppler shift, the recession
can also be quantified by the {\it redshift} $z$ relating
$\lambda_{\rm obs} = (1+z)
\lambda_{\rm em}$. At low redshifts, we can relate velocity and
redshift with $v=cz$. The Hubble recession is what we mean when we say
the universe is expanding. A rough calculation of the age of the
universe from this expansion yields 14 billion years, which is
remarkably close to the right answer.

Essentially every luminous galaxy has a supermassive black hole at its
center. Larger galaxies tend to have larger black holes, ranging up to
about $10^9$ $M_\odot$. These black holes presumably grew through
accretion. During episodes of accretion, these black holes can become
much more luminous than their host galaxies. Accreting black holes are
referred to as {\it active galactic nuclei} (AGN), and the most
luminous ones are referred to as {\it quasars}. These quasars were
most common about 10 billion years ago, corresponding to redshifts of
$z\sim 2$--$3$. They can be used as backlights upon which gas
aborption signatures are imprinted, and thus reveal the gas
distribution throughout the universe.

The relative abundance of different atomic elements in the Universe is
known. By mass, the hydrogren fraction is $X \sim 0.75$, the helium
fraction is about $Y \sim 0.24$, and the fraction of other elements is
$Z \sim 0.01$, with about half of that in oxygen. The elements more
massive than helium are collectively known in astronomy as {\it
metals} and their fraction in any system is known as the {\it
metallicity} of the system, even though only roughly half of those
elements would be referred to by a chemist as a metal or
metalloid. Metallicity is usually quantified on a log scale relative
to solar metallicity, with notation:
\begin{equation}
[Z/H] = \log_{10} \frac{N_Z / N_H}{N_{Z, \odot} /
N_{H, \odot}}, 
\end{equation}
with similar definitions for individual elemental abundances
(e.g. [Fe/H], [O/H], etc.). For the Sun, $Z\approx 0.02$, but note
that the elemental abundances of the Sun are still subject to periodic
revision.

\section{Important numbers}

\begin{itemize}
\item $1 {\rm ~eV} = 1.602 \times 10^{-12} {\rm ~erg}$
\item $c = 2.99792 \times 10^8  {\rm ~m} {\rm ~s}^{-1} $
\item $G = 6.6738 \times 10^{-11} {\rm ~m}^3 {\rm ~kg}^{-1} {\rm s}^{-2} $
\item $h = 2\pi \hbar = 6.626 \times 10^{-27} {\rm ~erg} {\rm ~Hz}^{-1} =
4.136 \times 10^{-15} {\rm ~eV} {\rm ~Hz}^{-1}$
\item $k_B = 1.3806503 \times 10^{-23} {\rm ~m}^2 {\rm ~kg} {\rm
~s}^{-2} {\rm ~K}^{-1} = 8.617 \times 10^{-5} {\rm ~eV} {\rm ~K}^{-1}$
\item $m_p =  1.6726 \times 10^{-27} {\rm ~kg} $
\item $m_n =  1.6749 \times 10^{-27} {\rm ~kg} $
\item $m_e =  9.1049 \times 10^{-31} {\rm ~kg} $
\item $M_{\rm Earth} = 5.974 \times 10^{24} {\rm ~kg} $
\item $M_{\odot} = 1.989 \times 10^{30} {\rm ~kg} $
\item $R_{\odot} = 6.955 \times 10^{8} {\rm ~m} $
\item $T_{\odot} = 5500 {\rm ~K} $
\item $L_{\odot} = 3.828 \times 10^{33} {\rm ~erg} {\rm ~s}^{-1}$
\item $L_{\ast} \sim 10^{10} L_{\odot} $
\item ${\rm 1~AU} = 1.496 \times 10^{11} {\rm ~m} $
\item ${\rm lightyear} = 9.461 \times 10^{15} {\rm ~m} $
\item ${\rm parsec} = 3.086 \times 10^{16} {\rm ~m} $
\item ${\rm year} = 3.156 \times 10^{7} {\rm ~s} $
\item $H_0 \approx 70 {\rm ~km} {\rm ~s}^{-1} {\rm ~Mpc}^{-1}$     
\end{itemize}

\section{Key References}

\begin{itemize}
  \item {\it Extragalactic Astronomy and Cosmology: An Introduction,
    \citet{schneider15a}}
\end{itemize}

\section{Order-of-magnitude Exercises}

\begin{enumerate} 
\item Estimate the mean distance between stars in the Milky Way disk
  in units of the solar radius. Are stellar collisions likely to be
  particularly common?

\begin{answer}
The disk extends to at least 8.5 kpc radius, is has about $10^{10}$
stars, and is a few hundred pc thick. The mean density of stars is
therefore:
\begin{equation}
n \sim \frac{10^{10}}{\pi (300 {\rm ~pc}) (10 {\rm ~kpc})^2} \sim
\frac{10^{10}}{10^{11} {\rm ~pc}^3} \sim  0.1 {\rm ~pc}^{-3} 
\end{equation}
This means the mean distance between stars is:
\begin{equation}
d \sim n^{-1/3} \sim 2 {\rm ~pc} \sim 2 {\rm ~pc} \times \frac{3 \times
  10^{16} {\rm ~m}}{1 {\rm ~pc}} \times \frac{R_\odot}{7 \times 10^8 {\rm
    m}}
\sim 10^8 R_\odot
\end{equation}
These distances are very large relative to stellar radii. We can go
further and ask for a relative velocity of $\sim 200$ km s$^{-1}$ (an
overestimate) what over 10 billion years is the probability that any
two stars will collide.
\begin{eqnarray}
  p &=& n (\pi R_\odot^2) v t = d^{-3} (\pi R_\odot^2) v t \cr
  &\sim& \pi (10^{-24} R_\odot^{-3}) R_\odot^2 \left(3 \times
  10^{-4} R_\odot {\rm s}^{-1}\right) \left(3 \times 10^{17} {\rm
    s}\right) \cr
  &\sim& 3 \times 10^{-10} 
\end{eqnarray}
Thus, the chances for any individual star to actually collide with
another is very small. Even accounting for the fact that there
$10^{10}$ in the Milky Way indicates that the total rate of encounters
is of order unity per 10 billion years.
\end{answer}
\item Use dimensional analysis to derive the virial relationship
between size and mass in an equilibrium gravitating system.

\begin{answer}
A gravitating system has a characteristic radius $R$, mass $M$, and
velocity $v$ of its orbiters.  Newton's laws introduce the constant
$G$, with units m$^3$ kg$^{-1}$ s$^{-2}$. The relationship between the
four relevant quantities can be expressed as:
\begin{equation}
v \propto G^\alpha M^\beta R^\gamma
\end{equation}
To match the mass, length, and time units respectively:
\begin{eqnarray}
0 &=& -\alpha + \beta \cr
1 &=& 3\alpha + \gamma \cr
- 1 &=& -2 \alpha
\end{eqnarray}
Therefore, $\alpha = 1/2$, and therefore $\gamma = - 1/2$ and
$\beta=1/2$. If we square both sides of the relationship we find:
\begin{equation}
v^2 \propto \frac{G M}{R},
\end{equation}
which up to a factor of order unity is the virial relation. 
\end{answer}
\item Estimate the approximate dynamical mass interior to the Sun. 

\begin{answer}
  Assuming a circular orbit at 8.5 kpc of 220 km s$^{-1}$, we can use
  the force law:
  \begin{equation}
   a = \frac{GM}{r^2} = \frac{v^2}{r}
  \end{equation}
  to infer:
  \begin{eqnarray}
   M &=& \frac{v^2 r}{G} =  \frac{(2.2 \times 10^5 {\rm ~m} {\rm
       ~s}^{-1})^2 (8500 \times 3 \times 10^{16} {\rm ~m})}
   {6.7 \times 10^{-11} {\rm ~m}^3 {\rm ~kg}^{-1} {\rm s}^{-2}} \cr
   &\sim& 2\times 10^{41} {\rm ~kg} \sim 10^{11} M_\odot 
  \end{eqnarray}
  This is only somewhat more than the total mass inferred in stars. The
  real evidence for  dark matter comes when you consider that the
  rotation curve remains constant (``flat'') to much larger radii,
  continuing to add to the discrepancy.
\end{answer}
\item Use the Hubble Law to estimate the age of the Universe.

  \begin{answer}
    We use the standard estimate of time from distance velocity:
    \begin{equation}
      t = \frac{d}{v} = \frac{d}{H_0 d} = H_0^{-1}
    \end{equation}
    That is, there is a time in the past at which all the galaxies
    were apparently at a single point. This is time of the Big Bang,
    and the {\it Hubble time} is:
    \begin{equation}
      H_0^{-1} = \left(70 \frac{{\rm ~km} {\rm ~s}^{-1}}{{\rm
          ~Mpc}}\right)^{-1} =  \frac{3 \times 10^{22} {\rm ~m}}{7
        \times 10^4 {\rm ~m} {\rm ~s}^{-1}} \sim 4.3 \times 10^{17}
      {\rm ~s} \sim 1.4 \times 10^{10} {\rm ~yr} \sim 14 {\rm
        ~billion~years}
    \end{equation}
    This estimate assumes that the galaxies have been traveling at
    constant velocity. However, in reality the mass density of the
    universe causes deceleration at early times, and at late times
    there is an unexplained acceleration called ``dark energy.'' At
    the current time, these two effects tend to cancel and the above
    estimate of the universe's age is correct to better than 5\%.
  \end{answer}
\end{enumerate}   

%\section{Numerical and Data Exercises}

%\begin{enumerate}   

% RA, Dec
% Galactic coordinates

%\end{enumerate}   

\bibliographystyle{apj}
\bibliography{exex}  
