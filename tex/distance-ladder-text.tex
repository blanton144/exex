\title{\bf Distance Ladder}

\section{Basics \& Nomenclature}

Determination of distance is a fundamental problem in astronomy, to
which is owed numerous breakthroughs. The distance to nearby galaxies
has been established through a calibrated {\it distance ladder},
refined over the past century.

The distance ladder begins with the measurement of {\it parallax},
which is the apparent motion of sufficiently nearby stars due to the
motion of the Earth around the Sun. Precise measurements of stars over
time in a fixed reference frame yield {\it proper motions} and
parallax-based distances. Most recently, the Gaia satellite has
provided a large sample of stars with measured parallaxes reaching
across our Galaxy.

To reach distances outside our galaxy, we must rely on less direct
measurements of distance. Usually this involves finding objects or
phenomena that have pairs of observables with a known relation but
with different dependences on distance.

Many of these techniques fall into the {\it standard candle} category:
a luminosity is estimated by some means, and a flux is measured, and
the distance is inferred from the inverse-square law $f=L/4\pi D^2$.
The most prominent standard candles are:
\begin{itemize}
\item {\it RR Lyrae} variable stars. These stars are relatively low mass
  horizontal branch stars. They vary due to the $\kappa$ mechanism,
  which is an instability associated with the reaction of the
  atmospheric opacity to changes in pressure and density that causes
  oscillations in stellar size and luminosity.  They have periods of
  order a day. They are prominent in globular clusters but can also be
  found throughout the Milky Way. Historically they have been most
  useful for determining distances within the Galaxy. However, the
  advent of Gaia and the James Webb Space Telescope opens up the
  possibility of observing these stars in nearby galaxies.
\item {\it Cepheid} variable stars. These stars are high mass and
  luminous.  They also vary due to the $\kappa$ mechanism but have
  longer periods. Historically, they have provided one of the most
  direct links between local distance scales and distant galaxies.
\item {\it Type Ia supernovae}. These supernovae result from the
  nuclear detonation of white dwarfs, probably due to accretion from
  or collision with a binary companion. The Phillips relation connects
  the time scale of the supernova light curve to the luminosity of the
  supernova. These sources were used to provide the first definitive
  evidence for the existence of cosmic acceleration.
\end{itemize}
Historically, a number of other distance measurements have also been
used that could be categorized as standard candle distances. Some
methods have relied on other supernova types. Other methods have
generally capitalized on scaling relations of galaxies, such as the
Tully-Fisher relation for spirals relating luminosity and circular
velocity, and the Fundamental Plane for ellipticals relating
luminosity, size, and velocity dispersion. The difficulty in
implementing these methods in a way that is free of systematic errors
generally has led to the use of supernovae for the most recent
measurements of the distance ladder.

Several standard candle techniques on the horizon may prove useful in
the coming years. I will discuss masers and gravitational waves:
\begin{itemize}
\item Under certain conditions the supermassive
black holes at the centers of galaxies may contain H$_2$O
masers. Under certain conditions, these systems can be sufficiently
clean that one may measure angular velocities of circular orbits and
angular sizes, which yields a period. Combined with a Doppler
velocity, this combination yields the mass of the black hole and the
distance. Only one clean enough system is known, about 8 Mpc away in
NGC 4258.
\item Colliding compact objects whose gravitational waves are detected can
yield constraints on distance from the gravitational waveform, because
the total mass of the system is related to both the waveform frequency
and the total energy. If the redshift is known from an electromagnetic
counterpart, then the rest-frame waveform frequency can be determined,
and the object can be placed on the redshift-distance relation.
\end{itemize}

A major application of these methods is to compare the redshifts and
distances of local galaxies. This comparison led Hubble to the Hubble
Law:
\begin{equation}
v = H_0 d
\end{equation}
where $v$ is velocity, $H_0$ is the Hubble constant, and $d$ is the
distance. Gravitational attraction causes galaxies to move toward each
other with respect to this flow, with {\it peculiar velocities} of
100s of km s$^{-1}$. For example, M31 is moving toward us at 400 km
s$^{-1}$, and large clusters, such as Virgo and Coma, have internal
motions and thus peculiar velocities of 1000s of km s$^{-1}$. Thus,
determinations of the Hubble constant need to use galaxies
sufficiently far to reduce this effect. Alternatively, they need to
estimate and correct for these peculiar velocities, using the density
field of galaxies itself to estimate the magnitudes and directions of
the expected velocities.

The Hubble Law is a fundamental measurement of the nature of the
universe we are in, but it also is a tool to map the universe on
larger scales, since measuring Doppler shifts is relatively easy
compared to galactic distances.

\section{Commentary}

Briefly, the best current estimates of the local Hubble constant come
from the Riess et al. papers. These use a number of Milky Way Cepheids
with parallax distances to anchor extragalactic Cepheids in galaxies
with SNe Ia, and use those to anchor a much larger SN Ia data set. As
of 2018, the results for $H_0$ from these analyses tend to be
considerably higher (several sigma) than those inferred from cosmic
microwave background and large scale structure measurements, using
standard cosmological models to extrapolate to $z=0$. The cause of
this discrepancy is not known.

\section{Key References}

\begin{itemize}
  \item
    {\it Riess et al}
  \item
    {\it Kitchin}
\end{itemize}

\citet{gunn06a}

\section{Order-of-magnitude Exercises}

\begin{enumerate} 
\item Typical parallax at 1pc, 10pc, 1kpc, 1 Mpc; diffraction limit necessary
  to determine them.
\item Translation between error of individual object to error in
Hubble
\item Estimate age of the universe from the Hubble constant
\end{enumerate} 

\section{Analytic Exercises}

\begin{enumerate}
\item Something about the masers?
\end{enumerate}

\section{Numerics and Data Exercises}

\begin{enumerate}
\item 
\end{enumerate}

\bibliographystyle{apj}
\bibliography{exex}  
