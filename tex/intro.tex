\documentclass[11pt, preprint, maxnames=9]{aastex}
\usepackage{hyperref}
\usepackage{rotating}
 
\setlength{\footnotesep}{9.6pt}

\newcounter{thefigs}
\newcommand{\fignum}{\arabic{thefigs}}

\newcounter{thetabs}
\newcommand{\tabnum}{\arabic{thetabs}}

\newcounter{address}

\begin{document}

\title{\bf Introduction}
~

Extragalactic astrophysics is a broad and rapidly evolving field. It
sits at the intersection of cosmology, stellar astrophysics, stellar
dynamics, and nuclear astrophysics. Much of our knowledge depends on
empirical observation or detailed numerical modeling, rather than
being derivable from first principles. These facts make good graduate
textbooks in the field extremely difficult to write. 

There is a great, broad textbook by Schneider that yields very
complete coverage of the field; because it is a textbook it is not
quite up to date, but it does contain all of the important
topics. There are a number of text books of somewhat narrower focus
that contain both fundamental astrophysics and thorough literature
reviews (particularly Binney \& Merrifield and Binney \& Tremaine).

The approach of my text here is to use exercises to provide a
practical introduction to the major aspects of theory and observation
underlying extragalactic astrophysics. This yields not quite as broad
a scope as Schneider, does not cover fundamentals as deeply as other
texts, and of course is not quite at the frontier of the field.  This
text necessarily relies on reference to a broader set of books that
provide a more complete background on physics and astrophysics. But
it does deliver essential information on experimental techniques,
analysis techniques, and particularly on phenomenology, that is hard
to deliver in any other form.

Inspiration for this approach came from the {\it Problem Book in
  Relativity and Gravitation}. However, in extragalactic astrophysics
the problems must include numerical and data-oriented questions to
introduce the subject and to demonstrate its empirical and
phenomenological aspects.

The topics that we are aiming to cover are:
\begin{enumerate}
  \item Astronomical measurements of light.
  \item The cosmological context for extragalactic observations.
  \item Surveys of resolved stellar populations in the Local Group.
  \item How stellar populations produce galaxy spectral energy
    distributions.
  \item Galaxy redshift surveys, and galaxy properties and evolution.
  \item Dynamics of stellar systems.
  \item Baryon cycle in the interstellar medium of galaxies.
  \item Measurements of galaxies from gravitational lensing.
  \item Supermassive black holes and active galactic nuclei.
  \item Galaxy formation theory.
\end{enumerate}

The activities in {\it Exercises in Extragalactic Astrophysics} fall
into three categories:
\begin{enumerate}
\item {\it Order-of-magnitude exercises}: These exercises are
  quantitative and meant to yield a sense of the units and
  quantities. They just require pencil and paper and usually no more
  information than is available in the material.
\item {\it Analytic exercises}: These exercises usually require
  physics and mathematics knowledge, in addition to the information
  available in the material. They just require pencil and paper.
\item {\it Numerics and Data Exercises}: These exercises require
  either minor computational work (computing integrals or derivatives)
  or minor analysis or plotting of astronomical data. 
\end{enumerate}

For the numerics and data exercises, in principle student can do these
any way they want. However, we provide answers using Python, under
Jupyter Notebooks. Some of the answers can be run on a laptop. Others
need to use the SciServer Compute system.  These solutions often take
a pedagogical approach, which is usually less efficient than the
approach one would take in a research problem.  However, where
convenient we illustrate the ``professional'' approach to solving the
problem. Note that the convenience of Python and the professional
approaches illustrated here will change over time, but hopefully the
solutions will still yield insight.

For solving the numerical and data problems, some students may need
extra introduction. In order of increasing sophistication we recommend
the following resources:
\begin{itemize}
\item If you need to get your feet wet in Python without actually
  installing it on your laptop, go to SciServer
  Compute\footnote{\tt \url{http://sciserver.org}} and learn how to
  open a notebook and start a Python shell.
\item Assuming you have some familiarity with some programming
  language, but perhaps don't know Python, the official Python
  tutorial\footnote{\tt \url{https://docs.python.org/3/tutorial/}} is as
  good a place to start as any.
\item You need 
  NumPy\footnote{\tt \url{
    https://docs.scipy.org/doc/numpy-dev/user/quickstart.html}} and
  SciPy\footnote{\tt \url{ 
    https://docs.scipy.org/doc/scipy-0.18.1/reference/tutorial/index.html}}. You
  may not want to go through all of those tutorials immediately but
  they are available if needed. 
\item You need AstroPy\footnote{\tt \url{
  http://astropy.org}}, which also has documentation and tutorials of
  use.
\item Some of the answers as given rely on SciScript to access the
  SDSS
  CAS.\footnote{\url{https://github.com/sciserver/SciScript-Python}}
\item You may want to use Python on your laptop. Chances are it is
  installed with Python, but you will still want to use a distribution
  that you can customize without administrative
  privileges. Anaconda\footnote{\tt \url{https://www.continuum.io/}} is as
  good a distribution as any, and good for scientific use. You can use
  conda or pip to further customize it. To do this, if you are on Unix
  or Mac OS X but not familiar with the Unix shell, do consult the
  Unix shell tutorial at Software Carpentry\footnote{\tt \url{
    https://software-carpentry.org/}}.
\end{itemize}
The solutions in this book will provide some information on Python
usage and capability. Note however that (at least partly by design for
clarity) they are not always in the most ``Pythonic'' form or use the
best software engineering practices.

There are a number of references in the notes to books and
articles. In general, the articles can be found with a search of the
\href{https://ui.adsabs.harvard.edu}{Astrophysics Data System}.

In the long term, the implementation used here will necessarily become
dated as new observations and tools develop. I can only hope that it
provides a useful set of exercises to students today and a durable
model for future efforts in this direction.

\end{document}

